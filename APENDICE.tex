%!TEX root = LIVRO.tex
\part{Apêndice}
\blankpage

\chapterspecial{Franz Kafka e o mundo invisível}{}{Otto Maria Carpeaux}
%Correio da Manhã (17 de abril de 1941) 

\noindent{}O mundo do contista Franz Kafka é uma casa burguesa, solidamente construída na aparência, com uma fachada um pouco descuidada. Entramos, e respiramos o ar das penúrias dolorosas, de quartos mal ventilados. Apodera-se de nós o sentimento
do \textit{déjà vu}, de já ter visto tudo isso. A escada range. O sótão é uma
loja de recordações. Um canto guarda os brinquedos esquecidos. Recordações,
recordações. Os mortos surgem. Os fantasmas que apavoravam a criança.
Figuras de demônios. Um labirinto. Delírio. Fuga. Nenhuma saída.
Voltamo-nos para o outro lado: aparece a face de Deus.

Franz Kafka não é um poeta religioso: não trata nunca de religião nas suas obras.
Mas é um espírito profundamente angustiado; e o seu mundo é cheio de
seres sobrenaturais, donde emana uma impressão inquietante, como o
encontro com uma mitologia desconhecida, que aparecesse, de repente, na
nossa vida quotidiana. Esta irrupção do sobrenatural no mundo não o
salva: enche o homem de terrores desconhecidos. O \textit{numen} de Kafka é um \textit{numen tremendum}. A religião de Kafka não é a religião fácil dos bem-pensantes, a quem o seu Deus garante todas as
ordens deste mundo; o Deus de Kafka faz estremecer os fundamentos do céu
e da terra. ``Minha fé é como uma guilhotina, assim leve e assim
pesada.'' É a ameaça mortal que antecede a esperança vital.

Esta é a religião daqueles que a psicologia religiosa de William James chama os \textit{twice-born},
aqueles que nascem duas vezes, aqueles cuja fé irrompe das convulsões duma agonia: Agostinho, Martinho Lutero, Blaise Pascal, Søren Kierkegaard.

Esses terrores e esses esplendores, Kafka os escondeu nos andrajos da vida quotidiana, pois ``quem vir descoberta a face de Deus morrerá''.

A pessoa e a vida de Franz Kafka acham-se também cobertas por um véu.
Nasceu em 1883 em Praga, filho de família pequeno-burguesa, dessa
nacionalidade incerta, germano-tcheco-judia, característica dos meios
intelectuais dessa cidade. Desde a sua infância, o humanismo alemão
desses meios é flanqueado pelo cabalismo
judaico e pelo misticismo eslavo.

\begin{verse}
Estou\\
Limitado ao norte pelos sentidos, ao sul pelo medo\\
A leste pelo apóstolo São Paulo, a oeste pela minha educação.\\
Murilo Mendes
\end{verse}

A vida corre-lhe nos quadros da burocracia subalterna.
Tísico, morre num sanatório de Viena, em 1924. No testamento ordena a destruição dos seus
manuscritos, que o executor, Max Brod, editará arbitrariamente.

A sua obra se compõe: de aforismos, que se alongam às vezes em parábolas; de parábolas, que se estendem às vezes em contos; de contos, dos quais três
se desenvolvem em romances, fragmentários, da mais alta concisão, e
cujo assunto se poderia condensar em parábola ou aforismo. A língua é
muito límpida, carregada de estranhas metáforas. Kafka descreve a vida
quotidiana dos escritórios, dos cafés, das casas de família; mas esses
lugares banais são cheios de potenciais demoníacos,
contra os quais o homem luta desesperadamente. Esse misto de clareza e
de mistério revela a fragilidade do nosso mundo, espreitado pela
catástrofe. Acontecimentos simples revestem-se de uma tensão febril. A língua lúcida faz
adivinhar um outro mundo. As personagens falam, comem, dormem, seguem os
caminhos escuros e estreitos; mas são os caminhos do inferno e do
paraíso, são os caminhos \textit{per realia ad realiora}.\footnote{``Pelas coisas reais ao mais real.''}

O primeiro romance publicado depois da morte do autor foi \textit{O processo}. O
seu herói chama-se K., simplesmente K. Um dia, na rua, K. é subitamente
preso. Explicam-lhe que fora instaurado contra ele um importante
processo criminal; aconselham-no a confessar e, em seguida, soltam-no
a fim de que possa prosseguir na sua defesa. A prisão não passava de uma
provocação por parte daquele estranho tribunal: o próprio K. tem de
criar pelas suas atitudes as razões da sua absolvição ou condenação. E
cria o delito mortal, prevalecendo-se obstinadamente da sua inocência.
Faz tudo o que se pode fazer: contrata um advogado e um médico, corrompe
o carcereiro e o escrivão. Nenhum destes compreende melhor o processo do
que K., mas todos estão convencidos da justiça e da onipotência do
tribunal; aconselham-no a confessar um crime que K. não conhece e não
quer conhecer. De maneira misteriosa, todos são empregados do tribunal, assim como nós outros
executamos, sem o saber e sem o querer, os desígnios da Providência. Pelas suas
atividades, K. não faz mais que jogar o processo contra si mesmo. Obstina-se.
Pelas suas providências apressa a catástrofe que será a sua condenação e
execução. O delito desconhecido está vingado.

\textit{O processo} é um apólogo e uma apologia, ao mesmo tempo. Sob o véu da
alegoria, Kafka instrui uma acusação contra a justiça do tribunal
divino. O delito desconhecido é o pecado original. A prisão é o signo da
predestinação. E o que K.
evita pelas suas falsas atividades é a graça. Há nesse romance uma
lembrança incerta de certas palavras do Senhor: ``Muitos serão os
chamados, mas poucos os eleitos'', e ``Aquele que quiser salvar sua vida
a perderá''. Mas as palavras evangélicas perdem-se neste mundo de
provação e desespero, onde a todo momento o
tribunal está presente e a força armada. ``É somente a noção que temos
do tempo'' --- diz Kafka --- ``que nos faz datar o juízo final; na verdade é uma corte
  marcial cuja audiência está aberta todos os nossos dias.'' Mas o céu
  negro se iluminará, um dia, sobre estas cenas de horror. No seu diário
  Kafka copiou as palavras de Lutero: ``Deus não é inimigo dos
  pecadores, mas somente dos descrentes que não reconhecem os próprios
  pecados nem procuram o apoio de Cristo, mas que procuram,
  temerariamente, a purificação em si mesmos''.

Em torno deste romance, alguns contos explicam a situação metafísica do homem. \textit{A colônia penitenciária} é uma como espécie de continuação de \textit{O
processo}. Nesta colônia, uma terrível máquina de precisão grava no corpo
dos forçados, por meio de agulhas incandescentes, os nomes dos delitos,
que são desconhecidos dos próprios condenados. A tortura pela qual a sua culpa lhes será revelada
é a única esperança, pois saber o nome do delito é a condição preliminar para
saber justificar-se.

Em \textit{A transformação}, um jovem é subitamente transformado num horrível inseto que os seus próprios parentes querem matar. O homem, submergido
pela vida banal de todos os dias, não é mais a imagem de Deus; não se
pode deter essa queda onde se desejaria, em alguma etapa propícia; e a queda torna-se radical até se perder o direito de existir.

A transformação tornou-se definitiva nesta pequena obra-prima chamada \textit{A preocupação do Pai Celeste}. É objeto da inquietação do Pai
misericordioso uma bobina, destituída de fios; coisa absolutamente inútil, sem nenhuma significação, mas
que não descansa nunca, que sobe e desce incessantemente a escada, até o
último dia. --- ``Como te chamas?'' --- ``Odradek''; palavra eslava, de
origem incerta, que significa ``apóstata''.

Em todas essas parábolas,
como em \textit{O processo}, o homem é a vítima passiva da perseguição celeste,
lembrando \textit{Hound of Heaven}, de Francis Thompson. Mas Kafka não condena a
atividade: ``Há dois pecados cardeais donde se poderiam deduzir todos
os outros: a impaciência e a preguiça. Por causa da impaciência foram
expulsos do paraíso; por causa da preguiça lá não podem voltar''. O que Kafka deseja excluir é a falsa direção das nossas atividades, no sentido
da segurança neste mundo. No conto \textit{A toca de texugo}, o animal, temendo a perseguição dos
cães, decide alargar e fortificar o seu edifício subterrâneo. Cava
buracos sobre buracos, corredores sobre corredores, até que afinal esquece a única saída. Então o animal agacha-se no seu canto,
aprisionado e sem saída, e espera, indefinidamente, numa estranha solidão, atento aos ruídos funestos do mundo exterior, ou ao silêncio,
ainda mais terrível.

A falsa direção das atividades humanas é o assunto
da obra-prima de Kafka: o romance inacabado \textit{O Castelo}.

Ainda aqui o herói chama-se K., simplesmente K. O seu adversário não é, desta vez, o
tribunal, mas o Castelo, o lugar onde a graça está concentrada. Ao pé
desse Castelo há uma aldeia, onde os camponeses, crentes humildemente submissos, executam as suas tarefas diárias. K. também desejaria ser camponês nessa
aldeia. É preciso frisar: ele o quer, ele o exige mesmo. Desejaria obrigar o Castelo a conceder-lhe o direito de permanência na aldeia. Quer forçar esta comunhão dos fiéis, sem ter obtido a graça.

Numa fria tarde de inverno, K. chega, contando
com a piedade, que não fará voltar o peregrino. Com efeito, o hospedeiro
acolhe-o. K. é modesto; quer somente achar um emprego de diarista.
Sim, há sempre possibilidades. Nesse ínterim o filho do castelão aparece
para expulsá-lo. K. desesperadamente recorre à mentira: ``O Castelo contratou-me como nivelador''. Resolvem telefonar para o Castelo. E o
Castelo responde de maneira surpreendente (``K. estremeceu um pouco''): ``Sim, K. é o nivelador contratado''. É o primeiro dom voluntário da graça: mas
contém uma punição. Pois o Castelo acrescenta: ``K. tem, portanto,
permissão para ficar; mas o seu contrato foi um lamentável engano, aqui
não temos trabalho para um nivelador. K. tem permissão para ficar, mas
não para trabalhar''.

Deste modo, K. encontra-se impossibilitado de
verificar o contrato surrupiado, justificar sua presença na aldeia. Sua
vida será vazia, destituída de qualquer sentido, como a nossa vida
quotidiana sem a vocação interior. K. não está contente. Não quer ser
tolerado. Quer o direito de permanecer, o direito. Quer extorquir a
graça. Recorre a meios impuros, perde-se em mentiras e subterfúgios.
Tudo em vão. Esgotado, enfim, cai gravemente doente. Espera a morte.

Eis-nos nas últimas linhas do fragmento. Uma anotação explica-nos o fim:
``Quando K. está à morte, chega a decisão definitiva do Castelo: K. não
tem nenhum direito de permanecer na aldeia; mas considerando-se certas
circunstâncias acessórias, ser-lhe-á permitido que aí permaneça até a
morte''.

Em \textit{O Processo}, o Céu instaura processo contra o homem. Em \textit{O Castelo}, o homem instaura processo contra o Céu. É o cúmulo da
temeridade titânica. ``Uns negam a miséria evocando o sol; outros negam
o sol evocando a miséria.'' O homem, em Kafka, não vê na sua miséria a
consequência da sua condição humana. Revolta-se. Acusa Deus, como Ivan Karamázov. A face de Deus, em sua obra, adquire traços blasfemos.

Em toda parte, no mundo desse Deus, há tribunais e forcas. Não parece que
esse Deus queira a redenção do homem. ``O verdadeiro caminho desdobra-se
sobre uma corda, lançada muito perto do chão; parece ser destinada
mais a fazer tropeçar que a ser transposta.'' Às vezes Kafka atinge uma
inversão diabólica: ``Leopardos forçavam o templo e esvaziavam os vasos
sagrados. Isto se repetia frequentemente. Até que conseguiram calcular a
hora em que chegavam e faziam do incidente uma parte do cerimonial''.
Tais blasfêmias lembram a zoolatria dos egípcios ou o Demiurgo mau dos
gnósticos. Mas um outro aforismo diz: ``O nosso mundo não é mais do que
um mau humor de Deus. Há esperança, muita esperança, mas não para nós
homens''. Este ``não para nós homens'' equivale a uma grande confissão,
que restabelece a ordem dos valores. ``Todas essas parábolas dizem somente que o incompreensível é incompreensível.'' Na aparência dessas
parábolas Deus não tem razão; mas esta falta de razão significa somente
uma incapacidade do homem em face do mandamento de Deus. Na aparência
dessas parábolas, Deus se cala; mas isto significa somente que o mundo
não o está escutando. Há, portanto, esperança, muita esperança. No fim
de \textit{O Castelo}, a graça aparece. Fato simbólico: Kafka não estava
destinado a escrever esse fim.

Franz Kafka, segundo uma frase de
Kierkegaard, ``aspirava a uma imortalidade mais alta que a da glória''.
Kafka desejava que a sua obra morresse com ele para servir de testemunha
em seu favor, perante o tribunal de Deus. A despeito dele, o seu dia chegará, se já não chegou.

À propagação dessa obra opõem-se obstáculos do destino. A sua publicação póstuma não encontrou nem leitores nem críticos. Dez anos depois da sua morte,
um André Gide, um Charles Du Bos deploram a inacessibilidade das obras,
a inexistência de traduções. Uma casa editora de Praga promete a publicação das obras completas, a \textit{Nouvelle Revue Française} traduz alguns contos. A
edição de Praga é interrompida pela derrota do Estado tcheco. A tradução
integral, prometida na França, talvez nunca apareça. A despeito de tudo,
o seu dia chegará, se já não chegou.

Todos esses obstáculos aprofundam
mais a virtude desse pensamento, em vez
de sufocá-lo. Existe uma herança que se deve conservar. A reflexão sobre o lugar de Kafka na literatura universal é o primeiro dever.

Feita a abstração de alguns pontos de contato com Heinrich von Kleist, o Kleist
do ensaio \textit{Sobre o teatro de bonecas}, e com E.\,T.\,A Hoffmann, a presença
de Kafka na literatura alemã é simplesmente ocasional. O seu lugar está
na literatura europeia do após-guerra.

O simbolismo de Kafka perturba
o mundo, pela estranha transposição dos acentos, pela desvalorização dos fatos tradicionais, pela revelação de um mundo mais real atrás do mundo real dos bem-pensantes: \textit{per realia ad realiora}. Eis o lema de Anton Tchecov, a quem Kafka deve a técnica do conto. Mas
um traço significativo distingue Kafka radicalmente deste grande
contista pessimista do \textit{fin de siècle}: a noção do tempo. Os homens de
Tchecov vivem no seu tempo, no tempo do seu
mundo. Mas o tempo, em Kafka, é um fato extramundano. Não é o tempo
psicológico de Proust. É antes um tempo religioso: o caminho da aldeia ao castelo, ``dois quilômetros mais ou menos'', leva séculos, eônios,\footnote{} para ser
percorrido; não se pode dizer a respeito de nenhuma obra de Kafka em que
século decorre a ação dela. A era dos deuses e a vida quotidiana dos
nossos dias se confundem. Não existe tempo, há unicamente uma data: a da
irrupção do divino no mundo, acontecimento que se repete todos os dias, todas as horas.
%Paulo: não encontrei a palavra ``eônio''; deve ter havido algum erro aqui

Esta ausência do tempo humano destrói a estrutura
normal do mundo e isola os
homens em desertos de eternidade glacial, tomando-os comparáveis às
personagens plásticas de um De Chirico, aos cantos ``homófonos'' de um
Stravinsky, aos anjos de um Rilke. A psicologia desses homens é uma
psicologia de monstros revoltados, como nos romances fantásticos de
Julien Green. A sua vida quotidiana é destituída de sentido, como nos
contos de um Bontempelli. E a sua vida real se
passa na atmosfera mágica dos romances de Marcel Jouhandeau. Enfim, este
mundo acha a sua expressão final nos poemas apocalípticos dum
Pierre-Jean Jouve que precedem a catástrofe. O dia de Kafka chegou.

Todas essas comparações só têm como fim estabelecer mais solidamente as oposições. A corrente literária do após-guerra acha-se diante de um montão de 
ruínas. O mundo é um cadáver que se decompõe porque o espírito abandonou
o corpo. A literatura e o pensamento modernos tentaram contentar-se
somente com os destroços, olhando-os primeiro como brinquedos de uma
nova infância, e em seguida como pedras para a construção do futuro;
eram as etapas do primitivismo e do construtivismo. Mas se reconhecerá o
verdadeiro estado de coisas e um profundo desespero prevalecerá. Este
desespero se conformará ou não se conformará: ele afirma e confirma a
decomposição do mundo por meio de uma nova psicologia, ou se insurge
contra essa \mbox{decomposição} pelas expressões de um pessimismo cínico. São
estas as posições do romance e da poesia modernos.

O que é comum a todas
essa correntes é o relativismo, que já não admite a integridade do
mundo, exceto a daqueles, não raros, que mergulham na fé tradicional.
A atitude de Franz Kafka é muito diferente. Não se contenta com os
destroços, como os ``fragmentistas'' italianos; não se conforma nem
decompõe. Não é nem tradicional nem relativista. Entre dois mundos e
entre duas épocas, coloca-se em caminho; está a caminho de Damasco.

Esta atitude o situa no meio de duas grandes correntes dos nossos tempos: uma
na França, os novos estudos pascalianos que giram em torno do problema da
graça e inspiram até o André Gide de \textit{L'École des femmes}; a outra na
Alemanha, a ``Teologia Dialética'' de Karl Barth e de Emil Brunner,
que gira em torno do abismo dialético, a incomensurabilidade entre Deus
e o mundo, e faz ressuscitar a obra esquecida de Søren Kierkegaard.

No abismo entre o Deus soberano dos dialéticos e o homem falido de Pascal, Kafka procura o lugar da graça. É Pascal quem define a situação. No
artigo \textsc{xv} das \textit{Pensées}, enumera as quatro possibilidades do homem. Primeiro, o homem
conhece
a Deus, mas não conhece a sua própria miséria; é o caso do farisaísmo
orgulhoso. Segundo, o homem conhece a sua miséria, mas não conhece a
Deus; é o desespero ateístico. Terceiro, o homem conhece a Deus e a
sua própria miséria, mas não a graça; é a angústia. Quarto, o homem
reconhece em Jesus Cristo seu Deus, sua miséria e sua graça.

A posição de Kafka é a terceira. É a posição do judaísmo perante o seu Messias
encarnado. Mas é também a posição atual do mundo apóstata, que renuncia
à graça e se declara pagão, cheio de orgulho e de angústia. Não se é
mais pagão depois de Jesus Cristo: a velha inocência desapareceu; ou
procuramo-Lo, ou renegamo-Lo. Em vão ``a angústia da lei'' maltrata o
rabino Saul antes de ter ele visto a luz do mundo. Uma fé vem nascer no
caos de uma alma em desespero. ``Como cumprir a vontade de Deus? Teme-se
que essa lei não seja mais do que uma tentação. E se o seu cumprimento
não representar nada perante Deus?'' É um aforismo de Kafka.
Mas o apóstolo Paulo poderia ter dito isso. É a confissão de um homem no caminho de Damasco.

O caminho de Damasco é a única saída desta prisão que é o nosso mundo envenenado. Todos os outros caminhos são subterfúgios inúteis, tergiversações que nos abismam cada vez mais, sem a possibilidade de uma libertação.
Sem a graça não se escapa deste mundo. Todas as seguranças exteriores são vãs.
Em vão nos entrincheiramos nas linhas Maginot da nossa ``toca de texugo''.
Enfim, somos os prisioneiros das nossas próprias prisões, para assistir, impotentes, à
nossa derrota decisiva. Só o caminho misterioso de Damasco é que liberta dos terrores exteriores,
para preparar ``o segundo nascimento'': é o caminho do apocalipse do
mundo para a escatologia da alma.

A obra de Franz Kafka é um indicador
na direção desse caminho. Nela se lê o
seu aforismo, cheio de aflição e de esperança: ``Quem procurar não
encontrará; quem não procurar, será encontrado''. E uma voz lhe responde, através de Pascal:
\textit{Console-toi, tu ne me chercherais pas si tu ne m'avais trouvé}.\footnote{``Consola-te: não me procurarias se já não me tivesses encontrado.''}


\chapterspecial{Franz Kafka\footnotemark}{A propósito do décimo aniversário de sua morte}{Walter Benjamin}

\footnotetext{Texto originalmente publicado em 1934. Tradução de Sergio Paulo Rouanet.}

\subsection{Potemkin}

Conta-se que Potemkin sofria de graves depressões, que se repetiam mais
ou menos regularmente, e durante essas crises ninguém podia aproximar-se
dele, sendo o acesso a seu quarto rigorosamente proibido. Esse fato não
era mencionado na corte, pois se sabia que a menor alusão a respeito
acarretava o desagrado da imperatriz Catarina. Uma dessas depressões do
chanceler teve uma duração excepcional, o que ocasionou sérios
embaraços. Os papéis se acumulavam, e os assuntos, cuja solução era
reclamada pela czarina, não podiam ser resolvidos sem a assinatura de
Potemkin. Os altos funcionários estavam perplexos. Por acaso, entrou na
antecâmara da Chancelaria um amanuense subalterno, Chuvalkin, que viu os
conselheiros reunidos, queixando-se como de hábito. ``O que se passa,
Excelências? Em que posso servir Vossas Excelências?'' perguntou o
zeloso funcionário. Explicaram-lhe o caso, lamentando que não pudessem
fazer uso dos seus serviços. ``Se é só isso, meus senhores'', respondeu
Chuvalkin, ``deem-me os papéis, por favor.'' Os conselheiros, que não
tinham nada a perder, concordaram, e Chuvalkin, carregando pilhas de
documentos, atravessou galerias e corredores em direção ao quarto de
Potemkin. Sem bater, girou imediatamente a maçaneta. O quarto não estava
fechado. Potemkin estava em seu leito na penumbra, roendo as unhas, num
velho roupão. Chuvalkin foi até a escrivaninha, mergulhou a pena na
tinta e sem uma palavra colocou-a na mão de Potemkin, pondo o primeiro
documento nos joelhos do chanceler. Depois de um olhar ausente sobre o
intruso, Potemkin assinou como um sonâmbulo o primeiro papel, depois o
segundo, e finalmente todos. Depois que o último papel estava assinado,
Chuvalkin deixou o quarto com a mesma sem-cerimônia com que entrara, os
maços debaixo do braço. Brandindo-os no ar, entrou triunfante na
antecâmara. Os conselheiros se precipitaram sobre ele, arrancando-lhe os
papéis das mãos. Com a respiração suspensa, inclinaram-se sobre os
documentos. Ninguém disse uma palavra; o grupo estava petrificado. Mais
uma vez Chuvalkin acorreu, indagando por que os conselheiros estavam tão
estupefatos. Nesse momento, leu as assinaturas. Todos os papéis estavam
assinados: Chuvalkin, Chuvalkin, Chuvalkin\ldots{}

Com dois séculos de antecipação, essa anedota anuncia a obra de Kafka. O
enigma que ela contém é o de Kafka. O mundo das chancelarias e dos
arquivos, das salas mofadas, escuras, decadentes, é o mundo de Kafka. O
zeloso Chuvalkin, para quem tudo parece tão fácil e que acaba voltando
de mãos vazias é K., de Kafka. Potemkin, semiadormecido e abandonado
num quarto distante cujo acesso é proibido, vegetando na penumbra, é um
antepassado daqueles seres todo-poderosos, que Kafka instala em sótãos,
na qualidade de juízes, ou em castelos, na qualidade de secretários, e
que por mais elevada que seja sua posição, têm sempre as características
de quem afundou, ou está afundando, mas que ao mesmo tempo podem surgir,
em toda a plenitude do seu poder, nas pessoas mais subalternas e
degradadas --- os porteiros e os empregados decrépitos. Em que pensam,
mergulhados na semiescuridão? São talvez os descendentes de Atlas, que
sustentam o globo sobre seus ombros? É por isso, talvez, que sua cabeça
``está tão inclinada no peito, que seus olhos mal podem ser vistos'',
como o castelão em seu retrato, ou Klamm, quando está sozinho? Não, não
é o globo terrestre que eles sustentam, pois o cotidiano já é
suficientemente pesado: ``Seu cansaço é o do gladiador depois do
combate, seu trabalho consistia em pintar o canto de uma sala de
funcionário''. Georg Lukács disse uma vez: para construir hoje uma mesa
decente, é preciso dispor do gênio arquitetônico de um Miguel Ângelo.
Lukács pensa em períodos históricos, Kafka em períodos cósmicos. Caiando
um pedaço de parede, o homem precisa pôr em movimento períodos cósmicos.
E isso nos gestos mais insignificantes. De muitas maneiras, e às vezes
nas ocasiões mais estranhas, os personagens batem suas mãos. Kafka disse
uma vez, casualmente, que essas mãos eram ``verdadeiros pilões a
vapor''.

Travamos conhecimento com esses poderosos, em seu movimento contínuo e
lento, ascendente ou descendente. Mas eles não são nunca mais terríveis
que quando se levantam da mais profunda degradação, como pais. O filho
tranquiliza o velho pai, senil, depois de o ter posto na cama, com toda
ternura: ``--- Fica tranquilo, estás bem coberto. --- Não, gritou o pai
(afastando o lençol com tanta força, que ele se desdobrou no ar por um
instante), e ergueu-se no leito, tocando o teto de leve com uma das
mãos. Tu querias me cobrir, bem o sei, mas ainda não estou coberto. Essa
é a última força que me resta, mas ela é suficiente para ti, excessiva
para ti\ldots{} Felizmente um pai não precisa aprender a desmascarar seu
filho\ldots{} Ele ficou de pé, perfeitamente livre, movendo as pernas. Seu
rosto irradiava inteligência\ldots{} --- Sabes agora o que existia fora de
ti, ao passo que até hoje sabias apenas o que te dizia respeito! É
verdade que eras uma criança inocente, mas a verdade mais profunda é que
eras um ser diabólico!''. Ao repelir o fardo das cobertas, o pai repele
com elas o fardo do mundo. Precisa pôr em movimento períodos cósmicos
inteiros, para tornar viva e rica de consequências a imemorial relação
entre pai e filho. Mas que consequências! Ele condena o filho a morrer
por afogamento. O pai é a figura que pune. A culpa o atrai, como atrai
os funcionários da Justiça. Há muitos indícios de que o mundo dos
funcionários e o mundo dos pais são idênticos para Kafka. Essa
semelhança não os honra. Ela é feita de estupidez, degradação e
imundície. O uniforme do pai é cheio de nódoas, sua roupa de baixo é
suja. A imundície é o elemento vital do funcionário. ``Ela não
compreendia por que as partes se movimentavam tanto. Para sujar a
escada, respondeu um funcionário, talvez por raiva, mas a resposta foi
para ela esclarecedora.'' A imundície é de tal modo um atributo dos
funcionários que eles podem ser vistos como gigantescos parasitas. Isso
não se refere, naturalmente, às relações econômicas, mas às forças da
razão e da humanidade, que permitem a esses indivíduos sobreviver. Do
mesmo modo, nas estranhas famílias de Kafka, o pai sobrevive às custas
do filho, sugando-o como um imenso parasita. Não consome apenas suas
forças, consome também seu direito de existir. O pai é quem pune, mas
também quem acusa. O pecado do qual ele acusa o filho parece ser uma
espécie de pecado original. A definição kafkiana do pecado original é
particularmente aplicável ao filho: ``O pecado original, o velho delito
cometido pelo homem, consiste na sua queixa incessante de que ele é
vítima de uma injustiça, de que foi contra ele que o pecado original foi
cometido''. Mas quem é acusado desse pecado original, hereditário --- o
pecado de haver engendrado um herdeiro --- senão o pai, pelo filho?
Assim, o pecador seria o filho. Porém não se pode concluir da frase de
Kafka que a acusação é pecaminosa, porque falsa. Em nenhum lugar Kafka
diz que essa acusação é injusta. Trata-se de um processo sempre
pendente, e nenhuma causa é mais suspeita que aquela para a qual o pai
pretende obter a solidariedade desses funcionários e empregados da
Justiça. O pior, neles, não é sua infinita corruptibilidade. Pois em seu
íntimo são feitos de tal maneira que sua venalidade é a única esperança
que a humanidade possa alimentar a seu respeito. É certo que os
tribunais dispõem de códigos. Mas eles não podem ser vistos. ``Faz parte
da natureza desse sistema judicial condenar não apenas réus inocentes,
mas também réus ignorantes'', presume Kafka. No mundo primitivo, as leis
e normas são não escritas. O homem pode transgredi-las sem o saber.
Contudo, por mais dolorosamente que elas afetem o homem que não tem
consciência de qualquer transgressão, sua intervenção, no sentido
jurídico, não é acaso, mas destino, em toda a sua ambiguidade. Segundo
Hermann Cohen, numa rápida análise da antiga concepção do destino, uma
ideia se impunha inelutavelmente: ``são os próprios decretos do destino
que parecem facilitar e ocasionar essa transgressão e essa queda''. O
mesmo ocorre com a instância que submete Kafka à sua jurisdição. Ela
remete a uma época anterior à lei das doze tábuas, a um mundo primitivo,
contra o qual a instituição do direito escrito representou uma das
primeiras vitórias. É certo que na obra de Kafka o direito escrito
existe nos códigos, mas eles são secretos, e através deles a
pré-história exerce seu domínio ainda mais ilimitadamente.

Entre a administração e a família, Kafka vê contatos múltiplos. Na
aldeia de Schlossberg havia para isso uma expressão eloquente. ``Temos
aqui uma expressão que você talvez conheça: as decisões administrativas
são tão tímidas quanto as moças. Bem observado, disse Kafka, bem
observado, e as decisões podem ter outras coisas em comum com as
moças.'' Entre essas qualidades comuns talvez a mais notável fosse a de
se prestar a tudo, como as jovens tímidas que Kafka encontra, em \textit{O
castelo} e em \textit{O processo} e que se revelam tão devassas no seio da
família como num leito. Ele as encontra a todo instante em seu caminho;
elas se oferecem com tão pouca cerimônia como a moça do albergue. ``Eles
se enlaçaram, o pequeno corpo ardia entre as mãos de Kafka, eles rolavam
em um frenesi do qual Kafka tentava salvar-se continuamente, mas em vão;
alguns passos adiante, os dois bateram surdamente na porta de Klamm e se
deitaram em seguida sobre as poças de cerveja e outras imundícies que
cobriam o chão. Ali permaneceram horas\ldots{} nas quais Kafka experimentou
constantemente a sensação de estar perdido, ou de estar num país
estrangeiro, como nenhum outro homem havia estado antes, tão estrangeiro
que mesmo o ar nada tinha em comum com o ar nativo, um ar asfixiante,
mas cujas loucas seduções eram tão irresistíveis que não havia remédio
senão ir mais longe, perder-se mais ainda.'' Voltaremos a essa terra
estrangeira. É digno de nota, contudo, que essas mulheres que se
comportam como prostitutas não são jamais belas. A beleza só aparece no
mundo de Kafka nos lugares mais obscuros: entre os acusados, por
exemplo. ``É um fenômeno notável, de certo modo científico\ldots{} Não pode
ser a culpa que os faz belos\ldots{} não pode ser também o castigo justo que
desde já os embeleza\ldots{} só pode ser o processo movido contra eles, que
de algum modo adere a seu corpo.''

Depreendemos de \textit{O processo} que esse procedimento judicial não deixa
via de regra nenhuma esperança aos acusados, mesmo quando subsiste a
esperança da absolvição. É talvez essa desesperança que faz com que os
acusados sejam os únicos personagens belos na galeria kafkiana. Essa
hipótese estaria de acordo com um fragmento de diálogo, narrado por Max
Brod. ``Recordo-me de uma conversa com Kafka, cujo ponto de partida foi
a Europa contemporânea e a decadência da humanidade. Somos, disse ele,
pensamentos niilistas, pensamentos suicidas, que surgem na cabeça de
Deus. Essa frase evocou em mim a princípio a visão gnóstica do mundo:
Deus como um demiurgo perverso, e o mundo como seu pecado original. Oh
não, disse ele, nosso mundo é apenas um mau humor de Deus, um dos seus
maus dias. Existiria então esperança, fora desse mundo de aparências que
conhecemos? Ele riu: há esperança suficiente, esperança infinita --- mas
não para nós.'' Essas palavras estabelecem um vínculo com aqueles
singulares personagens de Kafka, os únicos que fugiram do meio familiar
e para os quais talvez ainda exista esperança. Esses personagens não são
os animais, e nem sequer os seres híbridos ou imaginários, como o
Gato-carneiro ou Odradek, pois todos eles vivem ainda no círculo da
família. Não é por acaso que é exatamente na casa dos seus pais que
Gregor Samsa se transforma em inseto, não é por acaso que o estranho
animal, meio gato, meio carneiro, é um legado paterno, não é por acaso
que Odradek é a grande preocupação do pai de família. Mas os
``ajudantes'' conseguem escapar a esse círculo.

Eles pertencem a um grupo de personagens que atravessam toda a obra de
Kafka. Dele fazem parte o vigarista desmascarado em \textit{Betrachtung}
(Meditação), assim como o estudante, que aparece à noite no balcão
como vizinho de Karl Rossmann, e os loucos que moram na cidade do sul e
que não se cansam nunca. A penumbra em que transcorre sua vida lembra a
iluminação trêmula em que aparecem os personagens das pequenas peças de
Robert Walser, autor do romance \textit{Der Gehülfe} (O ajudante), admirado
por Kafka. As lendas indianas conhecem os \textit{Gandharwe}, criaturas
inacabadas, ainda em estado de névoa. É dessa natureza que são feitos os
``ajudantes'' de Kafka: não pertencem a nenhum dos outros grupos de
personagens e não são estranhos a nenhum deles --- são mensageiros que
circulam entre todos. Como diz Kafka, assemelham-se a Barnabás, também
um mensageiro. Ainda não abandonaram de todo o seio materno da natureza
e, por isso, ``instalaram-se num canto do chão, sobre dois velhos
vestidos de mulher. Sua ambição\ldots{} era ocupar um mínimo de espaço, e
para isso, sempre sussurrando e rindo, faziam várias experiências,
cruzavam seus braços e pernas, acocoravam-se uns ao lado dos outros e na
penumbra pareciam um grande novelo''. Para eles e seus semelhantes, os
inábeis e os inacabados, ainda existe esperança.

A mesma norma de comportamento que nesses mensageiros é suave e flexível
transforma-se em lei opressiva e sombria no restante da galeria
kafkiana. Nenhuma de suas criaturas tem um lugar fixo, um contorno fixo
e próprio, não há nenhuma que não esteja ou subindo ou descendo, nenhuma
que não seja intercambiável com um vizinho ou um inimigo, nenhuma que
não tenha consumido o tempo à sua disposição, permanecendo imatura,
nenhuma que não esteja profundamente esgotada, e ao mesmo tempo no
início de uma longa jornada. Impossível falar aqui de ordens e
hierarquias. O mundo mítico, à primeira vista próximo do universo
kafkiano, é incomparavelmente mais jovem que o mundo de Kafka, com
relação ao qual o mito já representa uma promessa de libertação. Uma
coisa é certa: Kafka não cedeu à sedução do mito. Novo Odisseus,
livrou-se dessa sedução graças ``ao olhar dirigido a um horizonte
distante'' \ldots{} ``as sereias desapareceram literalmente diante de tamanha
firmeza, e, no momento em que estava mais próximo delas, não as percebia
mais''. Entre os ancestrais de Kafka no mundo antigo, os judeus e os
chineses, que reencontraremos mais tarde, esse antepassado grego não
deve ser esquecido. Pois Odisseus está na fronteira do mito e do conto
de fadas. A razão e a astúcia introduziram estratagemas no mito; por
isso, os poderes míticos deixaram de ser invencíveis. O conto é a
tradição que narra a vitória sobre esses poderes. Kafka escreveu contos
para os espíritos dialéticos quando se propôs narrar sagas. Introduziu
pequenos truques nesses contos e deles extraiu a prova de que ``mesmo os
meios insuficientes e até infantis podem ser úteis para a salvação''. É
com essas palavras que ele inicia sua narrativa sobre \textit{O silêncio das
sereias}. Pois em Kafka as sereias silenciam; elas dispõem de ``uma arma
ainda mais terrível que o seu canto\ldots{} o seu silêncio''. Elas utilizaram
essa arma contra Odisseus. Mas ele, informa-nos Kafka, ``era tão astuto,
uma raposa tão fina, que nem sequer a deusa do destino conseguiu
devassar seu interior. Embora isso seja incompreensível para a
inteligência humana, talvez ele tenha de fato percebido que as sereias
estavam silenciosas, usando contra elas e contra os deuses o estratagema
que nos foi transmitido pela tradição apenas como uma espécie de
escudo''.

Em Kafka as sereias silenciam. Talvez porque a música e o canto são para
ele uma expressão ou pelo menos um símbolo da fuga. Um símbolo da
esperança que nos vem daquele pequeno mundo intermediário, ao mesmo
tempo inacabado e cotidiano, ao mesmo tempo consolador e absurdo, no
qual vivem os ajudantes. Kafka é como o rapaz que saiu de casa para
aprender a ter medo. Ele chegou ao palácio de Potemkin, mas acabou
encontrando em seu porão Josefine, aquela ratinha cantora, que ele
descreve assim: ``existe nela algo de uma infância breve e pobre, algo
de uma felicidade perdida e irrecuperável, mas também algo da vida ativa
de hoje, com suas pequenas alegrias, incompreensíveis, mas reais, e que
ninguém pode extinguir' '.

\subsection{Uma fotografia de criança}

Existe uma foto infantil de Kafka. Poucas vezes ``a pobre e breve
infância'' concretizou-se em imagem tão evocativa. A foto foi tirada num
desses ateliês do século \textsc{xix}, que, com seus cortinados e palmeiras,
tapeçarias e cavaletes, parecia um híbrido ambíguo de câmara de torturas
e sala do trono. O menino de cerca de seis anos é representado numa
espécie de paisagem de jardim de inverno, vestido com uma roupa de
criança, muito apertada, quase humilhante, sobrecarregada com rendas. No
fundo, erguem-se palmeiras imóveis. E, como para tornar esse acolchoado
ambiente tropical ainda mais abafado e sufocante, o modelo segura na mão
esquerda um chapéu extraordinariamente grande, com largas abas, do tipo
usado pelos espanhóis. Seus olhos incomensuravelmente tristes dominam
essa paisagem feita sob medida para eles, e a concha de uma grande
orelha escuta tudo o que se diz.

Essa tristeza profunda foi talvez um dia compensada pelo fervoroso
desejo de ``ser índio''. ``Como seria bom ser um índio, sempre pronto, a
galope, inclinado na sela, trepidante no ar, sobre o chão que trepida,
abandonando as esporas, porque não há esporas, jogando fora as rédeas,
porque não há rédeas, vendo os prados na frente, com a vegetação rala,
já sem o pescoço do cavalo, já sem a cabeça do cavalo.'' Esse desejo tem
um conteúdo muito rico. Ele revela seu segredo no momento em que se
realiza: na América. A importância de \textit{Amérika} na obra de Kafka é
demonstrada pelo próprio nome do herói. Enquanto nos primeiros romances
o autor se designava apenas, em surdina, por uma inicial, nesse livro
ele nasce de novo, no novo mundo, com seu nome completo. Experimenta
esse renascimento no teatro ao ar livre de Oklahoma. ``Karl viu numa
esquina um cartaz com os seguintes dizeres: `Na pista de corridas de
Clayton contratam-se, das seis da manhã de hoje até a meia-noite,
pessoas para o teatro de Oklahoma. O grande teatro de Oklahoma te chama!
Só hoje, só uma vez! Quem perder a ocasião hoje, a perderá para \mbox{sempre}!
Quem pensa em seu futuro, nos pertence! Todos são bem-vindos! Quem
quiser ser artista, que se apresente! Nosso teatro que pode utilizar
todos, cada um em seu lugar! Quem se decidir por nós, merece ser
felicitado! Mas apressem-se, para serem admitidos antes da meia-noite! À
meia-noite tudo estará fechado e não reabrirá mais! Maldito seja aquele
que não acredita em nós! Para Clayton!'\,'' O leitor dessas palavras é
Karl Rossmann, a terceira encarnação, a mais feliz de todas, de Kafka, o
herói dos romances de Kafka. A felicidade está à sua espera no teatro ao
ar livre de Oklahoma, uma verdadeira pista de corridas, do mesmo modo
que a infelicidade o tinha encontrado no estreito tapete de seu quarto,
quando ele ali entrara ``como numa pista de corridas''. Desde que Kafka
escrevera suas \textit{Reflexões para os cavaleiros}, desde que descreveu o
``novo advogado'' ``levantando até o alto as coxas e com um passo que
faz ressoar o mármore a seus pés, subindo os degraus do Foro'', e desde
que mostrou ``as crianças na estrada'' trotando pelos campos com grandes
saltos, os braços cruzados, essa figura se tornara familiar para ele.
Com efeito, às vezes ocorre que Karl Rossmann, ``distraído por falta de
sono, perca seu tempo dando pulos inutilmente altos''. Por isso, é
somente numa pista de corridas que ele pode chegar ao objeto dos seus
desejos.

Essa pista é ao mesmo tempo um teatro, e isso constitui um enigma. Mas o
lugar enigmático e a figura inteiramente transparente e não enigmática
de Karl Rossmann pertencem ao mesmo contexto. Pois, se Karl Rossmann é
transparente, límpido e mesmo desprovido de caráter, ele o é no sentido
utilizado por Franz Rosenzweig em seu \textit{Stern der Erlösung} (Estrela da
redenção). Na China, o homem interior é ``inteiramente desprovido de
caráter; o conceito do sábio, encarnado classicamente\ldots{} por Confúcio,
supõe um caráter totalmente depurado de todas as particularidades; ele é
o homem verdadeiramente sem caráter, isto é, o homem médio\ldots{} O que
define o chinês é algo de completamente distinto do caráter: uma pureza
elementar dos sentimentos''. Como quer que possamos traduzir
conceitualmente essa pureza de sentimentos --- talvez ela seja um
instrumento capaz de medir de forma especialmente sensível o
comportamento gestual ---, o fato é que o teatro ao ar livre de Oklahoma
remete ao teatro clássico chinês, que é um teatro gestual. Uma das
funções mais significativas desse teatro ao ar livre é a dissolução do
acontecimento no gesto. Podemos ir mais longe e dizer que muitos estudos
e contos menores de Kafka só aparecem em sua verdadeira luz quando
transformados, por assim dizer, em peças representadas no teatro ao ar
livre de Oklahoma. Somente então se perceberá claramente que toda a obra
de Kafka representa um código de gestos, cuja significação simbólica não
é de modo algum evidente, desde o início, para o próprio autor; eles só
recebem essa significação depois de inúmeras tentativas e experiências,
em contextos múltiplos. O teatro é o lugar dessas experiências. Num
comentário inédito sobre \textit{Brudermord} (O fratricídio), Werner Kraft
observou lucidamente que a ação dessa pequena história era de natureza
cênica. ``O espetáculo pode começar e é anunciado por uma campainha.
Este som se produz da forma mais natural, no momento em que Wese deixa a
casa em que se encontra seu escritório. Mas essa campainha, diz o autor
expressamente, \textit{toca alto demais para uma simples campainha de porta,
ela ressoa na cidade inteira, até o céu}.'' Assim como essa campainha,
que toca alto demais e chega até o céu, os gestos dos personagens
kafkianos são excessivamente enfáticos para o mundo habitual e
extravasam para um mundo mais vasto. Quanto mais se afirma a técnica
magistral do autor, mais ele desdenha adaptar esses gestos às situações
habituais e explicá-los. Na \textit{Verwandlung} (Metamorfose), lemos sobre
``a estranha maneira que tem o chefe de sentar-se em sua escrivaninha e
falar de cima para baixo com seu empregado, que além disso precisa
chegar muito perto, devido à surdez do patrão''. Mas no \textit{Prozess} (O
processo) não existem mais essas justificações. No penúltimo capítulo,
K. ``parou nos primeiros bancos, mas para o padre a distância ainda era
excessiva. Estendeu a mão e mostrou com o indicador um lugar mais
próximo do púlpito. K. o seguiu até esse lugar, precisando inclinar a
cabeça fortemente para trás a fim de ver o padre''.

Se é certo, como diz Max Brod, que ``era imenso o mundo dos fatos que
ele considerava importantes'', o mais imenso de todos era o mundo dos
gestos. Cada um é um acontecimento em si e por assim dizer um drama em
si. O palco em que se representa esse drama é o teatro do mundo, com o
céu como perspectiva. Por outro lado, este céu é apenas pano de fundo;
investigá-lo segundo sua própria lei significaria emoldurar um pano de
fundo teatral e pendurá-lo numa galeria de quadros. Como El Greco, Kafka
despedaça o céu, atrás de cada gesto; mas como em El Greco, padroeiro
dos expressionistas, o gesto é o elemento decisivo, o centro da ação. Os
que ouviram a batida no portão se afastam, curvados de terror. Um ator
chinês representaria assim o terror, mas não assustaria ninguém. Em
outra passagem o próprio K. faz teatro. Semiconsciente do que fazia, ele
``levantou cuidadosamente os olhos\ldots{} pegou um dos papéis, sem olhá-lo,
colocou-o na palma da mão e o ofereceu lentamente aos cavalheiros,
enquanto ele próprio se erguia. Ele não pensava em nada de preciso, mas
tinha apenas a sensação de que era assim que ele teria que se comportar,
quando terminasse a grande petição que deveria inocentá-lo
completamente''. Esse gesto supremamente enigmático e supremamente
simples é um gesto de animal. Podemos ler durante muito tempo as
histórias de animais de Kafka sem percebermos que elas não tratam de
seres humanos. Quando descobrimos o nome da criatura --- símio, cão ou
toupeira ---, erguemos os olhos, assustados, e verificamos que o mundo
dos homens já está longe. Kafka é sempre assim; ele priva os gestos
humanos dos seus esteios tradicionais e os transforma em temas de
reflexões intermináveis.

Porém elas também são intermináveis quando partem das histórias
alegóricas. Pense-se na parábola \textit{Vor dem Gesetz} (Diante da lei). O
leitor que a encontra no \textit{Landarzt} (Médico de aldeia) percebe os
trechos nebulosos que ela contém. Mas teria pensado nas inúmeras
reflexões que ocorrem a Kafka, quando ele a interpreta? É o que ele faz
em \textit{O processo}, por intermédio do padre, e num lugar tão oportuno que
poderíamos suspeitar que o romance não é mais que o desdobramento da
parábola. Mas a palavra ``desdobramento'' tem dois sentidos. O botão se
``desdobra'' na flor, mas o papel ``dobrado'' em forma de barco, na
brincadeira infantil, pode ser ``desdobrado'', transformando-se de novo
em papel liso. Essa segunda espécie de desdobramento convém à parábola,
e o prazer do leitor é fazer dela uma coisa lisa, cuja significação
caiba na palma da mão. Mas as parábolas de Kafka se desdobram no
primeiro sentido: como o botão se desdobra na flor. Por isso, são
semelhantes à criação literária. Apesar disso, elas não se ajustam
inteiramente à prosa ocidental e se relacionam com o ensinamento como a
\textit{hagadá} se relaciona com a \textit{halachá}. Não são parábolas e não podem
ser lidas no sentido literal. São construídas de tal modo que podemos
citá-las e narrá-las com fins didáticos. Porém conhecemos a doutrina
contida nas parábolas de Kafka e que é ensinada nos gestos e atitudes de
K. e dos animais kafkianos? Essa doutrina não existe; podemos dizer no
máximo que um ou outro trecho alude a ela. Kafka talvez dissesse: esses
trechos constituem os resíduos dessa doutrina e a transmitem. Mas
podemos dizer igualmente: eles são os precursores dessa doutrina, e a
preparam. De qualquer maneira, trata-se da questão da organização da
vida e do trabalho na comunidade humana. Essa questão preocupou Kafka
como nenhuma outra e era impenetrável para ele. Assim como, na célebre
conversa de Erfurt entre Goethe e Napoleão, o Imperador substituiu a
política pelo destino, Kafka poderia ter substituído a organização pelo
destino. A organização está constantemente presente em Kafka, não
somente nas gigantescas hierarquias de funcionários, em \textit{O processo} e
\textit{O castelo}, mas de modo ainda mais tangível nos incompreensíveis
projetos de construção, descritos em \textit{A muralha da China}.

``A muralha deveria servir de proteção durante séculos; por isso, o
 máximo de cuidado na construção, a utilização dos conhecimentos
 arquitetônicos de todos os tempos e de todos os povos e um duradouro
 sentimento de responsabilidade por parte dos construtores constituíam
 pressupostos indispensáveis para esse trabalho. Para as obras acessórias,
 assalariados ignorantes do povo podiam ser usados, homens, mulheres,
 crianças, enfim, todos os que se empregavam para ganhar dinheiro; mas já
 para dirigir quatro desses assalariados, um homem culto era necessário,
 especializado em arquitetura\ldots{} Nós --- estou falando aqui em nome de
 muitos --- somente aprendemos a nos conhecer soletrando as instruções
 dos nossos superiores, descobrindo que sem sua liderança nosso saber
 acadêmico e nosso bom senso não teriam sido suficientes para podermos
 executar a pequena tarefa que nos cabia no grande todo''. Essa
organização se assemelha ao destino. Em seu famoso livro \textit{A civilização
e os grandes rios históricos}, Metchnikov descreve o esquema dessa
organização com palavras que poderiam ser de Kafka. ``Os canais do
Yang-tsé-kiang e as represas do Huang-ho são provavelmente o resultado
de um trabalho comum, conscientemente organizado, de\ldots{} gerações\ldots{} A
menor desatenção na escavação de um fosso ou na sustentação de uma
represa, a menor negligência, uma atitude egoísta por parte de um homem
ou de um grupo de homens na tarefa de conservar os recursos hidráulicos
da comunidade, podem originar, nessas circunstâncias insólitas, grandes
males e desgraças sociais de consequências incalculáveis. Por isso, um
funcionário encarregado de administrar os rios exigia, com ameaças de
morte, uma estreita e duradoura solidariedade entre massas da população
que muitas vezes eram estranhas e mesmo inimigas entre si; ele condenava
todos a trabalhos cuja utilidade coletiva só se evidenciava com o tempo,
e cujo plano de conjunto era muitas vezes incompreensível para o homem
comum.''

Kafka queria ser incluído entre esses homens comuns. Pouco a pouco os
limites de sua compreensão se tornaram evidentes. Ele quer mostrar aos
outros esses limites. Às vezes ele se parece com o Grande Inquisidor, de
Dostoiévski: ``Estamos, portanto, em presença de um mistério, que não
podemos compreender. E, como se trata de um enigma, tínhamos o direito
de pregar, de ensinar aos homens que o que estava em jogo não era nem a
liberdade nem o amor, mas um enigma, um segredo, um mistério, ao qual
tinham que se submeter, sem qualquer reflexão, e mesmo contra sua
consciência''. Nem sempre Kafka resistiu às tentações do misticismo.
Sobre seu encontro com Rudolf Steiner possuímos uma página de diário,
que pelo menos na forma em que foi publicada não reflete a posição de
Kafka. Teria se recusado a revelar sua opinião? Sua atitude com relação
aos próprios textos sugere que essa hipótese não é de modo algum
impossível. Kafka dispunha de uma capacidade invulgar de criar
parábolas. Mas ele não se esgota nunca nos textos interpretáveis e toma
todas as precauções possíveis para dificultar essa interpretação. É com
prudência, com circunspecção, com desconfiança que devemos penetrar,
tateando, no interior dessas parábolas. Devemos ter presente sua maneira
peculiar de lê-las, como ela transparece na sua interpretação da
parábola citada. Precisamos pensar também em seu testamento. Suas
instruções para que sua obra póstuma fosse destruída são tão difíceis de
compreender e devem ser examinadas tão cuidadosamente como as respostas
do guardião da porta, diante da lei. Cada dia de sua vida confrontará
Kafka com atitudes indecifráveis e com explicações ininteligíveis, e é
possível que pelo menos ao morrer Kafka tivesse decidido pagar seus
contemporâneos na mesma moeda.

O mundo de Kafka é um teatro do mundo. Para ele, o homem está desde o
início no palco. E a prova é que todos são contratados no teatro de
Oklahoma. Impossível conhecer os critérios que presidem a essa
contratação. O talento de ator, que parece o critério mais óbvio, não
tem nenhuma importância. Podemos exprimir esse fato de outra forma: não
se exige dos candidatos senão que interpretem a si mesmos. Está
absolutamente excluído que eles \textit{sejam} o que \textit{representam}.
Representando seus papéis, os atores procuram um abrigo no teatro ao ar
livre, como os seis atores de Pirandello procuram um autor. Para uns e
outros, a cena constitui o último refúgio, e não é impossível que esse
refúgio seja também a salvação. A salvação não é uma recompensa
outorgada à vida, mas a última oportunidade de evasão oferecida a um
homem, como diz Kafka, ``cujo próprio crânio bloqueia\ldots{} o caminho''. A
lei desse teatro está numa frase escondida no \textit{Bericht für eine
Akademie} (Relatório à academia): ``\ldots{} eu imitava porque estava à
procura de uma saída, por nenhuma outra razão''. No final do seu
processo, K. parece ter um pressentimento de tudo isso. Ele se volta de
repente para os dois cavalheiros de cartola, que vieram levá-lo, e
pergunta: ``--- Em que teatro trabalham os Senhores? --- Teatro?
perguntou um deles, pedindo conselho ao outro, com os lábios trêmulos.
Este reagiu como um mudo, que luta com um organismo recalcitrante''.
Eles não responderam à pergunta, mas esses indícios fazem supor que
foram afetados por ela.

Todos os atores que se tornaram membros do teatro ao ar livre são
servidos num grande banco, recoberto com uma toalha branca. ``Todos
estavam alegres e excitados.'' Para celebrar, os figurantes fazem o
papel de anjos, em altos pedestais cobertos com panos ondulantes e que
têm uma escada em seu interior. Todos os elementos de uma quermesse
campestre, ou talvez de uma festa infantil, na qual o menino da foto,
vestido com sua roupa excessivamente pomposa, teria talvez perdido a
tristeza do seu olhar. Sem as asas postiças, talvez fossem anjos de
verdade. Eles têm precursores na obra de Kafka, entre eles o empresário
teatral, que sobe na rede para confortar o trapezista acometido da
``primeira dor'', acaricia-o e aperta o seu rosto contra o seu, de modo
que ``as lágrimas do trapezista o inundaram também''. Outro anjo, anjo
guardião ou guardião da lei, depois do ``fratricídio'' se encarrega do
assassino Schmar, que ``cola a boca no ombro do guarda'' e o leva
consigo, com passos leves. O último romance de Kafka termina nas
cerimônias campestres de Oklahoma. Segundo Soma Morgenstern, ``em Kafka,
como em todos os fundadores de religião, sopra um ar de aldeia''.
Devemos recordar aqui a concepção da piedade, sustentada por Lao-tsé, da
qual Kafka deu uma descrição completa em \textit{Nächste Dorf} (A aldeia
próxima): ``Duas aldeias vizinhas podem estar ao alcance da vista e
ouvir os galos e os cães uma da outra, mas seus habitantes morrem
velhos, sem jamais viajarem de uma para outra''. São palavras de
Lao-tsé. Kafka também compunha parábolas, mas não fundou nenhuma
religião.

Recordemos a aldeia ao pé do castelo, do qual K. recebe a confirmação
misteriosa e inesperada de sua designação como agrimensor. Em seu
posfácio a \textit{O castelo}, Brod informa que Kafka tinha pensado num
vilarejo específico ao criar essa aldeia: Zürau, no Erzgebirge. Mas
podemos reconhecer nela outro lugar. É a aldeia mencionada numa lenda
talmúdica, narrada por um rabino em resposta à pergunta: por que os
judeus preparam um banquete na noite de sexta-feira? É a história de uma
princesa exilada, longe dos seus compatriotas, que definha numa aldeia
cuja língua ela não compreende. Um dia ela recebe uma carta do seu
noivo, anunciando que não a tinha esquecido e que estava a caminho para
revê-la. O noivo, diz o rabino, é o Messias, a princesa a alma, e a
aldeia o corpo. Ignorando a língua falada na aldeia, seu único meio para
comunicar-lhe a alegria que sente é preparar para ela um festim. Essa
aldeia talmúdica está no centro do mundo kafkiano. O homem de hoje vive
em seu corpo como K. ao pé do castelo: ele desliza fora dele e lhe é
hostil. Pode ocorrer que o homem acorde um dia e verifique que se
transformou num inseto. O país de exílio --- o seu exílio ---
apoderou-se dele. É o ar dessa aldeia que sopra no mundo de Kafka, e é
por isso que ele nunca cedeu à tentação de fundar uma religião. É nesse
vilarejo que estão o chiqueiro de onde saem os cavalos para o médico de
aldeia, o sufocante quarto dos fundos onde Klamm está sentado diante de
um copo de cerveja, com o charuto na boca, e o portão no qual não se
pode bater sem desafiar a morte. O ar dessa aldeia é impuro, com a
mescla putrefata das coisas que não chegaram a existir e das coisas que
amadureceram demais. Em sua vida, Kafka teve que respirar essa
atmosfera. Não era nem adivinho nem fundador de religiões. Como
conseguiu suportar tal atmosfera?

\subsection{O homenzinho corcunda}

Há muito se sabe que Knut Hamsun tinha o hábito de publicar suas
opiniões na seção dos leitores do jornal que circulava na cidadezinha
perto da qual ele vivia. Há alguns anos foi instaurado nessa cidade um
processo contra uma jovem que assassinara seu filho recém-nascido. Ela
foi condenada à prisão. Pouco depois apareceu na folha local uma carta
de Hamsun. O autor dizia que daria as costas a uma cidade que aplicasse
a mães capazes de matar seus filhos outra pena que a mais severa: se não
a forca, pelo menos a prisão perpétua. Passaram-se alguns anos. Hamsun
publicou \textit{Benção da terra}, na qual havia a história de uma empregada
doméstica que comete o mesmo crime, recebe a mesma pena e certamente não
merecia um castigo mais severo, como o leitor percebe claramente.

As reflexões póstumas de Kafka, contidas em \textit{A grande muralha da China},
fazem lembrar esse episódio. Pois assim que apareceu o volume póstumo,
foi publicada uma exegese de Kafka, baseada apenas nessas reflexões e
que procurava interpretá-las, ignorando sumariamente a própria obra. Há
dois mal-entendidos possíveis com relação a Kafka: recorrer a uma
interpretação natural e a uma interpretação sobrenatural. As duas, a
psicanalítica e a teológica, perdem de vista o essencial. A primeira é
devida a Hellmuth Kaiser; a segunda foi praticada por numerosos autores,
como H. J. Schoeps, Bernhard Rang e Groethuysen. Willy Haas pode também
ser incluído nessa corrente, embora em outras ocasiões tenha escrito
comentários muito instrutivos sobre Kafka, como veremos a seguir. Isso
não o impediu de explicar a obra de Kafka em seu conjunto segundo certos
lugares-comuns teológicos. ``O poder superior'' --- escreve ele ---, ``a
esfera da graça, é descrito em seu grande romance \textit{O castelo}, enquanto
o poder inferior, a esfera do julgamento e da danação, é descrito em
outro grande livro, \textit{O processo}. Tentou descrever a terra, esfera
intermediária entre esses dois planos\ldots{} o destino terreno, com suas
difíceis exigências, e de modo altamente estilizado, num terceiro
romance, \textit{América}.'' O primeiro terço dessa interpretação constitui
hoje, a partir de Brod, patrimônio comum da exegese kafkiana. Assim, por
exemplo, escreve Bernhard Rang: ``Na medida em que o Castelo pode ser
visto como a sede da Graça, os vãos esforços e tentativas dos homens
significam, teologicamente falando, que eles não podem forçar e provocar
arbitrariamente, por um ato de vontade, a graça divina. A agitação e a
impaciência inibem e perturbam o silêncio grandioso de Deus''. É uma
interpretação cômoda, que se torna cada vez mais insustentável à medida
que se avança na mesma direção. Willy Haas é especialmente claro nessa
linha de argumentação: ``Kafka descende\ldots{} de Kierkegaard e de Pascal;
podemos considerá-lo o único descendente legítimo desses dois filósofos.
Os três partem, com a mesma dureza implacável, do mesmo tema religioso
de base: o homem nunca tem razão em face de Deus\ldots{} O mundo superior de
Kafka, o Castelo, com seus funcionários imprevisíveis, mesquinhos,
complicados e gananciosos, e seu estranho Céu, brincam com os homens
impiedosamente\ldots{} no entanto nem diante desse Deus o homem tem razão''.
Em suas especulações bárbaras, que de resto não são sequer compatíveis
com o próprio texto literal de Kafka, essa teologia fica muito aquém da
doutrina da justificação, de Anselmo de Salisbury. É exatamente em O
\textit{castelo} que encontramos a frase: ``Pode um só funcionário perdoar? No
máximo, a administração como um todo poderia fazê-lo, mas provavelmente
ela não pode perdoar, e sim julgar, apenas''. Esse tipo de interpretação
levou rapidamente a um beco sem saída. ``Nada disso'', diz Denis de
Rougemont, ``significa a miséria de um homem sem Deus, mas a miséria do
homem ligado a um Deus que ele não conhece, porque não conhece o
Cristo.''

É mais fácil extrair conclusões especulativas das notas póstumas de
Kafka que investigar um único dos temas que aparecem em seus contos e
romances. No entanto somente esses temas podem lançar alguma luz sobre
as forças arcaicas que atravessam a obra de Kafka --- forças,
entretanto, que com igual justificação poderíamos identificar no mundo
contemporâneo. Quem pode dizer sob que nome essas forças apareceram a
Kafka? O que é certo é que ele não se encontrou nelas. Não as conheceu.
No espelho da culpa, que o mundo primitivo lhe apresentou, ele viu
apenas o futuro, sob a forma do tribunal. Como representar esse
tribunal? Seria o julgamento final? O juiz não se converte em acusado? A
punição não está no próprio processo? Kafka não respondeu a essas
perguntas. Veria alguma utilidade nelas? Ou julgava preferível adiá-las?
Nas narrativas que ele nos deixou, a epopeia recuperou a significação
que lhe dera Scherazade: adiar o que estava por vir. O adiamento é em \textit{O
processo} a esperança dos acusados --- contanto que o procedimento
judicial não se transforme gradualmente na própria sentença. O adiamento
beneficiaria mesmo o Patriarca, e para isso deveria renunciar ao papel
que lhe cabe na tradição. ``Posso imaginar um outro Abraão, que não
chegaria evidentemente à condição de Patriarca, nem sequer à de
negociante de roupas usadas, que se disporia a cumprir a exigência do
sacrifício, obsequioso como um garçom, mas que não consumaria esse
sacrifício, porque não pode sair de casa, onde é indispensável, porque
seus negócios lhe impõem obrigações, porque há sempre alguma coisa a
arrumar, porque a casa não está pronta, e sem que ela esteja pronta não
pode sair, como a própria Bíblia admite, quando diz: ele pôs em ordem
sua casa.''

Abraão parece ``obsequioso como um garçom''. Só pelo gesto podia Kafka
fixar alguma coisa. É esse gesto, que ele não compreende, que constitui
o elemento nebuloso de suas parábolas. É dele que parte a obra literária
de Kafka. Sabe-se como ele era reticente com relação a essa obra. Em seu
testamento, ordenou que ela fosse destruída. Esse testamento, que nenhum
estudo sobre Kafka pode ignorar, mostra que o autor não estava
satisfeito; que ele considerava seus esforços malogrados; que ele se
incluía entre os que estavam condenados ao fracasso. Fracassada foi sua
grandiosa tentativa de transformar a literatura em doutrina,
devolvendo-lhe, sob a forma de parábolas, a consistência e a austeridade
que lhe convinham, à luz da razão. Nenhum escritor seguiu tão
rigorosamente o preceito de ``não construir imagens''.

``Era como se a vergonha devesse lhe sobreviver'' --- são as últimas
palavras de \textit{O processo}. A vergonha, que nele corresponde à ``pureza
elementar dos sentimentos'', é o mais forte gesto de Kafka. Ela tem uma
dupla face. A vergonha é ao mesmo tempo uma reação íntima do indivíduo e
uma reação social. Não é apenas vergonha dos outros, mas vergonha pelos
outros. A vergonha de Kafka é tão pouco pessoal quanto a vida e o
pensamento que ela determina e sobre os quais Kafka escreveu: ``Ele não
vive por causa de sua vida pessoal, nem pensa por causa do seu
pensamento pessoal. Tudo se passa como se ele vivesse e pensasse sob o
peso de uma obrigação familiar\ldots{} Por causa dessa família
desconhecida\ldots{} ele não podia ser despedido''. Não conhecemos a
composição dessa família desconhecida, constituída por homens e animais.
Só uma coisa é clara: é ela que o força, ao escrever, a movimentar
períodos cósmicos. Obedecendo às exigências dessa família, Kafka rola o
bloco do processo histórico, como Sísifo rola seu rochedo. Nesse
movimento, o lado de baixo desse bloco se torna visível. Não é um
espetáculo agradável. Mas Kafka consegue suportar essa visão. ``Ter fé
no progresso não significa julgar que o progresso já aconteceu. Isso não
seria mais fé.'' A época em que ele vive não representa para Kafka
nenhum progresso com relação ao começo primordial. Seus romances se
passam num lamaçal. A criatura para ele está no estágio que Bachofen
caracterizou como hetaírico. O fato de que esse estágio esteja esquecido
não significa que ele não se manifeste no presente. Ao contrário, é esse
esquecimento que o torna presente. Ele é descoberto por uma experiência
mais profunda que a do homem comum. Em uma de suas primeiras anotações,
escreve Kafka: ``Eu tenho experiência e não estou brincando quando digo
que essa experiência é uma espécie de enjoo em terra firme''. Não é por
acaso que a primeira \textit{Reflexão} parte de um balanço. Kafka é inesgotável
em sua descrição da natureza oscilante das experiências. Cada uma cede à
outra, mistura-se com a experiência contrária. ``Era um dia quente de
verão.'' --- começa \textit{Schlag ans Hoftor} (Batida no portão) ---
``Voltando para casa com minha irmã, passei diante de um portão. Não sei
se ela bateu no portão, por capricho ou distração, ou se apenas ameaçou
fazê-lo com o punho, sem bater.'' A mera possibilidade da terceira
hipótese faz as duas outras, aparentemente inocentes, aparecerem sob
outra luz. É do pântano dessas experiências que emergem os personagens
femininos de Kafka. São figuras de lodo, como Leni, que ``separa o dedo
médio e o anular de sua mão direita, de modo que `a película que une os
dois dedos' se estende quase até atingir a articulação superior do dedo
mínimo''. A ambígua Frieda se recorda de sua vida passada. ``Belos
tempos. Nunca me perguntaste nada sobre o meu passado.'' Esse passado se
estende até o ponto mais escuro das profundezas em que se dá aquela
cópula cuja ``voluptuosidade desenfreada'', para usar as palavras de
Bachofen, ``é abominada pelos poderes imaculados da luz divina e que
justifica a expressão \textit{luteae voluptates}, de Arnobius''.

Só a partir desse fato podemos compreender a técnica narrativa de Kafka.
Quando outros personagens têm algo que dizer a K., eles o dizem
casualmente, como se ele no fundo já soubesse do que se tratava, por
mais importante e surpreendente que seja a comunicação. É como se não
houvesse nada de novo, como se o herói fosse discretamente convidado a
lembrar-se de algo que ele havia esquecido. É nesse sentido que Willy
Haas interpreta, com razão, o movimento de \textit{O processo}, dizendo que ``o
objeto desse processo, o verdadeiro herói desse livro inacreditável, é o
esquecimento\ldots{} cujo principal atributo é o de esquecer-se a si mesmo\ldots{}
Ele se transformou em personagem mudo na figura do acusado, figura da
mais grandiosa intensidade''. Não podemos afastar de todo a hipótese de
que esse ``centro misterioso'' derive da ``religião judaica''. ``A
memória enquanto piedade desempenha aqui um papel supremamente
misterioso. O mais profundo atributo de Jeová é que ele se recorda, que
conserva uma memória infalível até `a terceira e quarta geração', até a
`centésima' geração; o momento mais sacrossanto do ritual é o apagamento
dos pecados no livro da memória''.

Mas o esquecimento --- e aqui atingimos um novo patamar na obra de Kafka
--- não é nunca um esquecimento individual. Tudo o que é esquecido se
mescla a conteúdos esquecidos do mundo primitivo, estabelece com ele
vínculos numerosos, incertos, cambiantes, para formar criações sempre
novas. O esquecimento é o receptáculo a partir do qual emergem à luz do
dia os contornos do inesgotável mundo intermediário, nas narrativas de
Kafka. ``Aqui a plenitude do mundo é considerada a única realidade. Todo
espírito precisa fazer-se coisa, ser isolado, para adquirir um lugar e
um direito à existência\ldots{} O espiritual, na medida em que ainda
desempenha um papel, pulveriza-se em espíritos. Os espíritos se tornam
entes completamente individuais, com os seus próprios nomes
estreitamente associados ao nome de quem os venera\ldots{} despreocupada, a
plenitude do mundo recebe da plenitude desses espíritos uma nova
plenitude\ldots{} Sem provocar nenhuma inquietação, aumenta a massa dos
espíritos\ldots{} aos antigos espíritos se acrescentam novos, todos com seu
nome e distintos uns dos outros.'' Essas palavras não se referem a
Kafka, e sim à China. É assim que Franz Rosenzweig descreve o culto dos
antepassados na \textit{Estrela da redenção}. Do mesmo modo que para Kafka o
mundo dos fatos importantes era imenso, também era imenso o mundo dos
seus ancestrais, e é certo que esse mundo, como o mastro totêmico dos
primitivos, chegava até os animais, em seu movimento descendente. De
resto, não é somente em Kafka que os animais são os receptáculos do
esquecimento. Na profunda obra de Tieck, \textit{Der Blonde Eckbert} (O louro
Eckbert), o nome esquecido de um cãozinho --- Strohmi --- figura como
símbolo de uma culpa enigmática. Podemos entender assim por que Kafka
não se cansava de escutar os animais para deles recuperar o que fora
esquecido. Eles não são um fim em si, mas sem eles nada seria possível.
Recorde-se o ``artista da fome'', que ``a rigor era apenas um obstáculo
no caminho que levava às estrebarias''. Não vemos, em \textit{Bau}
(Construção) ou no \textit{Riesenmaulwurf} (Toupeira gigante), o animal
refletindo e ao mesmo tempo cavando suas galerias subterrâneas? Por
outro lado, esse pensamento é algo de muito confuso. Indeciso, ele
oscila de uma preocupação para outra, saboreia todos os medos e tem a
inconstância do desespero. Por isso, em Kafka também existem borboletas;
o ``caçador Gracchus'', sob o peso de uma culpa da qual ele nada quer
saber, ``transforma-se em borboleta''. ``Não riam, diz o caçador
Gracchus.'' O que é certo é que de todos os seres de Kafka são os
animais os que mais refletem. O que é a corrupção no mundo do direito, a
angústia é no mundo do pensamento. Ela perturba o pensamento, mas
constitui o único elemento de esperança que ele contém. Porém em nosso
corpo o mais esquecido dos países estrangeiros é o nosso próprio corpo,
e, por isso, compreendemos a razão pela qual Kafka chamava ``o animal''
à tosse que irrompia das suas entranhas. Era o posto avançado da grande
horda.

Em Kafka, Odradek é o mais estranho bastardo gerado pelo mundo
pré-histórico com seu acasalamento com a culpa. ``À primeira vista ele
tem o aspecto de um carretel achatado, em forma de estrela, e de fato
parece ter alguma analogia com um novelo de fios: de qualquer maneira só
poderiam ser fios rasgados, velhos, interligados por nós, emaranhados um
no outro, dos mais diferentes tipos e cores. Mas não é apenas um
carretel, porque do centro da estrela sai um bastonete transversal, ao
qual se junta outro no canto direito. Com auxílio desse último bastonete
e de uma das pontas da estrela, a criatura pode ficar de pé, como se
tivesse duas pernas.'' Odradek ``fica alternadamente no sótão, na
escada, no corredor, no vestíbulo''. Ele frequenta, portanto, os mesmos
lugares que o investigador da Justiça, à procura da culpa. O sótão é o
lugar dos objetos descartados e esquecidos. A obrigação de comparecer ao
tribunal evoca talvez o mesmo sentimento que a obrigação de remexer
arcas antigas, deixadas no sótão durante anos. Se dependesse de nós,
adiaríamos a tarefa até o fim dos nossos dias, do mesmo modo que K. acha
que seu documento de defesa ``poderá um dia ocupar sua inteligência
senil, depois da aposentadoria''.

Odradek é o aspecto assumido pelas coisas em estado de esquecimento.
Elas são deformadas. Deformada é a ``preocupação do pai de família'',
que ninguém sabe em que consiste, deformado o inseto, que como sabemos é
na realidade Gregor Samsa, deformado o grande animal, meio carneiro e
meio gato, para o qual talvez ``a faca do carniceiro fosse uma
salvação''. Mas esses personagens de Kafka se associam, através de uma
longa série de figuras, com a figura primordial da deformação, o
corcunda. Entre as atitudes descritas por Kafka em suas narrativas
nenhuma é mais frequente que a do homem cuja cabeça se inclina
profundamente sobre seu peito. Ela é provocada pelo cansaço nos membros
do tribunal, pelo ruído nos porteiros do hotel, pelo teto excessivamente
baixo nos frequentadores das galerias. Contudo na \textit{Strafkolonie} (Colônia
penal) os dirigentes se servem de uma antiga máquina que grava letras
floreadas nas costas do culpado, aumenta as incisões, acumula os
ornamentos, até que suas costas se tornem clarividentes, possam elas
próprias decifrar as inscrições, descobrindo assim o nome da culpa
desconhecida. São, portanto, as costas que importam. São elas que
importam para Kafka, desde muito tempo. Lemos nas primeiras anotações do
\textit{Diário}: ``Para ficar tão pesado quanto possível, o que considero bom
para o sono, eu cruzava os braços e punha as mãos nos ombros, como um
soldado com sua mochila''. É claro que a ideia de estar carregado tem
relação com a de esquecer --- no sono. Uma canção popular --- \textit{O
homenzinho corcunda} --- concretiza essa relação. O homenzinho é o
habitante da vida deformada; desaparecerá quando chegar o Messias, de
quem um grande rabino disse que ele não quer mudar o mundo pela força,
mas apenas retificá-lo um pouco.

``Vou para o meu quartinho/ para fazer minha caminha/ e encontro um
homenzinho corcunda/ que começa a rir.'' É o riso de Odradek, que
``ressoa como o murmúrio de folhas caídas''. A canção continua: ``Quando
me ajoelho em meu banquinho/ para rezar um pouquinho/ encontro um
homenzinho corcunda/ que começa a falar./ Querida criancinha, por favor/
Reza também pelo homenzinho corcunda''. Assim termina a canção. Em suas
profundezas, Kafka toca o chão que não lhe era oferecido nem pelo
``pressentimento mítico'' nem pela ``teologia existencial''. É o chão do
mundo germânico e do mundo judeu. Se Kafka não rezava, o que ignoramos,
era capaz ao menos, como faculdade inalienavelmente sua, de praticar o
que Malebranche chamava ``a prece natural da alma'' --- a atenção. Como
os santos em sua prece, Kafka incluía na sua atenção todas as criaturas.

\subsection{Sancho Pança}

Conta-se que numa aldeia hassídica alguns judeus estavam sentados numa
pobre estalagem, num sábado à noite. Eram todos residentes do lugar,
menos um desconhecido, de aspecto miserável, mal vestido, escondido num
canto escuro, nos fundos. Conversava-se aqui e ali. Num certo momento,
alguém se lembrou de perguntar o que cada um desejaria, se um único
desejo pudesse ser atendido. Um queria dinheiro, outro um genro, outro
uma nova banca de carpinteiro, e assim por diante. Depois que todos
falaram, restava apenas o mendigo, em seu canto escuro. Interrogado, ele
respondeu, com alguma relutância: ``Gostaria de ser um rei poderoso,
governando um vasto país, e que uma noite, ao dormir em meu palácio, um
exército inimigo invadisse o meu reino, e que antes do nascer do dia os
cavaleiros tivessem entrado em meu castelo, sem encontrar resistência, e
que acordando assustado eu não tivesse tempo de me vestir, e com uma
simples camisa no corpo eu fosse obrigado a fugir, perseguido sem parar,
dia e noite, por montes, vales e florestas, até chegar a este banco,
neste canto, são e salvo. É o meu desejo''. Os outros se entreolharam
sem entender. ``--- E o que você ganharia com isso?'' perguntaram. ``---
Uma camisa'', foi a resposta.

Essa história conduz ao centro da obra de Kafka. Não está dito que as
deformações que um dia o Messias corrigirá são apenas as do nosso
espaço. Certamente são também as do nosso tempo. E certamente Kafka
pensou nisso. É com uma certeza desse gênero que seu avô diz: ``A vida é
surpreendentemente curta. Ela é mesmo tão curta em minha memória, que
mal posso compreender, por exemplo, como um jovem pode se decidir a
viajar para a próxima aldeia sem temer --- mesmo deixando de lado os
acidentes imprevisíveis --- que o tempo de toda uma vida normal e sem
imprevistos seja insuficiente para terminar essa viagem''. O mendigo é
um irmão desse velho. Em sua ``vida normal e sem imprevistos'' ele não
encontra tempo para um só desejo, mas na vida anormal e cheia de
imprevistos da fuga, que ele fantasia em sua história, ele renuncia a
qualquer desejo e o troca pela sua realização.

Entre as criaturas de Kafka existe uma tribo singularmente consciente da
brevidade da vida. Ela vem da ``cidade do sul'', que Kafka caracteriza
com o seguinte diálogo: ``Ali estão as pessoas! Imaginem, elas não
dormem! --- E por que não? --- Porque não se cansam nunca. --- E por que
não? --- Porque são tolos. --- Então os tolos não se cansam? --- Como
poderiam os tolos cansar-se?''. Como se vê, os tolos têm afinidades com
os infatigáveis ajudantes. Mas essa tribo tem ainda outras
características. De passagem, ouvimos um comentário segundo o qual os
rostos dos ajudantes ``lembravam os de adultos, talvez mesmo os de
estudantes''. Com efeito, os estudantes, que em Kafka aparecem nos
lugares mais estranhos, são os chefes e porta-vozes dessa tribo. ``---
Mas quando dormem vocês? perguntou Karl, olhando admirado os estudantes.
--- Ah, dormir! disse o estudante. Dormirei quando tiver acabado os meus
estudos.'' Pense-se nas crianças: com que relutância vão para a cama!
Pois enquanto dormem, alguma coisa interessante poderia acontecer. ``Não
se esqueça do melhor!'' é uma observação ``que nos é familiar a partir
de uma quantidade incerta de velhas narrativas, embora ela talvez não
ocorra em nenhuma.'' Porém o esquecimento diz sempre respeito ao melhor,
porque diz respeito à possibilidade da redenção. ``A ideia de querer
ajudar-me'', diz, ironicamente, o espírito sempre inquieto do caçador
Gracchus, ``é uma doença que deve ser curada na cama.'' Os estudantes
não dormem, por causa dos seus estudos, e talvez a maior virtude dos
estudos é mantê-los acordados. O artista da fome jejua, o guardião da
porta silencia e os estudantes velam: assim, ocultas, operam em Kafka as
grandes regras da ascese.

Os estudos são seu coroamento. Kafka os traz à luz do dia, resgatando-os
dos anos extintos de sua infância. ``Numa cena não muito diferente, há
muitos anos, Karl se sentava, em casa, à mesa dos seus pais, fazendo
seus deveres escolares, enquanto o pai lia o jornal ou fazia
contabilidade e redigia a correspondência para uma firma, e a mãe
costurava, levantando muito alto a linha. Para não incomodar o pai, Karl
só colocava na mesa o caderno e o material de escrever, arrumando os
livros necessários em cadeiras à direita e à esquerda. Como tudo era
tranquilo ali! Como era rara a visita dos estranhos!''. Talvez esses
estudos não tenham servido para nada. Mas esse ``nada'' é muito próximo
daquele ``nada'' taoista que nos permite utilizar ``alguma coisa''. É em
busca desse ``nada'' que Kafka formulava o desejo de ``fabricar uma mesa
com uma perícia exata e escrupulosa, e ao mesmo tempo não fazer nada, de
tal maneira que, em vez de dizerem: o martelo não é nada para ele, as
pessoas dissessem: o martelo é para ele um verdadeiro martelo e ao mesmo
tempo não é nada, e com isso o martelo se tornaria ainda mais audacioso,
mais decidido, mais real e, se se quiser, mais louco''. Em seus estudos,
os estudantes têm uma atitude igualmente resoluta e igualmente fanática.
Essa atitude não pode ser mais estranha. Escrevendo e estudando, as
pessoas perdem o fôlego. ``Muitas vezes o funcionário dita em voz tão
baixa que o escrevente não ouve nada se estiver sentado, e, por isso,
precisa pular, capturar as palavras ditadas, sentar-se depressa e
escrevê-las, em seguida pular de novo, e assim por diante. Como é
singular! É quase incompreensível.'' Mas talvez possamos compreender
melhor se voltarmos aos atores do teatro ao ar livre. Os atores têm que
ficar extremamente atentos às suas deixas. Eles se assemelham também sob
outros aspectos a essas pessoas zelosas. Para eles, com efeito, ``o
martelo é um verdadeiro martelo e ao mesmo tempo não é nada'', desde que
esse martelo faça parte do seu papel. Eles estudam esse papel; o ator
que esquecesse uma palavra ou um gesto seria um mau ator. Para os
integrantes da equipe de Oklahoma, contudo, esse papel é sua vida
anterior. Por isso, esse teatro ao ar livre é um teatro ``natural''. Os
atores estão salvos. O mesmo não ocorre com o estudante que Karl vê do
seu balcão, em silêncio, à noite, quando ele lê o seu livro: ``ele
virava as folhas, de vez em quando consultava outro livro, que ele
segurava rapidamente, fazia anotações frequentes em um caderno,
inclinando profundamente o rosto sobre ele.''

Kafka não se cansa de dar corpo ao gesto, em descrições desse tipo. Mas
sempre com assombro. Com razão, Kafka foi comparado ao soldado Schweyk;
porém o primeiro se assombra com tudo, e o segundo não se assombra com
nada. O cinema e o gramofone foram inventados na era da mais profunda
alienação dos homens entre si e das relações mediatizadas ao infinito,
as únicas que subsistiram. No cinema, o homem não reconhece seu próprio
andar e no gramofone não reconhece sua própria voz. Esse fenômeno foi
comprovado experimentalmente. A situação dos que se submetem a tais
experiências é a situação de Kafka. É ela que o obriga ao estudo. Nesse
processo, talvez ele encontre fragmentos da própria existência, que
talvez ainda estejam em relação com o papel. Ele recuperaria o gesto
perdido, com Schlemihl, a sombra perdida. Ele se compreenderia enfim,
mas com que esforço imenso! Pois o que sopra dos abismos do esquecimento
é uma tempestade. E o estudo é uma corrida a galope contra essa
tempestade. É assim que o mendigo em seu banco ao lado da lareira
cavalga em direção ao seu passado, para se apoderar de si mesmo, sob a
forma do rei fugitivo. À vida, que é curta demais para uma cavalgada,
corresponde a vida que é suficientemente longa para que o cavaleiro
``abandone as esporas, porque não há esporas, jogue fora as rédeas,
porque não há rédeas, veja os prados na frente, com a vegetação rala, já
sem o pescoço do cavalo, já sem a cabeça do cavalo!''. Assim se realiza
a fantasia do cavaleiro feliz, que galopa numa viagem alegre e vazia em
direção ao passado, sem pesar sobre sua montaria. Infeliz, no entanto, o
cavaleiro que está preso à sua égua porque se fixou um objetivo situado
no futuro, ainda que seja o futuro mais imediato, como o de atingir o
depósito de carvão. Infeliz também seu cavalo, infelizes os dois.
``Montado num balde, segurando a alça, a mais simples das rédeas, desço
penosamente as escadas; mas, quando chego embaixo, meu balde se levanta,
lindo, lindo; camelos deitados no chão não se levantariam de modo mais
belo, sacudindo-se sob o bastão do cameleiro.'' Nenhuma região é mais
desolada que a região da ``montanha de gelo'' em que se perde para
sempre o ``cavaleiro do balde''. Das ``regiões inferiores da morte''
sopra o vento, que lhe é favorável --- o mesmo que em Kafka sopra tão
frequentemente do mundo primitivo, e que impulsiona o barco do caçador
Gracchus. ``Ensina-se em toda parte'', diz Plutarco, ``em mistérios e
sacrifícios, tanto entre os gregos como entre os bárbaros\ldots{} que devem
existir duas essências distintas e duas forças opostas, uma que leva em
frente, por um caminho reto, e outra que interrompe o caminho e força a
retroceder.'' É para trás que conduz o estudo, que converte a existência
em escrita. O professor é Bucéfalo, o ``novo advogado'', que sem o
poderoso Alexandre --- isto é, livre do conquistador, que só queria
caminhar para frente --- toma o caminho de volta. ``Livre, com seus
flancos aliviados da pressão das coxas do cavaleiro, sob uma luz calma,
longe do estrépito das batalhas de Alexandre, ele lê e vira as páginas
dos nossos velhos livros.'' Há algum tempo, Wemer Kraft interpretou essa
narrativa. Depois de ter examinado com cuidado cada pormenor do texto,
observa o intérprete: ``Nunca antes na literatura foi o mito em toda a
sua extensão criticado de modo tão violento e devastador''. Segundo
Kraft, o autor não usa a palavra ``justiça''; não obstante, é da justiça
que parte a crítica do mito. Mas, já que chegamos tão longe, se
parássemos aqui, correríamos o risco de não entender Kafka. É
verdadeiramente o direito que em nome da justiça é mobilizado contra o
mito? Não; como jurista, Bucéfalo permanece fiel à sua origem: porém ele
não parece \textit{praticar} o direito, e nisso, no sentido de Kafka, está o
elemento novo, para Bucéfalo e para a advocacia. A porta da justiça é o
direito que não é mais praticado, e sim estudado.

A porta da justiça é o estudo. Mas Kafka não se atreve a associar a esse
estudo as promessas que a tradição associa no estudo da Torá. Seus
ajudantes são bedéis que perderam a igreja, seus estudantes são
discípulos que perderam a escrita. Ela não se impressiona mais com ``a
viagem alegre e vazia''. Contudo Kafka achou a lei na sua viagem; pelo
menos uma vez, quando conseguiu ajustar sua velocidade desenfreada a um
passo épico, que ele procurou durante toda a sua vida. O segredo dessa
lei está num dos seus textos mais perfeitos, e não apenas por se tratar
de uma interpretação. ``Sancho Pança, que aliás nunca se vangloriou
disso, conseguiu no decorrer dos anos afastar de si o seu demônio, que
ele mais tarde chamou de Dom Quixote, fornecendo-lhe, para ler de noite
e de madrugada, inúmeros romances de cavalaria e de aventura. Em
consequência, esse demônio foi levado a praticar as proezas mais
delirantes, mas que não faziam mal a ninguém, por falta do seu objeto
predeterminado, que deveria ter sido o próprio Sancho Pança. Sancho
Pança, um homem livre, seguia Dom Quixote em suas cruzadas com
paciência, talvez por um certo sentimento de responsabilidade, daí
derivando até o fim de sua vida um grande e útil entretenimento.''

Sancho Pança, tolo sensato e ajudante incapaz de ajudar, mandou na
frente o seu cavaleiro. Bucéfalo sobreviveu ao seu. Homem ou cavalo,
pouco importa, desde que o dorso seja aliviado do seu fardo.
