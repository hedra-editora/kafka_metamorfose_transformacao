\part{Apêndice}

\chapterspecial{Franz Kafka e o mundo invisível}{}{Otto Maria Carpeaux}

O mundo do contista Franz Kafka é uma casa burguesa, solidamente construída na aparência, com uma fachada um pouco descuidada. Entramos, e respiramos o ar das penúrias dolorosas, de quartos mal ventilados. Apodera-se de nós o sentimento
do \textit{déjà vu}, de já ter visto tudo isso. A escada range. O sótão é uma
loja de recordações. Um canto guarda os brinquedos esquecidos. Recordações,
recordações. Os mortos surgem. Os fantasmas que apavoravam a criança.
Figuras de demônios. Um labirinto. Delírio. Fuga. Nenhuma saída.
Voltamo-nos para o outro lado: aparece a face de Deus.

Franz Kafka não é um poeta religioso: não trata nunca de religião nas suas obras.
Mas é um espírito profundamente angustiado; e o seu mundo é cheio de
seres sobrenaturais, donde emana uma impressão inquietante, como o
encontro com uma mitologia desconhecida, que aparecesse, de repente, na
nossa vida quotidiana. Esta irrupção do sobrenatural no mundo não o
salva: enche o homem de terrores desconhecidos. O \textit{numen} de Kafka é um \textit{numen tremendum}. A religião de Kafka não é a religião fácil dos bem-pensantes, a quem o seu Deus garante todas as
ordens deste mundo; o Deus de Kafka faz estremecer os fundamentos do céu
e da terra. ``Minha fé é como uma guilhotina, assim leve e assim
pesada.'' É a ameaça mortal que antecede a esperança vital.

Esta é a religião daqueles que a psicologia religiosa de William James chama os \textit{twice-born},
aqueles que nascem duas vezes, aqueles cuja fé irrompe das convulsões duma agonia: Agostinho, Martinho Lutero, Blaise Pascal, Soeren Kierkegaard.

Esses terrores e esses esplendores, Kafka os escondeu nos andrajos da vida quotidiana, pois ``quem vir descoberta a face de Deus morrerá''.

A pessoa e a vida de Franz Kafka acham-se também cobertas por um véu.
Nasceu em 1883 em Praga, filho de família pequeno-burguesa, dessa
nacionalidade incerta, germano-tcheco-judia, característica dos meios
intelectuais dessa cidade. Desde a sua infância, o humanismo alemão
desses meios é flanqueado pelo cabalismo
judaico e pelo misticismo eslavo.

\begin{verse}
Estou\\
Limitado ao norte pelos sentidos, ao sul pelo medo\\
A leste pelo apóstolo São Paulo, a oeste pela minha educação.\\
Murilo Mendes
\end{verse}

A vida corre-lhe nos quadros da burocracia subalterna.
Tísico, morre num sanatório de Viena, em 1924. No testamento ordena a destruição dos seus
manuscritos, que o executor, Max Brod, editará arbitrariamente.

A sua obra se compõe: de aforismos, que se alongam às vezes em parábolas; de parábolas, que se estendem às vezes em contos; de contos, dos quais três
se desenvolvem em romances, fragmentários, da mais alta concisão, e
cujo assunto se poderia condensar em parábola ou aforismo. A língua é
muito límpida, carregada de estranhas metáforas. Kafka descreve a vida
quotidiana dos escritórios, dos cafés, das casas de família; mas esses
lugares banais são cheios de potenciais demoníacos,
contra os quais o homem luta desesperadamente. Esse misto de clareza e
de mistério revela a fragilidade do nosso mundo, espreitado pela
catástrofe. Acontecimentos simples revestem-se de uma tensão febril. A língua lúcida faz
adivinhar um outro mundo. As personagens falam, comem, dormem, seguem os
caminhos escuros e estreitos; mas são os caminhos do inferno e do
paraíso, são os caminhos \textit{per realia ad realiora}.\footnote{``Pelas coisas reais ao mais real.''}

O primeiro romance publicado depois da morte do autor foi \textit{O processo}. O
seu herói chama-se K., simplesmente K. Um dia, na rua, K. é subitamente
preso. Explicam-lhe que fora instaurado contra ele um importante
processo criminal; aconselham-no a confessar e, em seguida, soltam-no
a fim de que possa prosseguir na sua defesa. A prisão não passava de uma
provocação por parte daquele estranho tribunal: o próprio K. tem de
criar pelas suas atitudes as razões da sua absolvição ou condenação. E
cria o delito mortal, prevalecendo-se obstinadamente da sua inocência.
Faz tudo o que se pode fazer: contrata um advogado e um médico, corrompe
o carcereiro e o escrivão. Nenhum destes compreende melhor o processo do
que K., mas todos estão convencidos da justiça e da onipotência do
tribunal; aconselham-no a confessar um crime que K. não conhece e não
quer conhecer. De maneira misteriosa, todos são empregados do tribunal, assim como nós outros
executamos, sem o saber e sem o querer, os desígnios da Providência. Pelas suas
atividades, K. não faz mais que jogar o processo contra si mesmo. Obstina-se.
Pelas suas providências apressa a catástrofe que será a sua condenação e
execução. O delito desconhecido está vingado.

\textit{O processo} é um apólogo e uma apologia, ao mesmo tempo. Sob o véu da
alegoria, Kafka instrui uma acusação contra a justiça do tribunal
divino. O delito desconhecido é o pecado original. A prisão é o signo da
predestinação. E o que K.
evita pelas suas falsas atividades é a graça. Há nesse romance uma
lembrança incerta de certas palavras do Senhor: ``Muitos serão os
chamados, mas poucos os eleitos'', e ``Aquele que quiser salvar sua vida
a perderá''. Mas as palavras evangélicas perdem-se neste mundo de
provação e desespero, onde a todo momento o
tribunal está presente e a força armada. ``É somente a noção que temos
do tempo'' --- diz Kafka --- ``que nos faz datar o juízo final; na verdade é uma corte
  marcial cuja audiência está aberta todos os nossos dias.'' Mas o céu
  negro se iluminará, um dia, sobre estas cenas de horror. No seu diário
  Kafka copiou as palavras de Lutero: ``Deus não é inimigo dos
  pecadores, mas somente dos descrentes que não reconhecem os próprios
  pecados nem procuram o apoio de Cristo, mas que procuram,
  temerariamente, a purificação em si mesmos''.

Em torno deste romance, alguns contos explicam a situação metafísica do homem. \textit{A colônia penitenciária} é uma como espécie de continuação de \textit{O
processo}. Nesta colônia, uma terrível máquina de precisão grava no corpo
dos forçados, por meio de agulhas incandescentes, os nomes dos delitos,
que são desconhecidos dos próprios condenados. A tortura pela qual a sua culpa lhes será revelada
é a única esperança, pois saber o nome do delito é a condição preliminar para
saber justificar-se.

Em \textit{A transformação}, um jovem é subitamente transformado num horrível inseto que os seus próprios parentes querem matar. O homem, submergido
pela vida banal de todos os dias, não é mais a imagem de Deus; não se
pode deter essa queda onde se desejaria, em alguma etapa propícia; e a queda torna-se radical até se perder o direito de existir.

A transformação tornou-se definitiva nesta pequena obra-prima chamada \textit{A preocupação do Pai Celeste}. É objeto da inquietação do Pai
misericordioso uma bobina, destituída de fios; coisa absolutamente inútil, sem nenhuma significação, mas
que não descansa nunca, que sobe e desce incessantemente a escada, até o
último dia. --- ``Como te chamas?'' --- ``Odradek''; palavra eslava, de
origem incerta, que significa ``apóstata''.

Em todas essas parábolas,
como em \textit{O processo}, o homem é a vitima passiva da perseguição celeste,
lembrando \textit{Hound of Heaven}, de Francis Thompson. Mas Kafka não condena a
atividade: ``Há dois pecados cardeais donde se poderiam deduzir todos
os outros: a impaciência e a preguiça. Por causa da impaciência foram
expulsos do paraíso; por causa da preguiça lá não podem voltar''. O que Kafka deseja excluir é a falsa direção das nossas atividades, no sentido
da segurança neste mundo. No conto \textit{A toca de texugo}, o animal, temendo a perseguição dos
cães, decide alargar e fortificar o seu edifício subterrâneo. Cava
buracos sobre buracos, corredores sobre corredores, até que afinal esquece a única saída. Então o animal agacha-se no seu canto,
aprisionado e sem saída, e espera, indefinidamente, numa estranha solidão, atento aos ruídos funestos do mundo exterior, ou ao silêncio,
ainda mais terrível.

A falsa direção das atividades humanas é o assunto
da obra-prima de Kafka: o romance inacabado \textit{O Castelo}.

Ainda aqui o herói chama-se K., simplesmente K. O seu adversário não é, desta vez, o
tribunal, mas o Castelo, o lugar onde a graça está concentrada. Ao pé
desse Castelo há uma aldeia, onde os camponeses, crentes humildemente submissos, executam as suas tarefas diárias. K. também desejaria ser camponês nessa
aldeia. É preciso frisar: ele o quer, ele o exige mesmo. Desejaria obrigar o Castelo a conceder-lhe o direito de permanência na aldeia. Quer forçar esta comunhão dos fiéis, sem ter obtido a graça.

Numa fria tarde de inverno, K. chega, contando
com a piedade, que não fará voltar o peregrino. Com efeito, o hospedeiro
acolhe-o. K. é modesto; quer somente achar um emprego de diarista.
Sim, há sempre possibilidades. Nesse ínterim o filho do castelão aparece
para expulsá-lo. K. desesperadamente recorre à mentira: ``O Castelo contratou-me como nivelador''. Resolvem telefonar para o Castelo. E o
Castelo responde de maneira surpreendente (``K. estremeceu um pouco''): ``Sim, K. é o nivelador contratado''. É o primeiro dom voluntário da graça: mas
contém uma punição. Pois o Castelo acrescenta: ``K. tem, portanto,
permissão para ficar; mas o seu contrato foi um lamentável engano, aqui
não temos trabalho para um nivelador. K. tem permissão para ficar, mas
não para trabalhar''.

Deste modo, K. encontra-se impossibilitado de
verificar o contrato surrupiado, justificar sua presença na aldeia. Sua
vida será vazia, destituída de qualquer sentido, como a nossa vida
quotidiana sem a vocação interior. K. não está contente. Não quer ser
tolerado. Quer o direito de permanecer, o direito. Quer extorquir a
graça. Recorre a meios impuros, perde-se em mentiras e subterfúgios.
Tudo em vão. Esgotado, enfim, cai gravemente doente. Espera a morte.

Eis-nos nas últimas linhas do fragmento. Uma anotação explica-nos o fim:
``Quando K. está à morte, chega a decisão definitiva do Castelo: K. não
tem nenhum direito de permanecer na aldeia; mas considerando-se certas
circunstâncias acessórias, ser-lhe-á permitido que aí permaneça até a
morte''.

Em \textit{O Processo}, o Céu instaura processo contra o homem. Em \textit{O Castelo}, o homem instaura processo contra o Céu. É o cúmulo da
temeridade titânica. ``Uns negam a miséria evocando o sol; outros negam
o sol evocando a miséria.'' O homem, em Kafka, não vê na sua miséria a
consequência da sua condição humana. Revolta-se. Acusa Deus, como Ivan Karamazov. A face de Deus, em sua obra, adquire traços blasfêmicos.

Em toda parte, no mundo desse Deus, há tribunais e forcas. Não parece que
esse Deus queira a redenção do homem. ``O verdadeiro caminho desdobra-se
sobre uma corda, lançada muito perto do chão; parece ser destinada
mais a fazer tropeçar que a ser transposta.'' Às vezes Kafka atinge uma
inversão diabólica: ``Leopardos forçavam o templo e esvaziavam os vasos
sagrados. Isto se repetia frequentemente. Até que conseguiram calcular a
hora em que chegavam e faziam do incidente uma parte do cerimonial''.
Tais blasfêmias lembram a zoolatria dos egípcios ou o Demiurgo mau dos
gnósticos. Mas um outro aforismo diz: ``O nosso mundo não é mais do que
um mau humor de Deus. Há esperança, muita esperança, mas não para nós
homens''. Este ``não para nós homens'' equivale a uma grande confissão,
que restabelece a ordem dos valores. ``Todas essas parábolas dizem somente que o incompreensível é incompreensível.'' Na aparência dessas
parábolas Deus não tem razão; mas esta falta de razão significa somente
uma incapacidade do homem em face do mandamento de Deus. Na aparência
dessas parábolas, Deus se cala; mas isto significa somente que o mundo
não o está escutando. Há, portanto, esperança, muita esperança. No fim
de \textit{O Castelo}, a graça aparece. Fato simbólico: Kafka não estava
destinado a escrever esse fim.

Franz Kafka, segundo uma frase de
Kierkegaard, ``aspirava a uma imortalidade mais alta que a da glória''.
Kafka desejava que a sua obra morresse com ele para servir de testemunha
em seu favor, perante o tribunal de Deus. A despeito dele, o seu dia chegará, se já não chegou.

À propagação dessa obra opõem-se obstáculos do destino. A sua publicação póstuma não encontrou nem leitores nem críticos. Dez anos depois da sua morte,
um André Gide, um Charles Du Bos deploram a inacessibilidade das obras,
a inexistência de traduções. Uma casa editora de Praga promete a publicação das obras completas, a \textit{Nouvelle Revue Française} traduz alguns contos. A
edição de Praga é interrompida pela derrota do Estado tcheco. A tradução
integral, prometida na França, talvez nunca apareça. A despeito de tudo,
o seu dia chegará, se já não chegou.

Todos esses obstáculos aprofundam
mais a virtude desse pensamento, em vez
de sufocá-lo. Existe uma herança que se deve conservar. A reflexão sobre o lugar de Kafka na literatura universal é o primeiro dever.

Feita a abstração de alguns pontos de contato com Heinrich von Kleist, o Kleist
do ensaio \textit{Sobre o teatro de bonecas}, e com E.\,T.\,A Hoffmann, a presença
de Kafka na literatura alemã é simplesmente ocasional. O seu lugar está
na literatura europeia do após-guerra.

O simbolismo de Kafka perturba
o mundo, pela estranha transposição dos acentos, pela desvalorização dos fatos tradicionais, pela revelação de um mundo mais real atrás do mundo real dos bem-pensantes: \textit{per realia ad realiora}. Eis o lema de Anton Tchecov, a quem Kafka deve a técnica do conto. Mas
um traço significativo distingue Kafka radicalmente deste grande
contista pessimista do \textit{fin de siècle}: a noção do tempo. Os homens de
Tchecov vivem no seu tempo, no tempo do seu
mundo. Mas o tempo, em Kafka, é um fato extramundano. Não é o tempo
psicológico de Proust. É antes um tempo religioso: o caminho da aldeia ao castelo, ``dois quilômetros mais ou menos'', leva séculos, eônios,\footnote{} para ser
percorrido; não se pode dizer a respeito de nenhuma obra de Kafka em que
século decorre a ação dela. A era dos deuses e a vida quotidiana dos
nossos dias se confundem. Não existe tempo, há unicamente uma data: a da
irrupção do divino no mundo, acontecimento que se repete todos os dias, todas as horas.

Esta ausência do tempo humano destrói a estrutura
normal do mundo e isola os
homens em desertos de eternidade glacial, tomando-os comparáveis às
personagens plásticas de um De Chirico, aos cantos ``homófonos'' de um
Stravinsky, aos anjos de um Rilke. A psicologia desses homens é uma
psicologia de monstros revoltados, como nos romances fantásticos de
Julien Green. A sua vida quotidiana é destituída de sentido, como nos
contos de um Bontempelli. E a sua vida real se
passa na atmosfera mágica dos romances de Marcel Jouhandeau. Enfim, este
mundo acha a sua expressão final nos poemas apocalípticos dum
Pierre-Jean Jouve que precedem a catástrofe. O dia de Kafka chegou.

Todas essas comparações só têm como fim estabelecer mais solidamente as oposições. A corrente literária do após-guerra acha-se diante de um montão de 
ruínas. O mundo é um cadáver que se decompõe porque o espirito abandonou
o corpo. A literatura e o pensamento modernos tentaram contentar-se
somente com os destroços, olhando-os primeiro como brinquedos de uma
nova infância, e em seguida como pedras para a construção do futuro;
eram as etapas do primitivismo e do construtivismo. Mas se reconhecerá o
verdadeiro estado de coisas e um profundo desespero prevalecerá. Este
desespero se conformará ou não se conformará: ele afirma e confirma a
decomposição do mundo por meio de uma nova psicologia, ou se insurge
contra essa decomposição pelas expressões de um pessimismo cínico. São
estas as posições do romance e da poesia modernos.

O que é comum a todas
essa correntes é o relativismo, que já não admite a integridade do
mundo, exceto a daqueles, não raros, que mergulham na fé tradicional.
A atitude de Franz Kafka é muito diferente. Não se contenta com os
destroços, como os ``fragmentistas'' italianos; não se conforma nem
decompõe. Não é nem tradicional nem relativista. Entre dois mundos e
entre duas épocas, coloca-se em caminho; está a caminho de Damasco.

Esta atitude o situa no meio de duas grandes correntes dos nossos tempos: uma
na França, os novos estudos pascalianos que giram em torno do problema da
graça e inspiram até o André Gide de \textit{L'École des femmes}; a outra na
Alemanha, a ``Teologia Dialética'' de Karl Barth e de Emil Brunner,
que gira em torno do abismo dialético, a incomensurabilidade entre Deus
e o mundo, e faz ressuscitar a obra esquecida de Soeren Kierkegaard.

No abismo entre o Deus soberano dos dialéticos e o homem falido de Pascal, Kafka procura o lugar da graça. É Pascal quem define a situação. No
artigo \textsc{xv} das \textit{Pensées} enumera as quatro possibilidades do homem. Primeiro, o homem
conhece
a Deus, mas não conhece a sua própria miséria; é o caso do farisaísmo
orgulhoso. Segundo, o homem conhece a sua miséria, mas não conhece a
Deus; é o desespero ateístico. Terceiro, o homem conhece a Deus e a
sua própria miséria, mas não a graça; é a angústia. Quarto, o homem
reconhece em Jesus Cristo seu Deus, sua miséria e sua graça.

A posição de Kafka é a terceira. É a posição do judaísmo perante o seu Messias
encarnado. Mas é também a posição atual do mundo apóstata, que renuncia
à graça e se declara pagão, cheio de orgulho e de angústia. Não se é
mais pagão depois de Jesus Cristo: a velha inocência desapareceu; ou
procuramo-Lo, ou renegamo-Lo. Em vão ``a angústia da lei'' maltrata o
rabino Saul antes de ter ele visto a luz do mundo. Uma fé vem nascer no
caos de uma alma em desespero. ``Como cumprir a vontade de Deus? Teme-se
que essa lei não seja mais do que uma tentação. E se o seu cumprimento
não representar nada perante Deus?'' É um aforismo de Kafka.
Mas o apóstolo Paulo poderia ter dito isso. É a confissão de um homem no caminho de Damasco.

O caminho de Damasco é a única saída desta prisão que é o nosso mundo envenenado. Todos os outros caminhos são subterfúgios inúteis, tergiversações que nos abismam cada vez mais, sem a possibilidade de uma libertação.
Sem a graça não se escapa deste mundo. Todas as seguranças exteriores são vãs.
Em vão nos entrincheiramos nas linhas Maginot da nossa ``toca de texugo''.
Enfim, somos os prisioneiros das nossas próprias prisões, para assistir, impotentes, à
nossa derrota decisiva. Só o caminho misterioso de Damasco é que liberta dos terrores exteriores,
para preparar ``o segundo nascimento'': é o caminho da apocalipse do
mundo para a escatologia da alma.

A obra de Franz Kafka é um indicador
na direção desse caminho. Nela se lê o
seu aforismo, cheio de aflição e de esperança: ``Quem procurar não
encontrará; quem não procurar, será encontrado''. E uma voz lhe responde, através de Pascal:
\textit{Console-toi, tu ne me chercherais pas si tu ne m'avais trouvé}.\footnote{``Consola-te: não me procurarias se já não me tivesses encontrado.''}