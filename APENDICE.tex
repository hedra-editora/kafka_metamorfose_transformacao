ENSAIOS REUNIDOS

percorreu, e ao fim do qual achou o mágico poema que, agora, já não será
misterioso:

``EIIe est retrouvée! Quoi? l 'Eternité! C'est la mer mêlée Au soleil.
Mon tlme éternelle Observe ton voeu Malgré la nuit seu/e t6S . Et le
Jour en feu.'' Há nesta poesia um fim e um começo. O espírito da
fronteira nela está, da

fronteira entre o dizível e o indizível; entre a vida e a morte. Só um
atravessou essa fronteira, a fronteira do país donde não se volta:
Arthur Rimbaud.

Franz Katka e o mundo invisível O MUNDO do contista Franz Kafka é uma
casa burguesa, solidamente construída

na aparência, com uma fachada um pouco descuidada. Entramos, e
respiramos o ar

das penúrias dolorosas, de quartos mal ventilados. Apodera-se de nós o
sentimento

do déjà vu, de já ter visto tudo isso. A escada range. O sótão é urna
loja de recorda16S A citação, tal como aparece em Carpeaux, est�
incorreta. Cito as (\}JuiJ1Yis Complites de Rimbaud, edição Pl�lade, que
�:

``EI/e est retroullée. Quoi? - L'Eternité. C'est la mer al/ée Avec /e
solei/.

Ame sentinelle, Murmurons /'aveu De la nuit si nulle Et du jour en feu
.. . ''

\begin{itemize}
\tightlist
\item
  o que � bastante diferente da citação, provavelmente de memória, feita
  por OMC. H� tradução, perfeita, de Ivo Barroso: '' Achada, � verdade?
\end{itemize}

Quem? A Eternidade.

É o mar que se evade

com o sol, ã tarde.

Alma sentinela, Murmura teu rogo De noite tão nula E um dia de fogo.�
U.W.)

1 53

ÜTrO MARIA CARPI!AUX

ções. Um canto guarda os brinquedos esquecidos. Recordações,
recordações. Os mortos surgem. Os fantasmas que apavoravam a criança.
Figuras de demônios. Um labirinto. Delírio. Fuga. Nenhuma saída.
Voltamo-nos para o outro lado: aparece a face de Deus.

Franz Kafka não é um poeta religioso: não trata nunca de religião nas
suas obras.

Mas é um espírito profundamente angustiado; e o seu mundo é cheio de
seres sobrenaturais, donde emana uma impressão inquietante, como o
encontro com uma mitologia desconhecida, que aparecesse, de repente, na
nossa vida quotidia­ na. Esta irrupção do sobrenatural no mundo não o
salva: enche o homem de terro­

res desconhecidos. O numen de Katka é um numen tremendum A religião de

Katka não é a religião fãcil dos bem-pensantes, a quem o seu Deus
garante todas as

ordens deste mundo; o Deus de Kafka faz estremécer os fundamentos do céu
e da terra. ``Minha fé é como uma guilhotina, assim leve e assim
pesada.'' É a ameaça mortal que antecede a esperança vital. Esta é a
religião daqueles que a psicologia religiosa de William james chama os

``twice-born

``,

aqueles que nascem duas vezes, aqueles cuja fé irrompe das convul­

sões duma agonia: Agostinho, Martinho Lutero, Blaise Pascal, Soeren
Kierkegaard. Esses terrores e esses esplendores, Kafka os escondeu nos
andrajos da vida quotidiana, pois ``quem vir descoberta a face de Deus
morrerã''. A pessoa e a vida de Franz Kafka acham-se também cobertas por
um véu.

Nasceu em 1883 em Praga, filho de família pequeno-burguesa, dessa
nacionalidade incerta, germano-tcheco-judia, caracteristica dos meios
intelectuais dessa cidade. Desde a sua infância, o humanismo alemão
desses meios é flanqueado pelo cabalismo

judaico e pelo misticismo eslavo.

``Estou Limitado ao norte pelos sentidos, ao sul pelo medo A lestepelo
apóstolo São Paulo, a oestepela minha educação.

''

(Murilo Mendes.) A vida corre-lhe nos quadros da burocracia subalterna.
Ttsico, morre num sana­

tório de Viena, em 1924. No testamento ordena a destruição dos seus
manuscritos, que o executor, Max Brod, editarã arbitrariamente.

A sua obra se compõe: de aforismos, que se alongam às vezes em
parãbolas; de

parãbolas, que se estendem às vezes em contos; de contos, dos quais três
se de­ senvolvem em romances, fragmentãrios, da mais alta concisão, e
cujo assunto se poderia condensar em parãbola ou aforismo. A língua é
muito límpida, carregada de estranhas metãforas. Katka descreve a vida
quotidiana dos escritórios, dos cafés, das casas de família; mas esses
lugares banais são cheios de potenciais demoníacos,

contra os quais o homem luta desesperadamente. Esse misto de clareza e
de mis­ tério revela a fragilidade do nosso mundo, espreitado pela
catãstrofe. Aconteci-

mentos simples revestem-se de uma tensão febril. A língua lúcida faz
adivinhar um outro mundo. As personagens falam, comem, dormem, seguem os
caminhos escu­ ros e estreitos; mas são os caminhos do inferno e do
paraíso, são os caminhos ``per

realia ad realtora � 166

O primeiro romance publicado depois da morte do autor foi Oprocesso. O
seu herói chama-se K., simplesmente K. Um dia, na rua, K. é subitamente
preso. Expli­ cam-lhe que fora instaurado contra ele um importante
processo criminal; aconse­ lham-no a confessar e, em seguida, soltam-no
a fun de que possa prosseguir na sua defesa. A prisão não passava de uma
provocação por parte daquele estranho tribu­ nal: o próprio K. tem de
criar pelas suas atitudes as razões da sua absolvição ou condenação. E
cria o delito mortal, prevalecendo-se obstinadamente da sua inocên­ cia.
Faz tudo o que se pode fazer: contrata um advogado e um médico, corrompe
o carcereiro e o escrivão. Nenhum destes compreende melhor o processo do
que K., mas todos estão convencidos da justiça e da onipotência do
tribunal; aconse­ lham-no a confessar um crime que K. não conhece e não
quer conhecer. De manei­

ra misteriosa, todos são empregados do tribunal, assim como nós outros
executa­

mos, sem o saber e sem o querer, os desígnios da Providência. Pelas suas
ativida­

des, K. não faz mais que jogar o processo contra si mesmo. Obstina-se.
Pelas suas providências apressa a catãstrofe que serã a sua condenação e
execução. O delito desconhecido está vingado.

O processo é um apólogo e uma apologia, ao mesmo tempo. Sob o véu da
alegoria, Kafka instrui uma acusação contra a justiça do tribunal
divino. O delito desconhecido é o pecado original. A prisão é o signo da
predestinação. E o que K.

evita pelas suas falsas atividades é a graça. Há nesse romance uma
lembrança incerta de certas palavras do Senhor: ``Muitos serão os
chamados, mas poucos os eleitos'', e ``Aquele que quiser salvar sua vida
a perderá''. Mas as palavras evangéli­ cas perdem-se neste mundo de
provação e desespero, onde a todo momento o

tribunal estã presente e a força armada. ``É somente a noção que temos
do tempo''

\begin{itemize}
\tightlist
\item
  diz Kafka - ``que nos faz datar o juízo final; na verdade é uma corte
  marcial cuja audiência estã aberta todos os nossos dias.'' Mas o céu
  negro se iluminarã, um dia, sobre estas cenas de horror. No seu diário
  Kafka copiou as palavras de Lutero: ``Deus não é inimigo dos
  pecadores, mas somente dos descrentes que não reco­ nhecem os próprios
  pecados nem procuram o apoio de Cristo, mas que procuram,
  temerariamente, a purificação em si mesmos.''
\end{itemize}

Em tomo deste romance, alguns contos explicam a situação metatisica d o
ho­

mem. A col6nia penuenciárta é uma como espécie de continuação de O
processo. Nesta colônia, uma tenivel máquina de precisão grava no corpo
dos forçados, por meio de agulhas incandescentes, os nomes dos delitos,
que são desconhecidos dos

próprios condenados. A tortura pela qual a sua culpa lhes será revelada
é a única 166 ``Pelas coisas reais ao mais real. • {[}N.E.)

1 55

OTro MARIA CARPI!.AUX

esperança, pois saber o nome do delito é a condição preliminar para
saber justifi­ car-se.

Em A transformação, 167 um jovem é subitamente transformado num horrivel

inseto que os seus próprios parentes querem matar. O homem, submergido
pela vida banal de todos os dias, não é mais a imagem de Deus; não se
pode deter essa queda onde se desejaria, em alguma etapa propícia; e a
queda torna-se radical até se perder o direito de existir.

A transformação tomou-se definitiva nesta pequena obra-prima chamada A
pre­ ocupaçtlo do Pai Celeste. É objeto da inquietação do Pai
misericordioso uma bobi­

na, destituída de fios; coisa absolutamente inútil, sem nenhuma
significação, mas

que não descansa nunca, que sobe e desce incessantemente a escada, até o
último dia. - ``Como te chamas?'' - ``Odradek''; palavra eslava, de
origem incerta, que significa ``apóstata''. Em todas essas parãbolas,
como em Oprocesso, o homem é a vitima passiva da pe rseguição celeste,
lembrando Hound of Heaven, de Francis Thompson. Mas Kaflca não condena a
atividade: ``Hã dois pe cados cardeais donde se poderiam deduzir todos
os outros: a impaciência e a preguiça. Por causa da impaciência foram
expulsos do paraíso; por causa da preguiça lã não podem voltar.'' O que

Katka deseja excluir é a falsa direção das nossas atividades, no sentido
da segurança

neste mundo. No conto A toca de texugo, o animal, temendo a perseguição
dos

cães, decide alargar e fortificar o seu edificio subterrâneo. Cava
buracos sobre bura­ cos, corredores sobre corredores, até que afmal
esquece a única saída. Então o animal agacha-se no seu canto,
aprisionado e sem saída, e espera, indefmidamente, numa estranha
solidão, atento aos ruídos funestos do mundo exterior, ou ao silên­ cio,
ainda mais tenivel. A falsa direção das atividades humanas é o assunto
da obra-prima de Kafka: o romance inacabado O Castelo. Ainda aqui o
herói chama-se K., simplesmente K. O seu adversãrio não é, desta vez, o
tribunal, mas o Castelo, o lugar onde a graça está concentrada. Ao pé
desse Castelo há uma aldeia, onde os camponeses, crentes humildemente
submissos,

executam as suas tarefas diãrias. K. também desejaria ser camponês nessa
aldeia. É preciso frisar: ele o quer, ele o exige mesmo. Desejaria
obrigar o Castelo a conce­ der-lhe o direito de permanência na aldeia.
Quer forçar esta comunhão dos fiéis,

sem ter obtido a graça. Numa fria tarde de inverno, K. chega, contando
com a piedade, que não fará voltar o peregrino. Com efeito, o hospedeiro
acolhe-o. K. é modesto; quer somen­ te achar um emprego de diarista.
Sim, há sempre possibilidades. Nesse ínterim o filho do castelão aparece
para expulsã-lo. K. desesperadamente recorre à mentira: ``O Castelo
contratou-me como nivelador.'' Resolvem telefonar para o Castelo. E o

Castelo responde de maneira surpreendente (``K. estremeceu um pouco''):
``Sim, 167 Publicado no Brasil como Metamorfose. IN.E.)

1 56

ENSAIOS REUNIDOS

K. é o nivelador contratado. • É o primeiro dom voluntário da graça: mas
contém uma punição. Pois o Castelo acrescenta: ``K. tem, portanto,
permissão para ficar; mas o seu contrato foi um lamentável engano, aqui
não temos trabalho para um nivelador. K. tem permissão para ficar, mas
não para trabalhar.'' Deste modo, K. encontra-se impossibilitado de
verificar o contrato surrupiado, justificar sua presença na aldeia. Sua
vida serã vazia, destituída de qualquer sentido, como a nossa vida
quotidiana sem a vocação interior. K. não está contente. Não quer ser
tolerado. Quer o direito de permanecer, o direito. Quer extorquir a
graça. Recorre a meios impuros, perde-se em mentiras e subterfúgios.
Tudo em vão. Esgotado, enfim, cai gravemente doente. Espera a morte.
Eis-nos nas últimas linhas do fragmento. Uma anotação explica-nos o fim:
``Quando K. está ã morte, chega a decisão definitiva do Castelo: K. não
tem nenhum direito de permanecer na aldeia; mas considerando-se certas
circunstâncias acessórias, ser­ lhe-á permitido que aí permaneça até a
morte.'' Em O Processo, o Céu instaura processo contra o homem. Em O
Castelo, o homem instaura processo contra o Céu. É o cúmulo da
temeridade titânica. ``Uns negam a miséria evocando o sol; outros negam
o sol evocando a miséria.'' O ho­ mem, em Katka, não vê na sua miséria a
conseqüência da sua condição humana. Revolta-se. Acusa Deus, como Ivan
Karamazov. A face de Deus, em sua obra, adquire traços blasfêmicos. Em
toda parte, no mundo desse Deus, há tribunais e forcas. Não parece que
esse Deus queira a redenção do homem. ``O verdadeiro caminho desdobra-se
so­ bre uma corda, lançada muito perto do chão; parece ser destinada
mais a fazer tropeçar que a ser transposta. • Às vezes Kafka atinge uma
inversão diabólica:''Le­ opardos forçavam o templo e esvaziavam os vasos
sagrados. Isto se repetia freqüentemente. Até que conseguiram calcular a
hora em que chegavam e faziam do incidente uma parte do cerimonial. ''
Tais blasfêmias lembram a zoolatria dos egípcios ou o Demiurgo mau dos
gnósticos. Mas um outro aforismo diz: ``O nosso mundo não é mais do que
um mau humor de Deus. Há esperança, muita esperan­ ça, mas não para nós
homens. • Este''não para nós homens'' equivale a uma grande confissão,
que restabelece a ordem dos valores. ``Todas essas parãbolas dizem so­
mente que o incompreensível é incompreensível.'' :-.la aparência dessas
parãbolas Deus não tem razão; mas esta falta de razão significa somente
uma incapacidade do homem em face do mandamento de Deus. Na aparência
dessas parãbolas, Deus se cala; mas isto significa somente que o mundo
não o estã escutando. Há, portanto, esperança, muita esperança. No fim
de O Castelo, a graça aparece. Pato simbólico: Kafka não estava
destinado a escrever esse fim. Pranz Kafka, segundo uma frase de
Kierkegaard, ``aspirava a uma imortalidade mais alta que a da glória''.
Kafka desejava que a sua obra morresse com ele para servir de testemunha
em seu favor, perante o tribunal de Deus. A despeito dele, o seu dia
chegará, se já não chegou. 1 57

Orro MwA CARPEAux

A propagação dessa obra opõem-se obstãculos do destino. A sua publicação

póstuma não encontrou nem leitores nem críticos. Dez anos depois da sua
morte,

um André Gide, um Charles Du Bos deploram a inacessibilidade das obras,
a inexistência de traduções. Uma casa editora de Praga promete a
publicação das

obras completas, a Nouvelle Revue Françatse traduz alguns contos. A
edição de Praga é interrompida pela derrota do Estado tcheco. A tradução
integral, prometida na França, talvez nunca apareça. A despeito de tudo,
o seu dia chegará, se já não chegou. Todos esses obstáculos aprofundam
mais a virtude desse pensamento, em vez

de sufocá-lo. Existe uma herança que se deve conservar. A reflexão sobre
o lugar de Kafka na literatura universal é o primeiro dever.

Feita a abstração de alguns pontos de contato com Heinrich von Kleist, o
Kleist

do ensaio Sobre o teatro de bonecas, e com E. T. A Hoffmann, a presença
de Kafka na literatura alemã é simplesmente ocasional. O seu lugar está
na literatura euro­ péia do após-guerra. O simbolismo de Kafka perturba
o mundo, pela estranha transposição dos acen­ tos, pela desvalorização
dos fatos tradicionais, pela revelação de um mundo mais

real atrás do mundo real dos bem-pensantes: '\,'jler realia ad realiora
� Eis o lema de Anton Tchecov, a quem Kafka deve a técnica do conto. Mas
um traço significativo distingue Kafka radicalmente deste grande
contista pessimista do fin de siecle: a noção do tempo. Os homens de
Tchecov vivem no seu tempo, no tempo do seu

mundo. Mas o tempo, em Kafka, é um fato extramundano. Não é o tempo
psico­

lógico de Proust. É antes um tempo religioso: o caminho da aldeia ao
castelo, ``dois

quilômetros mais ou menosn, leva séculos, eônios,168 para ser
percorrido; não se pode dizer a respeito de nenhuma obra de Kafka em que
século decorre a ação dela. A era dos deuses e a vida quotidiana dos
nossos dias se confundem. Não existe tempo, há unicamente uma data: a da
irrupção do divino no mundo, aconte­ cimento que se repete todos os
dias, todas as horas. Esta ausência do tempo humano destrói a estrutura
normal do mundo e isola os

homens em desertos de eternidade glacial, tomando-os comparáveis ãs
persona­

gens plásticas de um De Chirico, aos cantos ``homófonos'' de um
Stravinsky, aos anjos de um Rilke. A psicologia desses homens é uma
psicologia de monstros revoltados, como nos romances fantásticos de
julien Green. A sua vida quotidiana é destituída de sentido, como nos
contos de um Bontempelli. E a sua vida real se

passa na atmosfera mágica dos romances de Marcel )ouhandeau. Enfim, este
mun­ do acha a sua expressão final nos poemas apocalípticos dum
Pierre-)eanjouve que precedem a catástrofe. O dia de Kafka chegou. Todas
essas comparações só têm como fim estabelecer mais solidamente as

oposições. A corrente literária do após-guerra acha-se diante de um
montão de 68 A gra fia correta, em portugu�s. seria eões. As traduções
de livros de jung e Mircea Eliade acabaram popularizando a forma
errõnea, eons. (N.E.I 1

1 58

ENSAIOS REUNIDOS

ruínas. O mundo é um cadáver que se decompõe porque o espirito abandonou
o corpo. A literatura e o pensamento modernos tentaram contentar-se
somente com os destroços, olhando-os primeiro como brinquedos de uma
nova infância, e em seguida como pedras para a construção do futuro;
eram as etapas do primitivismo e do construti.ismo. Mas se reconhecerá o
verdadeiro estado de coisas e um pro­ fundo desespero prevalecerá. Este
desespero se conformará ou não se conformará: ele afirma e confirma a
decomposição do mundo por meio de uma nova psicolo­ gia, ou se insurge
contra essa decomposição pelas expressões de um pessimismo cínico. São
estas as posições do romance e da poesia modernos. O que é comum a todas
essa correntes é o relativismo, que jã não admite a integridade do
mundo, exceto a daqueles, não raros, que mergulham na fê tradici­ onal.
A atitude de Franz Kafka é muito diferente. Não se contenta com os
destro­ ços, como os •fragmentistas'' italianos; não se conforma nem
decompõe. Não é nem tradicional nem relativista. Entre dois mundos e
entre duas épocas, coloca-se em caminho; está a canúnho de Damasco.

Esta atitude o situa no meio de duas grandes correntes dos n ossos
tempos: uma

na França, os novos estudos pascalianos que giram em tomo do problema da
graça e inspiram até o André Gide de L 'Éco/e desfemmes; a outra na
Alemanha, a ``Teo­ logia Dialética'' de Karl Barth e de Ernil Brunner,
que gira em torno do abismo dialético, a incomensurabilidade entre Deus
e o mundo, e faz ressuscitar a obra esquecida de Soeren Kierkegaard. No
abismo entre o Deus soberano dos dialéticos e o homem falido de Pascal,

Kafka procura o lugar da graça. É Pascal quem define a situação. No
artigo XV das

Pensées enumera as quatro possibilidades do homem. Primeiro, o homem
conhece

a Deus, mas não conhece a sua própria miséria; é o caso do farisaísmo
orgulhoso. Segundo, o homem conhece a sua miséria, mas não conhece a
Deus; é o desespe­ ro ateístico. Terceiro, o homem conhece a Deus e a
sua própria miséria, mas não a graça; é a angústia. Quarto, o homem
reconhece em Jesus Cristo seu Deus, sua miséria e sua graça. A posição
de Kafka é a terceira. É a posição do judaísmo perante o seu Messias

encarnado. Mas é também a posição atual do mundo apóstata, que renuncia
à graça e se declara pagão, cheio de orgulho e de angústia. Não se é
mais pagão depois de Jesus Cristo: a velha inocência desapareceu; ou
procuramo-Lo, ou renegamo-Lo. Em vão ``a angústia da lei'' maltrata o
rabino Saul antes de ter ele visto a luz do mundo. Uma fé vem nascer no
caos de uma alma em desespero. ``Como cumprir a vontade de Deus? Teme-se
que essa lei não seja mais do que uma tentação. E se o seu cumprimento
não representar nada perante Deus?'' É um aforismo de Kafka.

Mas o apóstolo Paulo pode ria ter dito iss o. É a confissão de um homem
no caminho

de Damasco.

O caminho de Damasco é a única saída desta prisão que é o nosso mundo
envenenado. Todos os outros caminhos são subterfúgios inúteis,
tergiversações

1 59

OTro MARIA CAiu\textgreater I!A.UX

que nos abismam cada vez mais, sem a possibilidade de uma libertação.
Sem a

graça não se escapa deste mundo. Todas as seguranças exteriores são vãs.
Em vão

nos entrincheiramos nas linhas Maginot da nossa ``toca de texugo''.
Enfun, somos os

prisioneiros das nossas próprias prisões, para assistir, impotentes, à
nossa derrota

decisiva. S6 o caminho misterioso de Damasco é que liberta dos terrores
exteriores,

para preparar ``o segundo nascimento'': é o caminho da apocalipse do
mundo para a escatologia da alma. A obra de Franz Kafka é um indicador
na direção desse caminho. Nela se lê o

seu aforismo, cheio de aflição e de esperança: ``Quem procurar não
encontrará; quem não procurar, será encontrado.'' E uma voz lhe
responde, através de Pascal:

``Console-toi, tu ne me cbercberais pas si tu ne m 'avais trouvé.''169

Um enigma shakespeareano Exercício de literatura comparada CENSURA-SE
muitas vezes à jovem ciência da literatura comparada o valor puramente
histórico e pouco interpretativo dos seus estudos. O método do grande
critico e maior poeta inglês, T. S. Eliot, escapa a estas censuras:
``Método maravilho­ so que encara, em conjunto, toda a literatura
universal, e que compara as obras de diversos povos em diversas épocas,
sem consideração de pretendidas relações históricas, para tirar
conclusões gerais'' (Edmund Wilson). Devemos a este método a
redescoberta das poesias barrocas espanhola e inglesa. Eliot é
inimitável. Contu­ do, pode-se imaginar um método análogo, aplicado para
resolver certos problemas de critica, para explicar a profunda emoção
que emana de certas obras, em aparên­ cia menos bem-sucedidas. Obras que
fazem pressentir a presença escondida, ocul­ ta, duma força misteriosa
atrãs da superfície, como os contos de Franz Kafka; ou

corno aquela comédia Me�forMeasure(``Medida por medida''), de
Shakespeare:

um enredo, banal ou esquisito segundo o ponto de vista, escondendo uma
amere­

pensée metaf''tsica, explicável só pela comparação, sem consideração de
relações históricas, com obras com as quais nunca foi comparada. Quase
nunca Shakespeare inventava os argumentos das suas peças. Contentava­ se
em dramatizar contos ou então retocar velhas peças, com ligeiras
modificações. Num conto medíocre do escritor George Whetstone achou
assunto para transfigurá­ lo no mundo completo, maravilhoso, enigmático,
de Medida por medida.

O Duque de Viena, reconhecendo que, sob o seu reino indulgente, as leis
caíam

em desuso e se aproximava a anarquia moral, resolve abandonar por algum
tempo o país e confiar o governo ao seu conselheiro Ângelo, homem
conceituado pela Ui!l

``Consola-te: não me procurarias se jã não me tivesses encontrado. •
{[}N.E.)

160


