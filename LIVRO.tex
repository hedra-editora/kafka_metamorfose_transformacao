\documentclass[showtrims,		% Mostra as marcas de corte em cruz
			   %trimframe,		% Mostra as marcas de corte em linha, para conferência
			   11pt				% 8pt, 9pt, 10pt, 11pt, 12pt, 14pt, 17pt, 20pt 
			   ]{memoir}
\usepackage[brazilian,
			% english,
			% italian,
			% ngerman,
			% french,
			% russian,
			% polutonikogreek
			]{babel}
\usepackage{anyfontsize}			    % para tamanhos de fontes maiores que \Huge 
\usepackage{relsize}					% para aumentar ou diminuir fonte por pontos. Ex. \smaller[1]
\usepackage{fontspec}					% para rodar fontes do sistema
\usepackage[switch]{lineno} 			% para numerar linhas
\usepackage{lipsum}						% para colocar textos lipsum
\usepackage{alltt}						% para colocar espaços duplos. Ex: verso livre
\usepackage{graphicx}					% para colocar imagens
\usepackage{float}						% para flutuar imagens e tabelas 		
\usepackage{lettrine}					% para capsulares
\usepackage{comment}					% para comentar o código em bloco \begin{comment}...
\usepackage{adforn}						% para adornos & glyphs
\usepackage{xcolor}					 	% para texto colorido
\usepackage[babel]{microtype}			% para ajustes finos na mancha
\usepackage{enumerate,enumitem}			% para tipos diferentes de enumeração/formatação ver `edlab-extra.sty`
\usepackage{url}					% para citar sites \url
\usepackage{marginnote}					% para notas laterais
\usepackage{titlesec}					% para produzir os distanciamentos entre pontos no \dotfill

%\usepackage{makeidx} 					% para índice remissivo

\usepackage{floatpag}
\usepackage{edlab-penalties}
\usepackage{edlab-git}
\usepackage{edlab-toc}					% define sumário
\usepackage{edlab-extra}				% define epígrafe, quote
\usepackage[largepost]{edlab-margins}
\usepackage[%semcabeco, 				% para remover cabeço, sobe mancha e mantem estilos
			]{edlab-sections}			% define pagestyle (cabeço, rodapé e seções)
\usepackage[%
			% notasemlinha 			
			% notalinhalonga
			chicagofootnotes			% para notas com número e ponto cf. man. de Chicago
			]{edlab-footnotes}

\babelprovide[transforms = hyphen.repeat]{portuguese}

% Medidas (ver: memoir p.11 fig.2.3)
\parindent=3ex			% Tamanho da indentação
\parskip=0pt			% Entre parágrafos
\marginparsep=1em		% Entre mancha e nota lateral 
\marginparwidth=4em		% Tamanho da caixa de texto da nota laterial

% Fontes
\newfontfamily\formular{Formular}
\newfontfamily\formularlight{Formular Light}
\newfontfamily\brabo{FS Brabo Pro Regular}
\newfontfamily\braboitalic{FS Brabo Pro Italic}
\setmainfont[Ligatures=TeX,Numbers=OldStyle]{Minion Pro}

% Estilos

\makeoddhead{baruch}{}{\scshape\MakeLowercase{\scshape\MakeTextLowercase{}}}{}
\makeevenhead{baruch}{}{\scshape{}}{}
\makeevenfoot{baruch}{}{\footnotesize\thepage}{}
\makeoddfoot{baruch}{}{\footnotesize\thepage}{}
\pagestyle{baruch}		
\headstyles{baruch}


% Testes
\begin{document}
%\input{MUSSUMIPSUM}  		% Teste da classe
% Tamanhos
% \tiny
% \scriptsize
% \footnotesize
% \small 
% \normalsize
% \large 
% \Large 
% \LARGE 
% \huge
% \Huge

% Posicionamento
% \centering 
% \raggedright
% \raggedleft
% \vfill 
% \hfill 
% \vspace{Xcm}   % Colocar * caso esteja no começo de uma página. Ex: \vspace*{...}
% \hspace{Xcm}

% Estilo de página
% \thispagestyle{<<nosso>>}
% \thispagestyle{empty}
% \thispagestyle{plain}  (só número, sem cabeço)
% https://www.overleaf.com/learn/latex/Headers_and_footers

% Compilador que permite usar fonte de sistema: xelatex, lualatex
% Compilador que não permite usar fonte de sistema: latex, pdflatex

% Definindo fontes
% \setmainfont{Times New Roman}  % Todo o texto
% \newfontfamily\avenir{Avenir}  % Contexto

\begingroup\thispagestyle{empty}\vspace*{.05\textheight} 

              \formular
              \Huge
              \noindent
              \textbf{A transformação}
              
              % \vspace{-0.35cm}

              % {\Large
              % \noindent{}com desenhos de}
              
              {\brabo\LARGE
              \noindent Franz Kafka}

\endgroup
\vfill
\pagebreak       % [Frontistício]
%\newcommand{\linhalayout}[2]{{\tiny\textbf{#1}\quad#2\par}}
\newcommand{\linha}[2]{\ifdef{#2}{\linhalayout{#1}{#2}}{}}

\begingroup\tiny
\parindent=0cm
\thispagestyle{empty}

\textbf{edição brasileira©}\quad			 {Hedra \the\year}\\
\textbf{tradução do alemão e introdução\,©}\quad{Celso Donizete Cruz}\\\medskip
\textbf{desenhos}\quad			 			 {Franz Kafka}\\
\textbf{textos de apêndice}\quad			 {Walter Benjamin, Otto Maria Carpeaux}\\
%\textbf{copyright}\quad 					 {Nome do autor}\\
%\textbf{ilustração}\quad			 		 {copyrightilustracao}\medskip

\textbf{título original}\quad			 	 {\emph{Die Verwandlung} (1915)}\\
\textbf{primeira edição}\quad			 	 {\emph{A metamorfose} (Hedra, 2009)}\\
%\textbf{edição consultada}\quad			 {}\\\medskip
%\textbf{agradecimentos}\quad			 	 {agradecimentos}\\
%\textbf{indicação}\quad			 		 {indicacao}\medskip

\textbf{edição}\quad			 			 {Paulo Henrique Pompermaier, Jorge Sallum,\\ Iuri Pereira e Bruno Costa}\\
\textbf{coedição}\quad						 {Suzana Salama}\\
%\textbf{editor assistente}\quad             {Suzana Salama}\\
\textbf{assistência editorial}\quad		 	 {Bruno Domingos e Thiago Lins}\\
\textbf{revisão}\quad			 			 {Rogério Duarte e Luis Dolhnikoff}\\
\textbf{capa}\quad			 				 {Lucas Kröeff}\\
%\textbf{iconografia}\quad					 {Suzana Salama}\\
%\textbf{pesquisa de imagem}\quad			 {Suzana Salama}\\
%\textbf{preparação}\quad			 		 {preparacao}\\
%\textbf{imagem da capa}\quad			 	 {imagemcapa}\medskip

\textbf{\textsc{isbn}}\quad			 				 {978-85-7715-963-5}\smallskip

\vfill

%ficha provisória a ser substituida
\begin{minipage}{6cm}
\textbf{Dados Internacionais de Catalogação na Publicação (\textsc{cip})\\
(Câmara Brasileira do Livro: \textsc{sp}, Brasil)}

\textbf{\hrule}\smallskip

Kafka, Franz, 1883--1924\\

\textit{A transformação}. Franz Kafka; tradução e introdução de Celso Donizete Cruz; apêndice de Walter Benjamin e Otto Maria Carpeaux. 3.\,ed. São Paulo, \textsc{sp}: Hedra, 2025. \\

\textsc{isbn} 978-85-7715-963-5\\

1.\,Ficção alemã \textsc{i}.\,Cruz, Celso Donizete.
\textsc{ii}.\,Benjamin, Walter. \textsc{iii}.\,Carpeaux, Otto Maria.
\textsc{iv}.\,Título.\\

24--213782 \hfill \textsc{cdd}: 833

\textbf{\hrule}\smallskip

\textbf{Elaborado por Aline Graziele Benitez (CRB--1/\,3129)}\\

\textbf{Índices para catálogo sistemático:}\\
1.\,Ficção: Literatura alemã (833)
\end{minipage}

\vfill

\textit{Grafia atualizada segundo o Acordo Ortográfico da Língua\\
Portuguesa de 1990, em vigor no Brasil desde 2009.}\\

\textit{Direitos reservados em língua\\ 
portuguesa somente para o Brasil}\\

\textsc{editora hedra ltda.}\\
Av.~São Luís, 187, Piso 3, Loja 8 (Galeria Metrópole)\\
01046--912 São Paulo \textsc{sp} Brasil\\
Telefone/Fax +55 11 3097 8304\\\smallskip
editora@hedra.com.br\\
www.hedra.com.br\\
\bigskip
Foi feito o depósito legal.

\endgroup
\pagebreak     % [Créditos]
% Tamanhos
% \tiny
% \scriptsize
% \footnotesize
% \small 
% \normalsize
% \large 
% \Large 
% \LARGE 
% \huge
% \Huge

% Posicionamento
% \centering 
% \raggedright
% \raggedleft
% \vfill 
% \hfill 
% \vspace{Xcm}   % Colocar * caso esteja no começo de uma página. Ex: \vspace*{...}
% \hspace{Xcm}

% Estilo de página
% \thispagestyle{<<nosso>>}
% \thispagestyle{empty}
% \thispagestyle{plain}  (só número, sem cabeço)
% https://www.overleaf.com/learn/latex/Headers_and_footers

% Compilador que permite usar fonte de sistema: xelatex, lualatex
% Compilador que não permite usar fonte de sistema: latex, pdflatex

% Definindo fontes
% \setmainfont{Times New Roman}  % Todo o texto
% \newfontfamily\avenir{Avenir}  % Contexto

\begingroup\thispagestyle{empty}\vspace*{.05\textheight} 

              \formular
              \Huge
              \noindent
              \textbf{A transformação}
              
              {\brabo\LARGE
              \noindent Franz Kafka}

              {\braboitalic\Large
              \noindent{}com desenhos do autor}
              
              \vfill
              
              \newfontfamily\minion{Minion Pro}
              {\fontsize{30}{40}\selectfont\minion\small
              \noindent{}Celso Donizete Cruz (\textit{tradução e introdução})\\
              
              \vspace{-2.5em}
              \noindent{}Walter Benjamin e Otto Maria Carpeaux (\textit{textos de apêndice})
              }

              \noindent
              {\fontsize{30}{40}\selectfont\minion\small\noindent 3ª edição}

              \vfill

              \newfontfamily\timesnewroman{Times New Roman}
              {\noindent\fontsize{30}{40}\selectfont \timesnewroman hedra}

              \vspace{-0.5cm}
              {\selectfont\minion\small\noindent São Paulo \quad\the\year}

\endgroup
\pagebreak
	       % [folha de rosto]

% nothing			is level -3
% \book				is level -2
% \part				is level -1
% \chapter 			is level 0
% \section 			is level 1
% \subsection 		is level 2
% \subsubsection 	is level 3
% \paragraph 		is level 4
% \subparagraph 	is level 5
\setcounter{secnumdepth}{-2}
\setcounter{tocdepth}{0}

\pagebreak
\begingroup \footnotesize \parindent0pt \parskip 5pt \thispagestyle{empty} \vspace*{-0.5\textheight}\mbox{} \vfill
\baselineskip=.92\baselineskip
\textbf{Franz Kafka} (Praga, 1883--Klosterneuburg, 1924) é um dos autores mais lidos 
e influentes do século \textsc{xx}. Oriundo de uma abastada família judaica de
comerciantes, sua infância é marcada pela relação conflituosa com o pai.
Frequenta na juventude uma escola alemã de Praga, cursa química e
posteriormente direito na Universidade Karl{}-Ferdinand, seguindo depois 
uma bem{}-sucedida carreira como funcionário público na área de segurança do trabalho. 
Além de \textit{A transformação}, Kafka publicou em vida \textit{O~foguista}, \textit{A~sentença} e \textit{O artista da fome}. \textit{O~processo}, \textit{O~castelo} e \textit{América} 
(este último, inacabado) foram publicados postumamente, graças à intervenção de 
seu amigo Max Brod, que se recusou a seguir o testamento de Kafka, no qual 
determinava a destruição de todos os seus escritos inéditos. A sua obra inclui
ainda contos, diários e uma significativa correspondência com sua noiva Felice
Bauer, que ele jamais desposaria. Falece ao 39 anos, vítima de tuberculose.

\textbf{A transformação} (\textit{Die Verwandlung}, 1915) foi publicada
originariamente na revista expressionista \textit{Weiße Blätter}, que abrigava
textos da nova geração de escritores alemães como Heinrich Mann, Ernst Bloch e
Rosa Luxemburgo. A novela narra a singular história de Gregor Samsa, um caixeiro
viajante que certo dia acorda em sua cama transformado em inseto. Plena de
significados simbólicos, a obra deu origem às mais diversas interpretações.
Famosas são as associações de ordem psicanalítica sobre a relação de Kafka 
com seu pai. Deve{}-se a Vladimir Nabokov a interpretação de que a obra não é
redutível a dramas familiares do autor mas, sim, a expressão da tensão do artista
em meio à sociedade burguesa de sua época. Uma das obras literárias mais lidas e comentadas
do século \textsc{xx}, \textit{A transformação} ganha aqui uma nova tradução direta do
alemão que procura recuperar o tom algo jocoso da obra alemã.
        
\textbf{Celso Donizete Cruz}, mestre em língua e literatura alemãs pela Universidade de São Paulo, foi professor da Universidade Federal de Sergipe. Além de obras traduzidas do alemão, inglês e italiano, é de sua autoria \textit{As metamorfoses de Kafka} (Annablume, 2008), um estudo comparativo das mais de doze traduções de \textit{A transformação} publicadas no Brasil.

\textbf{Walter Benjamin} (1892--1940), filósofo, crítico literário e ensaísta alemão, é amplamente conhecido por suas análises da modernidade, história e cultura de massa. Após a ascensão do nazismo na Alemanha, também atuou como crítico e tradutor de obras literárias, além de colaborador em jornais e revistas durante seu exílio em Paris, a partir de 1933.

\textbf{Otto Maria Carpeaux} (1900--1978) foi um crítico literário e historiador austríaco. Destacou-se por sua monumental \textit{História da literatura ocidental}. Fugindo do nazismo, se estabeleceu no Brasil em 1939, onde influenciou profundamente o pensamento literário e cultural.



\endgroup

\pagebreak
{\begingroup\mbox{}\pagestyle{empty}
\pagestyle{empty} 
\addtocontents{toc}{\protect\thispagestyle{empty}}
\tableofcontents*\clearpage\endgroup}

\chapter*{Introdução\smallskip\subtitulo{Franz Kafka, um personagem da\break
mitologia moderna}}
\addcontentsline{toc}{chapter}{Introdução, \emph{por Celso Donizete Cruz}}
%\markboth{Introdução}{}

\begin{flushright}
\textsc{celso donizete cruz}
\end{flushright}

\noindent{}\textls[-15]{\textit{A transformação},}\looseness=-1\footnote{Este texto foi adaptado para a terceira edição, em decorrência da mudança de título (de \textit{A metamorfose} para \textit{A transformação}) que na primeira edição já tinha sido reivindicada pelo tradutor e autor deste texto. \textsc{{[}n.\,e.{]}}} \textls[-15]{novo título dado à reedição da tradução da obra mais conhecida de Franz Kafka, recupera a repetição sonora do substantivo alemão do título original, \textit{Verwandlung}, que ecoa na forma verbal \textit{verwandelt} (em português, “transformado”) no fim da primeira frase da narrativa, considerada por muitos a sentença de abertura mais célebre de toda a literatura.}\looseness=-1\footnote{ “Als Gregor Samsa eines Morgens aus unruhigen Träumen erwachte, fande er sich in seinem Bett zu einem ungeheuren Ungeziefer \textit{verwandelt}”. A fim de manter a mesma correspondência, Modesto Carone, responsável pela primeira tradução brasileira diretamente do alemão, na década de 1980, propõe solução inversa: não mexe no título, mas traduz o verbo por “metamorfoseado”, opção desde então seguida por algumas traduções posteriores.} \textls[-15]{Dizem que o escritor argentino Jorge Luis Borges também criticava o título consagrado nas traduções, argumentando que a língua alemã possui a palavra \textit{Metamorphose}, e Kafka a adotaria se sua intenção fosse de fato privilegiar em sua narrativa a mutação biológica, o que não é o caso. Na recepção em espanhol do século \textsc{xxi}, em consonância com o reparo, propõe-se uma nova tradução do título, \textit{La Transformación}, nas edições da Editorial Funambulista, de Madri, e da Debolsillo, de Barcelona, ambas de 2005. Em língua inglesa, já no final do século \textsc{xx} surgia uma proposta conciliatória, \textit{The transformation (Metamorphosis)}, na edição da Penguin Classics, de 1995.}\looseness=-1

\textls[-10]{Os exemplos não são muitos, afinal, e houve também argumentos contrários
à adoção do novo título, todos no fundo receosos de afrontar gratuitamente a tradição (em português, \textit{tradução} e \textit{tradição} é que dão um trocadilho revelador das condições do campo). A experiência poderia ser desastrosa em mais de um sentido. Poderia levar a perder leitores interessados na obra, porém em busca do título tradicional, o que autoriza quando muito um parêntese, como na edição inglesa. A mudança de título poderia além do mais ser vista como tática meramente novidadeira, sem maiores implicações para a fruição da obra. Ou quem sabe angariasse para a tradução a pecha de enganadora, dando título desconhecido a uma obra mais do que famosa (fosse entendida a proposta só como brincadeira, e já estaria melhor). Mas, mesmo assim, achamos importante alterar o título desta reedição para \textit{A transformação}.}\looseness=-1\footnote{As imagens utilizadas ao longo desta edição são desenhos do próprio Kafka, parte de uma coleção particular encontrada em 2019. \textsc{{[}n.\,e.{]}}}

% Mantenha-se então
% \textit{A metamorfose}, a tradução consagrada do título em língua
% portuguesa. Não há por que polemizar, a questão é mesmo menor. O que
% importa vem depois do título, com ou sem eco, e aí logo se percebe que
% o foco não está na metamorfose, mas nas transformações que ela
% acarreta.

\section{kafka no brasil}

\textls[-5]{No Brasil, \textit{A transformação} vem funcionando como o carro-chefe
da recepção de Kafka, sobretudo de sua recepção popular. De 1956 a
2002, contam-se no país pelo menos 21 edições diferentes da
obra.}\looseness=-1\footnote{Cf.\,Celso Cruz. \textit{Metamorfoses de Kafka}. São
Paulo: Annablume, 2007.} \textls[-5]{É o livro que fisga o leitor e lhe abre as
portas para o universo kafkiano.}\looseness=-1\footnote{ Alguns colegas reivindicam
essa primazia para \textit{O processo}, às vezes até para \textit{O
castelo}, o que pode até acontecer entre o público mais intelectual.
Porém, o número de edições e traduções de \textit{A transformação} é bem
maior, o que confirma sua extensa popularização. \textit{O processo}
exige mais do leitor, e \textit{O castelo} ainda mais, daí a
dificuldade dessas obras de atingir o grande público. Elas tendem a
atrair o interesse desse público no caminho que \textit{A transformação}
pavimenta.} \textls[-5]{A mesma coisa deve se dar em outros países. A história do
homem que se transforma em inseto tem um forte apelo, nunca deixando de
inspirar novos lançamentos, cuja sucessão reafirma o notável sucesso da
obra entre leitores dos mais distintos estratos culturais. Difusão sem
dúvida louvável, quando se pensa no teor crítico do discurso kafkiano e
em seu poder desalienante. Contudo, vendo o que já se fez para sua
divulgação, pode-se supor também um leitor leigo, seduzido pelo
título e por algumas capas, a julgar que se trata de uma história de
terror cujo protagonista é um homem que vira uma barata gigante e
ameaça sua família. Tal leitor não estará absolutamente errado, só que
se acompanhar a narrativa há de topar com um terror estranho e
inesperado, por vezes mais engraçado que aflitivo (pode achar que
levou gato por lebre: queria \textit{A metamorfose}, e recebeu
\textit{A transformação}). A hipótese não é de todo descabida, ainda
que seja difícil um leitor se aproximar da obra assim tão
desavisadamente. O adjetivo derivado do nome de seu autor é presença
certa nos dicionários, além do que Kafka e os títulos de suas obras
mais famosas já são verbetes obrigatórios das enciclopédias. Se o
leitor vai ao livro, é porque em geral soube de antemão alguma coisa.}\looseness=-1

\textls[10]{Soube no mínimo do grande prestígio do escritor, um dos nossos maiores
ícones literários. Embora tenha escrito no começo do século \textsc{xx}, e
alcançado a glória póstuma após a metade desse mesmo século,
rapidamente ganhou posição ao lado dos clássicos imortais da literatura
de todos os tempos. O adjetivo “kafkiano” ultrapassou os círculos do
pensamento literário, vindo a servir para designar determinadas
situações de nossa vida prática. Tão entranhado assim ficou em nossa
cultura, que figura ao lado de outros adjetivos literários, como dantesco, quixotesco,
homérico. A uma tal altura no Olimpo das letras, não será difícil ao
leitor divisá-lo ao adentrar o pátio principal da literatura do
Ocidente. Mas o que justifica tamanho destaque? De onde virá a força
que lhe assegura de saída um lugar no panteão dos gênios indisputáveis?
Na resposta a essas questões, há a considerar o que Kafka fez, e o que
dele foi feito.}

\textls[10]{\textit{A transformação} é um bom exemplo para tanto. Em parte por ser sua
obra mais popular e ter sido publicada com o autor ainda vivo. Não
foi muita coisa que ele deixou vir a público enquanto vivia.}\looseness=-1\footnote{\textls[-10]{Há opinião diversa, como a de Osman Durrani, no \textit{\mbox{The Cambridge}
Companion to Kafka}, que procura desfazer o mito do autor tímido, avesso à
publicação de sua obra. Em relação ao mito, até que Kafka publicou
bastante. In: \textit{The Cambridge
Companion to Kafka}, org.\,Julian Preece, Cambridge University Press,
2002.}} \textls[10]{Três pequenos livros de narrativas curtas:
\textit{Betrachtung} (\textit{Contemplação}), de 1913; \textit{Ein
Landartz} (\textit{Um médico rural}), de 1919; e \textit{Ein
Hungerkünstler} (\textit{Um artista da fome}), publicado no ano de sua
morte, 1924. Três narrativas médias: \textit{Der Heizer} (\textit{O
foguista}), de 1913; \textit{Das Urteil} (\textit{O julgamento} ou
\textit{O veredito}), de 1916; e \textit{In der Strafkolonie}
(\textit{Na colônia penal}), de 1919. Além de \textit{Die Verwandlung}
(\textit{A transformação}), uma narrativa
longa, que saiu inicialmente na revista \textit{Weiße Blätter}
(\textit{Folhas brancas}) em 1915, depois em livro, em 1916, e alcançou
uma segunda edição em 1918. Em conjunto, os escritos publicados em vida
não ultrapassam quinhentas páginas.}\looseness=-1\footnote{ \textls[10]{São 447, numa contagem
mais recente, incluindo textos não literários. Cf.\,Osman Durrani,
“Editions, translations, adaptations”. In: \textit{The Cambridge
Companion to Kafka, \textit{op.\,cit.}}, p.\,208.}} \textls[10]{A economia narrativa e o rigor no acabamento
apresentam-se desde já como parte do projeto literário de Kafka. De
fato, apenas essas obras talvez fossem suficientes para garantir sua
posição entre os grandes mestres. As principais características de sua
ficção estão praticamente todas presentes. Só não se saberia então que
o material publicado era apenas parte do edifício.}\looseness=-1

\section{max brod e o espólio}

%<http://www.sololiteratura.com/sor/sorrenelkafkiano.htm>, acesso em 30/05/2008
 \textls[10]{Ficou mais do que notável uma das últimas vontades de Kafka, a de que,
após a sua morte, seu espólio literário fosse destruído. O amigo Max
Brod, incumbido pelo autor da realização dessa vontade, evidentemente
não cumpriu a promessa. Esse episódio biográfico já deu o que falar e é
um dos que contribuem para a construção de uma visão romantizada da vida de
Kafka. Pode-se imaginar o escritor em seu leito de morte, vencido
pela tuberculose, entre acessos de tosse e escarros de sangue,
encarecendo o amigo com a tarefa inglória; na cena seguinte o amigo, ao
abrir o baú, surpreso e maravilhado com a quantidade e a qualidade do
tesouro que encontra; no final feliz, o tesouro partilhado com os
próximos e os pósteros\ldots{} Deve ter sido mais ou menos isso o que
aconteceu, descontada a dose de má ficção. O já citado Jorge Luis
Borges é um dos que referem o episódio,}\looseness=-1\footnote{\textls[-10]{Num conhecido prólogo
publicado no Brasil na abertura de uma edição da Ediouro de \textit{A
metamorfose}, tradução de Torrieri Guimarães, de 1998, coleção
Biblioteca de Babel, dedicada à literatura fantástica, homônima porém
não a mesma dirigida por Borges e Bioy Casares na Argentina. A
propósito de Borges, ainda, também se acredita que tenha traduzido
\textit{A transformação}, fato entretanto desmentido por Fernando
Sorrentino, em ``\textit{La Metamorfosis} que Borges jamás tradujo”, \textit{La
Nación}, Buenos Aires, 9 de março de 1997 (disponível \textit{online} com o título “El kafkiano caso de la \textit{Verwandlung} que Borges
jamás tradujo”).}} evocando para efeito de comparação o caso de Virgílio
(outro escritor cujo similar último desejo também não se realizou) e a
seguir argumentando que se essa fosse realmente a vontade desses
autores, eles mesmos se encarregariam de riscar o fósforo. Ironias à
parte, o pedido não atendido de Kafka pode ser a tradução sincera de
sua dúvida quanto ao valor de suas páginas inacabadas. O rigor de seus
critérios de acabamento aumenta na medida da desproporção entre o muito
que escreveu e o pouco que publicou. \textit{A transformação}, todavia,
não deixa dúvidas, pois passou pelo crivo do autor, não sofreu as
interferências da organização e edição póstumas de Max Brod, não
padecendo assim da desconfiança, algo desmedida, diga-se, quanto à
autenticidade de alguns trechos de sua produção literária divulgada
\textit{post mortem}. Por isso vem a ser mesmo o livro ideal para um
contato inicial preciso com a mais pura ficção kafkiana.\footnote{
Entretanto, não se quer dizer que o que veio após sua morte deva ser
descartado, longe disso. Inclusive, acontece uma coisa interessante, a
partir da recepção das obras do espólio. O inacabado e o fragmentário
próprio desses papéis cuja redação não foi retomada, ou que não foram
revistos para publicação, são incorporados como matrizes da expressão
literária de Kafka, e revertem sobre suas produções anteriores.
Cumpre-se de certa forma a perspicaz observação, de novo de Borges,
agora em “Kafka e seus precursores”, de que os autores que influenciam
Kafka só vêm a surgir depois de sua morte.}

Proponho a distinção entre as narrativas póstumas e as publicadas em
vida apenas como tentativa de destacar o cuidado do autor com seus
escritos, sua consciência literária, seu senso crítico apurado, seu
compromisso vital com a literatura. Trata-se de um homem de letras,
que frequentou espaços sociais comuns a intelectuais e artistas, que
tinha uma visão particular da literatura, estava informado das
novidades de seu tempo, e certamente manifestaria suas opiniões em
encontros com os amigos nos cafés de Praga. Não corresponderia
unicamente à imagem do escritor desconhecido, enclausurado, sombrio,
gênio incompreendido e maldito --- visões românticas tantas vezes
propagadas nas biografias. Max Brod, conhecendo o amigo, por certo
estaria consciente de seu alto valor literário, e de antemão calcularia
a importância do que o aguardava no baú. Kafka não foi afinal o
escritor anônimo descoberto da noite para o dia, infelizmente quando
era tarde demais e já não podia desfrutar da fama. Não viu o sucesso de
nenhuma das obras que publicou, é certo, porém é igualmente correto que
alcançou de imediato com elas o reconhecimento de seus pares em Praga,
despertando reações positivas também em alguns círculos literários da
Alemanha.\footnote{\textls[15]{ Luiz Costa Lima comprova em \textit{Limites da voz:
Kafka} (Rocco, 1993) que o escritor “não foi um
ignorado”, e que sua “recepção inteligente” soube lhe destacar o valor,
além de ser em alguns casos muito feliz na caracterização de suas
peculiaridades.}} O seu talento de primeira grandeza não era popular, mas
foi notado. Consta que arrebatou em 1915 a terceira edição do Prêmio
Theodor Fontane de Arte e Literatura, instituído na Alemanha, embora
tenha sido uma vitória indireta: o vencedor oficial, o dramaturgo
alemão Carl Sternheim, repassou depois a premiação a Kafka. Episódio
emblemático de uma recepção restrita --- que tem o autor como escritor dos
escritores, conhecido apenas em pequenos círculos literários, condição
que após a sua morte sua obra superaria totalmente, chegando ao coração
das massas, o que é até espantoso, em face do desconforto inevitável
provocado por sua leitura.

\section{praga, a sombra de um cenário}

Franz Kafka nasceu em 1883 em Praga, capital da então Boêmia, hoje
República Tcheca. À época, a Boêmia fazia parte do Império
Austro-Húngaro, e seu idioma administrativo oficial era o alemão. A
submissão compulsória ao império obviamente não retirava aos tchecos o
sentimento de pertença à cultura de sua região, e logo os movimentos
nacionalistas desta e de outras regiões submetidas iriam dissolver o
império. Imagine-se a aversão pelo imperialismo, e a desconfiança
para com todos que parecessem mais fiéis ao império do que à Boêmia.
Este em parte devia ser o caso de Kafka que, apesar de seu local de
nascimento, não possuía identificação muito evidente com a cultura
tcheca. Era filho de pais judeus emigrados da Áustria, praticamente sem
laços afetivos ou nacionalistas com a Boêmia. Sua família fazia parte
da comunidade judaica de Praga e ao mesmo tempo flertava com os
oficiais alemães, tanto é que colocaram Kafka para estudar numa escola
alemã. Seu caso, evidentemente, não seria único, contudo não será
também de admirar a crise identitária e o sentimento de perseguição
decorrentes da situação. Kafka era tcheco, mas escreveu em alemão e
acabou órfão das duas culturas. Um dos maiores nomes da literatura
alemã de todos os tempos não era alemão. E Praga em sua obra é nada
mais que a sombra de um cenário ocasional. Judeu, mas desgarrado e
descrente, tampouco pode-se dizer que encontrasse sua identidade em
meio à comunidade judaica. Exilado das três pátrias, seria hoje cidadão
do mundo\ldots{} Mas a Europa era outra, e Kafka a viu antes, durante e logo
depois da Primeira Guerra. Foi testemunha desse acontecimento
traumático, que literalmente expôs as entranhas de uma sociedade
pretensamente racional. A condição marginal lhe possibilitaria observar
essa sociedade sem comprometimentos patrióticos. O que tinha para
dizer, e deixou por escrito, não se dirigia especificamente à cultura
tcheca, alemã ou judaica. A nenhuma das três em particular, mas a todas
a um só tempo --- ao humano em cada uma delas.

\textls[-5]{Sua posição à parte no conturbado cenário europeu de então deu-lhe uma
compreensão inusitada dos problemas do homem de seu tempo, o homem
contemporâneo, este que veio a ser o que ainda hoje somos. Sua obra
resulta dessa compreensão, um dos motivos elementares da importância a
ela atribuída. Kafka parece ter dito uma vez que concebia a literatura
como uma “expedição à verdade”.}\looseness=-1\footnote{ “Dichtung ist immer nur eine
Expedition nach der Wahrheit”, frase atribuída a Kafka por Gustav
Janouch em seu livro \textit{Conversas com Kafka}.} \textls[-5]{Essa concepção
acentua outro tanto o interesse pelo que deixou. Seus textos literários
são nesse sentido uma contribuição à filosofia (em sentido lato), que
de direito se ocupa dos problemas da verdade. A literatura kafkiana
demonstra que a reserva filosófica não impede a progressão do método
literário na exploração de um mesmo território. A diferença é que, onde
a filosofia explica, a literatura mostra. Não se recorre às premissas
que permitirão a dedução de uma situação absurda na qual o ser humano,
“barateado”, reduz-se à condição de inseto. Não se aciona o
pensamento lógico \textit{stricto sensu}. O absurdo é maior e mais
impactante com a eclosão inexplicável do inseto humano no seio de uma
típica família pequeno-burguesa. O fenômeno é incomum, e visível
apenas pelas lentes literárias. Mas não desperta nenhuma dúvida nas
personagens, que em nenhum momento questionam a impossibilidade do
fato. Note-se que não se trata de metáfora, o inseto está lá em toda
sua concretude, para quem quiser ver. Age como inseto: tem dificuldades
para se mover, não possui dentes, rasteja pelas paredes e pelo teto,
se alimenta de restos e, apesar de ainda raciocinar como humano e de
entender a língua dos humanos, estes não só não entendem o que ele
fala, como o julgam (com exceção talvez da faxineira) incapaz de
compreendê-los. Chama-se aqui a atenção para o irreal que afinal
aparece como a condição para que se enxergue a realidade, e aí temos um
método de desalienação. Didática de Kafka: os contrassensos não são
discutidos, são vistos, e é o impacto do que se vê que perturba o
entendimento sossegado do leitor.}\looseness=-1

\section{investigação através da literatura}

Vale falar de um propósito na dedicação extrema de Kafka à literatura. A
julgar pelo que relatou em diários e cartas, sua vida só adquiria
sentido em função da literatura. Acredito que seja possível confiar na
sinceridade desses escritos pessoais, embora a relação da biografia do
autor (em boa parte inspirada por esses mesmos escritos) com as obras
que deixou dê margem a interpretações muitas vezes equivocadas ou
ingênuas. Não é que tenha retratado episódios de sua vida pessoal.
Estes, no máximo, iriam lhe servir de inspiração. A literatura, como a
concebia, seria mais uma forma de flagrar as contradições da cultura
ocidental no princípio do século \textsc{xx}, de um modo eficaz, no entanto nada
confortável nem óbvio. Seria uma tentativa de entender o que acontece
com os humanos numa sociedade cada vez menos humanizada, se é que algum
dia houvesse sido mais\ldots{} Escrever lhe era vital, provavelmente porque
o punha em contato com a \textit{verdadeira vida}. A procura da
verdade, se por um lado enfeixa suas produções na confluência da
literatura com a filosofia --- e não por acaso os maiores filósofos do
século se dispuseram a interpretá-lo ---, por outro lado leva a
classificá-las como realistas. De um realismo que não se reduz à
descrição pitoresca da superfície do real, antes corresponde à
percepção objetiva da realidade. Com efeito, seu realismo é de tipo
expressionista, à medida que dá vazão a uma realidade desfigurada pela
percepção interna do sujeito. Entretanto, o propósito de objetivar essa
realidade impede a expressão puramente subjetiva. Como explica 
Luis Costa Lima,\footnote{\textit{Op.\,cit.}, pp.\,65--66} a ficção
de Kafka pressupõe uma mediação, “um meio interposto entre a
subjetividade e o mundo externo, que permita a objetivação daquela”:

\begin{quote}
Sua questão é converter as tematizações pessoais de próprias ao espaço
interno em capazes de se mover no externo; \textit{i.\,e.}, transformá-las de
fantasmas em objetos, cujos traços mostrariam a si e a seu tempo.
\end{quote} 

\textls[-20]{Se bem entendo a lição, diviso uma metodologia nessa busca de conversão do
interior em exterior, de “fantasmas em objetos”, da subjetividade em
objetividade, enfim. Só faz sentido falar em método quando se quer
atingir um objetivo, no caso \textit{mostrar} “a si e a seu tempo”, o
que vem a ser a confirmação de uma intenção realista.}

\textls[-10]{Aqui se cai de chofre na \textit{selva selvaggia} da fortuna crítica.
Missão impossível não recorrer ao paradoxo na descrição da
singularidade do autor. O subjetivo objetivo, a ação que é inação, o
estranho familiar\ldots{} Nomeia-se pela contradição uma obra que se
realiza no limite, sempre na dúvida entre o que \textit{é} e o que
\textit{não é}. Tal indecisão retira as bases de qualquer juízo crítico
absoluto. E o mistério sempre se mantém um mistério, mesmo depois de
aberto com as diferentes chaves forjadas pela crítica. Ora, não será
demasiado supor que era esse precisamente o ponto visado pela
literatura de Kafka, a apresentação de situações numa perspectiva
ambígua, trágica e cômica ao mesmo tempo, próxima e distante, real e
fantástica (termos e contratermos se sucedem\ldots{}). Toda representação
kafkiana sustenta-se na evocação de sua face contrária. O caso de Gregor
Samsa, personagem principal de \textit{A transformação}, é mais uma vez
exemplar. Ele só toma consciência de sua alienação ao ser alienado de
sua forma humana. Não é o fato de se transformar em inseto o que o
aliena, isso só lhe revela sua real alienação. Já era inseto quando
ainda era humano, se ainda é humano quando já é inseto? 
Se para descobrir sua humanidade é
preciso que a perca, a \textit{transformação} é a condição de sua consciência. A
exposição do humano é levada ao extremo com a oposição do inseto. No
choque dos opostos é possível viver uma verdade.}

Não devia mesmo ser fácil ao autor sustentar tal ponto de vista. Kafka
sempre se queixou da falta de espaço e tempo para se dedicar à
literatura como gostaria e, de acordo com seus critérios, deveria. A
biografia em quadrinhos de Robert Crumb e David Zane
Mairowitz\footnote{ \textit{Kafka de Crumb}, trad.\,José Gradel. Rio de
Janeiro: Relume-Dumará, 2006.} retrata enfaticamente a ausência de
privacidade na casa dos pais, onde residiu durante quase toda a sua
vida, e também o grau de concentração exigido em sua prática literária.
No traço de Crumb, o escritor entra em transe ao escrever, os olhos
esbugalhados, como se transportado para um outro plano existencial. O
transe é ainda ambíguo, pois significa a necessidade tanto de superar
um entorno desfavorável à prática (situação do sujeito) quanto de
aceder ao plano de perseguição da verdade (condição do objeto). Não
admira que o esforço exaurisse o autor, solicitando-lhe uma
disposição que somente teria se pudesse abandonar o trabalho e demais
compromissos sociais. Kafka se dizia um fraco. Para poder escrever, se
viu obrigado a abdicar de possíveis casamentos e a buscar a solidão.
Ainda assim, o mínimo exigido de vida social já lhe parecia muito e
roubava-lhe as forças de que necessitava para completar suas obras.
Em que pese a fantasia \textit{underground}, a representação proposta
por Crumb sintetiza os apuros do escritor, que se sentia hábil e capaz
apenas para o trabalho literário que sua vida lhe dificultava exercer.
O transe místico é ainda o simulacro de sua obsessão com a literatura,
à qual sacrificava a vida.

\section{kafka, mítico e místico}

Nesse ponto tocamos a esfera do mito. Não é fácil acreditar que um autor
se dispusesse a tanto, nem que a literatura exija pacto tão radical.
Mas fato é que o próprio Kafka cultivou a ideia do escritor abnegado. A
divulgação de suas obras póstumas também pôs em circulação seus
escritos pessoais, e é nesses que se acham declarações do autor sobre
seu envolvimento com a criação literária. Essas declarações alimentam o
mito. De acordo com elas, o escritor dedica-se à literatura como a um
sacerdócio. A literatura seria sua religião mais cara, se de fato lhe
revelasse a verdade. Daí a enxergá-lo como profeta é um passo. Isso
sem contar a simpatia despertada por sua situação pessoal precária.
Note-se, porém, que Kafka fala de uma \textit{expedição} à verdade,
não de uma \textit{revelação}. \textls[-10]{Uma expedição é uma viagem, uma
aventura, e só se vive uma verdadeira aventura quando não se pode
prever o final. Para abandonar o mito, é preciso compreender o
compromisso com a verdade do ponto de vista ético, não religioso. Os
resultados das expedições nunca são conclusivos. Mas o rigoroso relato
do percurso é a prova da dedicação e da fidelidade ao compromisso
assumido. A literatura é, assim, seu instrumento de busca da verdade,
ao encontro da qual não é necessário ir com a alma pura dos crentes
inocentes. Também não é preciso deixar ao cinismo o papel principal.
Parece haver em Kafka, como nos grandes autores, um compromisso com a
sinceridade. A tortura ou o transe da criação podem, sim, se associar antes ao
rigor do que à mística. Essa reivindicação, contudo, no fundo também
obedece a imperativos associados a uma visão específica da arte,
entendida aqui mais como construção e cálculo do que como magia e
inspiração. Reclama-se um escritor consciente de sua proposta
literária, antes que um “médium” da expressão de forças superiores.}

\textls[-5]{Entretanto, a preferência parece tender ao místico. 
Com a popularização de sua recepção, Franz Kafka
passa a habitar uma outra dimensão, e se transforma em um personagem da
mitologia moderna, cujos círculos inevitavelmente se misturam à
mitologia dos tempos imemoriais. E faz pouco mais de oitenta anos que
veio a falecer. Morria quando nossos pais ou avós nasciam, há duas
gerações apenas. Como esse intervalo é relativamente pequeno, ainda é
possível testar, com base em documentação histórica, a verdade de
alguns relatos biográficos.}\looseness=-1\footnote{ Cf., por exemplo, Anthony
Northey, “Myths and realities in Kafka biography”. In: \textit{The
Cambridge Companion to Kafka}, \textit{op.\,cit.}} \textls[-5]{Mas a própria disputa pela verdade biográfica tende a confirmar
o mito. Algumas correções não perturbam a imagem geral, pelo contrário.
A voracidade do mito traga qualquer migalha de veracidade histórica.}\looseness=-1

Creio que o que acontece com a recepção de Kafka no Brasil repete em
escala doméstica, ressalvado o atraso, o movimento internacional de sua
popularização. De início, sua leitura é privilégio de pequenos círculos,
mas logo suas produções vêm a ser difundidas para todos os estratos
sociais. \textit{A transformação} é sua obra mais divulgada, porta de
entrada de sua recepção, como observado. Sua primeira tradução
brasileira, de Brenno Silveira, data de 1956, e foi feita a partir do inglês. É só
a partir dos anos 1960 que o autor passa a ser popularizado, já
então como clássico. \textls[5]{Em muito concorre para sua popularização a
história do homem que se transforma em um inseto. Na época de sua
primeira publicação em livro, em 1916, Kafka instou para que o inseto
de modo algum fosse sugerido na capa. Logo na primeira edição
brasileira, contudo, já aparecem justapostos, de costas um para o
outro, os perfis de um homem e de uma barata, esta mais detalhada que
aquele. A tônica no inseto descortina uma estratégia de difusão que
potencializa em demasia certos apelos popularescos da história
original, e tende a preservar o mito. \textit{A transformação} é
normalmente divulgada como clássico da literatura moderna ou universal.
Só por isso já deveria ser lida. Mas a presença do homem-inseto é um
incentivo a mais, considerada a curiosidade que o fantástico e o
sobrenatural em geral despertam no público. Em nome desse apelo é que
se coloca a \textit{transformação} do homem em primeiro plano, quando na verdade a
narrativa trata é das transformações de seu entorno em face de uma
situação totalmente inesperada. Todas as nuanças possíveis decorrentes
desse evento inicial são exploradas. \textit{A transformação} desperta reações em
cadeia, e são essas reações que a narrativa de Kafka acompanha. Daí, na
verdade pouco importa o inseto, basta frisar sua inadequação e a
repulsa que ele provoca. Se em lugar do inseto houver uma massa amorfa
e gosmenta, nada muda, a não ser a forma de suas pegadas.}

\section{humor em «a transformação»}

Por isso a sugestão de deslocar o foco. Trocar a \textit{metamorfose} pela
\textit{transformação}. As reações ao asco são mais interessantes que o objeto
asqueroso. Aí é que está o humor, um humor não autorizado pelo horror
da situação descrita e que entretanto comparece como possível matriz do
modo de narração. O distanciamento narrativo é máximo, mesmo que
tenhamos acesso direto aos pensamentos das personagens. O narrador é
onisciente e não se compromete. Mantém a objetividade ainda que o
evento narrado seja o maior dos absurdos. Faz questão de chamar a
atenção para detalhes periféricos das situações principais, e tais
detalhes acabam sendo reveladores das reais motivações das personagens.
Ora, o absurdo, o inesperado, o grotesco, o ínfimo que se revela
fundamental, essas ocorrências são comuns ao reino do cômico, isso sem
falar que o distanciamento é a condição da comédia, pois são poucos os
que acham graça quando são os objetos de derrisão. Não admira pois que,
conforme reza a lenda, Kafka tenha chegado às gargalhadas ao ler a
narrativa em primeira mão para os amigos. Acredito que falte uma pitada
maior desse humor nas edições e traduções brasileiras, o qual todavia
está presente nas ilustrações do pintor Walter Levy para a primeira
edição brasileira de 1956. Esse veio interpretativo ficou meio esquecido nas
várias edições posteriores. A tônica foi mais para o horror ou para o
trágico, que fazem justiça à obra, mas não a esgotam.

Constata-se, por mais incrível que pareça, apesar de toda a avalanche
interpretativa a que o autor esteve e continua sujeito, a existência de
espaços ainda a explorar, afora a necessidade de revisão de algumas
ideias prontas herdadas de recepções passadas. O maior desafio da
crítica kafkiana talvez seja escapar aos mitos e às múltiplas
interpretações preestabelecidas de sua obra. De qualquer modo, todo
clássico acaba se impondo por si só quando nos dispomos à sua leitura.
Uma nova tradução é só mais uma proposta de interpretação, sempre
possível porque o contexto de recepção nunca é estável. O clássico
atravessa as gerações, tendo sempre o que dizer a cada uma delas. Há
portanto sempre uma oportunidade de renovação a comprovar o seu vigor
atemporal.

% \section{nota editorial}

% Em 2019, centenas de desenhos de Franz Kafka foram encontrados em uma coleção particular. Seus desenhos são figuras fluidas e flutuantes, feitas principalmente com tinta nanquim sobre papel. As figuras transitam do realista ao fantástico, do grotesco ao estranho e ao carnavalesco, iluminando um aspecto até então desconhecido do autor. Kafka desenhou durante muito tempo, mas a maior parte de sua produção artística aconteceu entre 1901 e 1907. Em 2022, esses desenhos foram publicados pela primeira vez pela Yale University Press. Utilizamos algumas dessas imagens ao longo desta edição.


\chapter*{}
\section{I}

\noindent{}Certa manhã, ao despertar de um sonho inquieto, Gregor Samsa descobriu-se
em sua cama transformado num insuportável inseto. Deitado de costas, duras
como um casco, ele viu, ao erguer um pouco a cabeça, sua barriga arredondada,
pardacenta, repartida por pregas arqueadas, do alto da qual a coberta, já
quase toda caída, escorregava. Diante de seus olhos moviam-se
desesperadas suas várias pernas, ridiculamente finas em comparação com
suas proporções de antes.

“O que aconteceu comigo?”, pensou. Não era um sonho. Seu quarto, abrigo
humano e normal em tudo, só um tanto quanto pequeno, jazia em silêncio
entre as quatro paredes velhas conhecidas. Acima da mesa, onde se
espalhavam pacotes desembrulhados de amostras de tecidos --- Samsa era
caixeiro-viajante ---, pendia a ilustração que ele recortara há pouco tempo
de uma revista e havia encaixilhado numa moldura linda, dourada. Era o
retrato de uma dona elegante, sentada, aprumada, ornamentada com um
barrete e uma estola de peles, que elevava na direção do observador um
pesado regalo também de pele, no interior do qual quase todo o seu
antebraço desaparecia.

O olhar de Gregor voltou-se então para a janela, e o tempo fechado ---
ouviam-se gotas de chuva batendo no peitoril de metal --- deixou-o bastante
melancólico. “Como seria bom dormir um pouco mais e esquecer
essas maluquices”, pensou, mas isso era inexequível, pois estava
acostumado a dormir do lado direito, e no seu estado atual não conseguia
ficar nessa posição. Por mais força que fizesse ao se projetar para a
direita, acabava sempre mandado de volta à posição inicial, de costas. Já
havia tentado umas cem vezes, fechava os olhos para não ter que ver a
movimentação das pernas, e só parou quando começou a sentir na
lateral do corpo uma
ligeira dor, surda, nunca antes sentida.

“Deus do céu”, pensou, “que profissão mais desgastante eu fui escolher! É
viajar todo santo dia. A tensão desse comércio é de fato muito maior do
que o trabalho na loja, e além disso a mim me toca ainda esse tormento das
viagens, a preocupação com as conexões dos trens, a comida péssima, sem
hora certa, o contato humano sempre alternado, nunca permanente, nunca
chegando a ser afetuoso. Que isso tudo vá pro inferno!” Sentiu uma
coceirinha na parte de cima, na barriga; empurrou as costas devagar para
junto da armação da cama, a fim de poder erguer melhor a cabeça; divisou a
região que coçava, coberta de minúsculos pontinhos brancos pronunciados,
não chegou a atinar o que fossem; e quis cutucar o local com uma perna,
mas retirou-a na mesma hora, pois ao contato foi acometido por um
calafrio.

Deslizou de volta à sua posição anterior. “Esse negócio de acordar tão
cedo”, pensou, “deixa a pessoa apalermada. Um homem deve ter direito a
suas horas de sono. Os outros vendedores levam vida de princesa. Quando,
por exemplo, eu volto para a hospedaria no meio da manhã, e vou passar a
limpo os pedidos, só então esses cidadãos se sentam para tomar o café. Vou
eu tentar a mesma coisa com o chefe que eu tenho; iria parar no olho da
rua. Aliás, vai saber se isso não seria mesmo o melhor para mim. Se eu,
por causa dos meus pais, não estivesse de mãos atadas, já tinha pedido as
contas há muito tempo, teria parado bem na frente do chefe e dito o que
penso com absoluta franqueza. Era capaz dele cair da mesa! Essa é outra
mania esquisita do chefe, sentar na mesa e falar com os funcionários
olhando de cima, sem contar que, por causa de sua audição sofrível, a
gente precisa chegar bem pertinho. Mas não perdi de todo as esperanças;
assim que juntar o dinheiro e saldar a dívida que os meus pais têm com ele
--- o que deve durar ainda uns cinco ou seis anos ---, adoto a medida sem
falta. Então as amarras serão todas rompidas. Por enquanto, todavia, eu
tenho de me levantar, porque o meu trem parte às cinco.”

E olhou de lado na direção do despertador, que fazia tique-taque em cima
do guarda-roupa. “Minha Nossa Senhora!”, pensou. Eram seis e meia, e os
ponteiros seguiam mansos adiante, era até mais tarde, já estava perto de
quinze para as sete.
Será que o despertador não tinha tocado? Via-se da cama
que ele havia sido ajustado direitinho para as quatro; na certa, pois,
tocara. É, mas seria possível manter um sono tranquilo com esse barulho
que chegava a estremecer os móveis? Bem, tranquilo é que não fora o seu
sono, entretanto, talvez por conta disso mesmo, teria sido mais profundo.
Mas o que devia fazer agora? O próximo trem partia às sete; para
alcançá-lo, teria de se apressar feito um louco, e as amostras ainda não
estavam na mala, e ele próprio não se sentia inteiramente descansado e
disposto. E depois, se o alcançasse, não havia mais como evitar um
acesso furioso do chefe, pois o menino da loja teria aguardado o trem das
cinco e há muito já transmitira o informe de sua falta. Esse era uma cria
do chefe, sem fibra e sem discernimento. Não poderia avisar que estava
doente? Isso, porém, seria demasiado constrangedor e suspeito, pois
Gregor, durante os seus cinco anos de serviço, não ficara doente nem uma
vez sequer. Na certa o chefe viria com o médico da previdência,
repreenderia os pais por causa do filho preguiçoso e rejeitaria qualquer
objeção com base no palpite do médico, que era da opinião de que a boa
saúde nunca faltava aos homens, faltava era a disposição para o trabalho.
E, a propósito, nesse caso estaria ele tão errado assim? Na verdade,
descontada uma certa sonolência realmente superficial, devida ao longo
período de sono, Gregor se sentia muito bem e até estava com uma fome um
pouco além do comum.

Enquanto refletia aos atropelos sobre tudo isso, sem encontrar coragem
para deixar a cama --- o despertador marcava exatamente quinze para as sete
---, foram ouvidas leves batidas na porta ao lado da cabeceira. “Gregor”,
chamavam --- era a mãe ---, “já são quinze pras sete. Você não ia
viajar?” Que
voz mais doce! Ao responder, Gregor se assustou com a sua própria, que era
nitidamente a mesma voz, porém agora, como se vindo do fundo, mesclava-se
a ela um guincho aflitivo, impossível de reprimir, que só num primeiro
momento deixava as palavras soarem inteligíveis, para corrompê-las em
seguida com seu eco por trás de cada emissão, de uma tal maneira que a
pessoa não sabia se tinha escutado direito. Gregor teve vontade de
responder em detalhes e explicar tudo, entretanto, dadas as
circunstâncias, limitou-se a dizer: “É, ia; obrigado, mãe; já vou
levantar”. Na certa por causa da madeira da porta, a alteração na voz de
Gregor não foi notada do lado de fora, pois a mãe se satisfez com a
explicação e se afastou arrastando as chinelas. Contudo, a pequena troca
de palavras alertou os outros membros da família para o fato de que
Gregor, contrariando o previsto, ainda estava em casa, e logo o pai batia
em uma das portas laterais, sem força, mas com o punho. “Gregor, Gregor”,
chamou, “o que houve?” E depois de um breve intervalo voltou a exortar,
engrossando a voz: “Gregor! Gregor!”. Na outra porta lateral, por sua vez,
a irmã chamou baixinho, num tom queixoso: “Gregor? Está se sentindo bem?
Precisa de alguma coisa?”. Gregor respondeu para ambos os lados: “Já estou
indo”, e esforçou-se para que não reparassem em sua voz, pronunciando as
palavras com o máximo de cuidado e intercalando longas pausas de uma
emissão a outra. Também o pai deu meia volta e retornou ao seu café da
manhã, a irmã entretanto sussurrou: “Gregor, abra, eu insisto”. Mas Gregor
não pensava de forma alguma em abrir, pelo contrário, louvava a precaução
adquirida com as viagens, e até em casa mantinha todas as portas trancadas
durante a noite.

Agora o que queria era levantar-se com calma e sem ser incomodado,
vestir-se e, acima de tudo, tomar o café da manhã, para só então refletir
sobre o restante, pois ele via muito bem que ali deitado não conseguiria
conduzir seus pensamentos a nenhuma conclusão satisfatória. Lembrou-se de
já ter sentido várias vezes na cama uma certa dor leve, provável resultado
do sono em posição desajeitada, que logo em seguida, ao se levantar,
revelava-se pura imaginação, e estava curioso para ver como aos poucos se
dissipariam suas impressões do dia. Que a alteração na voz nada mais fosse
do que o sintoma de um belo resfriado, doença típica da sua profissão,
disso não tinha a menor dúvida.

Afastar a coberta foi muito fácil; bastou a ele inflar-se um pouco que ela
caiu por si só. Mas a partir daí ficou difícil, principalmente por causa
de sua largura tão excepcional. Ele teria necessitado de braços e mãos
para se pôr de pé; em vez disso, porém, tinha apenas as várias perninhas
que se movimentavam sem trégua para todos os lados, e que além do mais ele
não conseguia dominar. Queria fazer uma delas dobrar, e essa era a
primeira a ficar esticada; conseguia afinal realizar o que queria com
essa, e já todas as outras, nesse meio tempo, trabalhavam por conta
própria, numa agitação maior e mais aflitiva ainda. “Não vai ficar
estendido aí como um inútil”, Gregor disse a si mesmo.

Primeiro quis deixar a cama com a parte de baixo de seu corpo, mas essa
parte, que aliás ele ainda não vira e da qual também não conseguia fazer
uma ideia muito precisa, mostrou-se muito difícil de mover; avançava tão
devagar; e quando ele afinal, numa fúria quase animalesca, projetou-se
para frente com todas as forças, sem prestar atenção, calculou a direção, bateu contra a armação dos pés da cama, e a dor
lancinante que sentiu o ensinou que justo a parte de baixo de seu corpo
talvez fosse no momento a mais sensível.

Em vista disso procurou sair primeiro com a parte de cima, e virou a
cabeça com cuidado para a beira da cama. Pareceu fácil e, apesar da
largura e do peso, seu corpanzil acabou por acompanhar aos poucos a
movimentação da cabeça. Porém, quando a sustentou fora da cama, em pleno
ar, ficou com medo de continuar avançando dessa maneira, porque, se no fim
ele se deixasse cair assim, seria um verdadeiro milagre que sua cabeça não
saísse seriamente machucada. E ele não podia de modo algum se arriscar a
perder os sentidos agora; preferia permanecer na cama.

Mas quando, bufando com a repetição do esforço, voltou à mesma posição de
antes, e tornou a enxergar suas perninhas em luta umas contra as outras,
ainda mais agitadas, se isso era possível, e não viu nenhuma chance de
acalmar e organizar aquela anarquia, repetiu consigo mesmo que era
inadmissível continuar deitado e que o mais lógico seria sacrificar tudo,
ainda que fosse ínfima a esperança de que assim conseguisse sair da cama.
Ao mesmo tempo, porém, não se esquecia de lembrar que naquela situação uma
reflexão calma, a mais calma, seria muito melhor do que resoluções
desesperadas. Nesses momentos buscava com os olhos a janela, aguçando ao
máximo o olhar, mas infelizmente muito pouca confiança e inspiração havia
a extrair da visão da neblina matutina, que ocultava até o lado oposto da
rua estreita. “Sete horas já”, ele disse ao ouvir o clique do despertador,
“sete horas já, e ainda uma neblina dessas.” E durante alguns instantes
permaneceu deitado quieto, quase sem respirar, como se esperasse da
quietude total o retorno das condições normais de realidade.

Mas depois falou para si mesmo: “Antes das sete e quinze, é indispensável
que eu tenha deixado a cama de uma vez. Aliás, até lá terá vindo alguém da
loja perguntar por mim, porque ela abre antes das sete”. E decidiu então
sair da cama com o corpo todo, movendo-o em toda a sua extensão num
balanço calculado e uniforme. Se chegasse a cair da cama dessa forma, era
de supor que conservaria ilesa a cabeça, que pretendia manter firmemente
ereta durante a queda. As costas pareciam duras; na certa nada sofreriam
no choque com o tapete. O maior receio lhe vinha da atenção que o alto
estrondo despertaria e das preocupações, quando não do susto, que deveria
causar atrás de todas as portas. Mas era preciso correr o risco.

Quando já estava suspenso pela metade para fora da cama --- o novo método
era mais uma diversão que um esforço, ele só precisava de pequenos
arrancos ao balançar ---, ocorreu-lhe como seria simples se alguém viesse em
seu auxílio. Duas pessoas fortes --- pensava em seu pai e na empregada ---
seriam mais do que suficientes; bastava-lhes enfiar os braços por baixo de
suas costas redondas, removê-lo da cama nessa posição, inclinar-se com a
carga e então apenas assisti-lo com cautela para que completasse o giro no
chão, onde enfim as perninhas oxalá iriam adquirir algum sentido. No
entanto, descontado o fato de que as portas estavam trancadas, ele deveria
mesmo pedir ajuda? Apesar de toda a aflição, não pôde deixar de sorrir a
esse pensamento.

Tanto já se deslocara que a um balanço mais forte mal poderia
manter o equilíbrio, e em muito pouco tempo teria de se decidir de uma
vez, pois dali a cinco minutos seriam sete e quinze --- quando a campainha
do apartamento tocou. “É alguém da loja”, falou para si e quase congelou,
enquanto suas perninhas em resposta dançavam ainda mais depressa. Por um
momento tudo continuou quieto. “Não vão abrir”, disse, agarrando-se a
alguma esperança absurda. Mas então, como sempre, naturalmente, a
empregada dirigiu-se com o passo firme até a porta e a abriu. Gregor só
precisou ouvir a primeira palavra do cumprimento da visita e já soube quem
era --- nada menos que o gerente. Por que será que Gregor estava condenado a
ser empregado numa firma onde a menor falta era vista com a maior das
desconfianças? Todos os funcionários eram, portanto, sem exceção, uns
mandriões, não havia entre eles uma só pessoa dedicada e fiel, que, se por
acaso apenas umas poucas horas da manhã não havia utilizado em prol da
loja, estaria se moendo de remorsos e definitivamente sem condições de
deixar a cama? Já não seria suficiente mandar um aprendiz pedir
informações --- se é que um tal interrogatório fosse mesmo necessário ---,
precisava vir o gerente em pessoa, e precisava com isso ser mostrado a
todos os familiares inocentes que a investigação desse caso suspeito só
podia ser confiada ao juízo do gerente? E mais por causa da irritação a
que fora levado através desses pensamentos do que em decorrência de uma
decisão tomada, lançou-se com todas as forças para fora da cama. Houve um
barulho alto de pancada, mas de fato um estrondo é que não foi. O tapete
amorteceu um pouco a queda, e também as costas tinham mais elasticidade do
que Gregor supunha, daí o som abafado, que nem chamava a atenção tanto
assim. Só não erguera a cabeça com o cuidado necessário, e por isso a
contundira; girou-a e, cheio de raiva e dor, esfregou-a no tapete.

“Alguma coisa caiu lá dentro”, disse o gerente no cômodo da esquerda.
Gregor procurou imaginar se ao menos uma vez não poderia se passar com o
gerente algo semelhante ao que lhe acontecera hoje; a possibilidade não
deveria de modo algum ser descartada. Porém, como se fosse uma resposta
seca a essa hipótese, o gerente deu alguns passos firmes e ouviu-se o
rangido de suas botas de verniz. No quarto da direita, a irmã sussurrou
para adverti-lo: “Gregor, o gerente está aí”. “Já sei”, disse Gregor de si
para si; mas erguer a voz a uma altura tal que pudesse ser ouvida pela
irmã, isso ele não arriscou.

“Gregor”, agora falava o pai, do cômodo à esquerda, “o senhor gerente veio
até aqui e quer saber por que você não partiu com o primeiro trem. Nós não
sabemos o que dizer a ele. E ele também quer falar com você em particular.
Então faça o favor de abrir a porta. Ele terá a bondade de desculpar a
desarrumação do quarto.” “Bom dia, senhor Samsa”, intrometeu-se o gerente,
chamando com voz amigável. “Ele não está bem”, a mãe disse ao gerente,
enquanto o pai ainda discursava para a porta, “ele não está bem, acredite,
senhor gerente. Se não, como Gregor iria perder um trem! O rapaz não tem
mais nada na cabeça a não ser a loja. Eu até fico irritada, porque ele
nunca sai à noite; agora mesmo, ele esteve oito dias seguidos na cidade,
mas ficou em casa todas as noites. Ele senta à mesa com a gente, e fica
quieto lendo o jornal ou estudando o horário dos trens. Já é uma grande
distração quando se ocupa com algum trabalho de marcenaria. Agora mesmo,
por exemplo, em duas ou três noites ele acabou de entalhar uma pequena
moldura; o senhor vai ficar admirado de ver como ela ficou bonita; está
pendurada lá dentro, no quarto; o senhor vai ver, assim que Gregor abrir.
Eu, aliás, fico contente que o senhor esteja aqui, senhor gerente; só a
gente não seria capaz de fazer o Gregor abrir a porta; ele é tão teimoso;
e com certeza não está nada bem, apesar de ter dito o contrário hoje de
manhã.” “Já vai”, disse Gregor, calculadamente lento, e não se mexeu, para
não perder nenhuma palavra da conversa. “De outro modo, prezada senhora,
eu também não saberia explicar”, disse o gerente, “tomara que não seja
nada sério. Embora, por outro lado, eu seja também obrigado a dizer que
nós, homens de negócios --- feliz ou infelizmente, como queira ---, muitas
vezes, em atenção às obrigações comerciais, devemos simplesmente ignorar
qualquer indisposição passageira.” “Então, o senhor gerente já pode
entrar?”, perguntou o pai com impaciência, e voltou a bater na porta.
“Não”, disse Gregor. No cômodo da esquerda sobreveio um silêncio
perturbador, no quarto da direita, a irmã começou a soluçar.

Por que será que a irmã não ia se juntar aos outros? Na certa só agora ela
havia saído da cama e ainda não se vestira. E por que chorava? Por que ele
não se levantava e não deixava o gerente entrar, por que se arriscava a
perder o emprego e por que desse jeito o chefe voltaria a perseguir os pais
com as antigas cobranças? Por enquanto, porém, essas eram preocupações de
todo desnecessárias. Gregor ainda estava presente e não tinha a menor
intenção de abandonar sua família. É certo que no momento ele continuava
lá, estendido no tapete, e ninguém que tivesse conhecimento de sua
situação iria lhe pedir a sério que deixasse o gerente entrar. Contudo,
por causa dessa pequena descortesia, para a qual mais tarde seria fácil
achar uma desculpa aceitável, Gregor não poderia ser mandado embora assim,
sumariamente. E lhe parecia muito mais lógico deixá-lo em paz
neste momento, em vez de perturbá-lo com choros e exortações. Mas era sem
dúvida a incerteza que afligia os outros e lhes desculpava o
comportamento.

“Senhor Samsa”, chamou então o gerente, em voz alta, “o que se passa? O
senhor fica entrincheirado aí em seu quarto, responde apenas com
monossílabos, deixa, sem necessidade, seus pais gravemente preocupados e ---
isso seja dito só de passagem --- falta às suas obrigações comerciais de uma
maneira realmente nunca vista. Eu falo aqui em nome de seus pais e do seu
chefe e lhe solicito, com toda a seriedade, o favor de uma explicação
clara e imediata. Estou atônito, estupefato. Eu acreditava conhecê-lo como
um homem pacato, ajuizado, e agora o senhor de repente dá mostras de
querer começar a exibir tais caprichos. É verdade que o chefe hoje de
manhã me insinuou uma possível explicação para a sua negligência --- dizia
respeito à cobrança recentemente confiada ao senhor ---, mas eu interpus a
bem da verdade quase a minha palavra de honra, dizendo que essa explicação
não tinha cabimento. Agora, contudo, que vejo sua obstinação
incompreensível, perco por completo a vontade de intervir o mínimo que
seja a seu favor. E sua posição não é em absoluto das mais garantidas. Eu
tinha a princípio a intenção de lhe dizer isso tudo em particular, porém,
já que o senhor me faz vir aqui desperdiçar inutilmente o meu tempo, não
sei por que também os seus pais não devam ouvir. Seus resultados nos
últimos tempos, aliás, não foram muito satisfatórios; claro que esta não é
a época do ano em que se fecham grandes negócios, nós reconhecemos; mas
uma época do ano em que não se fecha negócio algum, isso terminantemente
não existe, senhor Samsa, não pode existir.”

“Mas, senhor gerente”, Gregor gritou fora de si, esquecendo tudo o mais no
alvoroço, “eu abro agora mesmo, num instante. Um pequeno mal-estar, uma
tontura, impediu que eu me levantasse. Ainda estou aqui deitado. Mas já me
sinto mais disposto. Acabo mesmo de me levantar da cama. Só um minutinho
de paciência! Ainda não estou tão bem como pensava. Mas já me sinto
melhor. Como um homem pode ser pego assim de surpresa! Ainda ontem à noite
estava tudo bem comigo, meus pais são testemunha, ou melhor, já ontem à
noite eu tive um leve pressentimento. Deviam ter reparado em mim. Por que
não mandei logo avisar na loja?! Mas a gente sempre pensa que vai vencer a
doença sem precisar ficar em casa. Senhor gerente! Poupe os meus pais!
Todas as acusações que o senhor me faz agora, elas não têm fundamento;
também não me disseram uma palavra a esse respeito. Talvez o senhor não
tenha tomado conhecimento dos últimos pedidos que eu despachei. A
propósito, ainda saio para viajar com o trem das oito, essas poucas horas
de descanso me fortaleceram. Não é preciso se demorar mais, senhor
gerente; agora mesmo eu vou para a loja, e o senhor tenha a bondade de
transmitir esse recado e apresentar os meus cumprimentos ao senhor chefe.”

E enquanto Gregor despejava tudo isso às pressas, mal sabendo o que dizia,
havia se aproximado do guarda-roupa com facilidade, graças à prática
adquirida antes na cama, e procurava se levantar apoiado nele.
Queria muito abrir a porta, queria de fato mostrar-se e falar com o
gerente; estava ansioso para saber o que iriam dizer quando o vissem, eles
que agora tanto reclamavam sua presença. Se tomassem um susto, Gregor não
precisava justificar mais nada, e podia ficar descansado. E se aceitassem
tudo com calma, então também não havia motivo para se preocupar, e ele
poderia, se se apressasse, estar de fato às oito horas na estação
ferroviária. No começo ele escorregou algumas vezes no guarda-roupa liso,
mas ao final deu um último arranco e conseguiu se erguer; já não prestava
atenção à dor na parte de baixo de seu corpo, por mais que ardesse.
Deixou-se depois cair em direção ao encosto de uma cadeira próxima, a cujas
bordas se agarrou com suas perninhas. Só aí voltou a recuperar o domínio
sobre si mesmo e emudeceu, pois agora precisava ouvir o gerente.

“Os senhores entenderam uma única palavra?”, perguntou o gerente aos pais,
“não estará ele nos pregando uma peça?” “Meu Deus do céu”, exclamou a mãe
já começando a chorar, “ele pode estar muito doente, e nós aqui o
atormentando. Grete! Grete!”, gritou então. “Mamãe?”, respondeu a irmã do
outro lado. Elas se comunicavam através do quarto de Gregor. “Você tem que
chamar o médico agora mesmo. Gregor está doente. Corre até o médico. Você
ouviu como ele falou?” “Era uma voz de animal”, disse o gerente num tom
estranhamente baixo, em contraste com os gritos da mãe. “Anna! Anna!”,
chamou o pai batendo palmas da antessala para a cozinha, “vai já buscar um
chaveiro!” Logo as duas moças atravessavam a antessala correndo num frufru
de saias --- como é que a irmã havia se vestido tão rápido? --- e abriam
precipitadas a porta do apartamento. Não se ouviu a porta bater de volta;
decerto a tinham deixado aberta, como é de costume nas casas onde
aconteceu uma grande desgraça.

Gregor, porém, ficou bem mais tranquilo. É fato que suas palavras já não
eram compreendidas, embora tenham lhe parecido claras, mais claras até do
que antes, talvez porque seu ouvido logo se ajustara a elas. Mas, ainda
assim, agora já sabiam que nem tudo estava em ordem com ele, e se
dispunham a ajudá-lo. Fizeram-lhe bem a confiança e a firmeza com que as
primeiras providências foram tomadas. Ele se sentiu de novo integrado ao
círculo humano e esperava de ambos, médico e chaveiro, sem distingui-los
com muita precisão, realizações grandiosas e surpreendentes. A fim de
participar das discussões decisivas que estavam por vir com a voz o mais
clara possível, tossiu limpando a garganta, todavia se esforçou para
abafar o ruído, que provavelmente também teria um som distinto da tosse
humana, algo que ele mesmo já não tinha competência para discernir. No
cômodo do lado, nesse meio tempo, fez-se completo silêncio. Talvez os pais
tenham ido se sentar à mesa com o gerente, e cochichavam, talvez
estivessem todos pregados à porta, na escuta.

Gregor moveu-se até lá empurrando a cadeira devagar, depois soltou-a,
jogou-se contra a porta, segurou-se a ela mantendo-se na vertical --- as
pontas de suas perninhas tinham uma espécie de grude --- e descansou ali um
instante do esforço realizado. A seguir, contudo, foi tentar girar com a
boca a chave na fechadura. Era de lamentar que não tivesse uns dentinhos
de verdade --- com o que mais iria se agarrar à chave? ---, mas em compensação
as mandíbulas eram muito fortes, com certeza; com a ajuda delas de fato
conseguiu movimentar a chave e nem reparou que assim fatalmente infligia a
si mesmo algum ferimento, dado que um líquido marrom saía-lhe da boca,
escorria pela chave e pingava no chão. “Prestem atenção”, disse o gerente
no cômodo ao lado, “ele está virando a chave.” Para Gregor, esse foi um
grande incentivo; mas todos deveriam apoiá-lo, inclusive o pai e a mãe:
“Ânimo, Gregor”, deveriam gritar, “não desiste, dá duro na fechadura!”
Então, imaginando que todos acompanhavam seus esforços com interesse,
aferrou-se à chave sem pensar em mais nada, reunindo todas as forças que
podia. Bailava em torno da fechadura conforme o avanço da volta da chave;
acabou segurando-se na vertical apenas com a boca, e de acordo com a
necessidade pendurava-se na chave ou a forçava mais uma vez para baixo com
todo o peso do seu corpo. O claro estalo da fechadura que enfim destravava
tirou-o do transe. Tomando fôlego, ele disse: “Pois então, nem precisei do
chaveiro”, e deitou a cabeça na maçaneta, para abrir de par em par as
folhas da porta.

Como só podia abri-las puxando daquela maneira, uma delas já estava
praticamente toda aberta e ele mesmo ainda não podia ser visto. Devia
primeiro contornar devagar essa folha, e sempre com muita cautela, se não
quisesse fazer o papelão de cair de costas bem na entrada do quarto.
Estava ainda ocupado com essa difícil operação e não tinha tempo para
prestar atenção em outra coisa, quando ouviu o gerente soltar um sonoro
“Oh!” --- soava como o sopro do vento --- e então Gregor o enxergou também,
viu como ele, que era o que estava mais próximo da porta, comprimia com a
mão a boca aberta e retrocedia aos poucos, parecia puxado por uma força de
atração contínua, invisível. A mãe --- que apesar da presença do gerente
apresentava-se com os cabelos ainda soltos, desgrenhados pela noite de
sono --- juntou as mãos e olhou primeiro para o pai, depois deu dois passos
na direção de Gregor e despencou no meio do círculo formado por suas saias
esparramadas, o rosto encoberto pendendo contra o peito. O pai cerrou o
punho com uma expressão ameaçadora, como se quisesse forçar Gregor a
voltar para dentro do quarto, então olhou a sala em torno de si, indeciso,
cobriu os olhos com as mãos e caiu num choro que chegou a sacudir seu
peito forte.

Gregor contudo não havia nem saído do quarto, ainda se esticava de dentro
na direção da folha que permanecia trancada, de modo que se avistavam
apenas metade de seu corpo e acima, inclinada para o lado, a cabeça com a
qual olhava de soslaio para os outros. Havia clareado bastante nesse meio
tempo; no outro lado da rua projetava-se nítido um recorte do imenso
edifício da frente, cinza escuro --- era um hospital ---, com suas janelas
rigorosamente simétricas rasgando a fachada; a chuva ainda caía, mas eram
apenas gotas grossas, esparsas, e que assim espaçadas atingiam o solo num
ritmo regular. A louça do café espalhava-se em grande número sobre a mesa,
pois para o pai o café da manhã era a refeição mais importante do dia, e
ele a prolongava horas a fio com a leitura de diferentes jornais. Na
parede oposta pendia uma fotografia de Gregor da época do exército,
posando como tenente, as mãos na espada, sorrindo despreocupado, invocando
respeito por sua postura e sua farda. A porta da antessala estava aberta
e, como a porta da frente também continuava aberta, era possível enxergar
na parte de fora o corredor e o começo das escadas que conduziam para
baixo.

“Bem”, disse Gregor, com a plena consciência de que era o único que havia
mantido a calma, “agora mesmo vou me vestir, empacotar as amostras e
partir. Vocês vão querer, por favor, me deixar partir? Como vê, senhor
gerente, não estou sendo teimoso e trabalho com vontade; as viagens são
incômodas, mas eu não poderia viver sem viajar. Para onde está indo,
senhor gerente? Para a loja? É? O senhor vai reportar tudo direitinho? Uma
pessoa pode estar num momento incapacitada para o trabalho, mas essa é
exatamente a hora certa de recordar suas realizações passadas e de pensar
que depois, afastado o impedimento, com certeza ela virá a trabalhar até
mesmo com mais aplicação e concentração do que antes. O senhor sabe muito
bem o tanto que eu devo ao chefe. E ainda tenho de cuidar dos meus pais e
da minha irmã. Estou na penúria, mas com o trabalho vou conseguir dar a
volta por cima. Não torne as coisas mais difíceis do que já são para mim.
Tome o meu partido na loja! Eu sei que ninguém gosta do caixeiro-viajante.
Pensam que ele ganha rios de dinheiro e além disso leva uma vida folgada.
Ninguém nem mesmo toma a iniciativa de discutir mais a fundo esse
preconceito. Mas o senhor, senhor gerente, tem uma visão geral da situação
melhor que a dos outros empregados, até mesmo, seja dito em absoluto
segredo, melhor que a do próprio chefe, que em sua posição de patrão às
vezes se deixa levar por juízos equivocados, prejudicando um funcionário.
O senhor também sabe muito bem que o caixeiro-viajante, que passa quase o
ano inteiro fora da loja, torna-se facilmente vítima de intrigas,
maledicências e queixas infundadas, das quais é impossível que se defenda,
porque na maioria das vezes ele nem chega a tomar conhecimento delas e é só
quando volta para casa, esgotado após outra viagem, que vem a receber de
corpo presente suas graves consequências, cujas causas originais não tem
mais como descobrir. Senhor gerente, não vá embora sem me dizer uma
palavra demonstrando que ao menos em parte o senhor me dá alguma razão!”

Mas o gerente, logo às primeiras palavras, já havia lhe dado as costas e,
com um esgar de lábios, só ousava olhar em sua direção por cima dos ombros
trêmulos. E durante o discurso de Gregor não permaneceu parado um segundo
sequer, ao contrário, sem perdê-lo de vista, retrocedeu em direção à
porta, porém muito devagar, como se houvesse uma lei misteriosa que o
proibisse de deixar a sala. Aos poucos chegou à antessala e, a julgar pelo
movimento repentino com que retirou o pé no último passo para fora da
sala, era possível acreditar que tivesse pisado em brasas. Já na
antessala ele esticava a mão direita cada vez mais na direção da escada,
como se lá o aguardasse uma salvação decididamente extraterrena.

Gregor sabia que em hipótese alguma devia deixar o gerente partir naquele
estado de espírito, se não quisesse que sua posição na loja ficasse
comprometida. Os pais não entendiam direito o que acontecia; ao longo dos
anos, haviam formado a convicção de que Gregor estava garantido na loja
até o fim da vida, e agora, ainda por cima, com a urgência da situação,
tinham tanto a fazer que uma conjetura dessas lhes passava despercebida.
Mas a Gregor não passava. Era preciso reter o gerente, apaziguá-lo,
persuadi-lo e por fim convencê-lo; o futuro de Gregor e de sua família
dependia muito disso! Quem dera a irmã estivesse aqui! Ela era
inteligente; chorou por antecipação quando Gregor ainda estava só deitado
quieto, virado de costas. E na certa o gerente, esse conquistador, iria se
deixar levar por ela; que fecharia a porta do apartamento e ali mesmo na
antessala o acalmaria. Mas a irmã nem ao menos estava lá, Gregor mesmo
devia cuidar do assunto. E sem pensar que nada sabia de sua real
capacidade de se movimentar, sem pensar também que era possível, ou
melhor, muito provável que seu discurso mais uma vez não houvesse sido
compreendido, ele largou a folha da porta e se lançou pela abertura;
queria correr para junto do gerente que, de um modo patético, já se
agarrava com ambas as mãos ao corrimão do corredor; porém, no instante
seguinte, procurando algum apoio, Gregor deu um gritinho e caiu por cima
de suas perninhas. Mal isso aconteceu, ele experimentou, pela primeira vez
naquela manhã, algum conforto físico; as perninhas encontraram chão firme
abaixo de si; e obedeciam de pronto, como ele notou com satisfação; até
faziam força para levá-lo aonde quisesse; e ele logo acreditou que era
iminente a melhora definitiva de todos aqueles incômodos. Porém, no mesmo
momento em que se viu ali no chão, agitado pelo desejo de se movimentar,
não muito afastado de sua mãe, justamente à sua frente, ela, que parecia
tão recolhida dentro de si mesma, de um pulo se levantou, os braços
esticados, apontava com o dedo, gritando: “Socorro, meu Deus do céu,
socorro!”, mantinha a cabeça inclinada, como se quisesse observar Gregor
com mais atenção, entretanto, num gesto contraditório, recuava sem pensar em
mais nada; esquecera que atrás de si a mesa estava posta; quando encostou
nela, como que distraída, sentou-se logo sobre o tampo; e pareceu nem
reparar que a seu lado, saindo do grande bule que virara, o café derramava
em grandes golfadas sobre o tapete.

“Minha mãe”, Gregor disse baixinho, e olhou na direção dela. Por um
instante, o gerente foi varrido de seus pensamentos; em contrapartida,
ante a visão do café que escorria, não pôde deixar de estalar várias vezes
as mandíbulas, com cobiça. Diante disso, a mãe soltou um novo grito, saiu
correndo da mesa e caiu nos braços do pai, que veio depressa ao seu
encontro. Gregor, contudo, agora não tinha tempo para os pais; o gerente
já descia a escada; o queixo na altura do corrimão, ainda voltava o olhar
pela última vez. Gregor se preparou para correr, na expectativa de alcançá-lo;
o gerente deve ter pressentido alguma coisa, pois de um salto desceu
vários degraus e desapareceu; mas ainda soltou um grito, “Ahh!”, que ecoou
em toda a escadaria. Por infelicidade, então, parece que a evasão do
gerente também deixou o pai, que até aqui havia se comportado de modo
relativamente calmo, bastante transtornado, pois, em vez de
correr ele mesmo atrás do homem ou de pelo menos permitir que Gregor saísse em seu
encalço, agarrou com a mão direita a bengala do gerente, por este deixada
na cadeira, junto com o chapéu e o sobretudo, apanhou com a mão esquerda
um grosso jornal de cima da mesa e, com passadas pesadas, sacudindo a
bengala e o jornal, pôs-se a tocar Gregor de volta para o quarto. Nenhuma
súplica lhe foi de valia, nenhuma súplica sequer fora entendida, Gregor
quis baixar a cabeça de modo ainda mais humilde, e o pai só fez bater os
pés com mais força ainda. Do lado oposto, a mãe, apesar do tempo frio,
havia aberto uma janela e, debruçada o mais para fora possível, apertava o
rosto contra as mãos. Da viela, pela escadaria, veio uma forte corrente de
ar, as cortinas esvoaçaram, os jornais farfalharam sobre a mesa, algumas
folhas voaram e foram parar no chão. Implacável, o pai insistia e passou a
silvar como um selvagem. Mas Gregor ainda não tinha a menor prática em
andar para trás, e ia realmente muito devagar. Se ao menos pudesse dar
meia volta, na mesma hora estaria dentro do quarto, mas ele temia
impacientar o pai com uma manobra muito demorada, e a todo instante via-se
ameaçado pelo golpe fatal da bengala, desferido em suas costas ou na
cabeça. Porém, no fim não restou nenhuma outra alternativa, pois ele notou
assustado que, ao andar para trás, nem uma vez sequer conseguira manter a
direção; e assim, entre olhadelas furtivas, medrosas e incessantes na
direção do pai, começou a se virar, o mais rápido que podia, na
verdade, contudo, ainda muitíssimo lento. Talvez o pai tenha notado a sua boa
vontade, porque não o atrapalhou nessa hora, pelo contrário, até
direcionou a volta aqui e ali, de longe, com a ponta da bengala. Se não
fossem aqueles silvos insuportáveis! Por causa deles, Gregor perdia a
cabeça. Já havia dado quase toda a volta quando, atordoado por aqueles
ruídos incessantes, chegou a se enganar e recuou um bom pedaço no sentido
contrário. Mas quando afinal, contente, viu-se diante da abertura da
porta, percebeu que seu corpo era muito largo para passar por ela sem
dificuldades. Ao pai, é lógico, naquele estado de ânimo, não ocorria nem
de longe abrir um pouco a outra folha da porta, de modo a deixar espaço
suficiente para a passagem de Gregor. Sua ideia fixa resumia-se a fazê-lo
entrar no quarto o mais rápido possível. Jamais toleraria também os
preparativos minuciosos de que precisava para se erguer e desse modo
tentar passar pela porta. Em vez de ajudar, agora fazendo um barulho infernal,
forçava o avanço de Gregor como se não houvesse nenhum obstáculo; soava
como se já não fosse mais a voz de um único pai apenas; com efeito, a
coisa deixou de ser brincadeira, e Gregor --- seja lá o que acontecesse ---
jogou-se contra a porta. Um dos lados de seu corpo subiu, ele ficou
atravessado na abertura, uma parte de suas costas foi toda esfolada, na
tinta branca da porta restaram manchas repulsivas, logo se viu prensado e
sozinho não teria podido mais se mexer, as perninhas do lado de cima
pendiam vibrando no ar, as do outro lado eram dolorosamente pressionadas
contra o chão --- foi quando o pai, de trás, deu-lhe um empurrão forte,
desta vez deveras libertador, e ele, sangrando a valer, entrou voando para
dentro do quarto. A porta ainda foi fechada com a bengala e então enfim
tudo ficou quieto. 

\chapter*{}
\section{II}

\noindent{}Só ao crepúsculo Gregor acordou de seu sono pesado, que mais parecia um
desmaio. Com certeza, se não fosse perturbado, também não teria acordado
muito mais tarde, pois sentia que já dormira e descansara o suficiente,
embora tivesse a impressão de haver sido despertado por alguns passos
furtivos e pelo ruído da porta de acesso à antessala, que fora trancada
por precaução. A fraca luz das lâmpadas elétricas da rua iluminava
palidamente alguns pedaços do teto do quarto e a parte de cima dos móveis,
mas Gregor embaixo estava às escuras. Aos poucos, tateando ainda
desajeitado com suas antenas, às quais só agora dava o devido valor,
deslocou-se até a porta, para ver o que havia ocorrido. Seu lado esquerdo
estava que era uma única cicatriz, comprida, esticada, incômoda, e por
isso ele tinha de andar mancando mesmo com suas duas fileiras de pernas.
Uma perninha, aliás, saíra seriamente machucada dos incidentes da manhã ---
era um milagre que apenas uma tivesse se machucado --- e pendia inerte,
arrastada pelas outras.

Só ao se aproximar da porta é que foi perceber o que o atraíra na verdade;
era o cheiro de alguma coisa comestível. Com efeito, lá estava uma tigela
cheia de leite fresco, no qual algumas migalhas de pão boiavam. Por pouco
não riu de tanta alegria, pois continuava com uma fome ainda maior do que
estava pela manhã, e na mesma hora mergulhou a cabeça no leite, quase até
os olhos. No instante seguinte porém a retirou, desapontado; não apenas
comer lhe era dificultoso, por causa do lado esquerdo avariado --- e ele só
podia comer com a colaboração ofegante de todo o corpo ---, mas também
acima de tudo não lhe apeteceu em absoluto o leite, que antes era sua
bebida favorita, e na certa por isso a irmã lho trouxera, de modo que ele,
meio enojado, deixou de lado a tigela, voltando a se arrastar para o
centro do quarto.

Na sala, como Gregor via pelas frinchas da porta, o gás fora aceso, porém,
enquanto antigamente a essa hora o pai se dedicava à leitura em voz alta
do jornal vespertino, para a mãe e à vezes também para a irmã, agora não
se ouvia ruído algum. Pode ser que essa leitura, que a irmã sempre lhe
descrevia e comentava nas cartas, tivesse saído da rotina nos últimos
tempos. Em todo caso, o silêncio prevalecia, embora com toda a certeza o
apartamento não estivesse vazio. “Mas que vida mais tranquila a família
leva”, disse Gregor com seus botões e, enquanto olhava fixo a escuridão à
sua frente, sentiu um grande orgulho de que pudesse proporcionar a seus
pais e a sua irmã uma vida dessas em um apartamento tão bom. Mas e se
agora todo o sossego, todo o bem-estar, toda a paz tivessem de chegar a um
terrível fim? Para não se perder em tais pensamentos, Gregor preferiu se
movimentar, e ficou se arrastando no quarto, de um lado para o outro.

Durante a longa noite, as duas portas laterais, primeiro uma, depois a
outra, foram abertas uma frestinha apenas, e fechadas rapidamente em
seguida; alguém sem dúvida sentia necessidade de entrar, mas resolvera
pensar duas vezes. Gregor se deteve então bem na frente da porta da sala,
determinado a trazer a indecisa visita para dentro de alguma maneira, ou
pelo menos disposto a descobrir quem era; a porta porém não voltou a ser
aberta e ele esperou em vão. De manhã, quando as portas estavam trancadas,
todos queriam entrar, agora, depois que ele abrira sozinho uma delas, e as
outras ao que tudo indicava teriam sido abertas no decorrer do dia,
ninguém mais vinha, mesmo com as chaves do lado de fora.

Só bem tarde da noite é que desligaram a luz da sala, e nesse momento foi
fácil comprovar que os pais e a irmã haviam ficado acordados até aquela
hora, pois dava para ouvir direitinho como todos os três se afastavam na
ponta dos pés. Na certa até amanhã ninguém mais viria à procura de Gregor;
ele tinha assim bastante tempo para refletir, sem ser incomodado, sobre o
modo como devia reorganizar sua vida a partir de agora. Contudo, o quarto
alto e despojado, no qual era forçado a permanecer deitado contra o rés do
chão, angustiava-o, e ele não conseguia descobrir por que, uma vez que era
o mesmo quarto que habitava havia já cinco anos --- então, com uma meia
volta quase involuntária, e não sem certa vergonha, correu para debaixo do
canapé, onde se sentiu muito bem acomodado, apesar de suas costas meio
espremidas e apesar de não poder mais erguer a cabeça, lamentando apenas
que seu corpo fosse tão largo que não coubesse inteiro embaixo do móvel.

Ali ele passou toda a noite, uma parte em cochilos dos quais era seguidas
vezes despertado de súbito pela fome, uma parte tomado por preocupações
e incertas esperanças de que tudo afinal encontraria uma solução, de que o
certo seria agir com calma nesse meio tempo e, com paciência e o máximo de
respeito aos familiares, tentar tornar suportável o desgosto que ele, em
suas atuais circunstâncias, excepcionalmente era obrigado a lhes causar.

Já de manhã bem cedo, madrugadinha ainda, Gregor teve oportunidade de pôr
à prova suas ponderadas decisões, pois vindo da antessala a irmã, vestida
quase dos pés à cabeça, abriu a porta e examinou o interior do quarto com
apreensão. Ela não o descobriu na hora, mas quando o divisou ali embaixo
do canapé --- Deus do céu, ele tinha de estar em algum lugar, não podia sair
voando por aí --- levou um susto tão grande que, sem que pudesse se
controlar, voltou a fechar a porta no mesmo instante. Porém, parecendo
arrependida de sua atitude, tornou a abri-la em seguida e entrou na ponta
dos pés, como se ali estivesse alguém muito doente ou fosse um estranho.
Gregor estendeu um tantinho a cabeça até a borda do canapé e observou a
irmã. Iria ela notar que ele tinha deixado o leite intacto, lógico que de
modo algum por falta de apetite, e será que traria uma outra comida que
lhe fosse mais adequada? Se ela não tomasse a iniciativa, ele preferia
morrer de fome a ter de chamar sua atenção para isso, embora no fundo
sentisse uma vontade urgente de deixar o canapé, se atirar aos pés da irmã
e pedir a ela algo de bom para comer. Mas a irmã, surpresa, logo notou a
tigela ainda cheia, ao redor da qual havia apenas um pouco de leite
derramado, recolheu-a no mesmo instante, claro que não com as mãos nuas, e
sim com um pedaço de pano, e a levou para fora. Gregor ficou bastante
curioso para saber o que ela traria em troca, e teceu as mais diversas
suposições a respeito. Nunca, porém, teria adivinhado o que a irmã, em sua
grande bondade, realmente fez. A fim de testar o seu paladar, ela lhe
trouxe várias coisas sortidas, que dispôs em uma folha de jornal velho.
Havia ali legumes passados já meio apodrecidos; ossos da última ceia
recobertos de molho branco ressecado; uma porção de passas e amêndoas; um
queijo que há dois dias Gregor julgara intragável; um pedaço de pão duro,
uma fatia de pão com manteiga e outra fatia com manteiga e sal. Além
disso, à frente de tudo voltou a colocar a tigela, pelo visto destinada de
uma vez por todas a ele, agora cheia de água. E por delicadeza, sabendo
que Gregor não iria comer diante dela, afastou-se às pressas e chegou a
dar a volta na chave, a fim de que ele notasse que poderia se pôr tão à
vontade quanto quisesse. As perninhas de Gregor zuniram quando ele partiu
na direção da comida. Suas feridas, aliás, já deviam ter sarado de todo,
ele não sentia mais nenhum desconforto, o que o deixou pasmo pois lembrou
como, mais de um mês atrás, havia feito com a faca um cortinho de nada no
dedo, e esse machucado ainda antes de ontem doía um bocado. “Será que
minha sensibilidade diminuiu?”, pensou, e sorveu ávido o queijo, que acima
de todas as outras comidas o atraíra imediata e energicamente. Com
rapidez, um após o outro, vertendo lágrimas de contentamento, ele devorou
o queijo, os legumes e o molho; os alimentos frescos, ao contrário, não
agradavam o seu paladar, ele mal podia suportar-lhes o cheiro, e teve até
de afastar para o lado as coisas que queria comer. Já dera cabo de tudo há
algum tempo, e continuava largado no mesmo lugar, apenas descansando,
quando a irmã, para sinalizar que ele devia retroceder, deu uma volta na
chave. Isso o despertou de pronto, quando estava a ponto de cochilar, e
ele voltou correndo para debaixo do canapé. Mas precisou de muito
autocontrole para permanecer ali embaixo durante o curto espaço de tempo
em que a irmã esteve no quarto, porque, após a lauta refeição, seu corpo
ficou mais redondo e ele mal conseguia respirar naquele aperto. Entre
breves crises de asfixia, viu com os olhos um pouco saltados a irmã, sem
desconfiar de nada, juntar com uma vassoura não apenas as sobras, mas
inclusive os alimentos em que ele nem sequer chegara a tocar, como se
esses também não pudessem mais ser aproveitados, e a viu ainda despejar
tudo depressa em um balde que cobriu com uma tampa de madeira e levou para
fora do quarto. Mal ela lhe deu as costas, Gregor avançou logo para fora
do canapé e relaxou, voltando a estufar-se.

Dessa maneira recebia doravante Gregor diariamente sua refeição, a
primeira vez de manhã, quando os pais e a empregada ainda dormiam, a
segunda vez depois que todos almoçavam, pois era o momento em que os dois
dormiam de novo ainda um bocadinho, e a empregada saía, encarregada pela
irmã de uma incumbência qualquer. Com certeza os pais não queriam que
Gregor morresse de fome, porém talvez só suportassem, da experiência de
suas refeições, no máximo ouvir dizer que aconteciam, ou quem sabe a irmã
quisesse lhes poupar mais uma aflição, ainda que pequena, pois era visto
que já sofriam o bastante.

Gregor não chegou a saber com quais desculpas o médico e o chaveiro foram
dispensados naquela manhã inicial, porque, como não o compreendiam,
ninguém presumia, nem mesmo a irmã, que ele pudesse compreender os outros,
e por isso, quando ela estava no quarto, ele era obrigado a se contentar
apenas com os suspiros e os lamentos que de vez em quando escutava. Só
depois que ela se acostumou um pouco com tudo --- uma aceitação completa,
claro, não poderia nunca ser cogitada ---, é que Gregor chegou a flagrar
algumas vezes uma observação mais simpática ou que assim poderia ser
entendida. “Acho que hoje estava bom”, ela falava, quando Gregor havia
devorado com gosto a refeição, ou, caso contrário, o que gradativamente
foi se repetindo com maior frequência, costumava dizer, meio entristecida:
“De novo não tocou em nada”.

Mas, conquanto não conseguisse saber de nenhuma novidade diretamente,
Gregor captava quase tudo o que vinha dos cômodos adjacentes, e assim que
escutava alguma voz corria no ato até a porta respectiva e colava-se a ela
com o corpo todo. Sobretudo nos primeiros tempos, não havia uma conversa
que, de algum modo, ainda que só às escondidas, não tratasse dele. Dois
dias seguidos, durante todas as refeições, só se ouviram deliberações
sobre como deviam se comportar daí em diante; mas também entre as
refeições falava-se do mesmo assunto, pois havia sempre no mínimo dois
membros da família em casa, dado que ninguém queria ficar sozinho ali e
não tinham a menor condição de abandonar o apartamento logo de uma vez.
Ainda nos primeiros dias --- não estava muito claro nem o que nem o quanto
ela sabia do incidente --- a empregada pediu de joelhos à mãe que a mandasse
embora de imediato, e quando, quinze minutos depois, deixava o emprego,
agradeceu com lágrimas nos olhos tanto a demissão quanto o imenso favor
que nessa hora lhe prestavam, e jurou de pé junto, sem que isso lhe fosse
solicitado, não revelar o menor detalhe a ninguém.

Então a irmã se viu também obrigada a ajudar a mãe na cozinha; em todo
caso, não era muito o esforço exigido, pois não se comia quase nada.
Gregor ouvia vezes seguidas como um deles incentivava em vão o outro a
comer, sempre obtendo como única resposta: “Obrigado, estou satisfeito”,
ou algo semelhante. Tampouco deviam beber. Em várias ocasiões a irmã
perguntava ao pai se ele não queria cerveja, e se punha com alegria à
disposição para ela mesma ir buscar ou então, ante o silêncio do pai e
para eliminar qualquer suspeita de inconveniência, dizia que podia mandar
a zeladora ir no lugar dela, mas daí o pai respondia com um poderoso “Não”
final, e não se falava mais no assunto.

Já no decorrer desses primeiros dias o pai expôs a ambas, mãe e irmã, suas
reais perspectivas e condições financeiras. Aqui e ali ele deixava a mesa
para ir buscar algum comprovante ou alguma caderneta no pequeno e valioso
cofre que conseguira salvar da bancarrota de seu negócio, ocorrida há
cinco anos. Ouvia-se ele destravar a complicada fechadura e, depois de
retirar o que buscava, voltar a trancá-la. Essas explicações do pai, por
um lado, foram as primeiras boas notícias que Gregor chegou a ouvir desde
o seu confinamento. Sempre achara que nada havia sobrado para o pai do
antigo negócio, pelo menos o pai nunca lhe dissera o contrário, e de
qualquer modo Gregor também nunca lhe perguntara nada a respeito. Sua
única preocupação na época havia sido fazer de tudo para que a família
superasse o mais rápido possível o contratempo comercial, que deixara
todos no mais completo desalento. E foi assim que ele começou a trabalhar
com uma disposição fora do comum, e passou da noite para o dia de simples
balconista a caixeiro-viajante, função que evidentemente lhe abria muitas
outras oportunidades de ganho, e cujos ótimos resultados, sob a forma de
comissões, rápido se transformaram em dinheiro vivo que podia ser
espalhado sobre a mesa em casa, diante da família assombrada e satisfeita.
Foram bons tempos aqueles, que depois nunca mais se repetiram, pelo menos
não com igual esplendor, apesar de Gregor ter recebido mais tarde ainda
muito dinheiro, o que o capacitava a assumir, como assumiu, as despesas de
toda a família. Tanto a família quanto Gregor se acostumaram logo a essa
situação, eles aceitavam o dinheiro agradecidos, ele o entregava de bom
grado, mas não houve mais nenhuma manifestação efusiva. Só a irmã
permaneceu próxima a Gregor, ela que diferente dele amava a música e
aprendera a tocar violino de um modo enternecedor, e ele planejava em
segredo matriculá-la no conservatório no próximo ano, sem ligar para os
altos custos que isso devia acarretar, e que seriam compensados de uma ou
outra maneira. Várias vezes, nas conversas com a irmã, durante as curtas
estadias de Gregor na cidade, o conservatório era mencionado, porém sempre
como se fosse apenas um belo sonho, cuja realização era impensável, e os
pais não ouviam essas fantasias inocentes com muita satisfação; mas Gregor
havia refletido a sério sobre o assunto e pretendia anunciar seus planos
oficialmente na noite de Natal.

Esses pensamentos, inúteis na sua situação atual, passavam-lhe pela cabeça
enquanto permanecia lá erguido e colado à porta, escutando. Às vezes,
devido ao cansaço de todo o corpo, não conseguia mais prestar atenção e
deixava sem querer a cabeça bater na porta, mas a endireitava no mesmo
instante, pois até o mínimo ruído que dessa forma provocava era ouvido do
lado de fora e fazia todos se calarem. “Que será que tanto se mexe”, dizia
o pai momentos depois, a voz alta dirigida para a porta, e só aos poucos a
conversa interrompida era retomada.

Gregor ficou então cansado de saber --- pois o pai fazia questão de repetir
sucessivas vezes suas explicações, em parte porque há muito que ele
próprio já não se ocupava dessas coisas, em parte também porque a mãe não
compreendia tudo logo na primeira vez --- que apesar de toda a desgraça
ainda tinham disponível uma sobra orçamentária dos velhos tempos, em todo
caso bastante pequena, mas que nesse ínterim os juros acumulados haviam
feito crescer um pouquinho. Além disso também havia o dinheiro que Gregor
todo mês deixava em casa --- retirava para si apenas algumas notas ---, que
não era integralmente gasto e acumulara-se, originando um pequeno capital.
Atrás da porta, Gregor balançou a cabeça em sinal de aprovação, contente
com essa inesperada prudência e economia. Era bem verdade que, com esse
excedente de dinheiro, já poderia ter quitado outras parcelas da dívida
que o pai tinha com o chefe, e aquele dia em que poderia enfim se libertar
do emprego já estaria mais próximo, porém agora sem dúvida as coisas
estavam melhor assim, do jeito que o pai havia organizado.

Contudo, a quantia não era em absoluto suficiente para que se pudesse
viver de seus rendimentos; talvez bastasse para manter a família durante
um, no máximo dois anos, para mais não dava. Eram portanto apenas
economias de que não podiam dispor, e deviam conservar intactas para um
caso de maior necessidade; o dinheiro do dia-a-dia teria de ser ganho.
Acontece, no entanto, que o pai, embora ainda com saúde, era um homem
idoso, que estava há cinco anos totalmente afastado do trabalho e além do
mais já não podia fazer muito esforço; nesses cinco anos, que foram as
primeiras férias de sua vida diligente porém malsucedida, ele engordara
muito e por causa disso havia ficado mole e pesadão. E devia a velha mãe,
porventura, ir atrás do dinheiro, ela que sofria de asma e ficava cansada
só de andar pelo apartamento e que, dia sim dia não, passava deitada no
sofá, de janelas abertas, com crises respiratórias? Ou iria conseguir
dinheiro a irmã, ainda uma criança com seus dezessete anos e que até o
momento tivera uma vida invejável, nada mais que cuidar de se vestir,
dormir até tarde, ajudar um pouco na casa, uma ou outra atividade de lazer
modesta e tocar violino acima de tudo? Sempre que a conversa chegava na
necessidade de ganhar dinheiro, Gregor primeiro se soltava e em seguida se
jogava sobre o frio sofá de couro encostado ao lado da porta, pois ficava
inflamado diante de tanta humilhação e infortúnio.

Era frequente ele passar a noite inteira ali em cima, sem dormir um só
minuto, horas seguidas apenas arranhando o couro. Salvo quando se dispunha
a encarar o grande esforço de empurrar uma cadeira até a janela, depois
subir rastejando até alcançar o parapeito e, apoiado na cadeira,
inclinar-se para frente, óbvio que apenas como uma espécie de recordação
da sensação de liberdade que antes tinha ao olhar pela janela. Pois era
patente que dia após dia via cada vez com menos nitidez as coisas que
estavam afastadas de si; não avistava mais o hospital do outro lado, cuja
visão cotidiana obrigatória ele antes amaldiçoava, e, se não tivesse
certeza absoluta de que morava na travessa Charlotte, uma viela tranquila,
não obstante sua total urbanidade, seria capaz de acreditar que sua janela
dava para um deserto no qual o cinza do céu e o cinza da terra se juntavam
e eram indistinguíveis. Bastou à atenciosa irmã ver em duas ocasiões a
cadeira encostada na janela para que todas as vezes, depois de arrumar o
quarto, voltasse a colocá-la no mesmo lugar, e a partir de então até a
folha interna da janela era deixada aberta.

Se Gregor pudesse falar com a irmã, e agradecê-la por tudo que estava
obrigada a fazer por ele, aceitaria os favores com mais facilidade; assim,
porém, ele sofria. A irmã decerto procurava disfarçar ao máximo o que
havia de penoso em tudo aquilo e, é claro, quanto mais o tempo passava,
tanto melhor ela o conseguia, entretanto, com o passar do tempo também
Gregor veio a perceber as coisas com maior exatidão. A entrada dela já lhe
parecia horrível. Mal havia entrado, sem sequer parar para fechar a porta,
ela que antes tanto se esforçava para esconder de todos a visão de Gregor,
disparava até a janela e a escancarava com as mãos apressadas, como se
estivesse sufocando, e ficava ali parada um instante, retomando o fôlego,
mesmo quando o frio era intenso. A correria e o barulho assustavam Gregor
duas vezes por dia; todos esses momentos passava tremendo embaixo do
canapé e todavia sabia muito bem que ela com certeza o dispensaria de tal,
se ao menos lhe fosse possível permanecer com as janelas fechadas num
quarto por ele ocupado.

Uma ocasião, passado já todo um mês desde a transformação, não havendo
mais nenhuma razão especial para a irmã se admirar com a aparência de
Gregor, ela veio um pouco antes do que costumava e o flagrou quando ele
ainda olhava pela janela, imóvel e dessa forma numa situação propensa a
provocar um susto. Se ela não entrasse, não teria sido uma atitude
inesperada para Gregor, uma vez que em sua posição ele a impedia de abrir
de imediato a janela, mas ela não apenas não entrou, como chegou a
retroceder e a trancar a porta; uma pessoa estranha teria o direito de
pensar que ele estava lá à espreita com a intenção de mordê-la. Gregor,
logicamente, foi logo se esconder embaixo do canapé, porém teve de esperar
até o meio-dia para que a irmã voltasse, e ela pareceu muito mais
desconfiada do que antes. Ele percebeu então que sua visão ainda era
insuportável para ela, e seguiria sendo insuportável no futuro, e que ela
devia se superar para não sair correndo ao avistar ainda que apenas uma ínfima parte
de seu corpo de sob o canapé. A fim de preservá-la inclusive
dessa mínima visão, ele um dia carregou o lençol nas costas até o canapé ---
precisou de quatro horas para a operação --- e o arranjou de um modo tal que
todo o seu corpo ficaria encoberto, e a irmã, ainda que se abaixasse, não
conseguiria vê-lo. Se em sua opinião o lençol não fosse necessário, então
ela mesma poderia retirá-lo, pois estava claro o bastante que não era nada
do agrado dele isolar-se assim tão completamente, porém ela deixou o
lençol do jeito que estava, e Gregor chegou a crer que divisara um olhar
de agradecimento quando uma vez cauteloso ergueu um pouquinho o pano com a
cabeça, para verificar como a irmã havia recebido o novo arranjo.

Nos primeiros quinze dias os pais não conseguiram superar o receio de
entrar no quarto dele, e Gregor muitas vezes ouviu como aprovavam sem
reservas o trabalho realizado pela irmã, apesar de até então terem vivido
meio aborrecidos com ela, que lhes parecia uma menina um tanto quanto
inútil. Mas agora os dois, pai e mãe, sempre esperavam diante da porta,
enquanto a irmã cuidava da arrumação, e mal ela saía tinha de descrever em
detalhes quais eram as condições do quarto, o que Gregor havia comido,
como agira dessa vez, e se dava para notar algum sinal de melhora. A mãe,
aliás, logo no começo teve vontade de ir vê-lo, mas o pai e a irmã a
detiveram, a princípio com argumentos sensatos, aos quais Gregor prestou
muita atenção e com os quais estava inteiramente de acordo. Depois, porém,
precisaram retê-la à força, e quando enfim ela gritou: “Me deixem entrar,
é o coitado do meu filho! Vocês não entendem que eu tenho que ver o meu
filho?”, Gregor então pensou que talvez fosse bom se a mãe viesse vê-lo,
não todo dia, é lógico, mas podia ser uma vez por semana; ela com certeza
compreendia tudo muito melhor do que a irmã, que apesar de todo o seu
empenho era ainda só uma criança e em última hipótese devia ter assumido
um encargo tão pesado apenas por causa de sua insensatez juvenil.

Não demorou para Gregor satisfazer o seu desejo de ver a mãe. De dia, em
atenção aos pais, ele já não queria aparecer à janela, mas também não
tinha como rastejar muito nos parcos metros quadrados do chão, era difícil
ficar parado durante a noite, a comida logo deixou de lhe proporcionar o
menor prazer, e assim, como distração, ele adquiriu o hábito de rastejar
de um lado para o outro pelas paredes e pelo forro. Gostava em especial de
se pendurar lá em cima, no teto; era bem diferente de ficar deitado no
chão; respirava-se com maior liberdade; uma vibração leve corria pelo
corpo; e no abandono quase feliz em que se encontrava lá no alto, podia
acontecer de se soltar, surpreendendo a si próprio, e se espatifar no
chão. Agora, porém, naturalmente, ele tinha sobre seu corpo um domínio
muito maior do que antes, e não se machucava, mesmo caindo de uma altura
dessas. A irmã percebeu de imediato o novo divertimento que Gregor havia
descoberto --- ele deixava um rastro de grude aqui e ali ---, e aí lhe veio à
cabeça dar a ele o máximo de espaço para rastejar, removendo os móveis que
o atrapalhavam, ou seja, principalmente o guarda-roupa e a escrivaninha.
Ocorre que não seria capaz de fazer tudo isso sozinha; não ousava pedir a
colaboração do pai; a empregada com toda a certeza não a ajudaria pois,
para uma mocinha de mais ou menos dezesseis anos, esta até que resistia
com bravura desde a dispensa da antiga cozinheira, contudo havia
solicitado permissão para manter a cozinha o tempo todo trancada, e só ser
obrigada a abrir em caso de extrema necessidade; assim só restou à irmã
como opção ir atrás da mãe, em uma das ausências do pai. A mãe a seguiu
com arroubos de uma alegria incontida, mas ficou muda diante da porta.
Lógico que a irmã conferiu primeiro, para ver se estava tudo em ordem no
quarto; só depois é que a mãe pôde entrar. Gregor na pressa havia puxado o
lençol mais para baixo, deixando-o mais amarrotado, o conjunto dava mesmo
a impressão de um lençol atirado ao acaso sobre o canapé. Gregor também,
nessa ocasião, absteve-se de erguer o pano para espiar; renunciava a ver a
mãe logo dessa vez, e contentava-se só por ela estar ali de verdade. “Pode
vir, ele não está à vista”, disse a irmã, ao que tudo indica puxando a mãe
pela mão. Gregor ouviu então como as duas mulheres magras empurraram o
guarda-roupa, todavia pesado, de seu lugar, e como a irmã insistiu em
tomar para si a maior parte da tarefa, sem dar ouvidos às advertências da
mãe, receosa de que a filha se cansasse demais. Demorava muito. Após uns
bons quinze minutos, a mãe disse que o melhor seria deixar o móvel ali
mesmo, em primeiro lugar porque ele era muito pesado, não conseguiriam
terminar antes da chegada do pai e com o guarda-roupa no meio do quarto
iriam bloquear toda a passagem a Gregor, em segundo lugar, porém, porque
não havia certeza de que a retirada dos móveis fosse do agrado dele. A ela
lhe parecia o inverso; a visão das paredes vazias era de oprimir o
coração; e por que Gregor não teria a mesma sensação, ele que já estava há
tanto tempo acostumado com os móveis e por isso se sentiria abandonado no
quarto vazio? “E não será o caso”, concluiu a mãe bem baixinho, já quase
num sussurro, como se quisesse impedir que Gregor, cujo esconderijo exato
ela desconhecia, ouvisse sequer o rumor de sua voz, pois que ele não
compreendia as palavras, disso estava convencida, “e não será o caso, ao
retirar os móveis, de demonstrar com isso que renunciamos a qualquer
esperança de recuperação e o deixamos, sem a menor consideração, entregue
à própria sorte? Eu penso que seria melhor tentar manter o quarto do
jeitinho que estava antes, para que Gregor, quando voltar de novo para a
gente, encontre tudo inalterado e possa ainda mais rápido esquecer esse
intervalo de tempo.”

Ao escutar essas palavras da mãe, Gregor admitiu que a privação total,
nesses últimos dois meses, de qualquer conversação humana direta,
associada ao convívio regular no seio da família, devia ter perturbado seu
juízo, pois de outro modo não saberia explicar como pôde sinceramente
desejar que seu quarto fosse esvaziado. Tinha de fato vontade de que esse
quarto aconchegante, com seus móveis antigos dispostos de forma tão
cômoda, fosse transformado numa toca onde ele poderia rastejar livre e
despreocupado em todas as direções, todavia sob o risco do esquecimento
paralelo, rápido e rasteiro, de seu passado humano? Já estava aliás bem
próximo desse esquecimento, e só mesmo a voz da mãe, que há muito não
ouvia, para alertá-lo. Nada tinha de ser removido; tudo deveria ficar como
estava; ele não podia dispensar as influências benéficas dos móveis sobre
sua condição; e se estes o impediam de praticar o seu rastejar absurdo,
isso não era uma perda, e sim uma grande vantagem.

Infelizmente, porém, a irmã era de outra opinião; ela se acostumara, a
propósito não sem um certo direito, a se contrapor aos pais nas discussões
que tinham Gregor como tema, comportando-se como uma especialista no
assunto, e assim naquela hora o conselho da mãe foi motivo suficiente para
que insistisse, não só na remoção do guarda-roupa e da escrivaninha, como
a princípio pensara, mas também na retirada de todo o mobiliário, com
exceção do canapé, indispensável. O que a aferrava a essa exigência era,
claro, não apenas uma teimosia infantil e a autoconfiança tão inesperada,
adquirida a duras penas nos últimos tempos; ela com efeito também
observara que Gregor precisava de bastante espaço para rastejar, ao passo
que dos móveis, pelo contrário, tanto quanto se podia ver, não aproveitava
quase nada. Contudo, talvez colaborasse para sua atitude  o espírito
entusiástico das jovens de sua idade, que não perde ocasião de se
manifestar e com o qual Grete era no momento instigada a querer tornar
ainda mais assustadora a situação de Gregor, para que então pudesse
realizar por ele mais do que fizera até agora. Pois em um local no qual
apenas Gregor imperasse entre as paredes vazias, é seguro que ninguém, a
não ser Grete, jamais se atreveria a entrar.

E assim ela não se deixou dissuadir de sua decisão pela mãe que, presa de
pura inquietação, se sentindo muito insegura dentro do quarto, logo se
calou e ajudou a irmã, na medida de suas forças, a levar o guarda-roupa
para fora. Ora, em último caso Gregor ainda podia dispensar este móvel,
mas a escrivaninha, esta tinha que ficar. E as mulheres mal haviam deixado
o quarto com o guarda-roupa, contra o qual se espremiam, gemendo, quando
Gregor projetou a cabeça por baixo do canapé, para ver como daria para
interferir com prudência e o máximo de delicadeza possível. Mas, por
infelicidade, foi bem a mãe quem retornou primeiro, enquanto Grete no
cômodo ao lado se atracava ao guarda-roupa e tentava balançá-lo de um lado
para o outro, é lógico que sem conseguir tirá-lo do lugar. A mãe, porém,
não estava habituada à aparência de Gregor, poderia ter um ataque, e por
isso espavorido ele retrocedeu depressa até o fundo do canapé, entretanto
não pôde mais evitar que o lençol se movimentasse um pouco na parte da
frente. Foi o suficiente para chamar a atenção da mãe. Ela estacou, ficou
um instante imobilizada e então voltou para junto de Grete.

Apesar de Gregor repetir consigo mesmo, sucessivamente, que nada de
excepcional acontecia, era tão-só a mudança de uns poucos móveis, esse vai
e vem das mulheres, as breves exortações entre elas, os riscos dos móveis
no assoalho, isso tudo o atingia, ele logo teve de reconhecer, como se
fosse um grande tumulto, com ruídos vindo de todos os lados, e mesmo
mantendo cabeça e pernas encolhidas e o corpo colado ao chão, ele se via
forçado a admitir que logo não suportaria mais aquilo. Estavam esvaziando
seu quarto; retiravam tudo o que tinha de mais caro; o guarda-roupa, onde
guardava o arco de serra e as outras ferramentas, elas já haviam levado;
agora soltavam a escrivaninha que fora bem fixada no chão, e na qual ele
escrevera suas lições no tempo da escola de comércio, do colégio e até
mesmo do primário --- de modo que não sobrava muito tempo para que avaliasse
quão boas eram as intenções das duas mulheres, de cuja real existência
aliás ele já havia quase se esquecido, pois, de exaustas, elas agora
trabalhavam mudas, e ouviam-se apenas as passadas pesadas de seus pés.

E assim, pois --- na hora em que as mulheres, já no cômodo ao lado,
apoiavam-se à escrivaninha, procurando retomar o fôlego ---, ele avançou
para fora, mudou de posição quatro vezes, calculando a direção da corrida,
sem saber ao certo o que salvar primeiro, quando notou, em destaque na
parede já de todo nua, o quadro pendurado da dona vestida puramente de
peles, rastejou rápido até lá e apertou-se contra o vidro, que o prendeu e
refrescou o calor de sua barriga. Pelo menos esse quadro, que agora Gregor
cobria por completo, com certeza ninguém levaria mais embora. Ele girou a
cabeça na direção da porta da sala, para prestar atenção nas mulheres, que
retornavam.

Elas não haviam se permitido muito tempo de descanso e logo estavam de
volta; Grete vinha com o braço ao redor da mãe e como que a arrastava. “E
agora, o que levamos?”, disse e olhou ao seu redor. Foi então que se
cruzaram, seu olhar com o de Gregor na parede. Por certo só em razão da
presença materna é que ela conservou a calma, abaixou o rosto para junto
da mãe, na tentativa de impedi-la de olhar à sua volta, e falou, não
obstante trêmula e irrefletidamente: “Vamos voltar, não é melhor a gente
descansar mais um tempinho na sala?”. Para Gregor, a intenção de Grete era
clara, ela queria garantir a segurança da mãe para depois vir varrê-lo da
parede. Pois ela que experimentasse! Ele estava unido ao seu quadro e não
o entregaria. Preferia pular na cara dela.

Mas as palavras de Grete a rigor só conseguiram sobressaltar a mãe, que
deu um passo para o lado, avistou a monstruosa mancha marrom sobre as
flores do papel de parede, gritou, com uma voz gutural e aflita, antes
mesmo de tomar consciência de que aquilo que via era Gregor: “Meu Deus,
meu Deus!”, e caiu por cima do canapé, com os braços estendidos, como se
largasse mão de tudo, e não voltou a se mexer. “Você, hein, Gregor!”,
bradou a irmã de punho erguido e com o olhar severo. Eram as primeiras
palavras diretas que lhe dirigia desde a transformação. Ela correu para o
cômodo ao lado, em busca de alguma essência com que pudesse despertar a
mãe desfalecida; Gregor também queria ajudar --- a defesa do quadro podia
ser adiada ---, mas estava tão grudado ao vidro que precisou usar da força
para se soltar; em seguida correu também até o aposento vizinho,
acreditando poder dar à irmã algum conselho, como nos velhos tempos; não
conseguiu fazer nada, porém, além de ficar imóvel atrás dela; enquanto
ainda remexia em diferentes frasquinhos, ela se virou e tomou um baita
susto; um vidro caiu no chão e se espatifou; um caco feriu Gregor no
rosto, uma substância corrosiva escorreu sobre ele; Grete, sem mais perda
de tempo, simplesmente pegou tantos frascos quanto podia carregar e
disparou com eles para junto da mãe; ao sair bateu a porta com o pé.
Gregor foi assim isolado da mãe, que por culpa dele talvez estivesse a
ponto de morrer; a porta ele não devia abrir, se não quisesse espantar a
irmã, que precisava ficar ao lado da mãe; não tinha no momento nada a
fazer, a não ser esperar; então, presa de remorsos e preocupação, ele
começou a rastejar, passando por cima de tudo, paredes, móveis, teto, e
por fim, em seu desespero, logo que todo o aposento começou a girar em
torno de si, caiu bem no centro, em cima da mesa grande.

Um breve intervalo de tempo transcorreu, Gregor continuava ali deitado,
prostrado, ao redor o silêncio, talvez isso fosse um bom sinal. Então a
campainha tocou. A mocinha, logicamente, estava trancada na cozinha e por
isso Grete teve de ir abrir. O pai estava de volta. “O que aconteceu?”,
foram suas primeiras palavras; a expressão da filha por certo lhe revelara
tudo. Grete respondeu com a voz abafada, o rosto apertado contra o peito
do pai: “Mamãe teve um desmaio, mas já está melhor. Gregor escapuliu”. “Eu
já esperava por isso”, disse o pai, “era o que eu sempre dizia, mas vocês
mulheres não quiseram me dar ouvidos.” Para Gregor ficou nítido que o pai
entendera mal o relato demasiado conciso de Grete e supunha que ele
tivesse cometido alguma brutalidade. Por isso devia agora tentar
apaziguá-lo, pois para esclarecer as coisas não havia tempo, tampouco
possibilidade. De modo que se afastou até a porta do seu quarto e ficou
bem junto dela, para que o pai, assim que entrasse pela antessala, pudesse
ver que ele tinha toda a intenção de regressar imediatamente aos seus
aposentos, e que não seria preciso tocá-lo de volta, pelo contrário,
bastava apenas abrir a porta e na mesma hora ele desapareceria.

Mas o pai não estava em condições de perceber tais sutilezas; “Arrá!”, ele
gritou logo ao entrar, em um tom que parecia ao mesmo tempo satisfeito e
ameaçador. Gregor tirou a cabeça da porta e a suspendeu na direção do pai.
Na verdade nunca havia imaginado o pai assim, como se apresentava agora;
além disso, nos últimos tempos, por conta da empolgação com o rastejar,
deixara de se preocupar como antes com os acontecimentos no resto do
apartamento, e portanto deveria estar pronto para encontrar muita coisa
mudada. Apesar disso, apesar de tudo, ainda era o pai? O mesmo homem que
antigamente, nas manhãs em que Gregor saía para uma viagem de negócios,
permanecia enterrado na cama, esgotado; que nas noites de regresso o
recebia no cadeirão, de pijama; incapaz de se erguer direito, podendo
apenas levantar os braços como sinal de alegria, e que nos raros passeios
da família, em um ou outro domingo do ano e nas datas mais importantes,
entre Gregor e a mãe, que em atenção a ele andavam devagar, caminhava
sempre um pouco mais devagar ainda, embrulhado em seu sobretudo gasto,
avançando a custo com a maior atenção ao apoiar a bengala, e quando queria
falar quase sempre se detinha e obrigava os outros a se reunir a seu
redor? Agora, porém, ele estava mesmo bem empertigado; metido em um
uniforme muito justo, azul com botões dourados, igual ao usado pelos
contínuos dos bancos; sobre o alto colarinho engomado do casaco
desdobrava-se sua papada dupla; sob as sobrancelhas espessas aflorava o
brilho radiante e atento dos olhos escuros; o cabelo branco, outrora todo
espetado, fora emplastrado e repartido rigorosamente ao meio em um
penteado meticuloso e luzidio. O quepe, no qual despontava um monograma
dourado, decerto de uma casa bancária, ele jogou em cima do canapé, num
lançamento em curva que cruzou todo o cômodo, depois marchou em direção a
Gregor com a cara amarrada, as abas do casaco do uniforme postas para
trás, as mãos nos bolsos da calça. Ele próprio não sabia direito como
proceder; ainda assim, erguia os pés a uma altura fora do comum, e Gregor
se espantou com as proporções colossais do solado de suas botas. Claro que
não ficaria apenas nisso, sempre soube, desde o primeiro dia de sua nova
vida, que o pai, para lidar com ele, só considerava apropriado o máximo de
rigor. Então saiu da sua frente correndo, estacando quando o pai ficava
parado e voltando a se apressar mal o pai se mexia. Desse modo deram mais
de uma vez a volta pelo cômodo, sem que algo de decisivo acontecesse,
inclusive sem que, no geral, em consequência do lento andamento, houvesse
a aparência de uma perseguição. Em vista disso, Gregor por enquanto também
se mantinha no chão, tanto mais temendo que o pai pudesse tomar uma fuga
pelas paredes ou pelo teto como uma provocação maldosa. Contudo, foi
obrigado a reconhecer que não suportaria essas corridinhas por muito mais
tempo, porque, enquanto o pai dava um passo, ele tinha de realizar um
sem-número de movimentos. A dificuldade de respirar logo começou a
aparecer, visto que, como nos velhos tempos, também não dispunha agora de
pulmões dignos de confiança. Por isso, ao se balançar para os lados, a fim
de reunir todas as forças para a corrida, nem abria os olhos; em sua
obtusidade, não conseguia pensar em outra solução a não ser correr; e
quase já havia esquecido que podia contar com as paredes, as quais,
todavia, eram aqui obstruídas por móveis finamente talhados, cheios de
pontas e quinas --- foi quando passou raspando ao lado dele alguma coisa
que, arremessada sem força, caiu e rolou à sua frente. Era uma maçã; logo
uma segunda voou em sua direção; Gregor ficou paralisado de medo; uma nova
corrida era inútil, pois o pai estava determinado a um bombardeio. Tinha
enchido os bolsos na fruteira do aparador e, sem caprichar na pontaria por
enquanto, atirava, maçã atrás de maçã. As pequenas frutas vermelhas
rolavam pelo chão, pareciam imantadas, e batiam umas contra as outras. Uma
maçã lançada de mansinho atingiu Gregor de leve, mas ricocheteou sem
maiores danos. A que veio voando logo em seguida, ao contrário, penetrou
firme em suas costas; Gregor quis ainda se arrastar como se a dor
extraordinariamente incrível fosse passar com a mudança de posição; porém
ele se sentia como se estivesse pregado e desmoronou, em completo colapso
de todos os sentidos. Só num último relance ainda pôde ver como a porta de
seu quarto se escancarou e a mãe, à frente da irmã que gritava, entrou
depressa, sem o vestido, pois a irmã a despira após o desmaio para
permitir-lhe uma respiração mais livre, como em seguida a mãe correu ao
encontro do pai e suas anáguas desapertadas escorregaram uma após a outra
até o chão, e como ela, tropeçando nas vestes caídas, atirando-se sobre o
pai e o abraçando, em completa conjunção com ele --- mas nessa hora a visão
de Gregor já vacilava ---, cruzando as mãos em torno de seu pescoço, pediu
clemência pela vida de Gregor. 

\chapter*{}
\section{III}

\noindent{}O ferimento sério, com o qual padeceu mais de um mês --- a maçã continuou,
uma vez que ninguém se atreveu a retirar, enfiada na carne, como uma
recordação exposta ---, pareceu fazer até mesmo o pai se lembrar de que
Gregor, apesar de suas feições asquerosas e deprimentes, era um membro da
família, e não merecia ser tratado que nem um inimigo, pelo contrário, no
seu caso a lei das obrigações familiares mandava engolir a repulsa e
tolerar, nada além de tolerar.

E se então, devido ao ferimento, Gregor também havia perdido certa
mobilidade, talvez para sempre, e necessitava no momento, como um veterano
mutilado, de longos e longos minutos para cruzar seu quarto --- rastejar no
alto, nem pensar ---, agora, em troca desse sensível declínio em suas
condições, recebia uma compensação em sua opinião bastante satisfatória,
dado que à noitinha a porta da sala, que uma ou duas horas antes ele já
tratava de observar com atenção, era em geral aberta para que, deitado no
escuro do seu quarto, não visível da sala, ele pudesse ver a família
reunida na mesa iluminada e escutar suas conversas, de certo modo com o
consentimento de todos, e portanto bem diferente de antes.

Decerto não eram mais aquelas reuniões calorosas dos velhos tempos, que
nos minúsculos quartos de hotel Gregor imaginara, nunca sem uma ponta de
inveja, quando, exausto, era obrigado a se enfiar nos lençóis gelados.

Tudo agora transcorria quase sempre na maior monotonia. Depois do jantar o
pai logo pegava no sono em sua cadeira; a mãe e a irmã recomendavam
silêncio uma à outra; a mãe, bastante curvada embaixo da luz, costurava
fina roupa branca para uma loja de confecções; a irmã, que arranjara um
emprego de vendedora, aprendia taquigrafia e francês à noite, para quem
sabe mais tarde obter um cargo melhor. Às vezes o pai despertava e, sem
nem se dar conta de que estivera dormindo, dizia para a mãe: “Mas que
tanto você fica aí só costurando!”, e voltava a adormecer no instante
seguinte, enquanto mãe e irmã trocavam entre si um sorriso fatigado.

Com uma birra particular, o pai nem em casa aceitava tirar seu uniforme de
serviço; e enquanto o pijama pendia inútil no cabide ele cochilava no seu
canto completamente vestido, como se estivesse sempre a postos para o
trabalho e mesmo aqui esperasse pelas ordens do supervisor. Em
consequência disso, o uniforme, que já não era novo no começo, não parava
limpo, apesar de todo o cuidado da mãe e da irmã, e Gregor olhava várias
vezes a noite inteira para aquela roupa vistosa, com seus botões dourados
superlustrados, e a cada dia mais cheia de nódoas, com a qual o velho no
maior desconforto dormia, entretanto, bem tranquilo.

Na hora em que o relógio chegava às dez, a mãe, fingindo bronquear,
chamava o pai e em seguida tentava persuadi-lo a ir para a cama, pois ali não
era lugar de dormir e uma boa noite de sono seria imprescindível para ele,
que devia se apresentar às seis da manhã em seu emprego. Mas, com a teima
que tinha desde que estava empregado, o pai sempre insistia em permanecer
à mesa, embora invariavelmente adormecesse, e então, além de tudo, era a
maior mão-de-obra induzi-lo a trocar a cadeira pela cama. Por mais que a
mãe e a irmã o animassem com breves incentivos, ele ficava uns quinze
minutos balançando devagar a cabeça, mantinha os olhos fechados, e não se
levantava. A mãe o puxava pela manga, sussurrava palavras ternas ao pé do
seu ouvido, a irmã deixava as lições para ajudá-la, mas com o pai isso de
nada valia. Ele só afundava ainda mais na cadeira. Apenas quando as
mulheres o tomavam pelos braços é que ele entreabria os olhos, olhava
alternado para a mãe e para a irmã, fazendo questão de dizer: “Isto é que
é vida. Este é o sossego da minha velhice”. E apoiado nas duas se
levantava, com muita dificuldade, como se fosse ele mesmo o que mais lhe
pesava, deixava-se conduzir pelas mulheres até a porta, lá as dispensava e
prosseguia então sozinho, enquanto a mãe largava às pressas o estojo de
costura, a irmã a pena, para irem correndo atrás dele e continuar ajudando.

Nessa família esfalfada e moída pelo trabalho, quem teria tempo para se
ocupar com Gregor além do estritamente necessário? O orçamento doméstico
era cada vez mais apertado; a empregada já fora inclusive dispensada; uma
faxineira enorme e robusta, com uma cabeleira branca esvoaçando pela
cabeça, vinha agora pela manhã e no final da tarde para dar conta do
serviço pesado; o restante sobrava para a mãe, junto com suas muitas
obrigações de costura. Até aconteceu de diversas joias da família, que no
passado a mãe e a irmã chegaram a ostentar radiantes em recepções e
cerimônias, terem de ser vendidas, como Gregor veio a saber à noite pela
falação geral em torno do montante obtido. A maior reclamação, porém, era
sempre a de que não podiam abandonar aquele apartamento, grande demais nas
atuais circunstâncias, por não saberem como incluir Gregor na mudança. Mas
Gregor percebia muito bem que não era apenas a preocupação com ele que
impedia uma mudança, pois seria simples transportá-lo em uma caixa
apropriada, com alguns furos para a entrada de ar; o que realmente
refreava a troca de apartamento era muito mais a total falta de esperanças
da família e o pensamento de que uma desgraça tal os fustigava, como nunca
a ninguém em todo o seu círculo de parentes e conhecidos. O que o mundo
demandava aos
%Iuri:cobravam dos
pobres, eles ofereciam ao máximo,
%Iuri:cumpriam totalmente
o pai ia buscar café para
os funcionários do baixo escalão do banco, a mãe se sacrificava pela roupa
íntima de pessoas estranhas, a irmã corria de um lado para o outro do
balcão atrás das exigências dos fregueses, mas as forças da família já
estavam no limite. E a ferida aberta nas costas voltava a doer em Gregor
como na primeira vez, quando a mãe e a irmã, depois de colocarem o pai na
cama, regressavam, deixavam a ocupação de lado, aproximavam-se uma da
outra e sentavam-se já de rosto colado; quando a mãe, apontando para
o quarto de Gregor, dizia: “Fecha aquela porta, Grete”, e quando enfim
Gregor voltava à escuridão, enquanto do outro lado as mulheres misturavam
suas lágrimas ou então, sem lacrimejar, contemplavam a mesa desconsoladas.

Gregor desperdiçava as noites e os dias e já quase não dormia. Algumas
vezes pensava em retomar as rédeas dos assuntos da família, igualzinho
como antes, na próxima vez que a porta fosse aberta; em seus pensamentos
voltavam a aparecer, depois de muito tempo, o chefe e o gerente, os
balconistas e os aprendizes, o imbecil do menino de recados, dois ou três
colegas de outras lojas, uma camareira de um hotel no interior,
lembrança querida e fugaz, uma atendente de caixa de uma loja de chapéus,
a quem cortejara a sério porém de modo muito tímido --- todos eles apareciam
misturados com pessoas estranhas ou já esquecidas, só que não serviam de
ajuda a ele e à sua família, ao contrário, eram indistintamente
inapeláveis, e Gregor ficava contente quando desapareciam. Mas daí não
tinha mais nenhuma gana de se preocupar com a família, sobrevinha apenas a
raiva pelo desleixo e, apesar de não poder imaginar nada que atiçasse o
seu apetite, planejava um modo de alcançar a despensa para retirar de lá
ao menos o que lhe era devido, ainda que não estivesse com fome. A irmã
agora, de manhã e ao meio-dia, antes de sair correndo para a loja, sem nem
cogitar uma forma de fazer um agrado especial a ele, empurrava para dentro
do quarto às pressas, com o pé, algum resto de comida qualquer, que à
noitinha ela botava para fora com uma vassourada, nem reparando se a
comida por acaso havia sido provada ou --- caso mais frequente --- continuava
perfeitamente intacta. A arrumação do quarto, de que ela agora só se
ocupava à noite, não poderia ser feita de modo mais rápido. Trilhas de
sujeira se estendiam pelas paredes, em toda parte cresciam montinhos de pó
e lixo. Nas primeiras vezes, assim que a irmã entrava, Gregor se
posicionava de maneira a deixar patente a sua insatisfação. Mas poderia ficar ali parado uma
semana, sem que a irmã se emendasse; ela enxergava a imundície tanto
quanto ele, porém estava determinada a ignorá-la. Ademais, com uma
suscetibilidade há pouco tempo adquirida, que a propósito contaminara toda
a família, ela zelava para que a arrumação do quarto de Gregor ficasse
exclusivamente por sua conta. Uma vez a mãe submeteu o quarto a uma faxina
geral, tarefa somente possível com o emprego de várias tinas de
água --- a umidade em excesso, no fim, também fez mal a Gregor e ele foi
deitar em cima do canapé, meio grogue, amargurado e incapaz de se mover ---,
mas o castigo da mãe não tardou a chegar. Pois à noite a irmã, mal notara
a mudança no quarto, voltou correndo para a sala, ofendidíssima, e, não
obstante o gesto de súplica das mãos erguidas da mãe, irrompeu numa crise
de choro que os pais --- o pai naturalmente acordou de um pulo de sua
cadeira --- a princípio observaram aturdidos e sem ação; até que também se
condoeram; o pai repreendia a mãe à direita, por não ter deixado à irmã a
limpeza do quarto de Gregor; à esquerda, por outro lado, berrava com a
irmã, exigindo que ela nunca mais limpasse aquele quarto; enquanto a mãe
tentava levar para a cama o pai, que ficara fora de si nesse alvoroço, a
irmã, sacudida pelos soluços, martelava a mesa com seus punhos pequenos; e
Gregor silvava alto de raiva, porque ninguém havia se lembrado de fechar a
porta e poupá-lo da cena e de toda aquela barulheira.

Mas mesmo se a irmã, exaurida por sua atividade profissional, estivesse
cheia de cuidar dele como cuidava antes, a mãe não teria de modo algum
obrigação de rendê-la e Gregor ainda assim não precisava ficar ao
desamparo. Pois agora tinham a faxineira. Essa velha viúva, que em sua
longa vida devia ter superado as piores situações com a ajuda de sua
notável robustez, não nutria a rigor nenhuma aversão por Gregor. Não sendo
de modo algum curiosa, havia uma vez por acaso aberto a porta do quarto
dele, que pego de surpresa começou a correr de um lado para outro, embora
ninguém o perseguisse, e ao avistá-lo ela continuou de pé onde estava, as
mãos cruzadas no peito, admirada. Desde então, não passava um dia sem
entreabrir a porta, de manhã e no final da tarde, para espiar Gregor um
minutinho. No começo, ela também o chamava com palavras que devia julgar
simpáticas, tais como “Vem cá, bichão!” ou “Cadê o velho besourão?!”.
Gregor não atendia a esses chamados, ficava mais é quieto no seu canto,
como se a porta nem tivesse sido aberta. Se ao menos fosse ordenado a essa
faxineira que, em vez de satisfazer seus caprichos indo perturbá-lo para
nada, limpasse seu quarto todos os dias! Certa ocasião, de manhã cedinho ---
uma chuva pesada golpeava a vidraça, talvez já um sinal anunciando a
primavera ---, logo que a faxineira veio de novo com aquele jeito de falar,
Gregor ficou irritado de tal modo que se virou para ela, é certo que num
passo lento e debilitado, como se fosse partir para o ataque. A faxineira,
porém, em vez de se intimidar, simplesmente tomou uma cadeira que
encontrou ao lado da porta, ergueu-a bem no alto e, do modo como ficou
parada lá com a bocona escancarada, era clara sua intenção de só fechá-la
quando a cadeira em suas mãos tivesse baixado nas costas dele. “Não vai
avançar mais?”, ela perguntou, ao vê-lo retroceder, e pôs de volta
tranquila a cadeira em seu lugar.

Gregor já não comia quase nada. Só quando por acaso topava com a comida
disponível é que abocanhava de brincadeira uma pequena porção, que
mantinha na boca horas a fio e depois cuspia fora, na maior parte das
vezes. A princípio pensou que era a tristeza com o estado de seu quarto
que o impedia de comer, mas foi justo com as mudanças do quarto que ele se
conformou mais depressa. Coisas que não podiam ser alojadas em outra parte
eram jogadas lá dentro, e agora havia muitas dessas coisas, dado que um
quarto do apartamento havia sido alugado para três inquilinos. Esses
senhores muito sérios --- todos os três usavam barba, como Gregor averiguou
certa vez por uma fresta da porta --- eram cheios de escrúpulos quanto à
ordem, não apenas no quarto que ocupavam, mas também, já que agora estavam
hospedados ali, em toda a casa, particularmente na cozinha. Não toleravam
trastes inúteis, muito menos sujos. Além disso haviam trazido consigo, em
grande parte, seus próprios móveis e objetos de decoração. Por esse motivo
muitas coisas se tornaram supérfluas, coisas que na verdade não eram
vendáveis, mas que também não se queria jogar fora. Todas elas rumaram
para o quarto de Gregor. Mesmo destino da caixa de cinzas e do cesto de
lixo da cozinha. Bastava que algo não tivesse utilidade no momento para
que a faxineira, sempre com muita pressa, simplesmente o atirasse para
dentro do quarto de Gregor; na maioria das vezes, por sorte, Gregor via só
o objeto em questão e a mão que o segurava. Pode ser que a faxineira
tivesse a intenção, havendo tempo e oportunidade, de pegar as coisas de
volta ou então de jogar tudo fora de uma vez, mas na prática elas ficavam
por lá, largadas no mesmo local onde haviam sido atiradas, a não ser
quando Gregor se retorcia no meio da tralha e começava a deslocá-las, no
começo forçado, porque já não sobrava espaço livre para rastejar, mas
depois com uma alegria crescente, embora, ao final dessas excursões, morto
de cansaço e mágoa, ficasse horas sem poder se movimentar de novo.

Como os inquilinos às vezes jantavam em casa, na sala, que era de uso
comum, a porta que dava para lá permanecia fechada algumas noites, mas
Gregor abdicava de sua abertura sem dificuldade, já tendo inclusive
deixado de aproveitá-la em várias ocasiões em que fora franqueada,
preferindo ficar encolhido, sem que a família percebesse, no canto mais
escuro do seu quarto. Uma vez, porém, a faxineira deixou essa porta um
tanto entreaberta, e assim ela ficou, mesmo quando os inquilinos entraram
à noite e a luz foi acesa. Eles foram se sentar à ponta da mesa, onde
antigamente o pai, a mãe e Gregor comiam, desdobraram os guardanapos e
empunharam garfo e faca. Na mesma hora a mãe apareceu na porta com uma
travessa de carne e bem atrás dela a irmã com outra travessa transbordando
de batatas. A comida fervia numa nuvem de vapor. Os inquilinos se curvaram
sobre as travessas colocadas à sua frente, como se quisessem dar uma
conferida antes de comer, e de fato o que estava sentado no meio, e
parecia ter mais autoridade que os outros dois, deu um talho na carne
ainda na travessa, averiguando de modo ostensivo se estava tenra o
suficiente e se não era o caso de devolvê-la à cozinha. Ele ficou
satisfeito, e a mãe e a irmã, que observavam apreensivas, puderam suspirar
com um sorriso de alívio.

A família mesmo foi comer na cozinha. Apesar disso, antes de ir para lá o
pai veio até a sala e, com uma única mesura, o quepe na mão, deu a volta
em torno da mesa. Os inquilinos levantaram-se juntos e murmuraram de má
vontade qualquer coisa lá com suas barbas. Quando então ficaram a sós,
comeram quase que em total silêncio. Gregor achou estranho que, de todos
os diversos ruídos que acompanham uma refeição, o mais perceptível fosse o
som de dentes mastigando, como se com isso devesse ser mostrado a ele que
para comer eram necessários dentes, e que com mandíbulas banguelas, mesmo
as mais bonitas, não se obtém sucesso. “Eu tenho sim vontade de comer”,
Gregor disse a si mesmo, muito preocupado, “só que não essas
coisas. Como comem esses inquilinos, e eu aqui definhando!”

Bem naquela noite --- Gregor não se lembrava de tê-lo escutado em todo
aquele tempo --- veio da cozinha o som do violino. Os inquilinos já haviam
terminado sua refeição noturna, o do meio desdobrou um jornal, deu aos
outros dois uma folha cada, e agora eles liam recostados e fumavam. Quando
o violino começou a soar, tiveram a atenção despertada, ergueram-se e
foram na ponta dos pés até a porta da antessala, onde ficaram parados, um
espremido contra o outro. Devem ter sido ouvidos da cozinha, pois o pai
gritou de lá: “A música não agrada aos senhores? Ela pode parar agora
mesmo”. “Pelo contrário”, disse o senhor do meio, “a menina não quer vir
até aqui tocar na sala, que é muito mais cômodo e agradável?” “Oh, como
não!”, gritou o pai, como se fosse ele o violinista. Os senhores voltaram
para a sala e esperaram. Logo chegou o pai com a estante, a mãe com a
partitura e a irmã com o violino. A irmã calmamente ajeitou tudo para
tocar; os pais, que nunca antes haviam alugado quartos e por isso se
excediam na cortesia com os inquilinos, nem se atreveram a sentar nas
próprias cadeiras; o pai encostou na porta, a mão direita enfiada entre
dois botões do casaco abotoado; a mãe, porém, aceitou a cadeira oferecida
por um dos senhores e, como não a moveu do local onde ele por acaso a
colocara, acabou sentando afastada.

A irmã começou a tocar; pai e mãe, cada qual do seu lado, seguiam atentos
os movimentos das mãos dela. Gregor, atraído pela música, arriscou-se um
pouco mais à frente e logo estava com a cabeça na sala. Já nem se admirava
de que, nos últimos tempos, demonstrasse tão pouca consideração pelos
outros; antigamente esse tipo de respeito era para ele um motivo de
orgulho. E na verdade teria justo agora muito mais razões para não
aparecer, porque, por causa da poeira que tomava conta de seu quarto e que
ao menor movimento se espalhava no ar, ele também vivia coberto de pó;
arrastava consigo fiapos, pelos, restos grudados dos lados e nas costas;
sua apatia diante de tudo era muito grande para que se desse ao trabalho
de deitar de costas, como fazia antes várias vezes ao dia, e se esfregar
no tapete. E mesmo nesse estado ele não teve o menor pudor de invadir um
pedaço do piso imaculado da sala.

Em todo caso ninguém nem reparava nele. A família estava bastante absorta
pelo som do violino; ao contrário dos inquilinos, que a princípio, as mãos
nos bolsos das calças, haviam se posicionado bem atrás da estante, muito
próximos da irmã, de modo a que pudessem todos os três acompanhar as
notas, o que com certeza devia atrapalhá-la, mas logo, conversando a meia
voz, as cabeças abaixadas, se afastaram para junto da janela, onde
permaneceram, observados pelo olhar apreensivo pai. Essa atitude realmente
tinha a aparência mais do que clara de que eles haviam sido frustrados em
sua intenção de ouvir ao violino uma música bela ou animada, que estavam
fartos daquela apresentação e que apenas por educação ainda admitiam ter
sua paz perturbada. Sobretudo o modo como todos eles sopravam para o alto
a fumaça de seus charutos, pelo nariz e pela boca, dava a entender um alto
grau de irritação. E todavia a irmã tocava tão bonito. Seu rosto pendia
para o lado, seus olhos aguçados e tristes acompanhavam as linhas da
pauta. Gregor rastejou outro tanto adiante e manteve a cabeça bem junto ao
chão, para assim, quem sabe, poder interceptar o olhar dela. Era por ser
um animal que a música o atraía tanto? A ele parecia como se tivesse à sua
frente o caminho para o ansiado alimento desconhecido. Estava decidido a
avançar até a irmã, puxá-la pela saia, e desse modo fazê-la entender que
ela poderia com prazer vir ao quarto dele com o violino, pois ninguém aqui
valorizava a música como ele quisera valorizar. Queria então não mais
deixá-la sair do quarto, pelo menos não enquanto estivesse vivo; sua
figura medonha pela primeira vez lhe seria útil; queria estar ao mesmo
tempo diante de todas as portas, bufando feito uma fera contra os
opressores; a irmã, porém, não devia ficar com ele à força, e sim por
vontade própria; ela iria se sentar ao lado dele no canapé, inclinar o
ouvido em sua direção, e ele queria nesse momento confidenciar a ela que
tivera o firme propósito de mandá-la para o conservatório, e que, se nesse
meio tempo a desgraça não houvesse ocorrido, no último Natal --- o Natal já
tinha passado, aliás? ---, ele teria anunciado o fato a todos, sem se
importar com qualquer tipo de objeção. Depois dessa explicação, a irmã
iria desatar num choro comovido, e Gregor, erguendo-se até altura do ombro
dela, lhe daria um beijo no pescoço que, desde que entrara na loja, ela
trazia sem fita nem cordão.

“Senhor Samsa!”, gritou para o pai o senhor do meio e, sem desperdiçar
mais palavras, apontou o dedo indicador na direção de Gregor, que avançava
devagar. O violino ficou mudo, o inquilino intermediário, num único
movimento de cabeça, sorriu para seus amigos e então voltou a olhar para
Gregor. Ao pai pareceu que o mais urgente era, em vez de tocar Gregor,
primeiro acalmar os inquilinos, embora estes não estivessem nada nervosos
e Gregor parecesse animá-los bem mais do que a música do violino. Ele
correu para o lado deles e com os braços abertos tentou conduzi-los a seus
aposentos e ao mesmo tempo bloquear com o corpo a visão de Gregor. Aí eles
de fato se zangaram um pouco, não dava para saber se com o comportamento
do pai ou se com a descoberta que acabavam de fazer, de que, sem que
soubessem, tinham como vizinho de quarto um tipo igual a Gregor. Exigiram
explicações do pai, ergueram como ele os braços, beliscaram as barbas com
impaciência e só a custo foram aos poucos recuando aos seus aposentos.
Enquanto isso a irmã conseguiu sair do estado de alheamento em que havia
caído com a interrupção súbita da música, depois de sustentar mais um
instante o violino e o arco nas mãos que pendiam frouxas, e em seguida
olhar para a partitura como se ainda estivesse tocando, conseguiu se recompor
num átimo, colocou o instrumento no colo da mãe, que com dificuldades
respiratórias e intensa atividade pulmonar seguia sentada em sua cadeira,
e foi correndo para o quarto ao lado, do qual, empurrados pelo pai, os
inquilinos se aproximavam agora mais depressa. Era de ver como, sob as
mãos treinadas da irmã, cobertas e travesseiros voavam pelo ar e pousavam
já dispostos direitinho nas camas. Ainda antes dos senhores chegarem à
entrada do quarto, ela já dera conta da arrumação e saíra de fininho. O
pai pareceu de novo tomado por sua teimosia, de uma tal forma que olvidou
todo o respeito a que estava obrigado perante seus locatários. Ele só
insistia e voltava a empurrar, até que, na porta do quarto, o senhor do
meio bateu com força o pé no chão, e desse modo conseguiu deter o pai.
“Declaro para todos os fins”, ele falou, ergueu a mão e dirigiu o olhar
também para a mãe e a irmã, “que, em vista das condições repugnantes em
vigor nesta casa e nesta família” --- nesse ponto ele, decidido, deu uma
breve cusparada no chão ---, “dou por cancelada minha locação. É evidente
que não pagarei o mínimo que seja, nem pelos dias em que aqui estive
hospedado, muito pelo contrário, ainda hei de ponderar se não apresento
contra o senhor uma queixa formal que --- o senhor pode ter certeza --- seria
bem fácil de justificar.” Ele se calou e olhou fixo à sua frente, como se
esperasse alguma coisa. Ato contínuo, seus amigos vieram com as palavras:
“Nós também cancelamos a nossa”. A seguir ele agarrou a maçaneta e fechou
ruidosamente a porta.

O pai, tateando com as mãos, voltou cambaleando até sua cadeira e ali
desabou; dava a impressão de que se recostava para sua soneca noturna
cotidiana, mas os intensos meneios da cabeça, que parecia solta, mostravam
que de modo nenhum estava dormindo. Gregor ficou esse tempo todo quieto,
deitado no mesmo lugar em que fora surpreendido pelos inquilinos. O
desapontamento com o insucesso de seus planos mas talvez também a fraqueza
provocada pela fome extrema impossibilitavam que se movimentasse. Com uma
certa garantia, temia já para o instante seguinte a tempestade coletiva a
desabar em cima dele, e esperava. Não o perturbou nem mesmo o violino na
hora em que caiu do colo da mãe, depois de escorregar de seus dedos
trêmulos, e produziu um estampido retumbante.

“Meus pais”, falou a irmã e bateu a mão na mesa à guisa de introdução,
“não dá para continuar assim. Se vocês por acaso não enxergam desse jeito,
eu enxergo. Não quero usar o nome do meu irmão com esse monstro, então só
digo uma coisa: temos que nos livrar disso. Tentamos o que era humanamente
possível para criá-lo e suportá-lo, acho que ninguém tem o menor direito
de nos condenar.”

“Ela está coberta de razão”, o pai disse consigo mesmo. A mãe, que ainda
não conseguia respirar normalmente, levou a mão à boca e começou a tossir
para dentro, com uma expressão de insanidade no olhar.

A irmã correu para junto da mãe e amparou-lhe a testa. O pai, levado pelas
palavras da irmã, pareceu ter encontrado ideias mais precisas,
endireitou-se na cadeira, brincou com o quepe do uniforme em meio aos
pratos largados sobre a mesa, ainda do jantar dos inquilinos, e olhou uma
vez ou outra para Gregor, que permanecia quieto.

“Temos que dar um jeito de nos livrar dessa coisa”, a irmã falou, agora
exclusivamente para o pai, pois a mãe, com a tosse, não escutava nada,
“isso vai acabar matando vocês dois, só estou vendo a hora. Quem é
obrigado a trabalhar tanto quanto nós três não pode ainda ter que aturar
em casa esse tormento sem fim. Eu já não suporto mais.” E rompeu num choro
tão forte que suas lágrimas espirraram no rosto da mãe, que as enxugou com
movimentos mecânicos da mão.

“Filha”, falou o pai morrendo de pena e exagerando na compreensão, “mas o
que vamos fazer?”

A irmã apenas deu de ombros como sinal da sensação de impotência que a
acometera enquanto chorava, o que contrastava com sua segurança de antes.

“Se ele pudesse nos entender”, disse o pai com ar interrogativo; a irmã em
meio ao choro sacudiu a mão com veemência em sinal de que isso estava fora
de cogitação.

“Se ele pudesse nos entender”, repetiu o pai e fechou os olhos, para
assimilar a convicção da irmã quanto à impossibilidade do fato, “então
quem sabe fosse possível um acordo com ele. Mas assim ---”

“Que vá embora”, a irmã gritou, “é o único jeito, pai. Basta se livrar do
pensamento de que é o Gregor. Ter acreditado nisso durante tanto tempo,
essa no fundo é a nossa desgraça. Mas como essa coisa pode ser o Gregor?
Se fosse o Gregor, já teria entendido há muito tempo que é impossível a
convivência das pessoas com um bicho desses, e teria partido por vontade
própria. Então não haveria mais irmão, mas podíamos seguir tocando a vida
e preservar sua memória. Mas assim, fica esse bicho nos seguindo, expulsa
os inquilinos, com certeza quer tomar o apartamento e nos deixar dormindo
na rua. Olha aí, pai”, ela gritou de repente, “lá vem ele de novo!” E, num
pânico de todo incompreensível para Gregor, a irmã deixou até a mãe de
lado, chegou mesmo a empurrar sua cadeira, como se preferisse sacrificar a
mãe a ficar perto dele, e correu atrás do pai que também se levantou,
provocado única e exclusivamente pelo comportamento dela, e posicionou os
braços a meia altura à sua frente, na tentativa de protegê-la.

Mas Gregor não pretendia de modo algum amedrontar quem quer que fosse,
muito menos a irmã. Ele havia tão-só começado a se virar para regressar ao
seu quarto, o que em todo caso seria muito chamativo, dado que, devido às
suas condições lastimáveis, era difícil fazer a volta e ele precisava ajudar com a cabeça que por causa disso diversas vezes levantava e
batia contra o chão. Parou um pouco e olhou ao seu redor. Parece que
reconheciam sua boa intenção; tinha sido só um susto passageiro. Agora
todos o observavam calados e entristecidos. A mãe largada em sua cadeira,
as pernas estendidas uma por cima da outra, os olhos quase se fechando de
fadiga; o pai e a irmã sentados lado a lado, ela apoiando a mão atrás do
pescoço dele.

“Agora talvez me deixem completar a volta”, Gregor pensou e retomou sua
ocupação. Não conseguia evitar o ruído de sua respiração ofegante do
esforço e era obrigado a fazer uma pausa a todo momento. No mais, não era
pressionado por ninguém, havia sido deixado por sua própria conta. Depois
de completar a volta ele deu início ao retorno em linha reta. Ficou
admirado com a grande distância que o separava de seu quarto, e não
entendia como, com a sua fraqueza, tinha ainda há pouco, quase sem
perceber, percorrido o mesmo caminho. O tempo todo concentrado apenas em
rastejar depressa, ele mal reparava que não era incomodado por nenhuma
palavra, nenhum grito de sua família. Só depois de chegar à porta é que
ele virou a cabeça, não muito porque na hora sentiu um torcicolo, em todo
caso ainda viu que atrás de si nada havia mudado, apenas a irmã se
levantara. Seu último olhar avistou de relance a mãe, a essa altura já
totalmente adormecida.

Mal acabara de entrar no quarto e a porta foi fechada às pressas, com o
trinco e à chave. Gregor levou um susto tão grande com o barulho
inesperado às suas costas que suas perninhas vacilaram. Era a irmã que
tinha toda essa pressa. Já estava lá levantada só esperando, avançou então
saltitando na ponta dos pés, Gregor nem a ouviu se aproximar, e ao girar a
chave na fechadura ela gritou para os pais: “Até que enfim!”.

“E agora?”, perguntou-se Gregor e olhou a escuridão ao seu redor. Logo
veio a descobrir que não conseguia se mexer de jeito nenhum. Não ficou
surpreso, antes lhe parecia pouco natural que até então tivesse podido
andar de verdade com aquelas perninhas finas. De resto ele estava
relativamente bem. É certo que tinha dores por todo o corpo, mas para ele
era como se elas fossem ficando mais e mais fracas e estivessem a ponto de
sumir de uma vez por todas. Já mal sentia a maçã podre em suas costas e a
inflamação que a circundava, ambas cobertas por uma fina camada de poeira.
Lembrou-se de sua família com intensa ternura e amor. Sua posição a
respeito da necessidade de seu desaparecimento era na medida do possível
ainda mais convicta do que a da irmã. Ficou nesse estado meditabundo,
vazio e tranquilo, até que as três horas da manhã soaram no relógio da
torre. Ainda presenciou o despontar da claridade matutina do lado de fora
da janela. Então, independente de sua vontade, sua cabeça pendeu toda para
baixo e suas narículas expeliram sem força seu último suspiro.

Quando, de manhã cedo, a faxineira veio --- sempre afobada e com muita
energia, apesar de já lhe ter sido pedido várias vezes que o evitasse, ela
batia todas as portas de uma tal maneira que com a sua chegada já não era
mais possível dormir sossegado em nenhum cômodo do apartamento ---, a
princípio não achou nada de novo em sua habitual visitinha a Gregor.
Pensou que ele estivesse ali deitado imóvel de propósito, bancando o
ofendido; ela o julgava capaz de compreender tudo. Como por acaso estava
com a comprida vassoura nas mãos, tentou fazer cócegas nele ali mesmo da
porta. Quando isso também não resultou em nada, ela ficou brava e cutucou
um pouco mais fundo, e só após empurrá-lo de seu lugar sem encontrar
nenhuma resistência é que procurou ver melhor. Tão logo reconheceu a
realidade dos fatos, arregalou os olhos, soltou um assobio, mas não se
conteve muito tempo, e sim foi logo escancarar a porta do quarto dos pais,
gritando bem alto para dentro da escuridão: “Venham dar uma olhada, ele
bateu as botas; está lá, abotoou de vez o paletó!”.

Marido e mulher estavam sentados na cama de casal e tiveram primeiro que
se recuperar do susto com a faxineira, antes de virem a compreender o que
ela anunciava. Mas então, cada qual pelo seu lado, o senhor e a senhora
Samsa saíram depressa da cama, o senhor Samsa jogou a coberta por cima dos
ombros, a senhora Samsa apareceu só de camisola; assim entraram no quarto
de Gregor. Entrementes, fora igualmente aberta a porta da sala, onde Grete
dormia desde a chegada dos inquilinos; ela estava vestida dos pés à
cabeça, como se nem tivesse dormido, seu rosto pálido parecia demonstrar a
mesma coisa. “Morto?”, falou a senhora Samsa e olhou para a faxineira,
esperando uma resposta, embora pudesse confirmar por si só ou até
reconhecer o fato sem confirmação nenhuma. “É o que eu acho”, disse a
faxineira e como prova empurrou com a vassoura o cadáver de Gregor um bom
pedaço para o lado. A senhora Samsa reagiu como se fosse tentar parar a
vassoura, mas não fez nada. “Bom”, falou o senhor Samsa, “agora podemos
agradecer a Deus.” Ele fez o sinal da cruz, e as três mulheres seguiram o
seu exemplo. Grete, que não tirava os olhos do cadáver, disse: “Vejam como
ele estava magro. Já fazia um bom tempo que não comia nada. A comida
voltava do jeito que ia”. De fato o corpo de Gregor estava bem achatado e
seco, e só agora é que reparavam nisso, quando ele não mais se erguia
sobre suas perninhas e também nenhuma outra coisa distraía a atenção.

“Vamos, Grete, entra um pouquinho com a gente”, disse a senhora Samsa com
um sorriso tristonho, e Grete, não sem voltar a olhar o cadáver mais uma
vez, entrou atrás dos pais no quarto do casal. A faxineira fechou a porta
e abriu toda a janela. Embora ainda fosse cedo, uma certa tepidez
misturava-se ao ar fresco da manhã. Já era sem dúvida o final de março.

Os três inquilinos saíram de seu quarto e atônitos procuraram com os olhos
pelo café da manhã; haviam sido esquecidos. “Cadê o café da manhã?”, o
senhor do meio perguntou mal-humorado à faxineira. Esta porém pôs o
indicador na frente da boca e a seguir, em silêncio, fez apressada um
sinal indicando aos inquilinos que por favor passassem ao quarto de
Gregor. Eles passaram e ficaram lá de pé, as mãos nos bolsos de seus
paletós curtos e um tanto gastos, no quarto já totalmente iluminado, em
torno do cadáver.

Foi quando a porta do quarto dos pais se abriu, e o senhor Samsa apareceu
de uniforme, sua esposa apoiada em um braço, sua filha no outro. Todos com
cara de quem tinha chorado um pouco; Grete vez por outra pressionava o
rosto no braço do pai.

“Saiam agora mesmo da minha casa!”, disse o senhor Samsa e apontou para a
porta, sem soltar as mulheres. “O que significa isso?”, disse o senhor do
meio um pouco espantado e com um sorriso amarelo. Os outros dois mantinham
as mãos para trás e esfregavam uma na outra, na alegre expectativa de uma
grande peleja, cujo resultado entretanto deveria ser favorável a eles.
“Significa exatamente o que acabo de dizer”, respondeu o senhor Samsa e
avançou em coluna com suas duas companheiras para cima do inquilino. Este
a princípio ficou parado onde estava e olhou para baixo, como se as coisas
em sua cabeça assumissem uma nova configuração. “Se é assim, nós vamos
embora”, acabou dizendo e ergueu o olhar para o senhor Samsa, como se, num
acesso repentino de humildade, tivesse de pedir permissão até mesmo para
uma decisão dessas. O senhor Samsa, de olhos bem abertos, limitou-se a
acenar brevemente com a cabeça uma e outra vez. O inquilino, com efeito,
em obediência a isso, de imediato se dirigiu a passos largos para a
antessala; seus dois amigos já estavam de sobreaviso há algum tempo, com
as mãos bem quietinhas, e em dois pulos seguiram no seu encalço, como se
temendo que o senhor Samsa pudesse alcançar a antessala antes deles e
perturbar o vínculo que mantinham com o líder. Na antessala, todos os três
pegaram seus chapéus no cabideiro, retiraram suas bengalas do
porta-bengalas, curvaram-se num cumprimento mudo e deixaram o apartamento.
Com uma desconfiança que se revelou infundada, o senhor Samsa saiu com as
duas mulheres no corredor; debruçados no corrimão, ficaram observando como
os três senhores, é certo que devagar, porém de modo ininterrupto, desciam
a comprida escadaria, a cada andar desapareciam em uma determinada curva
da escada e alguns segundos depois voltavam a aparecer; quanto mais para
baixo iam, mais se perdia o interesse da família Samsa por eles, e quando
um empregado do açougue surgiu, cruzou com eles e, todo aprumado e
orgulhoso, continuou subindo com um pacote em cima da cabeça, logo o
senhor Samsa abandonou o corrimão ao lado das mulheres e todos, parecendo
aliviados, regressaram ao apartamento.

Decidiram aproveitar o dia para descansar e passear; não só tinham direito
a essa folga no trabalho, ela também lhes era mais do que necessária. E
assim foram se sentar à mesa e redigir três justificativas, o senhor Samsa
à direção do banco, a senhora Samsa ao dono da loja e Grete ao seu patrão.
Estavam escrevendo quando a faxineira veio dizer que já ia, pois havia
acabado o serviço da manhã. Ocupados com a escrita, os três a princípio
apenas acenaram com a cabeça, sem erguer os olhos, só quando notaram que a
faxineira demorava a sair é que resolveram olhar para ela, contrariados.
“O que foi?”, perguntou o senhor Samsa. A faxineira estava parada na
porta, sorridente, como se tivesse uma notícia muito boa para dar à
família, porém só faria isso se lhe fosse solicitado com alguma
insistência. Fincada quase reta em seu chapéu, a pequena pluma de
avestruz, que tanto irritava o senhor Samsa durante todo o expediente
dela, oscilava de leve em todas as direções. “Mas o que a senhora quer
afinal?”, perguntou a senhora Samsa, por quem a faxineira ainda nutria o
máximo de respeito. “Está bem”, respondeu a faxineira e não conseguiu
falar muito mais por causa da risada amistosa, “o negócio aí do lado, a
senhora não vai precisar se incomodar em pôr para fora. Está tudo
resolvido.” A senhora Samsa e Grete inclinaram-se sobre o papel, dando a
entender que queriam continuar escrevendo; o senhor Samsa, ao perceber que
a faxineira agora queria começar a descrever tudo em detalhes, cortou a
conversa levantando a mão de maneira decidida. Vendo que não tinha licença
para contar, lembrou que estava com pressa, falou bem alto, visivelmente
ofendida: “Passar bem”, deu meia volta enfurecida e saiu do apartamento
debaixo de um assombroso bater de portas.

“Hoje mesmo ela vai ser despedida”, disse o senhor Samsa, sem porém obter
resposta, nem de sua esposa, nem de sua filha, pois a faxineira parecia
ter perturbado o sossego que acabavam de alcançar. Elas se levantaram,
foram até a janela e lá ficaram, abraçadas uma à outra. O senhor Samsa
virou-se para elas sem sair de sua cadeira e ficou observando quieto um
momento. Então reclamou: “Ora, venham até aqui. Deixem em paz as coisas
passadas. E tenham um pouco de consideração por mim”. As mulheres o
obedeceram no ato, voltaram depressa para ele, fizeram-lhe festinhas, e
logo terminaram de escrever suas justificativas.

Então todos os três saíram juntos do apartamento, o que já não faziam há
meses, e seguiram de bonde rumo aos espaços abertos nos arredores da
cidade. O vagão em que se sentaram a sós reluzia com o brilho quente do
sol. Comodamente reclinados em seus assentos, eles avaliaram suas
perspectivas de futuro e acharam que, examinando mais de perto, elas não
seriam assim tão ruins, pois todos os três empregos eram, embora não
tivessem informações muito precisas um do outro a esse respeito,
sobremaneira oportunos e, a longo prazo, bastante promissores. A principal
melhoria em sua situação no momento deveria advir sem muita dificuldade
com a mudança de residência; queriam agora alugar um apartamento menor e
mais barato, porém mais bem localizado e acima de tudo mais prático do que
o atual, que havia sido ainda uma escolha de Gregor. Enquanto conversavam
sobre essas coisas, o senhor e a senhora Samsa, admirados com a vivacidade
crescente da filha, notaram praticamente no mesmo instante como ela nos
últimos tempos, apesar de todo o sofrimento responsável pela palidez em
suas faces, desabrochara e era agora uma moça bonita e cheia de viço. Cada
vez mais calados e entendendo-se quase que por instinto só com a troca de
olhares, pensaram que já era tempo de arranjar um bom marido para ela. E
foi para eles como que uma confirmação de seus novos sonhos e belos
projetos quando, no final da viagem, a filha se levantou primeiro e
espreguiçou seu corpo jovem.

%\movetooddpage
%\addcontentsline{toc}{chapter}{Notas}
%\theendnotes
%!TEX root = LIVRO.tex

\blankpage
\part{Apêndice}
\blankpage

\chapter*{Franz Kafka e o\break mundo invisível\footnotemark}
\addcontentsline{toc}{chapter}{Franz Kafka e o mundo invisível, \emph{por Otto Maria Carpeaux}}
%\markboth{Introdução}{}

\footnotetext{Texto originalmente publicado no Correio da Manhã, em 17 de abril de 1941.}

\begin{flushright}
\textsc{otto maria carpeaux}
\end{flushright}

\noindent{}\textls[15]{O mundo do contista Franz Kafka é uma casa burguesa, solidamente construída na aparência, com uma fachada um pouco descuidada. Entramos, e respiramos o ar das penúrias dolorosas, de quartos mal ventilados. Apodera-se de nós o sentimento
do \textit{déjà vu}, de já ter visto tudo isso. A escada range. O sótão é uma
loja de recordações. Um canto guarda os brinquedos esquecidos. Recordações,
recordações. Os mortos surgem. Os fantasmas que apavoravam a criança.
Figuras de demônios. Um labirinto. Delírio. Fuga. Nenhuma saída.
Voltamo-nos para o outro lado: aparece a face de Deus.}

\textls[-5]{Franz Kafka não é um poeta religioso: não trata nunca de religião nas suas obras.
Mas é um espírito profundamente angustiado; e o seu mundo é cheio de
seres sobrenaturais, donde emana uma impressão inquietante, como o
encontro com uma mitologia desconhecida, que aparecesse, de repente, na
nossa vida quotidiana. Esta irrupção do sobrenatural no mundo não o
salva: enche o homem de terrores desconhecidos. O \textit{numen} de Kafka é um \textit{numen tremendum}. A religião de Kafka não é a religião fácil dos bem-pensantes, a quem o seu Deus garante todas as
ordens deste mundo; o Deus de Kafka faz estremecer os fundamentos do céu
e da terra. ``Minha fé é como uma guilhotina, assim leve e assim
pesada.'' É a ameaça mortal que antecede a esperança vital.}

Esta é a religião daqueles que a psicologia religiosa de William James chama os \textit{twice-born},
aqueles que nascem duas vezes, aqueles cuja fé irrompe das convulsões duma agonia: Agostinho, Martinho Lutero, Blaise Pascal, Søren Kierkegaard.

\textls[15]{Esses terrores e esses esplendores, Kafka os escondeu nos andrajos da vida quotidiana, pois ``quem vir descoberta a face de Deus morrerá''.}

\textls[10]{A pessoa e a vida de Franz Kafka acham-se também cobertas por um véu.
Nasceu em 1883 em Praga, filho de família pequeno-burguesa, dessa
nacionalidade incerta, germano-tcheco-judia, característica dos meios
intelectuais dessa cidade. Desde a sua infância, o humanismo alemão
desses meios é flanqueado pelo cabalismo
judaico e pelo misticismo eslavo.}

\begin{quote}
\begin{verse}
\textit{Estou}\\
\textit{Limitado ao norte pelos sentidos, ao sul pelo medo}\\
\textls[-20]{\textit{A leste pelo apóstolo São Paulo, a oeste pela minha educação.}}\footnote{Excerto de poema de Murilo Mendes. \textsc{{[}n.\,e.{]}}}
\end{verse}
\end{quote}

\textls[15]{A vida corre-lhe nos quadros da burocracia subalterna.
Tísico, morre num sanatório de Viena, em 1924. No testamento ordena a destruição dos seus
manuscritos, que o executor, Max Brod, editará arbitrariamente.}

\textls[5]{A sua obra se compõe: de aforismos, que se alongam às vezes em parábolas; de parábolas, que se estendem às vezes em contos; de contos, dos quais três
se desenvolvem em romances, fragmentários, da mais alta concisão, e
cujo assunto se poderia condensar em parábola ou aforismo. A língua é
muito límpida, carregada de estranhas metáforas. Kafka descreve a vida
quotidiana dos escritórios, dos cafés, das casas de família; mas esses
lugares banais são cheios de potenciais demoníacos,
contra os quais o homem luta desesperadamente. Esse misto de clareza e
de mistério revela a fragilidade do nosso mundo, espreitado pela
catástrofe. Acontecimentos simples revestem-se de uma tensão febril. A língua lúcida faz
adivinhar um outro mundo. As personagens falam, comem, dormem, seguem os
caminhos escuros e estreitos; mas são os caminhos do inferno e do
paraíso, são os caminhos \textit{per realia ad realiora}.\footnote{``Pelas coisas reais ao mais real.''}}

\textls[5]{O primeiro romance publicado depois da morte do autor foi \textit{O processo}. O
seu herói chama-se \textsc{k.}, simplesmente \textsc{k.} Um dia, na rua, \textsc{k.} é subitamente
preso. Explicam-lhe que fora instaurado contra ele um importante
processo criminal; aconselham-no a confessar e, em seguida, soltam-no
a fim de que possa prosseguir na sua defesa. A prisão não passava de uma
provocação por parte daquele estranho tribunal: o próprio \textsc{k.} tem de
criar pelas suas atitudes as razões da sua absolvição ou condenação. E
cria o delito mortal, prevalecendo-se obstinadamente da sua inocência.
Faz tudo o que se pode fazer: contrata um advogado e um médico, corrompe
o carcereiro e o escrivão. Nenhum destes compreende melhor o processo do
que \textsc{k.}, mas todos estão convencidos da justiça e da onipotência do
tribunal; aconselham-no a confessar um crime que \textsc{k.} não conhece e não
quer conhecer. De maneira misteriosa, todos são empregados do tribunal, assim como nós outros
executamos, sem o saber e sem o querer, os desígnios da Providência. Pelas suas
atividades, \textsc{k.} não faz mais que jogar o processo contra si mesmo. Obstina-se.
Pelas suas providências apressa a catástrofe que será a sua condenação e
execução. O delito desconhecido está vingado.}

\textit{O processo} é um apólogo e uma apologia, ao mesmo tempo. Sob o véu da
alegoria, Kafka instrui uma acusação contra a justiça do tribunal
divino. O delito desconhecido é o pecado original. A prisão é o signo da
predestinação. E o que \textsc{k.}
evita pelas suas falsas atividades é a graça. Há nesse romance uma
lembrança incerta de certas palavras do Senhor: ``Muitos serão os
chamados, mas poucos os eleitos'', e ``Aquele que quiser salvar sua vida
a perderá''. Mas as palavras evangélicas perdem-se neste mundo de
provação e desespero, onde a todo momento o
tribunal está presente e a força armada. ``É somente a noção que temos
do tempo'' --- diz Kafka --- ``que nos faz datar o juízo final; na verdade é uma corte
  marcial cuja audiência está aberta todos os nossos dias.'' Mas o céu
  negro se iluminará, um dia, sobre estas cenas de horror. No seu diário,
  Kafka copiou as palavras de Lutero: 

\begin{quote}
  Deus não é inimigo dos
  pecadores, mas somente dos descrentes que não reconhecem os próprios
  pecados nem procuram o apoio de Cristo, mas que procuram,
  temerariamente, a purificação em si mesmos''.
\end{quote}

Em torno deste romance, alguns contos explicam a situação metafísica do homem. \textit{A colônia penitenciária} é uma como espécie de continuação de \textit{O
processo}. Nesta colônia, uma terrível máquina de precisão grava no corpo
dos forçados, por meio de agulhas incandescentes, os nomes dos delitos,
que são desconhecidos dos próprios condenados. A tortura pela qual a sua culpa lhes será revelada
é a única esperança, pois saber o nome do delito é a condição preliminar para
saber justificar-se.

\textls[15]{Em \textit{A transformação}, um jovem é subitamente transformado num horrível inseto que os seus próprios parentes querem matar. O homem, submergido
pela vida banal de todos os dias, não é mais a imagem de Deus; não se
pode deter essa queda onde se desejaria, em alguma etapa propícia; e a queda torna-se radical até se perder o direito de existir.}

A transformação tornou-se definitiva nesta pequena obra-prima chamada \textit{A preocupação do Pai Celeste}. É objeto da inquietação do Pai
misericordioso uma bobina, destituída de fios; coisa absolutamente inútil, sem nenhuma significação, mas
que não descansa nunca, que sobe e desce incessantemente a escada, até o
último dia. ``--- Como te chamas?'', ``--- Odradek''; palavra eslava, de
origem incerta, que significa ``apóstata''.

\textls[-10]{Em todas essas parábolas,
como em \textit{O processo}, o homem é a vítima passiva da perseguição celeste,
lembrando \textit{Hound of Heaven}, de Francis Thompson. Mas Kafka não condena a
atividade: }

\begin{quote}
Há dois pecados cardeais donde se poderiam deduzir todos
os outros: a impaciência e a preguiça. Por causa da impaciência foram
expulsos do paraíso; por causa da preguiça lá não podem voltar. 
\end{quote}

\textls[-10]{O que Kafka deseja excluir é a falsa direção das nossas atividades, no sentido
da segurança neste mundo. No conto \textit{A toca de texugo}, o animal, temendo a perseguição dos
cães, decide alargar e fortificar o seu edifício subterrâneo. Cava
buracos sobre buracos, corredores sobre corredores, até que afinal esquece a única saída. Então o animal agacha-se no seu canto,
aprisionado e sem saída, e espera, indefinidamente, numa estranha solidão, atento aos ruídos funestos do mundo exterior, ou ao silêncio,
ainda mais terrível.}

A falsa direção das atividades humanas é o assunto
da obra-prima de Kafka: o romance inacabado \textit{O castelo}.

\textls[-5]{Ainda aqui o herói chama-se \textsc{k.}, simplesmente \textsc{k.} O seu adversário não é, desta vez, o
tribunal, mas o castelo, o lugar onde a graça está concentrada. Ao pé
desse castelo há uma aldeia, onde os camponeses, crentes humildemente submissos, executam as suas tarefas diárias. \textsc{k.} também desejaria ser camponês nessa
aldeia. É preciso frisar: ele o quer, ele o exige mesmo. Desejaria obrigar o castelo a conceder-lhe o direito de permanência na aldeia. Quer forçar esta comunhão dos fiéis, sem ter obtido a graça.}

Numa fria tarde de inverno, \textsc{k.} chega, contando
com a piedade, que não fará voltar o peregrino. Com efeito, o hospedeiro
acolhe-o. \textsc{k.} é modesto; quer somente achar um emprego de diarista.
Sim, há sempre possibilidades. Nesse ínterim o filho do castelão aparece
para expulsá-lo. \textsc{k.} desesperadamente recorre à mentira: ``O castelo contratou-me como nivelador''. Resolvem telefonar para o castelo. E o
castelo responde de maneira surpreendente (``\textsc{k.} estremeceu um pouco''): ``Sim, \textsc{k.} é o nivelador contratado''. É o primeiro dom voluntário da graça: mas
contém uma punição. Pois o castelo acrescenta: 

\begin{quote}
\textls[10]{\textsc{k.} tem, portanto,
permissão para ficar; mas o seu contrato foi um lamentável engano, aqui
não temos trabalho para um nivelador. \textsc{k.} tem permissão para ficar, mas
não para trabalhar.}
\end{quote}

\textls[10]{Deste modo, \textsc{k.} encontra-se impossibilitado de
verificar o contrato surrupiado, justificar sua presença na aldeia. Sua
vida será vazia, destituída de qualquer sentido, como a nossa vida
quotidiana sem a vocação interior. \textsc{k.} não está contente. Não quer ser
tolerado. Quer o direito de permanecer, o direito. Quer extorquir a
graça. Recorre a meios impuros, perde-se em mentiras e subterfúgios.
Tudo em vão. Esgotado, enfim, cai gravemente doente. Espera a morte.}

\textls[10]{Eis-nos nas últimas linhas do fragmento. Uma anotação explica-nos o fim:}

\begin{quote}
\textls[15]{Quando \textsc{k.} está à morte, chega a decisão definitiva do castelo: \textsc{k.} não tem nenhum direito de permanecer na aldeia; mas considerando-se certas circunstâncias acessórias, ser-lhe-á permitido que aí permaneça até a morte.}
\end{quote}

\textls[5]{Em \textit{O processo}, o Céu instaura processo contra o homem. Em \textit{O castelo}, o homem instaura processo contra o Céu. É o cúmulo da
temeridade titânica. ``Uns negam a miséria evocando o sol; outros negam
o sol evocando a miséria.'' O homem, em Kafka, não vê na sua miséria a
consequência da sua condição humana. Revolta-se. Acusa Deus, como Ivan Karamázov. A face de Deus, em sua obra, adquire traços blasfemos.}

Em toda parte, no mundo desse Deus, há tribunais e forcas. Não parece que
esse Deus queira a redenção do homem. O verdadeiro caminho desdobra-se
sobre uma corda, lançada muito perto do chão; parece ser destinada
mais a fazer tropeçar que a ser transposta.'' Às vezes Kafka atinge uma
inversão diabólica: 

\begin{quote}
Leopardos forçavam o templo e esvaziavam os vasos
sagrados. Isto se repetia frequentemente. Até que conseguiram calcular a
hora em que chegavam e faziam do incidente uma parte do cerimonial.
\end{quote}

\textls[-5]{Tais blasfêmias lembram a zoolatria dos egípcios ou o Demiurgo mau dos
gnósticos. Mas um outro aforismo diz: ``O nosso mundo não é mais do que
um mau humor de Deus. Há esperança, muita esperança, mas não para nós
homens''. Este ``não para nós homens'' equivale a uma grande confissão,
que restabelece a ordem dos valores. ``Todas essas parábolas dizem somente que o incompreensível é incompreensível.'' Na aparência dessas
parábolas Deus não tem razão; mas esta falta de razão significa somente
uma incapacidade do homem em face do mandamento de Deus. Na aparência
dessas parábolas, Deus se cala; mas isto significa somente que o mundo
não o está escutando. Há, portanto, esperança, muita esperança. No fim
de \textit{O castelo}, a graça aparece. Fato simbólico: Kafka não estava
destinado a escrever esse fim.}

\textls[10]{Franz Kafka, segundo uma frase de
Kierkegaard, ``aspirava a uma imortalidade mais alta que a da glória''.
Kafka desejava que a sua obra morresse com ele para servir de testemunha
em seu favor, perante o tribunal de Deus. A despeito dele, o seu dia chegará, se já não chegou.}

À propagação dessa obra opõem-se obstáculos do destino. A sua publicação póstuma não encontrou nem leitores nem críticos. Dez anos depois da sua morte,
um André Gide, um Charles Du Bos deploram a inacessibilidade das obras,
a inexistência de traduções. Uma casa editora de Praga promete a publicação das obras completas, a \textit{Nouvelle Revue Française} traduz alguns contos. A
edição de Praga é interrompida pela derrota do Estado tcheco. A tradução
integral, prometida na França, talvez nunca apareça. A despeito de tudo,
o seu dia chegará, se já não chegou.

\textls[10]{Todos esses obstáculos aprofundam
mais a virtude desse pensamento, em vez
de sufocá-lo. Existe uma herança que se deve conservar. A reflexão sobre o lugar de Kafka na literatura universal é o primeiro dever.}

\textls[10]{Feita a abstração de alguns pontos de contato com Heinrich von Kleist, o Kleist
do ensaio \textit{Sobre o teatro de bonecas}, e com \textsc{e.\,t.\,a.}\,Hoffmann, a presença
de Kafka na literatura alemã é simplesmente ocasional. O seu lugar está
na literatura europeia do pós-guerra.}

\textls[-5]{O simbolismo de Kafka perturba
o mundo, pela estranha transposição dos acentos, pela desvalorização dos fatos tradicionais, pela revelação de um mundo mais real atrás do mundo real dos bem-pensantes: \textit{per realia ad realiora}. Eis o lema de Anton Tchecov, a quem Kafka deve a técnica do conto. Mas
um traço significativo distingue Kafka radicalmente deste grande
contista pessimista do \textit{fin de siècle}: a noção do tempo. Os homens de
Tchecov vivem no seu tempo, no tempo do seu
mundo. Mas o tempo, em Kafka, é um fato extramundano. Não é o tempo
psicológico de Proust. É antes um tempo religioso: o caminho da aldeia ao castelo, ``dois quilômetros mais ou menos'', leva séculos para ser
percorrido; não se pode dizer a respeito de nenhuma obra de Kafka em que
século decorre a ação dela. A era dos deuses e a vida quotidiana dos
nossos dias se confundem. Não existe tempo, há unicamente uma data: a da
irrupção do divino no mundo, acontecimento que se repete todos os dias, todas as horas.}
% Paulo: não encontrei a palavra ``eônio''; deve ter havido algum erro aqui
% Suzana: Acabei por retirar esta palavra daqui: ``eônios''

Esta ausência do tempo humano destrói a estrutura
normal do mundo e isola os
homens em desertos de eternidade glacial, tomando-os comparáveis às
personagens plásticas de um De Chirico, aos cantos ``homófonos'' de um
Stravinsky, aos anjos de um Rilke. A psicologia desses homens é uma
psicologia de monstros revoltados, como nos romances fantásticos de
Julien Green. A sua vida quotidiana é destituída de sentido, como nos
contos de um Bontempelli. E a sua vida real se
passa na atmosfera mágica dos romances de Marcel Jouhandeau. Enfim, este
mundo acha a sua expressão final nos poemas apocalípticos dum
Pierre-Jean Jouve que precedem a catástrofe. O dia de Kafka chegou.

Todas essas comparações só têm como fim estabelecer mais solidamente as oposições. A corrente literária do pós-guerra acha-se diante de um montão de 
ruínas. O mundo é um cadáver que se decompõe porque o espírito abandonou
o corpo. A literatura e o pensamento modernos tentaram contentar-se
somente com os destroços, olhando-os primeiro como brinquedos de uma
nova infância, e em seguida como pedras para a construção do futuro;
eram as etapas do primitivismo e do construtivismo. Mas se reconhecerá o
verdadeiro estado de coisas e um profundo desespero prevalecerá. Este
desespero se conformará ou não se conformará: ele afirma e confirma a
decomposição do mundo por meio de uma nova psicologia, ou se insurge
contra essa \mbox{decomposição} pelas expressões de um pessimismo cínico. São
estas as posições do romance e da poesia modernos.

O que é comum a todas
essa correntes é o relativismo, que já não admite a integridade do
mundo, exceto a daqueles, não raros, que mergulham na fé tradicional.
A atitude de Franz Kafka é muito diferente. Não se contenta com os
destroços, como os ``fragmentistas'' italianos; não se conforma nem
decompõe. Não é nem tradicional nem relativista. Entre dois mundos e
entre duas épocas, coloca-se em caminho; está a caminho de Damasco.

Esta atitude o situa no meio de duas grandes correntes dos nossos tempos: uma
na França, os novos estudos pascalianos que giram em torno do problema da
graça e inspiram até o André Gide de \textit{L'École des femmes}; a outra na
Alemanha, a teologia dialética de Karl Barth e de Emil Brunner,
que gira em torno do abismo dialético, a incomensurabilidade entre Deus
e o mundo, e faz ressuscitar a obra esquecida de Søren Kierkegaard.

\textls[15]{No abismo entre o Deus soberano dos dialéticos e o homem falido de Pascal, Kafka procura o lugar da graça. É Pascal quem define a situação. No
artigo \textsc{xv} das \textit{Pensées}, enumera as quatro possibilidades do homem. Primeiro, o homem
conhece
a Deus, mas não conhece a sua própria miséria; é o caso do farisaísmo
orgulhoso. Segundo, o homem conhece a sua miséria, mas não conhece a
Deus; é o desespero ateístico. Terceiro, o homem conhece a Deus e a
sua própria miséria, mas não a graça; é a angústia. Quarto, o homem
reconhece em Jesus Cristo seu Deus, sua miséria e sua graça.}

\textls[5]{A posição de Kafka é a terceira. É a posição do judaísmo perante o seu Messias
encarnado. Mas é também a posição atual do mundo apóstata, que renuncia
à graça e se declara pagão, cheio de orgulho e de angústia. Não se é
mais pagão depois de Jesus Cristo: a velha inocência desapareceu; ou
procuramo-Lo, ou renegamo-Lo. Em vão ``a angústia da lei'' maltrata o
rabino Saul antes de ter ele visto a luz do mundo. Uma fé vem nascer no
caos de uma alma em desespero. }

\begin{quote}
\textls[15]{Como cumprir a vontade de Deus? Teme-se
que essa lei não seja mais do que uma tentação. E se o seu cumprimento
não representar nada perante Deus? }
\end{quote}

É um aforismo de Kafka.
Mas o apóstolo Paulo poderia ter dito isso. É a confissão de um homem no caminho de Damasco.

O caminho de Damasco é a única saída desta prisão que é o nosso mundo envenenado. Todos os outros caminhos são subterfúgios inúteis, tergiversações que nos abismam cada vez mais, sem a possibilidade de uma libertação.
Sem a graça não se escapa deste mundo. Todas as seguranças exteriores são vãs.
Em vão nos entrincheiramos nas linhas Maginot da nossa ``toca de texugo''.
Enfim, somos os prisioneiros das nossas próprias prisões, para assistir, impotentes, à
nossa derrota decisiva. Só o caminho misterioso de Damasco é que liberta dos terrores exteriores,
para preparar ``o segundo nascimento'': é o caminho do apocalipse do
mundo para a escatologia da alma.

A obra de Franz Kafka é um indicador
na direção desse caminho. Nela se lê o
seu aforismo, cheio de aflição e de esperança: ``Quem procurar não
encontrará; quem não procurar, será encontrado''. E uma voz lhe responde, através de Pascal:
\textit{Console-toi, tu ne me chercherais pas si tu ne m'avais trouvé}.\footnote{``Consola-te: não me procurarias se já não me tivesses encontrado.''}


\chapterspecial{Franz Kafka\footnotemark}{A propósito do décimo aniversário\break de sua morte}{Walter Benjamin}

\footnotetext{Texto originalmente publicado em 1934. Tradução de Sergio Paulo Rouanet.}

\subsection{Potemkin}

\textls[-5]{Conta-se que Potemkin sofria de graves depressões, que se repetiam mais
ou menos regularmente, e durante essas crises ninguém podia aproximar-se
dele, sendo o acesso a seu quarto rigorosamente proibido. Esse fato não
era mencionado na corte, pois se sabia que a menor alusão a respeito
acarretava o desagrado da imperatriz Catarina. Uma dessas depressões do
chanceler teve uma duração excepcional, o que ocasionou sérios
embaraços. Os papéis se acumulavam, e os assuntos, cuja solução era
reclamada pela czarina, não podiam ser resolvidos sem a assinatura de
Potemkin. Os altos funcionários estavam perplexos. Por acaso, entrou na
antecâmara da Chancelaria um amanuense subalterno, Chuvalkin, que viu os
conselheiros reunidos, queixando-se como de hábito. ``O que se passa,
Excelências? Em que posso servir Vossas Excelências?'' perguntou o
zeloso funcionário. Explicaram-lhe o caso, lamentando que não pudessem
fazer uso dos seus serviços. ``Se é só isso, meus senhores'', respondeu
Chuvalkin, ``deem-me os papéis, por favor.'' Os conselheiros, que não
tinham nada a perder, concordaram, e Chuvalkin, carregando pilhas de
documentos, atravessou galerias e corredores em direção ao quarto de
Potemkin. Sem bater, girou imediatamente a maçaneta. O quarto não estava
fechado. Potemkin estava em seu leito na penumbra, roendo as unhas, num
velho roupão. Chuvalkin foi até a escrivaninha, mergulhou a pena na
tinta e sem uma palavra colocou-a na mão de Potemkin, pondo o primeiro
documento nos joelhos do chanceler. Depois de um olhar ausente sobre o
intruso, Potemkin assinou como um sonâmbulo o primeiro papel, depois o
segundo, e finalmente todos. Depois que o último papel estava assinado,
Chuvalkin deixou o quarto com a mesma sem-cerimônia com que entrara, os
maços debaixo do braço. Brandindo-os no ar, entrou triunfante na
antecâmara. Os conselheiros se precipitaram sobre ele, arrancando-lhe os
papéis das mãos. Com a respiração suspensa, inclinaram-se sobre os
documentos. Ninguém disse uma palavra; o grupo estava petrificado. Mais
uma vez Chuvalkin acorreu, indagando por que os conselheiros estavam tão
estupefatos. Nesse momento, leu as assinaturas. Todos os papéis estavam
assinados: \textit{Chuvalkin, Chuvalkin, Chuvalkin}\ldots{}}

\textls[-5]{Com dois séculos de antecipação, essa anedota anuncia a obra de Kafka. O
enigma que ela contém é o de Kafka. O mundo das chancelarias e dos
arquivos, das salas mofadas, escuras, decadentes, é o mundo de Kafka. O
zeloso Chuvalkin, para quem tudo parece tão fácil e que acaba voltando
de mãos vazias é \textsc{k.}, de Kafka. Potemkin, semiadormecido e abandonado
num quarto distante cujo acesso é proibido, vegetando na penumbra, é um
antepassado daqueles seres todo-poderosos, que Kafka instala em sótãos,
na qualidade de juízes, ou em castelos, na qualidade de secretários, e
que por mais elevada que seja sua posição, têm sempre as características
de quem afundou, ou está afundando, mas que ao mesmo tempo podem surgir,
em toda a plenitude do seu poder, nas pessoas mais subalternas e
degradadas --- os porteiros e os empregados decrépitos. Em que pensam,
mergulhados na semiescuridão? São talvez os descendentes de Atlas, que
sustentam o globo sobre seus ombros? É por isso, talvez, que sua cabeça
``está tão inclinada no peito, que seus olhos mal podem ser vistos'',
como o castelão em seu retrato, ou Klamm, quando está sozinho? Não, não
é o globo terrestre que eles sustentam, pois o cotidiano já é
suficientemente pesado: ``Seu cansaço é o do gladiador depois do
combate, seu trabalho consistia em pintar o canto de uma sala de
funcionário''. Georg Lukács disse uma vez: para construir hoje uma mesa
decente, é preciso dispor do gênio arquitetônico de um Miguel Ângelo.
Lukács pensa em períodos históricos, Kafka em períodos cósmicos. Caiando
um pedaço de parede, o homem precisa pôr em movimento períodos cósmicos.
E isso nos gestos mais insignificantes. De muitas maneiras, e às vezes
nas ocasiões mais estranhas, os personagens batem suas mãos. Kafka disse
uma vez, casualmente, que essas mãos eram ``verdadeiros pilões a
vapor''.}

Travamos conhecimento com esses poderosos, em seu movimento contínuo e
lento, ascendente ou descendente. Mas eles não são nunca mais terríveis
que quando se levantam da mais profunda degradação, como pais. O filho
tranquiliza o velho pai, senil, depois de o ter posto na cama, com toda
ternura: 

\begin{quote}
--- Fica tranquilo, estás bem coberto.

\noindent{}--- Não, gritou o pai (afastando o lençol com tanta força, que ele se desdobrou no ar por um instante), e ergueu-se no leito, tocando o teto de leve com uma das
mãos. Tu querias me cobrir, bem o sei, mas ainda não estou coberto. Essa
é a última força que me resta, mas ela é suficiente para ti, excessiva
para ti\ldots{} Felizmente um pai não precisa aprender a desmascarar seu
filho\ldots{} Ele ficou de pé, perfeitamente livre, movendo as pernas. Seu
rosto irradiava inteligência\ldots{} 

\noindent{}--- Sabes agora o que existia fora de
ti, ao passo que até hoje sabias apenas o que te dizia respeito! É
verdade que eras uma criança inocente, mas a verdade mais profunda é que
eras um ser diabólico!
\end{quote}

\textls[-5]{Ao repelir o fardo das cobertas, o pai repele
com elas o fardo do mundo. Precisa pôr em movimento períodos cósmicos
inteiros, para tornar viva e rica de consequências a imemorial relação
entre pai e filho. Mas que consequências! Ele condena o filho a morrer
por afogamento. O pai é a figura que pune. A culpa o atrai, como atrai
os funcionários da Justiça. Há muitos indícios de que o mundo dos
funcionários e o mundo dos pais são idênticos para Kafka. Essa
semelhança não os honra. Ela é feita de estupidez, degradação e
imundície. O uniforme do pai é cheio de nódoas, sua roupa de baixo é
suja. A imundície é o elemento vital do funcionário. Ela não
compreendia por que as partes se movimentavam tanto. Para sujar a
escada, respondeu um funcionário, talvez por raiva, mas a resposta foi
para ela esclarecedora.'' A imundície é de tal modo um atributo dos
funcionários que eles podem ser vistos como gigantescos parasitas. Isso
não se refere, naturalmente, às relações econômicas, mas às forças da
razão e da humanidade, que permitem a esses indivíduos sobreviver. Do
mesmo modo, nas estranhas famílias de Kafka, o pai sobrevive às custas
do filho, sugando-o como um imenso parasita. Não consome apenas suas
forças, consome também seu direito de existir. O pai é quem pune, mas
também quem acusa. O pecado do qual ele acusa o filho parece ser uma
espécie de pecado original. A definição kafkiana do pecado original é
particularmente aplicável ao filho: ``O pecado original, o velho delito
cometido pelo homem, consiste na sua queixa incessante de que ele é
vítima de uma injustiça, de que foi contra ele que o pecado original foi
cometido''. Mas quem é acusado desse pecado original, hereditário --- o
pecado de haver engendrado um herdeiro --- senão o pai, pelo filho?
Assim, o pecador seria o filho. Porém não se pode concluir da frase de
Kafka que a acusação é pecaminosa, porque falsa. Em nenhum lugar Kafka
diz que essa acusação é injusta. Trata-se de um processo sempre
pendente, e nenhuma causa é mais suspeita que aquela para a qual o pai
pretende obter a solidariedade desses funcionários e empregados da
Justiça. O pior, neles, não é sua infinita corruptibilidade. Pois em seu
íntimo são feitos de tal maneira que sua venalidade é a única esperança
que a humanidade possa alimentar a seu respeito. É certo que os
tribunais dispõem de códigos. Mas eles não podem ser vistos. ``Faz parte
da natureza desse sistema judicial condenar não apenas réus inocentes,
mas também réus ignorantes'', presume Kafka. No mundo primitivo, as leis
e normas são não escritas. O homem pode transgredi-las sem o saber.
Contudo, por mais dolorosamente que elas afetem o homem que não tem
consciência de qualquer transgressão, sua intervenção, no sentido
jurídico, não é acaso, mas destino, em toda a sua ambiguidade. Segundo
Hermann Cohen, numa rápida análise da antiga concepção do destino, uma
ideia se impunha inelutavelmente: ``são os próprios decretos do destino
que parecem facilitar e ocasionar essa transgressão e essa queda''. O
mesmo ocorre com a instância que submete Kafka à sua jurisdição. Ela
remete a uma época anterior à lei das doze tábuas, a um mundo primitivo,
contra o qual a instituição do direito escrito representou uma das
primeiras vitórias. É certo que na obra de Kafka o direito escrito
existe nos códigos, mas eles são secretos, e através deles a
pré-história exerce seu domínio ainda mais ilimitadamente.}

\textls[-10]{Entre a administração e a família, Kafka vê contatos múltiplos. Na
 aldeia de Schlossberg havia para isso uma expressão eloquente. ``Temos
 aqui uma expressão que você talvez conheça: as decisões administrativas
 são tão tímidas quanto as moças. Bem observado, disse Kafka, bem
 observado, e as decisões podem ter outras coisas em comum com as
 moças.'' Entre essas qualidades comuns talvez a mais notável fosse a de
 se prestar a tudo, como as jovens tímidas que Kafka encontra, em \textit{O
 castelo} e em \textit{O processo} e que se revelam tão devassas no seio da
 família como num leito. Ele as encontra a todo instante em seu caminho;
 elas se oferecem com tão pouca cerimônia como a moça do albergue.} 

\begin{quote}
\textls[-10]{Eles
se enlaçaram, o pequeno corpo ardia entre as mãos de Kafka, eles rolavam
em um frenesi do qual Kafka tentava salvar-se continuamente, mas em vão;
alguns passos adiante, os dois bateram surdamente na porta de Klamm e se
deitaram em seguida sobre as poças de cerveja e outras imundícies que
cobriam o chão. Ali permaneceram horas\ldots{} nas quais Kafka experimentou
constantemente a sensação de estar perdido, ou de estar num país
estrangeiro, como nenhum outro homem havia estado antes, tão estrangeiro
que mesmo o ar nada tinha em comum com o ar nativo, um ar asfixiante,
mas cujas loucas seduções eram tão irresistíveis que não havia remédio
senão ir mais longe, perder-se mais ainda.}
\end{quote}

Voltaremos a essa terra
estrangeira. É digno de nota, contudo, que essas mulheres que se
comportam como prostitutas não são jamais belas. A beleza só aparece no
mundo de Kafka nos lugares mais obscuros: entre os acusados, por
exemplo. 

\begin{quote}
\textls[10]{É um fenômeno notável, de certo modo científico\ldots{} Não pode
ser a culpa que os faz belos\ldots{} não pode ser também o castigo justo que
desde já os embeleza\ldots{} só pode ser o processo movido contra eles, que
de algum modo adere a seu corpo.}
\end{quote}

Depreendemos de \textit{O processo} que esse procedimento judicial não deixa
via de regra nenhuma esperança aos acusados, mesmo quando subsiste a
esperança da absolvição. É talvez essa desesperança que faz com que os
acusados sejam os únicos personagens belos na galeria kafkiana. Essa
hipótese estaria de acordo com um fragmento de diálogo, narrado por Max
Brod.

\begin{quote}
Recordo-me de uma conversa com Kafka, cujo ponto de partida foi
a Europa contemporânea e a decadência da humanidade. Somos, disse ele,
pensamentos niilistas, pensamentos suicidas, que surgem na cabeça de
Deus. Essa frase evocou em mim a princípio a visão gnóstica do mundo:
Deus como um demiurgo perverso, e o mundo como seu pecado original. Oh
não, disse ele, nosso mundo é apenas um mau humor de Deus, um dos seus
maus dias. Existiria então esperança, fora desse mundo de aparências que
conhecemos? Ele riu: há esperança suficiente, esperança infinita --- mas
não para nós. 
\end{quote}

\textls[10]{Essas palavras estabelecem um vínculo com aqueles
singulares personagens de Kafka, os únicos que fugiram do meio familiar
e para os quais talvez ainda exista esperança. Esses personagens não são
os animais, e nem sequer os seres híbridos ou imaginários, como o
Gato-carneiro ou Odradek, pois todos eles vivem ainda no círculo da
família. Não é por acaso que é exatamente na casa dos seus pais que
Gregor Samsa se transforma em inseto, não é por acaso que o estranho
animal, meio gato, meio carneiro, é um legado paterno, não é por acaso
que Odradek é a grande preocupação do pai de família. Mas os
``ajudantes'' conseguem escapar a esse círculo.}

\textls[5]{Eles pertencem a um grupo de personagens que atravessam toda a obra de
Kafka. Dele fazem parte o vigarista desmascarado em \textit{Betrachtung}
(em português, ``Meditação''), assim como o estudante, que aparece à noite no balcão
como vizinho de Karl Rossmann, e os loucos que moram na cidade do sul e
que não se cansam nunca. A penumbra em que transcorre sua vida lembra a
iluminação trêmula em que aparecem os personagens das pequenas peças de
Robert Walser, autor do romance \textit{Der Gehülfe} (em português, ``O ajudante''), admirado
por Kafka. As lendas indianas conhecem os \textit{Gandharwe}, criaturas
inacabadas, ainda em estado de névoa. É dessa natureza que são feitos os
``ajudantes'' de Kafka: não pertencem a nenhum dos outros grupos de
personagens e não são estranhos a nenhum deles --- são mensageiros que
circulam entre todos. Como diz Kafka, assemelham-se a Barnabás, também
um mensageiro. Ainda não abandonaram de todo o seio materno da natureza
e, por isso, }

\begin{quote}
\textls[10]{instalaram-se num canto do chão, sobre dois velhos
vestidos de mulher. Sua ambição\ldots{} era ocupar um mínimo de espaço, e
para isso, sempre sussurrando e rindo, faziam várias experiências,
cruzavam seus braços e pernas, acocoravam-se uns ao lado dos outros e na
penumbra pareciam um grande novelo. }
\end{quote}

\textls[10]{Para eles e seus semelhantes, os
inábeis e os inacabados, ainda existe esperança.}

\textls[-8]{A mesma norma de comportamento que nesses mensageiros é suave e flexível
transforma-se em lei opressiva e sombria no restante da galeria
kafkiana. Nenhuma de suas criaturas tem um lugar fixo, um contorno fixo
e próprio, não há nenhuma que não esteja ou subindo ou descendo, nenhuma
que não seja intercambiável com um vizinho ou um inimigo, nenhuma que
não tenha consumido o tempo à sua disposição, permanecendo imatura,
nenhuma que não esteja profundamente esgotada, e ao mesmo tempo no
início de uma longa jornada. Impossível falar aqui de ordens e
hierarquias. O mundo mítico, à primeira vista próximo do universo
kafkiano, é incomparavelmente mais jovem que o mundo de Kafka, com
relação ao qual o mito já representa uma promessa de libertação. Uma
coisa é certa: Kafka não cedeu à sedução do mito. Novo Odisseus,
livrou-se dessa sedução graças ``ao olhar dirigido a um horizonte
distante'' \ldots{} ``as sereias desapareceram literalmente diante de tamanha
firmeza, e, no momento em que estava mais próximo delas, não as percebia
mais''. Entre os ancestrais de Kafka no mundo antigo, os judeus e os
chineses, que reencontraremos mais tarde, esse antepassado grego não
deve ser esquecido. Pois Odisseus está na fronteira do mito e do conto
de fadas. A razão e a astúcia introduziram estratagemas no mito; por
isso, os poderes míticos deixaram de ser invencíveis. O conto é a
tradição que narra a vitória sobre esses poderes. Kafka escreveu contos
para os espíritos dialéticos quando se propôs narrar sagas. Introduziu
pequenos truques nesses contos e deles extraiu a prova de que ``mesmo os
meios insuficientes e até infantis podem ser úteis para a salvação''. É
com essas palavras que ele inicia sua narrativa sobre \textit{O silêncio das
sereias}. Pois em Kafka as sereias silenciam; elas dispõem de ``uma arma
ainda mais terrível que o seu canto\ldots{} o seu silêncio''. Elas utilizaram
essa arma contra Odisseus. Mas ele, informa-nos Kafka,}

\begin{quote}
\textls[20]{era tão astuto,
uma raposa tão fina, que nem sequer a deusa do destino conseguiu
devassar seu interior. Embora isso seja incompreensível para a
inteligência humana, talvez ele tenha de fato percebido que as sereias
estavam silenciosas, usando contra elas e contra os deuses o estratagema
que nos foi transmitido pela tradição apenas como uma espécie de
escudo.}
\end{quote}

\textls[-10]{Em Kafka as sereias silenciam. Talvez porque a música e o canto são para
ele uma expressão ou pelo menos um símbolo da fuga. Um símbolo da
esperança que nos vem daquele pequeno mundo intermediário, ao mesmo
tempo inacabado e cotidiano, ao mesmo tempo consolador e absurdo, no
qual vivem os ajudantes. Kafka é como o rapaz que saiu de casa para
aprender a ter medo. Ele chegou ao palácio de Potemkin, mas acabou
encontrando em seu porão Josefine, aquela ratinha cantora, que ele
descreve assim: }

\begin{quote}
\textls[20]{existe nela algo de uma infância breve e pobre, algo
de uma felicidade perdida e irrecuperável, mas também algo da vida ativa
de hoje, com suas pequenas alegrias, incompreensíveis, mas reais, e que
ninguém pode extinguir.}
\end{quote}

\subsection{Uma fotografia de criança}

\textls[-5]{Existe uma foto infantil de Kafka. Poucas vezes ``a pobre e breve
infância'' concretizou-se em imagem tão evocativa. A foto foi tirada num
desses ateliês do século \textsc{xix}, que, com seus cortinados e palmeiras,
tapeçarias e cavaletes, parecia um híbrido ambíguo de câmara de torturas
e sala do trono. O menino de cerca de seis anos é representado numa
espécie de paisagem de jardim de inverno, vestido com uma roupa de
criança, muito apertada, quase humilhante, sobrecarregada com rendas. No
fundo, erguem-se palmeiras imóveis. E, como para tornar esse acolchoado
ambiente tropical ainda mais abafado e sufocante, o modelo segura na mão
esquerda um chapéu extraordinariamente grande, com largas abas, do tipo
usado pelos espanhóis. Seus olhos incomensuravelmente tristes dominam
essa paisagem feita sob medida para eles, e a concha de uma grande
orelha escuta tudo o que se diz.}

\textls[-10]{Essa tristeza profunda foi talvez um dia compensada pelo fervoroso
desejo de ``ser índio''. ``Como seria bom ser um índio, sempre pronto, a
galope, inclinado na sela, trepidante no ar, sobre o chão que trepida,
abandonando as esporas, porque não há esporas, jogando fora as rédeas,
porque não há rédeas, vendo os prados na frente, com a vegetação rala,
já sem o pescoço do cavalo, já sem a cabeça do cavalo.'' Esse desejo tem
um conteúdo muito rico. Ele revela seu segredo no momento em que se
realiza: na América. A importância de \textit{Amérika} na obra de Kafka é
demonstrada pelo próprio nome do herói. Enquanto nos primeiros romances
o autor se designava apenas, em surdina, por uma inicial, nesse livro
ele nasce de novo, no novo mundo, com seu nome completo. Experimenta
esse renascimento no teatro ao ar livre de Oklahoma.}

\begin{quote}
Karl viu numa
esquina um cartaz com os seguintes dizeres: \textit{Na pista de corridas de
Clayton contratam-se, das seis da manhã de hoje até a meia-noite,
pessoas para o teatro de Oklahoma. O grande teatro de Oklahoma te chama!
Só hoje, só uma vez! Quem perder a ocasião hoje, a perderá para sempre!
Quem pensa em seu futuro, nos pertence! Todos são bem-vindos! Quem
quiser ser artista, que se apresente! Nosso teatro que pode utilizar
todos, cada um em seu lugar! Quem se decidir por nós, merece ser
felicitado! Mas apressem-se, para serem admitidos antes da meia-noite! À
meia-noite tudo estará fechado e não reabrirá mais! Maldito seja aquele
que não acredita em nós! Para Clayton!} 
\end{quote}

O leitor dessas palavras é
Karl Rossmann, a terceira encarnação, a mais feliz de todas, de Kafka, o
herói dos romances de Kafka. A felicidade está à sua espera no teatro ao
ar livre de Oklahoma, uma verdadeira pista de corridas, do mesmo modo
que a infelicidade o tinha encontrado no estreito tapete de seu quarto,
quando ele ali entrara ``como numa pista de corridas''. Desde que Kafka
escrevera suas \textit{Reflexões para os cavaleiros}, desde que descreveu o
``novo advogado'' ``levantando até o alto as coxas e com um passo que
faz ressoar o mármore a seus pés, subindo os degraus do Foro'', e desde
que mostrou ``as crianças na estrada'' trotando pelos campos com grandes
saltos, os braços cruzados, essa figura se tornara familiar para ele.
Com efeito, às vezes ocorre que Karl Rossmann, ``distraído por falta de
sono, perca seu tempo dando pulos inutilmente altos''. Por isso, é
somente numa pista de corridas que ele pode chegar ao objeto dos seus
desejos.

Essa pista é ao mesmo tempo um teatro, e isso constitui um enigma. Mas o
lugar enigmático e a figura inteiramente transparente e não enigmática
de Karl Rossmann pertencem ao mesmo contexto. Pois, se Karl Rossmann é
transparente, límpido e mesmo desprovido de caráter, ele o é no sentido
utilizado por Franz Rosenzweig em seu \textit{Stern der Erlösung} (em português, ``Estrela da
redenção''). Na China, o homem interior é ``inteiramente desprovido de
caráter; o conceito do sábio, encarnado classicamente\ldots{} por Confúcio,
supõe um caráter totalmente depurado de todas as particularidades; ele é
o homem verdadeiramente sem caráter, isto é, o homem médio\ldots{} O que
define o chinês é algo de completamente distinto do caráter: uma pureza
elementar dos sentimentos''. Como quer que possamos traduzir
conceitualmente essa pureza de sentimentos --- talvez ela seja um
instrumento capaz de medir de forma especialmente sensível o
comportamento gestual ---, o fato é que o teatro ao ar livre de Oklahoma
remete ao teatro clássico chinês, que é um teatro gestual. Uma das
funções mais significativas desse teatro ao ar livre é a dissolução do
acontecimento no gesto. Podemos ir mais longe e dizer que muitos estudos
e contos menores de Kafka só aparecem em sua verdadeira luz quando
transformados, por assim dizer, em peças representadas no teatro ao ar
livre de Oklahoma. Somente então se perceberá claramente que toda a obra
de Kafka representa um código de gestos, cuja significação simbólica não
é de modo algum evidente, desde o início, para o próprio autor; eles só
recebem essa significação depois de inúmeras tentativas e experiências,
em contextos múltiplos. O teatro é o lugar dessas experiências. Num
comentário inédito sobre \textit{Brudermord} (em português, ``O fratricídio''), Werner Kraft
observou lucidamente que a ação dessa pequena história era de natureza
cênica. 

\begin{quote}
O espetáculo pode começar e é anunciado por uma campainha.
Este som se produz da forma mais natural, no momento em que Wese deixa a
casa em que se encontra seu escritório. Mas essa campainha, diz o autor
expressamente, \textit{toca alto demais para uma simples campainha de porta,
ela ressoa na cidade inteira, até o céu}. 
\end{quote}

Assim como essa campainha,
que toca alto demais e chega até o céu, os gestos dos personagens
kafkianos são excessivamente enfáticos para o mundo habitual e
extravasam para um mundo mais vasto. Quanto mais se afirma a técnica
magistral do autor, mais ele desdenha adaptar esses gestos às situações
habituais e explicá-los. Na \textit{Verwandlung} (em português, ``A metamorfose''), lemos sobre
``a estranha maneira que tem o chefe de sentar-se em sua escrivaninha e
falar de cima para baixo com seu empregado, que além disso precisa
chegar muito perto, devido à surdez do patrão''. Mas no \textit{Prozess} (em português, ``O
processo'') não existem mais essas justificações. No penúltimo capítulo,
\textsc{k.} 

\begin{quote}
parou nos primeiros bancos, mas para o padre a distância ainda era
excessiva. Estendeu a mão e mostrou com o indicador um lugar mais
próximo do púlpito. \textsc{k.} o seguiu até esse lugar, precisando inclinar a
cabeça fortemente para trás a fim de ver o padre.
\end{quote}

Se é certo, como diz Max Brod, que ``era imenso o mundo dos fatos que
ele considerava importantes'', o mais imenso de todos era o mundo dos
gestos. Cada um é um acontecimento em si e por assim dizer um drama em
si. O palco em que se representa esse drama é o teatro do mundo, com o
céu como perspectiva. Por outro lado, este céu é apenas pano de fundo;
investigá-lo segundo sua própria lei significaria emoldurar um pano de
fundo teatral e pendurá-lo numa galeria de quadros. Como El Greco, Kafka
despedaça o céu, atrás de cada gesto; mas como em El Greco, padroeiro
dos expressionistas, o gesto é o elemento decisivo, o centro da ação. Os
que ouviram a batida no portão se afastam, curvados de terror. Um ator
chinês representaria assim o terror, mas não assustaria ninguém. Em
outra passagem o próprio \textsc{k.} faz teatro. Semiconsciente do que fazia, ele

\begin{quote}
\textls[15]{levantou cuidadosamente os olhos\ldots{} pegou um dos papéis, sem olhá-lo,
colocou-o na palma da mão e o ofereceu lentamente aos cavalheiros,
enquanto ele próprio se erguia. Ele não pensava em nada de preciso, mas
tinha apenas a sensação de que era assim que ele teria que se comportar,
quando terminasse a grande petição que deveria inocentá-lo
completamente. }
\end{quote}

Esse gesto supremamente enigmático e supremamente
simples é um gesto de animal. Podemos ler durante muito tempo as
histórias de animais de Kafka sem percebermos que elas não tratam de
seres humanos. Quando descobrimos o nome da criatura --- símio, cão ou
toupeira ---, erguemos os olhos, assustados, e verificamos que o mundo
dos homens já está longe. Kafka é sempre assim; ele priva os gestos
humanos dos seus esteios tradicionais e os transforma em temas de
reflexões intermináveis.

Porém elas também são intermináveis quando partem das histórias
alegóricas. Pense-se na parábola \textit{Vor dem Gesetz} (em português, ``Diante da lei''). O
leitor que a encontra no \textit{Landarzt} (em português, ``Médico de aldeia'') percebe os
trechos nebulosos que ela contém. Mas teria pensado nas inúmeras
reflexões que ocorrem a Kafka, quando ele a interpreta? É o que ele faz
em \textit{O processo}, por intermédio do padre, e num lugar tão oportuno que
poderíamos suspeitar que o romance não é mais que o desdobramento da
parábola. Mas a palavra ``desdobramento'' tem dois sentidos. O botão se
``desdobra'' na flor, mas o papel ``dobrado'' em forma de barco, na
brincadeira infantil, pode ser ``desdobrado'', transformando-se de novo
em papel liso. Essa segunda espécie de desdobramento convém à parábola,
e o prazer do leitor é fazer dela uma coisa lisa, cuja significação
caiba na palma da mão. Mas as parábolas de Kafka se desdobram no
primeiro sentido: como o botão se desdobra na flor. Por isso, são
semelhantes à criação literária. Apesar disso, elas não se ajustam
inteiramente à prosa ocidental e se relacionam com o ensinamento como a
\textit{hagadá} se relaciona com a \textit{halachá}. Não são parábolas e não podem
ser lidas no sentido literal. São construídas de tal modo que podemos
citá-las e narrá-las com fins didáticos. Porém conhecemos a doutrina
contida nas parábolas de Kafka e que é ensinada nos gestos e atitudes de
\textsc{k.} e dos animais kafkianos? Essa doutrina não existe; podemos dizer no
máximo que um ou outro trecho alude a ela. Kafka talvez dissesse: esses
trechos constituem os resíduos dessa doutrina e a transmitem. Mas
podemos dizer igualmente: eles são os precursores dessa doutrina, e a
preparam. De qualquer maneira, trata-se da questão da organização da
vida e do trabalho na comunidade humana. Essa questão preocupou Kafka
como nenhuma outra e era impenetrável para ele. Assim como, na célebre
conversa de Erfurt entre Goethe e Napoleão, o Imperador substituiu a
política pelo destino, Kafka poderia ter substituído a organização pelo
destino. A organização está constantemente presente em Kafka, não
somente nas gigantescas hierarquias de funcionários, em \textit{O processo} e
\textit{O castelo}, mas de modo ainda mais tangível nos incompreensíveis
projetos de construção, descritos em \textit{A muralha da China}.

\begin{quote}
A muralha deveria servir de proteção durante séculos; por isso, o
 máximo de cuidado na construção, a utilização dos conhecimentos
 arquitetônicos de todos os tempos e de todos os povos e um duradouro
 sentimento de responsabilidade por parte dos construtores constituíam
 pressupostos indispensáveis para esse trabalho. Para as obras acessórias,
 assalariados ignorantes do povo podiam ser usados, homens, mulheres,
 crianças, enfim, todos os que se empregavam para ganhar dinheiro; mas já
 para dirigir quatro desses assalariados, um homem culto era necessário,
 especializado em arquitetura\ldots{} Nós --- estou falando aqui em nome de
 muitos --- somente aprendemos a nos conhecer soletrando as instruções
 dos nossos superiores, descobrindo que sem sua liderança nosso saber
 acadêmico e nosso bom senso não teriam sido suficientes para podermos
 executar a pequena tarefa que nos cabia no grande todo. 
\end{quote}

 \textls[15]{Essa
 organização se assemelha ao destino. Em seu famoso livro \textit{A civilização
 e os grandes rios históricos}, Metchnikov descreve o esquema dessa
 organização com palavras que poderiam ser de Kafka. }

\begin{quote}
Os canais do
Yang-tsé-kiang e as represas do Huang-ho são provavelmente o resultado
de um trabalho comum, conscientemente organizado, de\ldots{} gerações\ldots{} A
menor desatenção na escavação de um fosso ou na sustentação de uma
represa, a menor negligência, uma atitude egoísta por parte de um homem
ou de um grupo de homens na tarefa de conservar os recursos hidráulicos
da comunidade, podem originar, nessas circunstâncias insólitas, grandes
males e desgraças sociais de consequências incalculáveis. Por isso, um
funcionário encarregado de administrar os rios exigia, com ameaças de
morte, uma estreita e duradoura solidariedade entre massas da população
que muitas vezes eram estranhas e mesmo inimigas entre si; ele condenava
todos a trabalhos cuja utilidade coletiva só se evidenciava com o tempo,
e cujo plano de conjunto era muitas vezes incompreensível para o homem
comum.
\end{quote}

Kafka queria ser incluído entre esses homens comuns. Pouco a pouco os
limites de sua compreensão se tornaram evidentes. Ele quer mostrar aos
outros esses limites. Às vezes ele se parece com o Grande Inquisidor, de
Dostoiévski: 

\begin{quote}
\textls[15]{Estamos, portanto, em presença de um mistério, que não
podemos compreender. E, como se trata de um enigma, tínhamos o direito
de pregar, de ensinar aos homens que o que estava em jogo não era nem a
liberdade nem o amor, mas um enigma, um segredo, um mistério, ao qual
tinham que se submeter, sem qualquer reflexão, e mesmo contra sua
consciência. }
\end{quote}

\textls[-5]{Nem sempre Kafka resistiu às tentações do misticismo.
Sobre seu encontro com Rudolf Steiner possuímos uma página de diário,
que pelo menos na forma em que foi publicada não reflete a posição de
Kafka. Teria se recusado a revelar sua opinião? Sua atitude com relação
aos próprios textos sugere que essa hipótese não é de modo algum
impossível. Kafka dispunha de uma capacidade invulgar de criar
parábolas. Mas ele não se esgota nunca nos textos interpretáveis e toma
todas as precauções possíveis para dificultar essa interpretação. É com
prudência, com circunspecção, com desconfiança que devemos penetrar,
tateando, no interior dessas parábolas. Devemos ter presente sua maneira
peculiar de lê-las, como ela transparece na sua interpretação da
parábola citada. Precisamos pensar também em seu testamento. Suas
instruções para que sua obra póstuma fosse destruída são tão difíceis de
compreender e devem ser examinadas tão cuidadosamente como as respostas
do guardião da porta, diante da lei. Cada dia de sua vida confrontará
Kafka com atitudes indecifráveis e com explicações ininteligíveis, e é
possível que pelo menos ao morrer Kafka tivesse decidido pagar seus
contemporâneos na mesma moeda.}

O mundo de Kafka é um teatro do mundo. Para ele, o homem está desde o
início no palco. E a prova é que todos são contratados no teatro de
Oklahoma. Impossível conhecer os critérios que presidem a essa
contratação. O talento de ator, que parece o critério mais óbvio, não
tem nenhuma importância. Podemos exprimir esse fato de outra forma: não
se exige dos candidatos senão que interpretem a si mesmos. Está
absolutamente excluído que eles \textit{sejam} o que \textit{representam}.
Representando seus papéis, os atores procuram um abrigo no teatro ao ar
livre, como os seis atores de Pirandello procuram um autor. Para uns e
outros, a cena constitui o último refúgio, e não é impossível que esse
refúgio seja também a salvação. A salvação não é uma recompensa
outorgada à vida, mas a última oportunidade de evasão oferecida a um
homem, como diz Kafka, ``cujo próprio crânio bloqueia\ldots{} o caminho''. A
lei desse teatro está numa frase escondida no \textit{Bericht für eine
Akademie} (em português, ``Relatório à academia''): ``\ldots{} eu imitava porque estava à
procura de uma saída, por nenhuma outra razão''. No final do seu
processo, \textsc{k.} parece ter um pressentimento de tudo isso. Ele se volta de
repente para os dois cavalheiros de cartola, que vieram levá-lo, e
pergunta: ``--- Em que teatro trabalham os Senhores?; --- Teatro?
perguntou um deles, pedindo conselho ao outro, com os lábios trêmulos.
Este reagiu como um mudo, que luta com um organismo recalcitrante''.
Eles não responderam à pergunta, mas esses indícios fazem supor que
foram afetados por ela.

Todos os atores que se tornaram membros do teatro ao ar livre são
servidos num grande banco, recoberto com uma toalha branca. ``Todos
estavam alegres e excitados.'' Para celebrar, os figurantes fazem o
papel de anjos, em altos pedestais cobertos com panos ondulantes e que
têm uma escada em seu interior. Todos os elementos de uma quermesse
campestre, ou talvez de uma festa infantil, na qual o menino da foto,
vestido com sua roupa excessivamente pomposa, teria talvez perdido a
tristeza do seu olhar. \textls[15]{Sem as asas postiças, talvez fossem anjos de
verdade. Eles têm precursores na obra de Kafka, entre eles o empresário
teatral, que sobe na rede para confortar o trapezista acometido da
``primeira dor'', acaricia-o e aperta o seu rosto contra o seu, de modo
que ``as lágrimas do trapezista o inundaram também''. Outro anjo, anjo
guardião ou guardião da lei, depois do ``fratricídio'' se encarrega do
assassino Schmar, que ``cola a boca no ombro do guarda'' e o leva
consigo, com passos leves. O último romance de Kafka termina nas
cerimônias campestres de Oklahoma. Segundo Soma Morgenstern, ``em Kafka,
como em todos os fundadores de religião, sopra um ar de aldeia''.
Devemos recordar aqui a concepção da piedade, sustentada por Lao Zi, da
qual Kafka deu uma descrição completa em \textit{Nächste Dorf} (em português, ``A aldeia próxima''): ``Duas aldeias vizinhas podem estar ao alcance da vista e
ouvir os galos e os cães uma da outra, mas seus habitantes morrem
velhos, sem jamais viajarem de uma para outra''. São palavras de
Lao Zi. Kafka também compunha parábolas, mas não fundou nenhuma
religião.}

\textls[-5]{Recordemos a aldeia ao pé do castelo, do qual \textsc{k.} recebe a confirmação
misteriosa e inesperada de sua designação como agrimensor. Em seu
posfácio a \textit{O castelo}, Brod informa que Kafka tinha pensado num
vilarejo específico ao criar essa aldeia: Zürau, no Erzgebirge. Mas
podemos reconhecer nela outro lugar. É a aldeia mencionada numa lenda
talmúdica, narrada por um rabino em resposta à pergunta: por que os
judeus preparam um banquete na noite de sexta-feira? É a história de uma
princesa exilada, longe dos seus compatriotas, que definha numa aldeia
cuja língua ela não compreende. Um dia ela recebe uma carta do seu
noivo, anunciando que não a tinha esquecido e que estava a caminho para
revê-la. O noivo, diz o rabino, é o Messias, a princesa a alma, e a
aldeia o corpo. Ignorando a língua falada na aldeia, seu único meio para
comunicar-lhe a alegria que sente é preparar para ela um festim. Essa
aldeia talmúdica está no centro do mundo kafkiano. O homem de hoje vive
em seu corpo como \textsc{k.} ao pé do castelo: ele desliza fora dele e lhe é
hostil. Pode ocorrer que o homem acorde um dia e verifique que se
transformou num inseto. O país de exílio --- o seu exílio ---
apoderou-se dele. É o ar dessa aldeia que sopra no mundo de Kafka, e é
por isso que ele nunca cedeu à tentação de fundar uma religião. É nesse
vilarejo que estão o chiqueiro de onde saem os cavalos para o médico de
aldeia, o sufocante quarto dos fundos onde Klamm está sentado diante de
um copo de cerveja, com o charuto na boca, e o portão no qual não se
pode bater sem desafiar a morte. O ar dessa aldeia é impuro, com a
mescla putrefata das coisas que não chegaram a existir e das coisas que
amadureceram demais. Em sua vida, Kafka teve que respirar essa
atmosfera. Não era nem adivinho nem fundador de religiões. Como
conseguiu suportar tal atmosfera?}

\subsection{O homenzinho corcunda}

\textls[5]{Há muito se sabe que Knut Hamsun tinha o hábito de publicar suas
opiniões na seção dos leitores do jornal que circulava na cidadezinha
perto da qual ele vivia. Há alguns anos foi instaurado nessa cidade um
processo contra uma jovem que assassinara seu filho recém-nascido. Ela
foi condenada à prisão. Pouco depois apareceu na folha local uma carta
de Hamsun. O autor dizia que daria as costas a uma cidade que aplicasse
a mães capazes de matar seus filhos outra pena que a mais severa: se não
a forca, pelo menos a prisão perpétua. Passaram-se alguns anos. Hamsun
publicou \textit{Benção da terra}, na qual havia a história de uma empregada
doméstica que comete o mesmo crime, recebe a mesma pena e certamente não
merecia um castigo mais severo, como o leitor percebe claramente.}

\textls[5]{As reflexões póstumas de Kafka, contidas em \textit{A grande muralha da China},
 fazem lembrar esse episódio. Pois assim que apareceu o volume póstumo,
 foi publicada uma exegese de Kafka, baseada apenas nessas reflexões e
 que procurava interpretá-las, ignorando sumariamente a própria obra. Há
 dois mal-entendidos possíveis com relação a Kafka: recorrer a uma
 interpretação natural e a uma interpretação sobrenatural. As duas, a
 psicanalítica e a teológica, perdem de vista o essencial. A primeira é
 devida a Hellmuth Kaiser; a segunda foi praticada por numerosos autores,
 como H. J. Schoeps, Bernhard Rang e Groethuysen. Willy Haas pode também
 ser incluído nessa corrente, embora em outras ocasiões tenha escrito
 comentários muito instrutivos sobre Kafka, como veremos a seguir. Isso
 não o impediu de explicar a obra de Kafka em seu conjunto segundo certos
 lugares-comuns teológicos. ``O poder superior'', escreve ele,} 

\begin{quote}
\textls[-15]{a
esfera da graça, é descrito em seu grande romance \textit{O castelo}, enquanto
o poder inferior, a esfera do julgamento e da danação, é descrito em
outro grande livro, \textit{O processo}. Tentou descrever a terra, esfera
intermediária entre esses dois planos\ldots{} o destino terreno, com suas
difíceis exigências, e de modo altamente estilizado, num terceiro
romance, \textit{Amérika}. }
\end{quote}

\textls[10]{O primeiro terço dessa interpretação constitui
hoje, a partir de Brod, patrimônio comum da exegese kafkiana. Assim, por
exemplo, escreve Bernhard Rang: }

\begin{quote}
\textls[10]{Na medida em que o castelo pode ser
visto como a sede da Graça, os vãos esforços e tentativas dos homens
significam, teologicamente falando, que eles não podem forçar e provocar
arbitrariamente, por um ato de vontade, a graça divina. A agitação e a
impaciência inibem e perturbam o silêncio grandioso de Deus. }
\end{quote}

É uma
interpretação cômoda, que se torna cada vez mais insustentável à medida
que se avança na mesma direção. Willy Haas é especialmente claro nessa
linha de argumentação: 

\begin{quote}
Kafka descende\ldots{} de Kierkegaard e de Pascal;
podemos considerá-lo o único descendente legítimo desses dois filósofos.
Os três partem, com a mesma dureza implacável, do mesmo tema religioso
de base: o homem nunca tem razão em face de Deus\ldots{} O mundo superior de
Kafka, o castelo, com seus funcionários imprevisíveis, mesquinhos,
complicados e gananciosos, e seu estranho Céu, brincam com os homens
impiedosamente\ldots{} no entanto nem diante desse Deus o homem tem razão.
\end{quote}

\textls[15]{Em suas especulações bárbaras, que de resto não são sequer compatíveis
com o próprio texto literal de Kafka, essa teologia fica muito aquém da
doutrina da justificação, de Anselmo de Salisbury. É exatamente em O
\textit{castelo} que encontramos a frase: ``Pode um só funcionário perdoar? No
máximo, a administração como um todo poderia fazê-lo, mas provavelmente
ela não pode perdoar, e sim julgar, apenas''. Esse tipo de interpretação
levou rapidamente a um beco sem saída. ``Nada disso'', diz Denis de
Rougemont, ``significa a miséria de um homem sem Deus, mas a miséria do
homem ligado a um Deus que ele não conhece, porque não conhece o
Cristo.''}

\textls[10]{É mais fácil extrair conclusões especulativas das notas póstumas de
Kafka que investigar um único dos temas que aparecem em seus contos e
romances. No entanto somente esses temas podem lançar alguma luz sobre
as forças arcaicas que atravessam a obra de Kafka --- forças,
entretanto, que com igual justificação poderíamos identificar no mundo
contemporâneo. Quem pode dizer sob que nome essas forças apareceram a
Kafka? O que é certo é que ele não se encontrou nelas. Não as conheceu.
No espelho da culpa, que o mundo primitivo lhe apresentou, ele viu
apenas o futuro, sob a forma do tribunal. Como representar esse
tribunal? Seria o julgamento final? O juiz não se converte em acusado? A
punição não está no próprio processo? Kafka não respondeu a essas
perguntas. Veria alguma utilidade nelas? Ou julgava preferível adiá-las?
Nas narrativas que ele nos deixou, a epopeia recuperou a significação
que lhe dera Scherazade: adiar o que estava por vir. O adiamento é em \textit{O
processo} a esperança dos acusados --- contanto que o procedimento
judicial não se transforme gradualmente na própria sentença. O adiamento
beneficiaria mesmo o Patriarca, e para isso deveria renunciar ao papel
que lhe cabe na tradição. }

\begin{quote}
Posso imaginar um outro Abraão, que não
chegaria evidentemente à condição de Patriarca, nem sequer à de
negociante de roupas usadas, que se disporia a cumprir a exigência do
sacrifício, obsequioso como um garçom, mas que não consumaria esse
sacrifício, porque não pode sair de casa, onde é indispensável, porque
seus negócios lhe impõem obrigações, porque há sempre alguma coisa a
arrumar, porque a casa não está pronta, e sem que ela esteja pronta não
pode sair, como a própria Bíblia admite, quando diz: ele pôs em ordem
sua casa.
\end{quote}

Abraão parece ``obsequioso como um garçom''. Só pelo gesto podia Kafka
fixar alguma coisa. É esse gesto, que ele não compreende, que constitui
o elemento nebuloso de suas parábolas. É dele que parte a obra literária
de Kafka. Sabe-se como ele era reticente com relação a essa obra. Em seu
testamento, ordenou que ela fosse destruída. Esse testamento, que nenhum
estudo sobre Kafka pode ignorar, mostra que o autor não estava
satisfeito; que ele considerava seus esforços malogrados; que ele se
incluía entre os que estavam condenados ao fracasso. Fracassada foi sua
grandiosa tentativa de transformar a literatura em doutrina,
devolvendo-lhe, sob a forma de parábolas, a consistência e a austeridade
que lhe convinham, à luz da razão. Nenhum escritor seguiu tão
rigorosamente o preceito de ``não construir imagens''.

\textls[5]{``Era como se a vergonha devesse lhe sobreviver'' --- são as últimas
palavras de \textit{O processo}. A vergonha, que nele corresponde à ``pureza
elementar dos sentimentos'', é o mais forte gesto de Kafka. Ela tem uma
dupla face. A vergonha é ao mesmo tempo uma reação íntima do indivíduo e
uma reação social. Não é apenas vergonha dos outros, mas vergonha pelos
outros. A vergonha de Kafka é tão pouco pessoal quanto a vida e o
pensamento que ela determina e sobre os quais Kafka escreveu: }

\begin{quote}
\textls[10]{Ele não
vive por causa de sua vida pessoal, nem pensa por causa do seu
pensamento pessoal. Tudo se passa como se ele vivesse e pensasse sob o
peso de uma obrigação familiar\ldots{} Por causa dessa família
desconhecida\ldots{} ele não podia ser despedido. }
\end{quote}

\textls[5]{Não conhecemos a
composição dessa família desconhecida, constituída por homens e animais.
Só uma coisa é clara: é ela que o força, ao escrever, a movimentar
períodos cósmicos. Obedecendo às exigências dessa família, Kafka rola o
bloco do processo histórico, como Sísifo rola seu rochedo. Nesse
movimento, o lado de baixo desse bloco se torna visível. Não é um
espetáculo agradável. Mas Kafka consegue suportar essa visão. ``Ter fé
no progresso não significa julgar que o progresso já aconteceu. Isso não
seria mais fé.'' A época em que ele vive não representa para Kafka
nenhum progresso com relação ao começo primordial. Seus romances se
passam num lamaçal. A criatura para ele está no estágio que Bachofen
caracterizou como hetaírico. O fato de que esse estágio esteja esquecido
não significa que ele não se manifeste no presente. Ao contrário, é esse
esquecimento que o torna presente. Ele é descoberto por uma experiência
mais profunda que a do homem comum. Em uma de suas primeiras anotações,
escreve Kafka: ``Eu tenho experiência e não estou brincando quando digo
que essa experiência é uma espécie de enjoo em terra firme''. Não é por
acaso que a primeira \textit{Reflexão} parte de um balanço. Kafka é inesgotável
em sua descrição da natureza oscilante das experiências. Cada uma cede à
outra, mistura-se com a experiência contrária. ``Era um dia quente de
verão.'' --- começa \textit{Schlag ans Hoftor} (em português, ``Batida no portão'') --- ``Voltando para casa com minha irmã, passei diante de um portão. Não sei
se ela bateu no portão, por capricho ou distração, ou se apenas ameaçou
fazê-lo com o punho, sem bater.'' A mera possibilidade da terceira
hipótese faz as duas outras, aparentemente inocentes, aparecerem sob
outra luz. É do pântano dessas experiências que emergem os personagens
femininos de Kafka. São figuras de lodo, como Leni, que ``separa o dedo
médio e o anular de sua mão direita, de modo que `a película que une os
dois dedos' se estende quase até atingir a articulação superior do dedo
mínimo''. A ambígua Frieda se recorda de sua vida passada. ``Belos
tempos. Nunca me perguntaste nada sobre o meu passado.'' Esse passado se
estende até o ponto mais escuro das profundezas em que se dá aquela
cópula cuja ``voluptuosidade desenfreada'', para usar as palavras de
Bachofen, ``é abominada pelos poderes imaculados da luz divina e que
justifica a expressão \textit{luteae voluptates}, de Arnobius''.}

Só a partir desse fato podemos compreender a técnica narrativa de Kafka.
Quando outros personagens têm algo que dizer a \textsc{k.}, eles o dizem
casualmente, como se ele no fundo já soubesse do que se tratava, por
mais importante e surpreendente que seja a comunicação. É como se não
houvesse nada de novo, como se o herói fosse discretamente convidado a
lembrar-se de algo que ele havia esquecido. É nesse sentido que Willy
Haas interpreta, com razão, o movimento de \textit{O processo}, dizendo que 

\begin{quote}
\textls[10]{o objeto desse processo, o verdadeiro herói desse livro inacreditável, é o
esquecimento\ldots{} cujo principal atributo é o de esquecer-se a si mesmo\ldots{}
Ele se transformou em personagem mudo na figura do acusado, figura da
mais grandiosa intensidade. }
\end{quote}

Não podemos afastar de todo a hipótese de
que esse ``centro misterioso'' derive da ``religião judaica''. ``A
memória enquanto piedade desempenha aqui um papel supremamente
misterioso. O mais profundo atributo de Jeová é que ele se recorda, que
conserva uma memória infalível até `a terceira e quarta geração', até a
`centésima' geração; o momento mais sacrossanto do ritual é o apagamento
dos pecados no livro da memória''.

Mas o esquecimento --- e aqui atingimos um novo patamar na obra de Kafka
--- não é nunca um esquecimento individual. Tudo o que é esquecido se
mescla a conteúdos esquecidos do mundo primitivo, estabelece com ele
vínculos numerosos, incertos, cambiantes, para formar criações sempre
novas. O esquecimento é o receptáculo a partir do qual emergem à luz do
dia os contornos do inesgotável mundo intermediário, nas narrativas de
Kafka. 

\begin{quote}
\textls[10]{Aqui a plenitude do mundo é considerada a única realidade. Todo
espírito precisa fazer-se coisa, ser isolado, para adquirir um lugar e
um direito à existência\ldots{} O espiritual, na medida em que ainda
desempenha um papel, pulveriza-se em espíritos. Os espíritos se tornam
entes completamente individuais, com os seus próprios nomes
estreitamente associados ao nome de quem os venera\ldots{} despreocupada, a
plenitude do mundo recebe da plenitude desses espíritos uma nova
plenitude\ldots{} Sem provocar nenhuma inquietação, aumenta a massa dos
espíritos\ldots{} aos antigos espíritos se acrescentam novos, todos com seu
nome e distintos uns dos outros. }
\end{quote}

Essas palavras não se referem a
Kafka, e sim à China. É assim que Franz Rosenzweig descreve o culto dos
antepassados na \textit{Estrela da redenção}. Do mesmo modo que para Kafka o
mundo dos fatos importantes era imenso, também era imenso o mundo dos
seus ancestrais, e é certo que esse mundo, como o mastro totêmico dos
primitivos, chegava até os animais, em seu movimento descendente. De
resto, não é somente em Kafka que os animais são os receptáculos do
esquecimento. Na profunda obra de Tieck, \textit{Der Blonde Eckbert} (em português, ``O louro Eckbert''), o nome esquecido de um cãozinho --- Strohmi --- figura como
símbolo de uma culpa enigmática. Podemos entender assim por que Kafka
não se cansava de escutar os animais para deles recuperar o que fora
esquecido. Eles não são um fim em si, mas sem eles nada seria possível.
Recorde-se o ``artista da fome'', que ``a rigor era apenas um obstáculo
no caminho que levava às estrebarias''. Não vemos, em \textit{Bau}
(em português, ``Construção'') ou no \textit{Riesenmaulwurf} (em português, ``Toupeira gigante''), o animal
refletindo e ao mesmo tempo cavando suas galerias subterrâneas? Por
outro lado, esse pensamento é algo de muito confuso. Indeciso, ele
oscila de uma preocupação para outra, saboreia todos os medos e tem a
inconstância do desespero. Por isso, em Kafka também existem borboletas;
o ``caçador Gracchus'', sob o peso de uma culpa da qual ele nada quer
saber, ``transforma-se em borboleta''. ``Não riam, diz o caçador
Gracchus.'' O que é certo é que de todos os seres de Kafka são os
animais os que mais refletem. O que é a corrupção no mundo do direito, a
angústia é no mundo do pensamento. Ela perturba o pensamento, mas
constitui o único elemento de esperança que ele contém. Porém em nosso
corpo o mais esquecido dos países estrangeiros é o nosso próprio corpo,
e, por isso, compreendemos a razão pela qual Kafka chamava ``o animal''
à tosse que irrompia das suas entranhas. Era o posto avançado da grande
horda.

Em Kafka, Odradek é o mais estranho bastardo gerado pelo mundo
pré-histórico com seu acasalamento com a culpa. 

\begin{quote}
À primeira vista ele
tem o aspecto de um carretel achatado, em forma de estrela, e de fato
parece ter alguma analogia com um novelo de fios: de qualquer maneira só
poderiam ser fios rasgados, velhos, interligados por nós, emaranhados um
no outro, dos mais diferentes tipos e cores. Mas não é apenas um
carretel, porque do centro da estrela sai um bastonete transversal, ao
qual se junta outro no canto direito. Com auxílio desse último bastonete
e de uma das pontas da estrela, a criatura pode ficar de pé, como se
tivesse duas pernas.
\end{quote}

\textls[-10]{Odradek ``fica alternadamente no sótão, na
escada, no corredor, no vestíbulo''. Ele frequenta, portanto, os mesmos
lugares que o investigador da Justiça, à procura da culpa. O sótão é o
lugar dos objetos descartados e esquecidos. A obrigação de comparecer ao
tribunal evoca talvez o mesmo sentimento que a obrigação de remexer
arcas antigas, deixadas no sótão durante anos. Se dependesse de nós,
adiaríamos a tarefa até o fim dos nossos dias, do mesmo modo que \textsc{k.} acha
que seu documento de defesa ``poderá um dia ocupar sua inteligência
senil, depois da aposentadoria''.}

Odradek é o aspecto assumido pelas coisas em estado de esquecimento.
Elas são deformadas. Deformada é a ``preocupação do pai de família'',
que ninguém sabe em que consiste, deformado o inseto, que como sabemos é
na realidade Gregor Samsa, deformado o grande animal, meio carneiro e
meio gato, para o qual talvez ``a faca do carniceiro fosse uma
salvação''. Mas esses personagens de Kafka se associam, através de uma
longa série de figuras, com a figura primordial da deformação, o
corcunda. Entre as atitudes descritas por Kafka em suas narrativas
nenhuma é mais frequente que a do homem cuja cabeça se inclina
profundamente sobre seu peito. Ela é provocada pelo cansaço nos membros
do tribunal, pelo ruído nos porteiros do hotel, pelo teto excessivamente
baixo nos frequentadores das galerias. Contudo na \textit{Strafkolonie} (em português, ``Colônia
penal'') os dirigentes se servem de uma antiga máquina que grava letras
floreadas nas costas do culpado, aumenta as incisões, acumula os
ornamentos, até que suas costas se tornem clarividentes, possam elas
próprias decifrar as inscrições, descobrindo assim o nome da culpa
desconhecida. São, portanto, as costas que importam. São elas que
importam para Kafka, desde muito tempo. Lemos nas primeiras anotações do
\textit{Diário}: ``Para ficar tão pesado quanto possível, o que considero bom
para o sono, eu cruzava os braços e punha as mãos nos ombros, como um
soldado com sua mochila''. É claro que a ideia de estar carregado tem
relação com a de esquecer --- no sono. Uma canção popular --- \textit{O
homenzinho corcunda} --- concretiza essa relação. O homenzinho é o
habitante da vida deformada; desaparecerá quando chegar o Messias, de
quem um grande rabino disse que ele não quer mudar o mundo pela força,
mas apenas retificá-lo um pouco.

``Vou para o meu quartinho; para fazer minha caminha; e encontro um
homenzinho corcunda; que começa a rir.'' É o riso de Odradek, que
``ressoa como o murmúrio de folhas caídas''. A canção continua: ``Quando
me ajoelho em meu banquinho; para rezar um pouquinho; encontro um
homenzinho corcunda; que começa a falar.; Querida criancinha, por favor;
Reza também pelo homenzinho corcunda''. Assim termina a canção. Em suas
profundezas, Kafka toca o chão que não lhe era oferecido nem pelo
``pressentimento mítico'' nem pela ``teologia existencial''. É o chão do
mundo germânico e do mundo judeu. Se Kafka não rezava, o que ignoramos,
era capaz ao menos, como faculdade inalienavelmente sua, de praticar o
que Malebranche chamava ``a prece natural da alma'' --- a atenção. Como
os santos em sua prece, Kafka incluía na sua atenção todas as criaturas.

\subsection{Sancho Pança}

Conta-se que numa aldeia hassídica alguns judeus estavam sentados numa
pobre estalagem, num sábado à noite. Eram todos residentes do lugar,
menos um desconhecido, de aspecto miserável, mal vestido, escondido num
canto escuro, nos fundos. Conversava-se aqui e ali. Num certo momento,
alguém se lembrou de perguntar o que cada um desejaria, se um único
desejo pudesse ser atendido. Um queria dinheiro, outro um genro, outro
uma nova banca de carpinteiro, e assim por diante. Depois que todos
falaram, restava apenas o mendigo, em seu canto escuro. Interrogado, ele
respondeu, com alguma relutância: 

\begin{quote}
\textls[-15]{Gostaria de ser um rei poderoso,
governando um vasto país, e que uma noite, ao dormir em meu palácio, um
exército inimigo invadisse o meu reino, e que antes do nascer do dia os
cavaleiros tivessem entrado em meu castelo, sem encontrar resistência, e
que acordando assustado eu não tivesse tempo de me vestir, e com uma
simples camisa no corpo eu fosse obrigado a fugir, perseguido sem parar,
dia e noite, por montes, vales e florestas, até chegar a este banco,
neste canto, são e salvo. É o meu desejo. }
\end{quote}

Os outros se entreolharam
sem entender. ``--- E o que você ganharia com isso?'' perguntaram. ``---
Uma camisa'', foi a resposta.

\textls[15]{Essa história conduz ao centro da obra de Kafka. Não está dito que as
deformações que um dia o Messias corrigirá são apenas as do nosso
espaço. Certamente são também as do nosso tempo. E certamente Kafka
pensou nisso. É com uma certeza desse gênero que seu avô diz: }

\begin{quote}
\textls[-20]{A vida é
surpreendentemente curta. Ela é mesmo tão curta em minha memória, que
mal posso compreender, por exemplo, como um jovem pode se decidir a
viajar para a próxima aldeia sem temer --- mesmo deixando de lado os
acidentes imprevisíveis --- que o tempo de toda uma vida normal e sem
imprevistos seja insuficiente para terminar essa viagem. }
\end{quote}

\textls[-20]{O mendigo é
um irmão desse velho. Em sua ``vida normal e sem imprevistos'' ele não
encontra tempo para um só desejo, mas na vida anormal e cheia de
imprevistos da fuga, que ele fantasia em sua história, ele renuncia a
qualquer desejo e o troca pela sua realização.}

\textls[-5]{Entre as criaturas de Kafka existe uma tribo singularmente consciente da
brevidade da vida. Ela vem da ``cidade do sul'', que Kafka caracteriza
com o seguinte diálogo: ``Ali estão as pessoas! Imaginem, elas não
dormem!; --- E por que não?; --- Porque não se cansam nunca.; --- E por que
não?; --- Porque são tolos.; --- Então os tolos não se cansam?; --- Como
poderiam os tolos cansar-se?''. Como se vê, os tolos têm afinidades com
os infatigáveis ajudantes. Mas essa tribo tem ainda outras
características. De passagem, ouvimos um comentário segundo o qual os
rostos dos ajudantes ``lembravam os de adultos, talvez mesmo os de
estudantes''. Com efeito, os estudantes, que em Kafka aparecem nos
lugares mais estranhos, são os chefes e porta-vozes dessa tribo. ``---
Mas quando dormem vocês? perguntou Karl, olhando admirado os estudantes;
--- Ah, dormir! disse o estudante. Dormirei quando tiver acabado os meus
estudos.'' Pense-se nas crianças: com que relutância vão para a cama!
Pois enquanto dormem, alguma coisa interessante poderia acontecer. ``Não
se esqueça do melhor!'' é uma observação ``que nos é familiar a partir
de uma quantidade incerta de velhas narrativas, embora ela talvez não
ocorra em nenhuma.'' Porém o esquecimento diz sempre respeito ao melhor,
porque diz respeito à possibilidade da redenção. ``A ideia de querer
ajudar-me'', diz, ironicamente, o espírito sempre inquieto do caçador
Gracchus, ``é uma doença que deve ser curada na cama.'' Os estudantes
não dormem, por causa dos seus estudos, e talvez a maior virtude dos
estudos é mantê-los acordados. O artista da fome jejua, o guardião da
porta silencia e os estudantes velam: assim, ocultas, operam em Kafka as
grandes regras da ascese.}

Os estudos são seu coroamento. Kafka os traz à luz do dia, resgatando-os
dos anos extintos de sua infância. 

\begin{quote}
\textls[-15]{Numa cena não muito diferente, há
muitos anos, Karl se sentava, em casa, à mesa dos seus pais, fazendo
seus deveres escolares, enquanto o pai lia o jornal ou fazia
contabilidade e redigia a correspondência para uma firma, e a mãe
costurava, levantando muito alto a linha. Para não incomodar o pai, Karl
só colocava na mesa o caderno e o material de escrever, arrumando os
livros necessários em cadeiras à direita e à esquerda. Como tudo era
tranquilo ali! Como era rara a visita dos estranhos!}
\end{quote}

\textls[15]{Talvez esses
estudos não tenham servido para nada. Mas esse ``nada'' é muito próximo
daquele ``nada'' taoista que nos permite utilizar ``alguma coisa''. É em
busca desse ``nada'' que Kafka formulava o desejo de }

\begin{quote}
\textls[15]{fabricar uma mesa
com uma perícia exata e escrupulosa, e ao mesmo tempo não fazer nada, de
tal maneira que, em vez de dizerem: o martelo não é nada para ele, as
pessoas dissessem: o martelo é para ele um verdadeiro martelo e ao mesmo
tempo não é nada, e com isso o martelo se tornaria ainda mais audacioso,
mais decidido, mais real e, se se quiser, mais louco.}
\end{quote}

Em seus estudos,
os estudantes têm uma atitude igualmente resoluta e igualmente fanática.
Essa atitude não pode ser mais estranha. Escrevendo e estudando, as
pessoas perdem o fôlego. 

\begin{quote}
Muitas vezes o funcionário dita em voz tão
baixa que o escrevente não ouve nada se estiver sentado, e, por isso,
precisa pular, capturar as palavras ditadas, sentar-se depressa e
escrevê-las, em seguida pular de novo, e assim por diante. Como é
singular! É quase incompreensível. 
\end{quote}

\textls[-5]{Mas talvez possamos compreender
melhor se voltarmos aos atores do teatro ao ar livre. Os atores têm que
ficar extremamente atentos às suas deixas. Eles se assemelham também sob
outros aspectos a essas pessoas zelosas. Para eles, com efeito, ``o
martelo é um verdadeiro martelo e ao mesmo tempo não é nada'', desde que
esse martelo faça parte do seu papel. Eles estudam esse papel; o ator
que esquecesse uma palavra ou um gesto seria um mau ator. Para os
integrantes da equipe de Oklahoma, contudo, esse papel é sua vida
anterior. Por isso, esse teatro ao ar livre é um teatro ``natural''. Os
atores estão salvos. O mesmo não ocorre com o estudante que Karl vê do
seu balcão, em silêncio, à noite, quando ele lê o seu livro: ``ele
virava as folhas, de vez em quando consultava outro livro, que ele
segurava rapidamente, fazia anotações frequentes em um caderno,
inclinando profundamente o rosto sobre ele.''}

\textls[5]{Kafka não se cansa de dar corpo ao gesto, em descrições desse tipo. Mas
sempre com assombro. Com razão, Kafka foi comparado ao soldado Schweyk;
porém o primeiro se assombra com tudo, e o segundo não se assombra com
nada. O cinema e o gramofone foram inventados na era da mais profunda
alienação dos homens entre si e das relações mediatizadas ao infinito,
as únicas que subsistiram. No cinema, o homem não reconhece seu próprio
andar e no gramofone não reconhece sua própria voz. Esse fenômeno foi
comprovado experimentalmente. A situação dos que se submetem a tais
experiências é a situação de Kafka. É ela que o obriga ao estudo. Nesse
processo, talvez ele encontre fragmentos da própria existência, que
talvez ainda estejam em relação com o papel. Ele recuperaria o gesto
perdido, com Schlemihl, a sombra perdida. Ele se compreenderia enfim,
mas com que esforço imenso! Pois o que sopra dos abismos do esquecimento
é uma tempestade. E o estudo é uma corrida a galope contra essa
tempestade. É assim que o mendigo em seu banco ao lado da lareira
cavalga em direção ao seu passado, para se apoderar de si mesmo, sob a
forma do rei fugitivo. À vida, que é curta demais para uma cavalgada,
corresponde a vida que é suficientemente longa para que o cavaleiro
``abandone as esporas, porque não há esporas, jogue fora as rédeas,
porque não há rédeas, veja os prados na frente, com a vegetação rala, já
sem o pescoço do cavalo, já sem a cabeça do cavalo!''. Assim se realiza
a fantasia do cavaleiro feliz, que galopa numa viagem alegre e vazia em
direção ao passado, sem pesar sobre sua montaria. Infeliz, no entanto, o
cavaleiro que está preso à sua égua porque se fixou um objetivo situado
no futuro, ainda que seja o futuro mais imediato, como o de atingir o
depósito de carvão. Infeliz também seu cavalo, infelizes os dois.}

\begin{quote}
Montado num balde, segurando a alça, a mais simples das rédeas, desço
penosamente as escadas; mas, quando chego embaixo, meu balde se levanta,
lindo, lindo; camelos deitados no chão não se levantariam de modo mais
belo, sacudindo-se sob o bastão do cameleiro. 
\end{quote}

\textls[5]{Nenhuma região é mais
desolada que a região da ``montanha de gelo'' em que se perde para
sempre o ``cavaleiro do balde''. Das ``regiões inferiores da morte''
sopra o vento, que lhe é favorável --- o mesmo que em Kafka sopra tão
frequentemente do mundo primitivo, e que impulsiona o barco do caçador
Gracchus. ``Ensina-se em toda parte'', diz Plutarco, ``em mistérios e
sacrifícios, tanto entre os gregos como entre os bárbaros\ldots{} que devem
existir duas essências distintas e duas forças opostas, uma que leva em
frente, por um caminho reto, e outra que interrompe o caminho e força a
retroceder.'' É para trás que conduz o estudo, que converte a existência
em escrita. O professor é Bucéfalo, o ``novo advogado'', que sem o
poderoso Alexandre --- isto é, livre do conquistador, que só queria
caminhar para frente --- toma o caminho de volta. ``Livre, com seus
flancos aliviados da pressão das coxas do cavaleiro, sob uma luz calma,
longe do estrépito das batalhas de Alexandre, ele lê e vira as páginas
dos nossos velhos livros.'' Há algum tempo, Wemer Kraft interpretou essa
narrativa. Depois de ter examinado com cuidado cada pormenor do texto,
observa o intérprete: ``Nunca antes na literatura foi o mito em toda a
sua extensão criticado de modo tão violento e devastador''. Segundo
Kraft, o autor não usa a palavra ``justiça''; não obstante, é da justiça
que parte a crítica do mito. Mas, já que chegamos tão longe, se
parássemos aqui, correríamos o risco de não entender Kafka. É
verdadeiramente o direito que em nome da justiça é mobilizado contra o
mito? Não; como jurista, Bucéfalo permanece fiel à sua origem: porém ele
não parece \textit{praticar} o direito, e nisso, no sentido de Kafka, está o
elemento novo, para Bucéfalo e para a advocacia. A porta da justiça é o
direito que não é mais praticado, e sim estudado.}

\textls[15]{A porta da justiça é o estudo. Mas Kafka não se atreve a associar a esse
estudo as promessas que a tradição associa no estudo da Torá. Seus
ajudantes são bedéis que perderam a igreja, seus estudantes são
discípulos que perderam a escrita. Ela não se impressiona mais com ``a
viagem alegre e vazia''. Contudo Kafka achou a lei na sua viagem; pelo
menos uma vez, quando conseguiu ajustar sua velocidade desenfreada a um
passo épico, que ele procurou durante toda a sua vida. O segredo dessa
lei está num dos seus textos mais perfeitos, e não apenas por se tratar
de uma interpretação.}

\begin{quote}
\textls[10]{Sancho Pança, que aliás nunca se vangloriou
disso, conseguiu no decorrer dos anos afastar de si o seu demônio, que
ele mais tarde chamou de Dom Quixote, fornecendo-lhe, para ler de noite
e de madrugada, inúmeros romances de cavalaria e de aventura. Em
consequência, esse demônio foi levado a praticar as proezas mais
delirantes, mas que não faziam mal a ninguém, por falta do seu objeto
predeterminado, que deveria ter sido o próprio Sancho Pança. Sancho
Pança, um homem livre, seguia Dom Quixote em suas cruzadas com
paciência, talvez por um certo sentimento de responsabilidade, daí
derivando até o fim de sua vida um grande e útil entretenimento.}
\end{quote}

\textls[15]{Sancho Pança, tolo sensato e ajudante incapaz de ajudar, mandou na
frente o seu cavaleiro. Bucéfalo sobreviveu ao seu. Homem ou cavalo,
pouco importa, desde que o dorso seja aliviado do seu fardo.}

\pagebreak
\blankpage
%\blankAteven

\pagestyle{empty}

\begingroup
\fontsize{7}{8}\selectfont
{\large\textsc{coleção <<hedra edições>>}}\\

\begin{enumerate}
\setlength\parskip{4.2pt}
\setlength\itemsep{-1.4mm}
\item \textit{A metamorfose}, Kafka
\item \textit{O príncipe}, Maquiavel
\item \textit{Jazz rural}, Mário de Andrade
\item \textit{O chamado de Cthulhu}, H.\,P.\,Lovecraft
\item \textit{Ludwig Feuerbach e o fim da filosofia clássica alemã}, Friederich Engels
\item \textit{Hino a Afrodite e outros poemas}, Safo de Lesbos 
\item \textit{Pr\ae terita}, John Ruskin
\item \textit{Manifesto comunista}, Marx e Engels
\item \textit{Rashômon e outros contos}, Akutagawa
\item \textit{Memórias do subsolo}, Dostoiévski
\item \textit{Teogonia}, Hesíodo
\item \textit{Trabalhos e dias}, Hesíodo
\item \textit{O contador de histórias e outros textos}, Walter Benjamin
\item \textit{Diário parisiense e outros escritos}, Walter Benjamin
\item \textit{Fábula de Polifemo e Galateia e outros poemas}, Góngora
\item \textit{Pequenos poemas em prosa}, Baudelaire
\item \textit{Ode ao Vento Oeste e outros poemas}, Shelley
\item \textit{Poemas}, Byron
\item \textit{Sonetos}, Shakespeare
\item \textit{Cântico dos cânticos}, [Salomão]
\item \textit{Balada dos enforcados e outros poemas}, Villon
\item \textit{Ode sobre a melancolia e outros poemas}, Keats
\item \textit{Robinson Crusoé}, Daniel Defoe
\item \textit{Dissertação sobre as paixões}, David Hume
\item \textit{A morte de Ivan Ilitch}, Lev Tolstói 
\item \textit{Don Juan}, Molière
\item \textit{Contos indianos}, Mallarmé
\item \textit{Triunfos}, Petrarca
\item \textit{O retrato de Dorian Gray}, Wilde
\item \textit{A história trágica do Doutor Fausto}, Marlowe
\item \textit{Os sofrimentos do jovem Werther}, Goethe
\item \textit{Dos novos sistemas na arte}, Maliévitch
\item \textit{Metamorfoses}, Ovídio
\item \textit{Micromegas e outros contos}, Voltaire
\item \textit{O sobrinho de Rameau}, Diderot
\item \textit{Carta sobre a tolerância}, Locke
\item \textit{Discursos ímpios}, Sade
\item \textit{Dao De Jing}, Lao Zi
\item \textit{O fim do ciúme e outros contos}, Proust
\item \textit{Fé e saber}, Hegel
\item \textit{Joana d'Arc}, Michelet
\item \textit{Livro dos mandamentos: 248 preceitos positivos}, Maimônides
\item \textit{Eu acuso!}, Zola | \textit{O processo do capitão Dreyfus}, Rui Barbosa
\item \textit{Apologia de Galileu}, Campanella 
\item \textit{Sobre verdade e mentira}, Nietzsche
\item \textit{A vida é sonho}, Calderón
\item \textit{Sagas}, Strindberg
\item \textit{O mundo ou tratado da luz}, Descartes
\item \textit{A vênus das peles}, Sacher{}-Masoch
\item \textit{Escritos sobre arte}, Baudelaire
\item \textit{Americanismo e fordismo}, Gramsci
\item \textit{Sátiras, fábulas, aforismos e profecias}, Da Vinci
\item \textit{O cego e outros contos}, D.H.~Lawrence
\item \textit{Imitação de Cristo}, Tomás de Kempis
\item \textit{O casamento do Céu e do Inferno}, Blake
\item \textit{Flossie, a Vênus de quinze anos}, [Swinburne]
\item \textit{Teleny, ou o reverso da medalha}, [Wilde et al.]
\item \textit{A filosofia na era trágica dos gregos}, Nietzsche
\item \textit{No coração das trevas}, Conrad
\item \textit{Viagem sentimental}, Sterne
\item \textit{Arcana C\oe lestia} e \textit{Apocalipsis revelata}, Swedenborg
\item \textit{Saga dos Volsungos}, Anônimo do séc.~\textsc{xiii}
\item \textit{Um anarquista e outros contos}, Conrad
\item \textit{A monadologia e outros textos}, Leibniz
\item \textit{Cultura estética e liberdade}, Schiller
\item \textit{Poesia basca: das origens à Guerra Civil} 
\item \textit{Poesia catalã: das origens à Guerra Civil} 
\item \textit{Poesia espanhola: das origens à Guerra Civil} 
\item \textit{Poesia galega: das origens à Guerra Civil} 
\item \textit{O pequeno Zacarias, chamado Cinábrio}, E.T.A.~Hoffmann
\item \textit{Um gato indiscreto e outros contos}, Saki
\item \textit{Viagem em volta do meu quarto}, Xavier de Maistre 
\item \textit{Hawthorne e seus musgos}, Melville
\item \textit{Feitiço de amor e outros contos}, Ludwig Tieck
\item \textit{O corno de si próprio e outros contos}, Sade
\item \textit{Investigação sobre o entendimento humano}, Hume
\item \textit{Sobre os sonhos e outros diálogos}, Borges | Osvaldo Ferrari
\item \textit{Sobre a filosofia e outros diálogos}, Borges | Osvaldo Ferrari
\item \textit{Sobre a amizade e outros diálogos}, Borges | Osvaldo Ferrari
\item \textit{A voz dos botequins e outros poemas}, Verlaine 
\item \textit{Gente de Hemsö}, Strindberg 
\item \textit{Senhorita Júlia e outras peças}, Strindberg 
\item \textit{Correspondência}, Goethe | Schiller
\item \textit{Poemas da cabana montanhesa}, Saigy\=o
\item \textit{Autobiografia de uma pulga}, [Stanislas de Rhodes]
\item \textit{A volta do parafuso}, Henry James
\item \textit{Carmilla --- A vampira de Karnstein}, Sheridan Le Fanu
\item \textit{Pensamento político de Maquiavel}, Fichte
\item \textit{Inferno}, Strindberg
\item \textit{Contos clássicos de vampiro}, Byron, Stoker e outros
\item \textit{O primeiro Hamlet}, Shakespeare
\item \textit{Noites egípcias e outros contos}, Púchkin
\item \textit{Jerusalém}, Blake
\item \textit{As bacantes}, Eurípides
\item \textit{Emília Galotti}, Lessing
\item \textit{Viagem aos Estados Unidos}, Tocqueville
\item \textit{Émile e Sophie ou os solitários}, Rousseau 
\item \textit{A fábrica de robôs}, Karel Tchápek 
\item \textit{Sobre a filosofia e seu método --- Parerga e paralipomena (v.~\textsc{ii}, t.~\textsc{i})}, Schopenhauer 
\item \textit{O novo Epicuro: as delícias do sexo}, Edward Sellon
\item \textit{Sobre a liberdade}, Mill
\item \textit{A velha Izerguil e outros contos}, Górki
\item \textit{Pequeno-burgueses}, Górki
\item \textit{Primeiro livro dos Amores}, Ovídio
\item \textit{Educação e sociologia}, Durkheim
\item \textit{A nostálgica e outros contos}, Papadiamántis 
\item \textit{Lisístrata}, Aristófanes 
\item \textit{A cruzada das crianças/ Vidas imaginárias}, Marcel Schwob
\item \textit{O livro de Monelle}, Marcel Schwob
\item \textit{A última folha e outros contos}, O. Henry
\item \textit{Romanceiro cigano}, Lorca
\item \textit{Sobre o riso e a loucura}, [Hipócrates]
\item \textit{Ernestine ou o nascimento do amor}, Stendhal
\item \textit{Odisseia}, Homero
\item \textit{O estranho caso do Dr. Jekyll e Mr. Hyde}, Stevenson
\item \textit{Sobre a ética --- Parerga e paralipomena (v.~\textsc{ii}, t.~\textsc{ii})}, Schopenhauer 
\item \textit{Contos de amor, de loucura e de morte}, Horacio Quiroga
\item \textit{A arte da guerra}, Maquiavel
\item \textit{Elogio da loucura}, Erasmo de Rotterdam
\item \textit{Oliver Twist}, Charles Dickens
\item \textit{O ladrão honesto e outros contos}, Dostoiévski
\item \textit{Sobre a utilidade e a desvantagem da histório para a vida}, Nietzsche
\item \textit{Édipo Rei}, Sófocles
\item \textit{Fedro}, Platão
\item \textit{A conjuração de Catilina}, Salústio
\item \textit{Escritos sobre literatura}, Sigmund Freud
\item \textit{O destino do erudito}, Fichte
\item \textit{Diários de Adão e Eva}, Mark Twain
\item \textit{Diário de um escritor (1873)}, Dostoiévski
\item \textit{Perversão: a forma erótica do ódio}, Stoller
\item \textit{Explosao: romance da etnologia}, Hubert Fichte
\end{enumerate}\medskip

{\large\textsc{coleção <<metabiblioteca>>}}\\

\begin{enumerate}
\setlength\parskip{4.2pt}
\setlength\itemsep{-1.4mm}
\item \textit{O desertor}, Silva Alvarenga
\item \textit{Tratado descritivo do Brasil em 1587}, Gabriel Soares de Sousa
\item \textit{Teatro de êxtase}, Pessoa
\item \textit{Oração aos moços}, Rui Barbosa
\item \textit{A pele do lobo e outras peças}, Artur Azevedo
\item \textit{Tratados da terra e gente do Brasil}, Fernão Cardim 
\item \textit{O Ateneu}, Raul Pompeia
\item \textit{História da província Santa Cruz}, Gandavo
\item \textit{Cartas a favor da escravidão}, Alencar
\item \textit{Pai contra mãe e outros contos}, Machado de Assis
\item \textit{Crime}, Luiz Gama
\item \textit{Direito}, Luiz Gama
\item \textit{Democracia}, Luiz Gama
\item \textit{Liberdade}, Luiz Gama
\item \textit{A escrava}, Maria Firmina dos Reis
\item \textit{Contos e novelas}, Júlia Lopes de Almeida 
% \item \textit{A família Medeiros}, Júlia Lopes de Almeida
% \item \textit{A viúva Simões}, Júlia Lopes de Almeida
% \item \textit{Memórias de Marta}, Júlia Lopes de Almeida
% \item \textit{A falência}, Júlia Lopes de Almeida
% \item \textit{Poesia completa}, Florbela Espanca
% \item \textit{Memória}, Florbela Espanca
% \item \textit{Esaú e Jacó}, Machado de Assis
% \item \textit{Helena}, Machado de Assis
% \item \textit{Memorial de Aires}, Machado de Assis
% \item \textit{Casa Velha}, Machado de Assis
% \item \textit{Um suplício moderno e outros contos}, Monteiro Lobato
\item \textit{Transposição}, Orides Fontela
\item \textit{Helianto}, Orides Fontela
% \item \textit{Alba}, Orides Fontela
% \item \textit{Rosácea}, Orides Fontela
% \item \textit{Teia e Poemas inéditos}, Orides Fontela
\item \textit{Iracema}, Alencar
\item \textit{Auto da barca do Inferno}, Gil Vicente
\item \textit{Poemas completos de Alberto Caeiro}, Pessoa
\item \textit{A cidade e as serras}, Eça
\item \textit{Mensagem}, Pessoa
\item \textit{Utopia Brasil}, Darcy Ribeiro
\item \textit{Bom Crioulo}, Adolfo Caminha
\item \textit{Índice das coisas mais notáveis}, Vieira
\item \textit{A carteira de meu tio}, Macedo
\item \textit{Elixir do pajé --- poemas de humor, sátira e escatologia}, Bernardo Guimarães
\item \textit{Eu}, Augusto dos Anjos
\item \textit{Farsa de Inês Pereira}, Gil Vicente
\item \textit{O cortiço}, Aluísio Azevedo
\item \textit{O que eu vi, o que nós veremos}, Santos-Dumont
\item \textit{Poesia Vaginal}, Glauco Mattoso 
\end{enumerate}

\medskip
{\large\textsc{coleção <<que horas são?>>}}

\begin{enumerate}
\setlength\parskip{4.2pt}
\setlength\itemsep{-1.4mm}
\item \textit{Lulismo, carisma pop e cultura anticrítica}, Tales Ab'Sáber
\item \textit{Crédito à morte}, Anselm Jappe
\item \textit{Universidade, cidade e cidadania}, Franklin Leopoldo e Silva
\item \textit{O quarto poder: uma outra história}, Paulo Henrique Amorim
\item \textit{Dilma Rousseff e o ódio político}, Tales Ab'Sáber
\item \textit{Descobrindo o Islã no Brasil}, Karla Lima
\item \textit{Michel Temer e o fascismo comum}, Tales Ab'Sáber
\item \textit{Lugar de negro, lugar de branco?}, Douglas Rodrigues Barros
\item \textit{Machismo, racismo, capitalismo identitário}, Pablo Polese
\item \textit{A linguagem fascista}, Carlos Piovezani \& Emilio Gentile
\item \textit{A sociedade de controle}, J.\,Souza; R.\,Avelino; S.\,Amadeu (orgs.)
\item \textit{Ativismo digital hoje}, R.\,Segurado; C.\,Penteado; S.\,Amadeu (orgs.)
\item \textit{Desinformação e democracia}, Rosemary Segurado
\item \textit{Labirintos do fascismo, vol.\,1}, João Bernardo
\item \textit{Labirintos do fascismo, vol.\,2}, João Bernardo
\item \textit{Labirintos do fascismo, vol.\,3}, João Bernardo
\item \textit{Labirintos do fascismo, vol.\,4}, João Bernardo
\item \textit{Labirintos do fascismo, vol.\,5}, João Bernardo
\item \textit{Labirintos do fascismo, vol.\,6}, João Bernardo
\item \textit{8/1: A rebelião dos manés}, Pedro Fiori Arantes, Fernando Frias e Maria Luiza Meneses
\end{enumerate}

\medskip
{\large\textsc{coleção <<mundo indígena>>}}

\begin{enumerate}
\setlength\parskip{4.2pt}
\setlength\itemsep{-1.4mm}
\item \textit{A árvore dos cantos}, Pajés Parahiteri
\item \textit{O surgimento dos pássaros}, Pajés Parahiteri
\item \textit{O surgimento da noite}, Pajés Parahiteri
\item \textit{Os comedores de terra}, Pajés Parahiteri
\item \textit{A terra uma só}, Timóteo Verá Tupã Popyguá
\item \textit{Os cantos do homem-sombra}, Patience Epps e Danilo Paiva Ramos
\item \textit{A mulher que virou tatu}, Eliane Camargo
\item \textit{Crônicas de caça e criação}, Uirá Garcia
\item \textit{Círculos de coca e fumaça}, Danilo Paiva Ramos
\item \textit{Nas redes guarani}, Valéria Macedo \& Dominique Tilkin Gallois
\item \textit{Os Aruaques}, Max Schmidt
\item \textit{Cantos dos animais primordiais}, Ava Ñomoandyja Atanásio Teixeira
\item \textit{Não havia mais homens}, Luciana Storto
\item \textit{Xamanismos ameríndios}, Aristoteles Barcelos Neto, Laura Pérez Gil e Danilo Paiva Ramos (orgs.)
\end{enumerate}

%\medskip
%{\large\textsc{coleção <<artecrítica>>}}

%\begin{enumerate}
%\setlength\parskip{4.2pt}
%\setlength\itemsep{-1.4mm}
%\item \textit{Dostoiévski e a dialética}, Flávio Ricardo Vassoler
%\item \textit{O renascimento do autor}, Caio Gagliardi
%\item \textit{O homem sem qualidades à espera de Godot}, Robson de Oliveira
%\end{enumerate}

%\medskip
%{\large\textsc{coleção <<ecopolítica>>}}

%\begin{enumerate}
%\setlength\parskip{4.2pt}
%\setlength\itemsep{-1.4mm}
%\item \textit{Dostoiévski e a dialética}, Flávio Ricardo Vassoler
%\item \textit{O renascimento do autor}, Caio Gagliardi
%\item \textit{O homem sem qualidades à espera de Godot}, Robson de Oliveira
%\end{enumerate}

\pagebreak
%\medskip
{\large\textsc{coleção <<narrativas da escravidão>>}}

\begin{enumerate}
\setlength\parskip{4.2pt}
\setlength\itemsep{-1.4mm}
\item \textit{Incidentes da vida de uma escrava}, Harriet Jacobs
\item \textit{Nascidos na escravidão: depoimentos norte-americanos}, \textsc{wpa}
\item \textit{Narrativa de William W. Brown, escravo fugitivo}, William Wells Brown
\end{enumerate}

\medskip
{\large\textsc{coleção <<anarc>>}}

\begin{enumerate}
\setlength\parskip{4.2pt}
\setlength\itemsep{-1.4mm}
\item \textit{Sobre anarquismo, sexo e casamento}, Emma Goldman
\item \textit{Ação direta e outros escritos}, Voltairine de Cleyre
\item \textit{O indivíduo, a sociedade e o Estado, e outros ensaios}, Emma Goldman
\item \textit{O princípio anarquista e outros ensaios}, Kropotkin
\item \textit{Os sovietes traídos pelos bolcheviques}, Rocker
\item \textit{Escritos revolucionários}, Malatesta
\item \textit{O princípio do Estado e outros ensaios}, Bakunin
\item \textit{História da anarquia (vol.\,1)}, Max Nettlau
\item \textit{História da anarquia (vol.\,2)}, Max Nettlau
\item \textit{Entre camponeses}, Malatesta
\item \textit{Revolução e liberdade: cartas de 1845 a 1875}, Bakunin
\item \textit{Anarquia pela educação}, Élisée Reclus 
\end{enumerate}

\pagebreak	   % [lista de livros publicados]
\pagebreak


\ifodd\thepage\blankpage\fi

\mbox{}\vfill
\thispagestyle{empty}

\begin{center}
		\begin{minipage}{.7\textwidth}\tiny\noindent{}
		\centering\tiny
		Adverte-se aos curiosos que se imprimiu 1\,000 exemplares 
		deste livro na gráfica Expressão \& Arte, 
		em \today{} em papel Pólen Soft 80, em tipologia 
		Minion Pro, 11 pt, 
		com diversos sofwares livres, 
		entre eles, Lua\LaTeX, git.\\ 
		\ifdef{\RevisionInfo{}}{\par(v.\,\RevisionInfo)}{}\medskip\\\
		\adforn{64}
		\end{minipage}
\end{center}		   % [colofon]

\checkandfixthelayout
\end{document}
