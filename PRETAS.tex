\textbf{Franz Kafka} (Praga, 1883--Klosterneuburg, 1924) é um dos autores mais lidos 
e influentes do século \textsc{xx}. Oriundo de uma abastada família judaica de
comerciantes, sua infância é marcada pela relação conflituosa com o pai.
Frequenta na juventude uma escola alemã de Praga, cursa química e
posteriormente direito na Universidade Karl{}-Ferdinand, seguindo depois 
uma bem{}-sucedida carreira como funcionário público na área de segurança do trabalho. 
Além de \textit{A transformação}, Kafka publicou em vida \textit{O~foguista}, \textit{A~sentença} e \textit{O artista da fome}. \textit{O~processo}, \textit{O~castelo} e \textit{América} 
(este último, inacabado) foram publicados postumamente, graças à intervenção de 
seu amigo Max Brod, que se recusou a seguir o testamento de Kafka, no qual 
determinava a destruição de todos os seus escritos inéditos. A sua obra inclui
ainda contos, diários e uma significativa correspondência com sua noiva Felice
Bauer, que ele jamais desposaria. Falece ao 39 anos, vítima de tuberculose.

\textbf{A transformação} (\textit{Die Verwandlung}, 1915) foi publicada
originariamente na revista expressionista \textit{Weiße Blätter}, que abrigava
textos da nova geração de escritores alemães como Heinrich Mann, Ernst Bloch e
Rosa Luxemburgo. A novela narra a singular história de Gregor Samsa, um caixeiro
viajante que certo dia acorda em sua cama transformado em inseto. Plena de
significados simbólicos, a obra deu origem às mais diversas interpretações.
Famosas são as associações de ordem psicanalítica sobre a relação de Kafka 
com seu pai. Deve{}-se a Vladimir Nabokov a interpretação de que a obra não é
redutível a dramas familiares do autor mas, sim, a expressão da tensão do artista
em meio à sociedade burguesa de sua época. Uma das obras literárias mais lidas e comentadas
do século \textsc{xx}, \textit{A transformação} ganha aqui uma nova tradução direta do
alemão que procura recuperar o tom algo jocoso da obra alemã.
        
\textbf{Celso Donizete Cruz}, mestre em língua e literatura alemãs pela Universidade de São Paulo, foi professor da Universidade Federal de Sergipe. Além de obras traduzidas do alemão, inglês e italiano, é de sua autoria \textit{As metamorfoses de Kafka} (Annablume, 2008), um estudo comparativo das mais de doze traduções de \textit{A transformação} publicadas no Brasil.

\textbf{Walter Benjamin} (1892--1940), filósofo, crítico literário e ensaísta alemão, é amplamente conhecido por suas análises da modernidade, história e cultura de massa. Após a ascensão do nazismo na Alemanha, também atuou como crítico e tradutor de obras literárias, além de colaborador em jornais e revistas durante seu exílio em Paris, a partir de 1933.

\textbf{Otto Maria Carpeaux} (1900--1978) foi um crítico literário e historiador austríaco. Destacou-se por sua monumental \textit{História da literatura ocidental}. Fugindo do nazismo, se estabeleceu no Brasil em 1939, onde influenciou profundamente o pensamento literário e cultural.


